\documentclass[article,12pt,onesidea,4paper,english,brazil]{abntex2}

\usepackage{lmodern, indentfirst, nomencl, color, graphicx, microtype, lipsum}			
\usepackage[T1]{fontenc}		
\usepackage[utf8]{inputenc}		

\setlrmarginsandblock{2cm}{2cm}{*}
\setulmarginsandblock{2cm}{2cm}{*}
\checkandfixthelayout

\setlength{\parindent}{1.3cm}
\setlength{\parskip}{0.2cm}

\SingleSpacing

\begin{document}
	
	\selectlanguage{brazil}
	
	\frenchspacing 
	
	\begin{center}
		\LARGE UTILIZAÇÃO DE HERBICIDAS PRÉ-EMERGENTES PARA CONTROLE DE
		PLANTAS DANINHAS\\NA CULTURA DA SOJA\footnote{Trabalho realizado dentro da área de Conhecimento CNPq: Ciências Agrárias com financiamento do Conselho Nacional de Desenvolvimento Científico e Tecnológico (CNPq).}
		
		\normalsize
		Rafael dos Santos Oliveira\footnote{Bolsista (Superior), rafaelsantosoliveira16@gmail.com, \textit{Campus} Colorado do Oeste.} 
		Vitor Freitas Silva\footnote{Colaborador(a), vitoragro7@gmail.com, \textit{Campus} Colorado do Oeste.} \\
		Marcos Aurélio Anequine de Macedo\footnote{Orientador(a), marcos.anequine@ifro.edu.br, \textit{Campus} Colorado do Oeste.} 
		Hugo de Almeida Dan\footnote{Co-orientador(a), halmeidadan@gmail.com .} 
	\end{center}
	
	\noindent O uso de herbicidas pré-emergentes vem se demonstrando cada vez mais
	necessário para o uso na agricultura contemporânea. Assim, objetivou-se com esse
	trabalho realizar a avaliação de diferentes tipos de produtos com eficácia residual
	fazendo-se uso de aplicação em pós-emergência para o controle de plantas
	daninhas. O delineamento experimental utilizado foi em blocos casualizados com
	três repetições, arranjado em esquema de parcelas subdivididas 2x5+1
	correspondendo a dois modos de aplicação, pré e pós-emergência, 5 tipos de
	herbicidas: imazethapyr, clomazone+carfentrazone,
	clomazone+carfentrazone+sulfentrazone, diclosulam, imazethapyr+flumioxazin e
	uma 1 testemunha. A princípio aplicou-se diferentes tipos de herbicidas em pré-
	emergência para área total das parcelas, passado cerca de vinte e cinco dias
	realizou-se a divisão das parcelas ao meio e aplicado glyphosate em pós-
	emergência, caracterizando-se como um tratamento sequencial ou dobrado. A área
	utilizada para ensaio apresentou basicamente a incidência de quatro espécies de
	plantas daninhas sendo estas: Corda de viola (\textit{commelina benghalensis}), capim colchão (\textit{Digitaria horizontalis}), fedegoso (\textit{Senna occidentalis}) e mata pasto (\textit{Diodia teres}). Devido a maior incidência de mata pasto realizou-se avaliação específica sobre o seu controle nos diferentes tipos de tratamentos. Para a utilização de
	somente pré-emergentes a estatística demonstrou que os tratamentos onde foram
	utilizados clomazone+carfentrazone e diclosulam obteve-se os menores índices de
	controle para mata pasto, atingindo um percentual entre 23 e 28%. O uso de
	imazethapyr e clomazone+carfentrazone+sulfentrazone apresentou resultados
	semelhantes aos outros dois anteriormente citados e também sendo considerados
	estatisticamente iguais ao uso de imazethapyr+flumioxazin no qual apresentou
	médias superiores a todos os outros tratamentos. Para utilização de sequencial
	observou-se que o uso de diclosulam passou a ser estatisticamente igual ao
	imazethaphyr+flumioxazin. A utilização de glyphosate em pós-emergência contribuiu
	para o aumento de cerca de 50\% para todos os tratamentos em questão. Avaliando-
	se o uso de somente pré-emergente com uso de pré + pós-emergente concluiu-se
	que todos os tratamentos mostraram-se superiores quando utilizado o glyphosate
	em pós-emergência. De acordo com a discussão, conclui-se que, o uso de
	imazethapyr+flumioxazin foi o tratamento no qual apresentou maior percentual de
	controle de mata pasto e o uso de pós-emergente mostrou-se como importante
	ferramenta para um manejo mais eficiente de plantas daninhas.
	
	\vspace{\onelineskip}
	
	\noindent
	\textbf{Palavras-chave}: Mata pasto. Período residual. Sequencial.	
	
\end{document}