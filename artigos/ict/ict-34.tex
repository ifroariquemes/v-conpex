\documentclass[article,12pt,onesidea,4paper,english,brazil]{abntex2}

\usepackage{lmodern, indentfirst, nomencl, color, graphicx, microtype, lipsum,textcomp}			
\usepackage[T1]{fontenc}		
\usepackage[utf8]{inputenc}		

\setlrmarginsandblock{2cm}{2cm}{*}
\setulmarginsandblock{2cm}{2cm}{*}
\checkandfixthelayout

\setlength{\parindent}{1.3cm}
\setlength{\parskip}{0.2cm}

\SingleSpacing

\begin{document}
	
	\selectlanguage{brazil}
	
	\frenchspacing 
	
	\begin{center}
		\LARGE TESTE DE TETRAZÓLIO EM SEMENTES DE FEIJÃO-MACUCO\footnote{Trabalho realizado dentro da área de Ciências Agrárias, com financiamento do CNPq e do IFRO.}
		
		\normalsize
		Jessica Fernandes Dias\footnote{Bolsista PIBIC-Af, jessiagro12@gmail.com, \textit{Campus} Colorado do Oeste.} 
		Marcelo Resende da Silva\footnote{Bolsista PIBIC-EM, marcelo.resende.s2901@gmail,com , \textit{Campus} Colorado do Oeste.} 
		Ernando Balbinot\footnote{Orientador ernando.balbinot@ifro.edu.br , \textit{Campus} Colorado do Oeste.} 
		Fabio Batista de Lima\footnote{Co-Orientador fabio.lima@ifro.edu.br , \textit{Campus} Colorado do Oeste.} 
	\end{center}
	
	\noindent O feijão-macuco (\textit{Pachyrhizus tuberosus (Lam.) Spreng.}) é uma hortaliça não
	convencional pertencente à família Fabaceae, originária das cabeceiras do rio
	Amazonas. No entanto a viabilidade sobre a qualidade da semente de feijão-macuco
	pelo teste de tetrazólio é desconhecida. Nesse teste as sementes são embebidas
	em uma solução incolor de 2,3, 5 trifenil brometo de tetrazólio como um indicador
	para revelar o processo de redução que acontece dentro das células vivas das
	sementes, onde íons de hidrogênio são transferidos para o referido sal. O objetivo
	deste trabalho foi estabelecer critérios para condução do teste de tetrazólio em
	sementes de feijão macuco, determinando grupos de viabilidade. O trabalho foi
	conduzido no Instituto Federal de Educação Ciência e Tecnologia de Rondônia,
	Campus Colorado do Oeste. Para o teste de tetrazólio utilizou-se um mesmo lote de
	sementes, colocadas para hidratar entre papel umedecido por 16 horas a 20°C. Na
	etapa de coloração, após testes preliminares, foram avaliadas as seguintes
	combinações de concentração do sal de tetrazólio e tempo de coloração: 0,5\%,
	0,75\% e 1\%, em dois períodos, 2 e 4 horas, à temperatura de 40°C. Para fazer a
	relação da coloração com a viabilidade foi realizado o teste de germinação onde as
	sementes foram mantidas em câmara de germinação tipo BOD em rolo de papel por
	10 dias, com 4 repetições de 50 sementes a temperatura de 25°C. Houve interação
	entre os tempos e concentrações de tetrazólio. A melhor combinação foi observada
	na concentração de 0,05\% por 4 horas. Assim, conclui-se que a avaliação da
	viabilidade das sementes de feijão-macuco por meio do teste de tetrazólio é eficiente
	mediante hidratação das sementes por 16 horas a 20°C, coloração por imersão em
	solução de tetrazólio a 0,5\%, durante 4 horas, a 40°C, sendo estabelecido e
	ilustrado três grupos de viabilidade.
	
	\vspace{\onelineskip}
	
	\noindent
	\textbf{Palavras-chave}: \textit{Pachyrhizus tuberosus}. Concentração de Tetrazólio. Período de coloração.	
	
\end{document}