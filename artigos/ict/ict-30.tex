\documentclass[article,12pt,onesidea,4paper,english,brazil]{abntex2}

\usepackage{lmodern, indentfirst, nomencl, color, graphicx, microtype, lipsum, textcomp}			
\usepackage[T1]{fontenc}		
\usepackage[utf8]{inputenc}		

\setlrmarginsandblock{2cm}{2cm}{*}
\setulmarginsandblock{2cm}{2cm}{*}
\checkandfixthelayout

\setlength{\parindent}{1.3cm}
\setlength{\parskip}{0.2cm}

\SingleSpacing

\begin{document}
	
	\selectlanguage{brazil}
	
	\frenchspacing 
	
	\begin{center}
		\LARGE POTENCIAL ALELOPÁTICO DE RAIZ E/OU PARTE ÁREA DE \textit{ELIUSINE} INDICA SOBRE O DESENVOLVIMENTO INICIAL DE\\ \textit{LACTUCA SATIVA} (ALFACE)
		
		\normalsize
		Josiel Faustino da Cruz\footnote{Discente do Curso de Engenharia Agronômica. Bolsista CNPq, Modalidade PIBIC – IFRO \textit{Campus} Colorado do Oeste. e-mail: josielfaustino@gmail.com .} 
		Rafael dos Santos Oliveira\footnote{Discente do Curso de Engenharia Agronômica. Colaborador - IFRO \textit{Campus} Colorado do Oeste. e-mail: rafaelsantosoliveira16@gmail.com .} 
		Edna Gomes Oliveira\footnote{Discente do Curso de Licenciatura em ciências Biológicas. Colaboradora - IFRO \textit{Campus} Colorado do Oeste. e-mail: ednaoliveira187@gmail.com.} \\
		Juliana Juchnievski de Oliveira\footnote{Discente do Curso de Licenciatura em ciências Biológicas. Colaboradora - IFRO \textit{Campus} Colorado do Oeste. julychuvineski@gmail.com .}
		Nelio Ranielly de Paula\footnote{Professor do IFRO \textit{Campus} Colorado do Oeste. Orientador. e-mail: nelio.ferreira@ifro.edu.br}
		Marcos Aurélio Anequine Macedo\footnote{Professor do IFRO \textit{Campus} Colorado do Oeste. Co-Orientador. e-mail: marcos.anequine@ifro.edu.br}
		 
	\end{center}
	
	\noindent Alelopatia são substâncias químicas, capazes de apresentar efeitos
	benéficos ou maléficos sobre as plantas. Pouco se sabe sobre a liberação de
	aleloquímicos na natureza, baseando neste preceito, o estudo teve por objetivo
	avaliar o potencial alelopático da planta daninha \textit{Eliusine} indica no crescimento do
	de \textit{Lactuca sativa}. Para isso o estudo foi realizado em duas etapas; sendo esta a
	coleta realizada na zona rural do município de Colorado do Oeste/RO em março de
	2017 e após coletada a planta foi identificada e a exsicata e depositada no herbário
	da instituição. O material vegetal \textit{in natura} da espécie selecionada foi submetido ao processo de secagem em estufa de circulação e renovação de ar à 45°C por 48 horas. A planta seca foi pulverizada e armazenada em frasco âmbar. O experimento
	foi montado no Laboratório de química do Instituto Federal de Ciência e Tecnologia
	de Rondônia extração de por meio de secagem e decocção para obtenção do
	extrato aquoso nas concentrações de 100\%, e depois denominado \textit{Eliusine}$_{100}$. Para
	o ensaio de concentração 50\%, água destilada foi adicionada à solução, \textit{Eliusine}$_1$ na
	proporção 1:1, aqui e depois denominada \textit{Eliusine}$_{50}$. Para o ensaio de concentração
	25\%, água destilada foi utilizada na proporção 3:1, aqui e depois denominado
	\textit{Eliusine}$_{25}$. Finalmente, para a concentração 10\%, água destilada foi adicionada na
	proporção 9:1, aqui e depois denominada \textit{Eliusine}$_{10}$. Posteriormente foi realizado
	ensaio de germinação em delineamento experimental inteiramente casualizado, com
	4 repetições de 5 sementes cada, a verificação do potencial alelopático de \textit{Eliusine}
	nas diversas concentrações foi realizada realizada medindo-se o comprimento da
	raiz e do hipocótilo de plantulas de alface. Os dados coletados foram tratados no
	software estatístico, Assistat versão 7.7. Ao final do projeto foi possível identificar
	que \textit{E. indica} (capim pe de galinha) apresentou resposta negativa para crescimento
	da \textit{Lactuca sativa} (alface) em concentrações menores que 50\% e para crescimento
	do hipocótilo não apresenta diferença significativa em concentração de 100\%.
	
	\vspace{\onelineskip}
	
	\noindent
	\textbf{Palavras-chave}: Planta daninha. Alelopatia. Hipocótilo. Extratos vegetais.	
	
\end{document}