\documentclass[article,12pt,onesidea,4paper,english,brazil]{abntex2}

\usepackage{lmodern, indentfirst, nomencl, color, graphicx, microtype, lipsum}			
\usepackage[T1]{fontenc}		
\usepackage[utf8]{inputenc}		

\setlrmarginsandblock{2cm}{2cm}{*}
\setulmarginsandblock{2cm}{2cm}{*}
\checkandfixthelayout

\setlength{\parindent}{1.3cm}
\setlength{\parskip}{0.2cm}

\SingleSpacing

\begin{document}
	
	\selectlanguage{brazil}
	
	\frenchspacing 
	
	\begin{center}
		\LARGE MONITORAMENTO DA RESISTÊNCIA DE \textit{DIGITARIA INSULARIS} A
		HERBICIDAS INIBIDORES DE EPSPS NO CONE SUL DE RONDÔNIA
		
		\normalsize
		Vitor Freitas Silva\footnote{Bolsista (PIBIB-AF) vitoragro7@gmail.com. Colorado do oeste.} 
		Heliabe Silva dos Santos\footnote{Colaborador(a) heliabesantiin@gmail.com, Colorado do oeste.} \\
		Marcos Aurelio Aniquine de Macedo\footnote{Orientador(a), marcos.anequine@ifro.edu.br, Colorado do oeste.} 
		Hugo de Almeida Dan\footnote{Co-orientador(a), halmeidadan@gmail.com Colorado do oeste.} 
	\end{center}
	
	\noindent A resistência de plantas daninhas aos herbicidas é a capacidade natural e herdável
	de alguns biótipos, dentro de uma determinada população, de sobreviver. O objetivo
	desse trabalho foi determinar a existência de \textit{Digitaria insularis} resistente a
	herbicidas inibidores de EPSPs no Cone Sul de Rondônia. foi realizado no campus
	experimental do instituto (IFRO) \textit{Campus} Colorado do Oeste. De início realizou-se a
	coleta das semestes em cada município no qual apresentavam adequada
	maturidade fisiológica. Para o biótipo susceptível, coletou em áreas sem aplicação
	de herbicidas. Foram semeadas 3 sementes de cada biótipo por vasos de 15 cm de
	diâmetro e capacidade de 3 litros. A aplicação foi realizada com pulverizador de
	precisão equipado com três pontas XR 100.02, espaçadas de 0,50 m, mantendo a
	pressão de 40 libras, pressurizado e acoplado com cilindro de CO$_2$, aplicando-se
	volume de calda de 200 L/ha. O ensaio, foi conduzido em delineamento inteiramente
	casualizado (DIC), com esquema fatorial 2x3 correspondendo a dois biótipos
	(Resistente e Susceptível) e três doses de herbicida (0,0; 4,0 e 8,0 L/ha) com 4
	repetições. Realizou aplicação quando o capim apresentou quatro folhas
	verdadeiras. Para avaliação, utilizou-se a escala percentual de ALAM atribuindo-se
	notas de 0 a 100\%, onde 0 representa ausência de controle e 100\% a morte de
	todas as plantas. Aos 28 dias após a aplicação de 4L/ha constatou que os biótipos
	de Vilhena, Colorado do Oeste, Pimenteiras do Oeste e Cabixi apresentaram
	indícios de resistência, evidenciando insatisfatório nível de controle: 42, 78, 51 e
	48\% seguidamente. Para o biótipo susceptível e Corumbiara constatou-se nível de
	controle de 100\%. O biótipo de Cerejeiras apresentou nível de controle de cerca de
	89\% estando acima do mínimo exigido de controle de 80\%. A dose de 8L/\textit{há}
	observou-se que apesar do aumento, os municípios de Vilhena, Pimenteiras do
	Oeste e Cabixi apresentou nível insatisfatório de controle atingindo percentuais de
	52, 52 e 50\% respectivamente. Assim, conclui-se que os biótipos avaliados de \textit{D.
	insularis} nos municípios de Vilhena, Pimenteiras do Oeste e Cabixi demonstraram-se
	como resistentes ao uso de glyphosate, necessitando de dosagens mínimas de
	15;15 e 16 L/\textit{há} respectivamente para controlar.
	
	\vspace{\onelineskip}
	
	\noindent
	\textbf{Palavras-chave}: Capim-amargoso. Glyphosate. Tolerância.	

\end{document}