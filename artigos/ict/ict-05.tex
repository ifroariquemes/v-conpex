\documentclass[article,12pt,onesidea,4paper,english,brazil]{abntex2}

\usepackage{lmodern, indentfirst, nomencl, color, graphicx, microtype, lipsum}			
\usepackage[T1]{fontenc}		
\usepackage[utf8]{inputenc}		

\setlrmarginsandblock{2cm}{2cm}{*}
\setulmarginsandblock{2cm}{2cm}{*}
\checkandfixthelayout

\setlength{\parindent}{1.3cm}
\setlength{\parskip}{0.2cm}

\SingleSpacing

\begin{document}
	
	\selectlanguage{brazil}
	
	\frenchspacing 
	
	\begin{center}
		\LARGE ANÁLISE QUALITATIVA DA PRESENÇA DE METAIS PESADOS NO MATERIAL		
		PARTICULADO (MP$_{10}$) DO MUNICÍPIO DE JI-PARANÁ (RO)\footnote{Trabalho realizado dentro da Química com financiamento do CNPq.}
		
		\normalsize
		Camila Ellen Ferreira Oliveira\footnote{Bolsista (PIBIC - Af), camilaefoliveira@gmail.com, \textit{Campus} Ji-Paraná.} 
		Clarice Machado Ramos dos Santos\footnote{Bolsista (PIBIC – Júnior), clariceramos82@gmail.com, \textit{Campus} Ji-Paraná.} 
		Luis Fernando Lira Souto\footnote{Colaborador, luis.lira@ifro.edu.br, \textit{Campus} Calama.} \\
		M. S. Azevedo\footnote{Colaboradora, mari@unir.br, UNIR – \textit{Campus} Porto-Velho.}
		Alecsandra Oliveira de Souza\footnote{Co-orientadora, alecsandra.souza@ifro.edu.br, \textit{Campus} Ji-Paraná.}
		Luiz Américo da Silva do Vale\footnote{Orientador, luiz.americo@ifro.edu.br, \textit{Campus} Ji-Paraná.} 
	\end{center}
	
	\noindent O índice de poluição atmosférica a cada ano cresce, poluição essa não apenas
	advinda de fontes antropogênicas, mas também resultante de focos de queimadas,
	acréscimo em frota rodoviária ocasionada pelo crescimento demasiado das
	populações. Como contribuinte da massa atmosférica poluidora que circunda as
	correntes de ar do planeta, está o Material Particulado (MP), estes possui uma
	composição vasta. Os MP podem ser classificados de acordo com o seu tamanho,
	sendo os finos, ultrafinos e grosso, destes os mais estudados são os inaláveis os
	quais podem penetrar os alvéolos e ocasionar doenças no trato respiratório, sendo
	os metais os quais possuem índices tóxicos elevados à saúde humana, relacionado
	a enfermidades como câncer no pulmão, doenças cardiovasculares, dentre outras.
	Assim, considerando os elevados índices de MP na região norte, a presença natural
	de diferentes poluentes e a ausência de estudos acerca da qualidade do ambiente
	atmosférico da região de Ji-Paraná, avaliou-se a presença de Metais Pesados no
	MP da região por de Fluorescência de Raios X (FRX), análise qualitativas da matriz
	de ar. O MP fora coletado em membrana de policarbonato por exposição dos filtros
	por sete dias em três principais avenidas (Brasil, Marechal e Transcontinental) da
	cidade, por impactação natural. Após coletado, foram submetidos a Análise de FRX,
	qualificando assim a presença dos metais presentes no MP coletado. Como
	resultante da pesquisa, observou-se a presença de metais de perfil toxicológicos já
	caracterizados na literatura, sendo eles comuns nas três extensões de análise, Cu,
	Fe, Co e tendo Zn presente na rodovia de transporte de caminhões. Resultados
	estes, equivalentes a pesquisas em centros urbanos, cuja frota rodoviária
	caracteriza os metais suspensos no ar, em decorrência não apenas de fontes
	naturais, mas da queima dos combustíveis, estes podem estar relacionados a
	índices de doenças respiratórias a AVC registradas no estado nos últimos anos,
	sendo que estas atingem na sua maioria as zonas de risco (crianças e idosos).
	Assim, verifica-se que o fluxo veicular culmina nos índices de metais presentes no
	MP da região, bem como os fatores de risco toxicológicos que podem apresentar para os 
	envoltos a este.
	
	\vspace{\onelineskip}
	
	\noindent
	\textbf{Palavras-chave}: Metais pesados. Material particulado. Fluorescência de raios-X.
	
\end{document}