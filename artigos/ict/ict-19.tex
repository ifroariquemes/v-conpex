\documentclass[article,12pt,onesidea,4paper,english,brazil]{abntex2}

\usepackage{lmodern, indentfirst, nomencl, color, graphicx, microtype, lipsum}			
\usepackage[T1]{fontenc}		
\usepackage[utf8]{inputenc}		

\setlrmarginsandblock{2cm}{2cm}{*}
\setulmarginsandblock{2cm}{2cm}{*}
\checkandfixthelayout

\setlength{\parindent}{1.3cm}
\setlength{\parskip}{0.2cm}

\SingleSpacing

\begin{document}
	
	\selectlanguage{brazil}
	
	\frenchspacing 
	
	\begin{center}
		\LARGE DOSES E FONTES DE FÓSFORO SULÚVEL SOB A EFICIÊNCIA NUTRICIONAL
		DE GENÓTIPOS DE FEIJÃO-CAUPI\footnote{Trabalho realizado dentro da área de Ciências Agrárias com financiamento do CNPq.}
		
		\normalsize
		Caiqui Roni Gomes Ferreira\footnote{Bolsista Pibiti CNPq, caiquiraoni@gmail.com, \textit{Campus} Colorado do Oeste.} 
		Jéssica Schiochet\footnote{Colaboradora, jessicaschiochet11@gmail.com, \textit{Campus} Colorado do Oeste.} 
		Érica de Oliveira Araújo\footnote{Orientadora, erica.araujo@ifro.edu.br, Professora colaboradora \textit{Campus} Colorado do Oeste.} 
		Wilk Sampaio de Almeida\footnote{Co-orientador, wilk.almeida@ifro.edu.br, \textit{Campus} Colorado do Oeste.} 
	\end{center}
	
	\noindent Em virtude do alto custo dos fertilizantes fosfatados, e também, da sua adsorção,
	torna-se necessário à busca por genótipos, fontes e doses de P que possam
	maximizar a eficiência de utilização do nutriente pela cultura do feijão-caupi,
	refletindo em máximas produtividades com baixo custo. Neste sentido, objetivou-se
	com o presente trabalho avaliar os efeitos das doses e fontes de P solúvel sobre os
	parâmetros fitotécnicos e absorção de fósforo por diferentes genótipos de feijão-
	caupi cultivado em ambiente protegido. O delineamento experimental utilizado foi o
	inteiramente casualizado em esquema fatorial 4x2x2, sendo, quatro doses de fósforo
	(0, 40, 80 e 120 kg ha$^{-1}$), duas fontes solúveis de P$_2$O$_5$ (SFS (21\% P$_2$O$_5$) e SFT (44\% P$_2$O$_5$)) e duas cultivares de feijão-caupi (BRS Tumucumaque e BRS Novaera), com
	quatro repetições. Na ocasião da colheita foi determinada a altura de plantas,
	diâmetro do caule, volume de raiz, teor de N e P nas diferentes partes da planta (raiz
	e parte aérea) e as eficiências de absorção e utilização de N e P. Os dados foram
	submetidos à análise de variância, após os quais foram submetidos à análise de
	regressão. Os resultados permitiram concluir que há variabilidade quanto à eficiência
	e resposta à aplicação de Fósforo entre os genótipos de feijão-caupi. O genótipo
	BRS Novaera mostrou-se mais eficiente quanto à absorção e transporte de fósforo.
	O superfosfato triplo apresenta maior eficiência na absorção de P pelas raízes que o
	superfosfato simples. O aumento nas doses de fósforo promove incremento nas
	concentrações e na eficiência de utilização do fósforo por plantas de feijão-caupi.
	
	\vspace{\onelineskip}
	
	\noindent
	\textbf{Palavras-chave}: \textit{Vigna unguiculata} (L). Estado nutricional. P. Adubação.
	
\end{document}