\documentclass[article,12pt,onesidea,4paper,english,brazil]{abntex2}

\usepackage{lmodern, indentfirst, nomencl, color, graphicx, microtype, lipsum}			
\usepackage[T1]{fontenc}		
\usepackage[utf8]{inputenc}		

\setlrmarginsandblock{2cm}{2cm}{*}
\setulmarginsandblock{2cm}{2cm}{*}
\checkandfixthelayout

\setlength{\parindent}{1.3cm}
\setlength{\parskip}{0.2cm}

\SingleSpacing

\begin{document}
	
	\selectlanguage{brazil}
	
	\frenchspacing 
	
	\begin{center}
		\LARGE AVALIAÇÃO DA CULTURA DA SOJA PRODUZIDA NA INTEGRAÇÃO
		SILVIAGRÍCOLA COM EUCALIPTO NO SEGUNDO ANO DE IMPLANTAÇÃO DO SISTEMA ILPF\footnote{Trabalho realizado dentro das Ciências Agrárias com financiamento do CNPq.}
		
		\normalsize
		Roberto Dias Marinho\footnote{Bolsista (PIBIC-Af), roberto\_dias\_marinho@hotmail.com, Campus Colorado do Oeste.} 
		Pablo Pedro Fernandes de Sousa\footnote{Bolsista (Pibc EM), pablopedro.fernandes@gmail.com, Campus Colorado do Oeste.} 
		Gisely Cristina da Silva\footnote{Colaboradora, giseelys@gmail.com , Campus Colorado do Oeste.} 
		Ernando Balbinot\footnote{Orientador, ernando.balbinot@ifro.edu.br, Campus Colorado do Oeste.} 
	\end{center}
	
	\noindent Integração lavoura-pecuária-floresta é um sistema que visa integrar e diversificar as
	atividades agropecuárias, podendo proporcionar uma série de benefícios. Tal
	sistema consiste na produção de espécies florestais juntamente com culturas
	anuais, consorciadas ou não com plantas forrageiras, nos primeiros anos de
	implantação e posteriormente o pastejo das forrageiras por animais. É um sistema
	de produção sustentável e justifica-se, sobretudo para Rondônia, que tem muita
	pastagem em estádio avançado de degradação, com necessidade de renovação. O
	eucalipto é a espécie florestal que mais se destaca na integração e a soja é uma
	cultura anual de grande interesse econômico e viabilidade na integração. Assim, a
	pesquisa teve como objetivo avaliar a cultura da soja no segundo ano de
	implantação do sistema de ILPF, em função da distância da cultura à espécie
	arbórea. O trabalho foi realizado no Instituto Federal de Rondônia, microrregião de
	Colorado do Oeste. Foi utilizado o delineamento em blocos casualizados com quatro
	repetições. A soja foi cultivada entre renques de eucalipto, plantados em fileiras
	duplas, espaçadas em 26,0 m, na orientação Leste-Oeste. Os tratamentos
	constituíram de 11 linhas de plantio de soja, considerando a distância em relação ao
	eucalipto (linhas 1, 3, 6, 10 e 15, nas direções sul e norte, além da linha central).
	Foram avaliados o número de vagens por plantas; número de grãos por vagens;
	número de grãos por planta e a produtividade da soja. Os dados foram submetidos à
	análise de variância e as médias comparadas pelo teste de Tukey a 5\% de
	significância. Os resultados obtidos revelaram a redução na produtividade da soja
	nas linhas mais próximas dos renques, especificamente as linhas 1 e 3 na direção
	Sul e a linha 1 na direção Norte. A restrição da radiação solar em função do
	sombreamento, principalmente nas primeiras horas do dia está relacionada aos
	resultados.
	
	\vspace{\onelineskip}
	
	\noindent
	\textbf{Palavras-chave}: Produção integrada. Componente arbóreo. Componente agrícola.
	
\end{document}