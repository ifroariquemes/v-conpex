\documentclass[article,12pt,onesidea,4paper,english,brazil]{abntex2}

\usepackage{lmodern, indentfirst, nomencl, color, graphicx, microtype, lipsum}			
\usepackage[T1]{fontenc}		
\usepackage[utf8]{inputenc}		

\setlrmarginsandblock{2cm}{2cm}{*}
\setulmarginsandblock{2cm}{2cm}{*}
\checkandfixthelayout

\setlength{\parindent}{1.3cm}
\setlength{\parskip}{0.2cm}

\SingleSpacing

\begin{document}
	
	\selectlanguage{brazil}
	
	\frenchspacing 
	
	\begin{center}
		\LARGE PARÂMETROS AGRONÔMICOS, MORFOLÓGICOS E PRODUTIVIDADE DE
		MASSA VERDE DE SORGO \\BIOMASSA COLHIDO EM DIFERENTES\\ ÉPOCAS E
		ALTURAS DE CORTE PARA\\ PRODUÇÃO DE SILAGEM\footnote{Trabalho realizado dentro da área de Conhecimento CNPq: Ciências Agrárias com financiamento do IFRO.}
		
		\normalsize
		Natanael Maikon dos Santos\footnote{Bolsista PIBITI, natanaelmaikon42@gmail.com, \textit{Campus} Colorado do Oeste.} 
		Wender Mateus Peixoto\footnote{Colaborador, wender.mpeixoto@gmail.com, \textit{Campus} Colorado do Oeste.} \\
		Rafael Henrique Pereira dos Reis\footnote{Orientador, rafael.reis@ifro.edu.br, \textit{Campus} Colorado do Oeste.} 
		Ernando Balbinot\footnote{Co-orientador, ernando.balbinot@ifro.edu.br, \textit{Campus} Colorado do Oeste.} 
	\end{center}
	
	\noindent Devido à alta produtividade em comparação com outras culturas anuais utilizadas
	para ensilagem, o sorgo biomassa tem despertado o interesse de pecuaristas para
	sua utilização como alternativa na produção de volumoso, a fim de suprir as
	necessidades nutricionais no período de escassez de alimento a pasto. No entanto,
	para esta finalidade é necessário que se conheça os aspectos nutricionais da planta.
	A qualidade da silagem produzida pode estar relacionada com a altura de corte na
	plataforma da colhedora de forragens, além da época de colheita que deve ser
	definida para que se ensile a planta no seu ponto de maior valor nutritivo e
	produtividade. Objetivou-se avaliar parâmetros agronômicos, morfológicos e a
	produtividade de massa verde de sorgo biomassa, colhido em diferentes épocas e
	alturas de corte. O delineamento utilizado foi em blocos casualizados arranjado em
	esquema fatorial 3x4, com três repetições, sendo três épocas de colheita (estádio de
	emborrachamento, grão pastoso e grão farináceo) e quatro alturas de corte da
	planta (a partir do solo em: 0,30 m, 0,60 m, 0,90 m e 1,20 m). Os resultados foram
	submetidos à análise de variância, e quando o teste F foi significativo, foram
	submetidos ao teste de médias Tukey (P<0,05), utilizando-se o programa estatístico
	Assistat. Houve efeito significativo da época de colheita sobre o número de folhas
	verdes (NFV), com média de 7,18 folhas verdes.planta$^{-1}$, diferentemente do número
	de folhas mortas (NFM) que apresentou efeito para cada um dos fatores, com média
	de 3,05 folhas mortas.planta$^{-1}$. O NFV está relacionado com o estádio fenológico da planta, mas a altura de corte não deve influenciar nesse parâmetro. Já o NFM pode ser influenciado tanto pela época como pela altura de corte, considerando que o corte ocorre no terço inferior da planta, onde há maior concentração de folhas mortas. Observou-se que não houve efeito significativo dos fatores sobre a
	produtividade que obteve média de 24,48 t.ha$^{-1}$, resultado que pode estar
	relacionado com a época de semeadura que não favoreceu a planta desempenhar o
	seu máximo potencial produtivo, inviabilizando a utilização de sorgo biomassa para
	a produção de silagem.
	
	\vspace{\onelineskip}
	
	\noindent
	\textbf{Palavras-chave}: Energia renovável. Estacionalidade de produção forrageira.
	\textit{Sorghum bicolor}.	
	
\end{document}