\documentclass[article,12pt,onesidea,4paper,english,brazil]{abntex2}

\usepackage{lmodern, indentfirst, nomencl, color, graphicx, microtype, lipsum,textcomp}			
\usepackage[T1]{fontenc}		
\usepackage[utf8]{inputenc}		

\setlrmarginsandblock{2cm}{2cm}{*}
\setulmarginsandblock{2cm}{2cm}{*}
\checkandfixthelayout

\setlength{\parindent}{1.3cm}
\setlength{\parskip}{0.2cm}

\SingleSpacing

\begin{document}
	
	\selectlanguage{brazil}
	
	\frenchspacing 
	
	\begin{center}
		\LARGE VULNERABILIDADE AO ESTRESSE EM ESTUDANTES DO ENSINO TÉCNICO \\INTEGRADO AO MÉDIO\footnote{Trabalho realizado na área de Educação Física, co-financiamento da Fundação Rondônia de Amparo ao
		Desenvolvimento das Ações Científicas e Tecnológicas e à Pesquisa do Estado de Rondônia (FAPERO) e pelo Conselho Nacional de Desenvolvimento Científico e Tecnológico (CNPq).}
		
		\normalsize
		Maria Vitória Dunice Pereira\footnote{Colaboradora, mvitoriadunice@gmail.com, \textit{Campus} Porto Velho Calama.} 
		Iranira Geminiano de Melo\footnote{Orientadora, iranira.melo@ifro.edu.br, \textit{Campus} Porto Velho Calama.} 
		George Madson Dias Santos\footnote{Co-orientador, george.santos@ifro.edu.br, \textit{Campus} Porto Velho Calama.} 
	\end{center}
	
	\noindent O estudo da vulnerabilidade ao estresse em estudantes do ensino técnico
	integrado ao médio, realizado pelo grupo de pesquisa GESSTEC (Grupo de
	Estudos, Saúde, Sociedade e Tecnologia), teve por objetivo pesquisar o estilo de
	vida dos estudantes do Campus Calama. Metodologicamente, a pesquisa foi
	desenvolvida através de um questionário digital intitulado “Estresse”, composto por
	dez questões onde as respostas variavam entre: alternativa a) Sempre verdadeiro;
	b) Geralmente verdadeiro; c) Geralmente Falso e d) Sempre falso. Esse questionário
	foi desenvolvido em formato digital, hospedado no site do GESSTEC. Foram mais
	de 400 acessos, no entanto, para este estudo foram considerados apenas aqueles
	que iniciaram e concluíram o questionário, totalizando 134 estudantes com idades
	entre quatorze e dezenove anos, sendo eles 44,78\% meninos e 55,22\% meninas.
	Os resultados evidenciaram que 58,21\% dos estudantes devem reagir de forma
	mais tranquila, 39,55\% estão no caminho certo e 2,24\% possuem vulnerabilidade ao
	estresse. A utilização da ferramenta digital tem vantagem tanto para os
	pesquisadores quanto para os participantes, para os primeiros tem a possibilidade
	de obtenção dos dados já prontos para a análise e para os últimos o conhecimento
	de sua situação simultaneamente após responder ao questionário, pois há um
	\textit{feedback} imediato. Esta prática pode vir a despertar a mudança de alguns hábitos,
	sendo eles: a busca de mecanismos que estimulem a sensação de satisfação e
	relaxamento, por exemplo, através da prática de esportes, a adoção de uma
	alimentação mais saudável e o convívio em meios sociais sadios. A partir deste
	estudo é possível pensar em ambientes escolares que estimulem as práticas
	saudáveis. A pesquisa ainda pode sugerir que a tecnologia deve ser aproveitada ao
	máximo com a intenção de melhorar o desenvolvimento das pesquisas e dos
	pesquisados, e nesse caso, tendo o indivíduo com foco naquilo que é sem dúvidas o
	mais importante, a sua saúde.	
	
\end{document}
	
	\vspace{\onelineskip}
	
	\noindent
	\textbf{Palavras-chave}: Estresse. Saúde. Vulnerabilidade.