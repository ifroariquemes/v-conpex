\documentclass[article,12pt,onesidea,4paper,english,brazil]{abntex2}

\usepackage{lmodern, indentfirst, nomencl, color, graphicx, microtype, lipsum}			
\usepackage[T1]{fontenc}		
\usepackage[utf8]{inputenc}		

\setlrmarginsandblock{2cm}{2cm}{*}
\setulmarginsandblock{2cm}{2cm}{*}
\checkandfixthelayout

\setlength{\parindent}{1.3cm}
\setlength{\parskip}{0.2cm}

\SingleSpacing

\begin{document}
	
	\selectlanguage{brazil}
	
	\frenchspacing 
	
	\begin{center}
		\LARGE O CINEMA COMO EXPERIÊNCIA FORMATIVA\\PARA A CONSTRUÇÃO DE UMA\\
		ÉTICA SOCIOAMBIENTAL\footnote{Trabalho realizado dentro da área de Ciências Humanas com financiamento do PROPESP - IFRO.}
		
		\normalsize
		Aline Santos Oliveira\footnote{Acadêmica do curso de Licenciatura em Ciências Biológicas, bolsista do PIBIC, alinne.bioliveira@gmail.com, \textit{Campus} Ariquemes.} 
		Danielly A. Souza Higuti\footnote{Aluna do ensino médio, daniellyakeme@gmail.com, \textit{Campus} Ariquemes.} \\
		Alessandro Eleutério Oliveira\footnote{Doutor em Educação, alessandro.oliveira@ifro.edu.br, \textit{Campus} Ariquemes.} 
		Walfredo Oliveira Dias\footnote{Professor graduado em Filosofia, walfredo.dias@ifro.edu.br, \textit{Campus} Ariquemes.} 
	\end{center}
	
	\noindent O cinema é uma potência informativa que pode auxiliar no processo de resgate do
	ideário formativo proposto pelo Iluminismo. Todavia, a obra cinematográfica, na
	medida em que o processo de reprodutibilidade técnica, através de \textit{mass media} que
	correlaciona som, imagem e movimento, pode levar às massas novas formas de
	experiência estética e existencial, no âmbito de um processo educativo mais amplo e
	correlacionado a outras formas de apreciação e compreensão do mundo humano.
	Portanto, esse projeto de investigação objetivou realizar uma análise do potencial
	formativo da experiência cinematográfica para a construção de uma ética da
	responsabilidade socioambiental no processo ensino aprendizagem. O projeto foi
	realizado por alunos do ensino médio e acadêmicos do curso de Licenciatura de
	Ciências Biológicas do IFRO - \textit{Campus} Ariquemes. As atividades compreendeu de
	duas etapas, a análise bibliográfica e a análise cinematográfica. A primeira etapa
	constituiu-se de base teórica, no qual contribui com relevância para o
	desenvolvimento do projeto. Portanto, nas reuniões quinzenais da equipe de
	pesquisa, foram analisados e debatidos de pensadores Walter Benjamim, Theodor
	Adorno, Max Horkheimer, Francis Bacon, Hans Jonas e Immanuel Kant.
	Posteriormente, como segunda etapa da pesquisa de abordagem qualitativa, foram
	analisados 170 filmes de ficção e documentários de curta e longa metragens por
	meio de uma \textit{ficha técnica}, filmes e documentários, que constituíram, à luz do
	referencial teórico mencionado, uma base de dados para a construção de um
	manancial filmográfico, estabelecidos como recursos didáticos que poderão ser
	mobilizados na práxis pedagógica, levando-se em conta as questões
	socioambientais que se impõem à humanidade no mundo hodierno.
	
	\vspace{\onelineskip}
	
	\noindent
	\textbf{Palavras-chave}: Experiência cinematográfica. Ética da responsabilidade. Educação ambiental.	
	
\end{document}