\documentclass[article,12pt,onesidea,4paper,english,brazil]{abntex2}

\usepackage{lmodern, indentfirst, nomencl, color, graphicx, microtype, lipsum}			
\usepackage[T1]{fontenc}		
\usepackage[utf8]{inputenc}		

\setlrmarginsandblock{2cm}{2cm}{*}
\setulmarginsandblock{2cm}{2cm}{*}
\checkandfixthelayout

\setlength{\parindent}{1.3cm}
\setlength{\parskip}{0.2cm}

\SingleSpacing

\begin{document}
	
	\selectlanguage{brazil}
	
	\frenchspacing 
	
	\begin{center}
		\LARGE A MEMÓRIA DAS ENCHENTES EM JI-PARANÁ POR MEIO DA FOTOGRAFIA: DINÂMICA SOCIAL E AMBIENTAL\footnote{Trabalho realizado dentro da Ciências Humanas com financiamento do IFRO.}
		
		\normalsize
		Gustavo José Gregolin\footnote{Bolsista (Iniciação Científica): gustavossguto@gmail.com, \textit{campus} Ji-Paraná.} 
		Emanuel de Souza Alencar\footnote{Colaborador: emanuelalencarjipa@gmail.com, \textit{campus} Ji-Paraná.} 
		Lorelayne Evência da Silva\footnote{Colaboradora: loreevencia@gmail.com, \textit{campus} Ji-Paraná.} 
		Rogger Sidne Ribeiro\footnote{Colaborador: rogge.sidne@gmail.com, \textit{campus} Ji-Paraná.}
		Mônica do Carmo Apolinário de Oliveira\footnote{Orientadora: monica.oliveira@ifro.edu.br , \textit{campus} Ji-Paraná.} 
	\end{center}
	
	\noindent O estudo evidencia a relação do ji-paranaense com os rios Machado e Urupá que
	circundam a cidade, frente às enchentes que assolam o município. Registros
	fotográficos encontrados nos álbuns de família apontaram os caminhos para o
	estudo sobre a dinâmica histórica, social e ambiental imposta pelo ritmo das águas
	em períodos chuvosos que provocam calamidades urbanas. Inundações ocorridas
	devido aos fortes temporais, desde muito provocam tragédias na urbe, com
	desabamento de casas, alagamento de ruas, destruição do comércio, problemas de
	transporte, doenças, falta de comida e outras mazelas. Entrevistas permitiram
	elencar alguns indicadores sociais e econômicos dos grupos humanos que ocupam
	áreas comumente alagadas em períodos de enchentes no município. O estudo
	apontou que os maiores prejudicados são pessoas de baixa renda, com pouca
	escolaridade, que não possuem condições seguras e ideais de moradia, estando à
	mercê das precárias condições urbanísticas das cidades. O estudo de natureza
	qualitativa levou em consideração a análise de conteúdo, categorizando aspectos
	históricos, sociais e ambientais dos documentos fotográficos. Estudos sobre teoria
	da imagem, espaço, tempo, natureza e ocupação humana fomentaram a análise. A
	coleta foi feita por meio de seleção de fotografias, anotações e entrevistas, com o
	propósito de perceber como as sensibilidades despertadas pelas enchentes podem
	ser contadas por meio das fotografias e do discurso visual que elas criam. A
	relevância desta pesquisa configura-se na preservação da memória, ao mesmo
	tempo que interroga o presente como uma tentativa de alertar para os problemas
	sociais decorrentes das enchentes e preocupações ambientas resultantes das ações
	do homem sobre o espaço ocupado pelos rios. Constatamos que mesmo diante das
	dificuldades e incertezas existe muita esperança nos discursos dos moradores
	entrevistados. Esses consideram o rio como uma fonte de renda e alimento, mas
	sobretudo, local de memória, de histórias e vivências que os une culturalmente.
	Neste sentido esperamos que as informações coletadas neste estudo possam servir
	de base, em dados, para a tomada de decisões que auxiliem nas questões sociais e
	ambientais, nas localidades atingidas pelas enchentes.
	
	\vspace{\onelineskip}
	
	\noindent
	\textbf{Palavras-chave}: Memória. Fotografia. Enchente.
	
\end{document}