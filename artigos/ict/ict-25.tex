\documentclass[article,12pt,onesidea,4paper,english,brazil]{abntex2}

\usepackage{lmodern, indentfirst, nomencl, color, graphicx, microtype, lipsum}			
\usepackage[T1]{fontenc}		
\usepackage[utf8]{inputenc}		

\setlrmarginsandblock{2cm}{2cm}{*}
\setulmarginsandblock{2cm}{2cm}{*}
\checkandfixthelayout

\setlength{\parindent}{1.3cm}
\setlength{\parskip}{0.2cm}

\SingleSpacing

\begin{document}
	
	\selectlanguage{brazil}
	
	\frenchspacing 
	
	\begin{center}
		\LARGE ESTUDOS COMPARATIVOS DE VÁRIAS FONTES DE FERTILIZANTES NA
		PRODUÇÃO DE BETERRABA (\textit{BETA VULGARIS}) NO CONE SUL DE
		RONDÔNIA\footnote{Trabalho realizado dentro das Ciências Agrárias com financiamento da PROPESP.}
		
		\normalsize
		Bruno Emanuel Lemes do Nascimento\footnote{Bolsista Pibiti IFRO, darllan.junior@outlook.com, \textit{Campus} Colorado do Oeste.} 
		Darllan Junior L.S. F. de Oliveira\footnote{Bolsista Pibic EM, b.nascimento7456@gmail.com, \textit{Campus} Colorado do Oeste.} 
		Dayane Barbosa Pereira\footnote{Colaboradora, dayanebarbosa\_13@hotmail.com, \textit{Campus} Colorado do Oeste.} 
		Luiz Cobiniano de Melo Filho\footnote{Orientador, luiz.cobiniano@ifro.edu.br, \textit{Campus} Colorado do Oeste.}
		Marcos Aurelio Anequine Macedo\footnote{Co-orientador, marcos.anequi@ifro.edu.br, \textit{Campus} Colorado do Oeste.} 
	\end{center}
	
	\noindent A Beterraba é uma das principais hortaliças cultivada no Brasil, contudo sua
	produção esta concentrada na região centro-sul do país. Neste contexto, a
	realização de pesquisas que viabilizem conhecimentos e tecnologias alternativas de
	adubação para garantir a viabilidade econômica de seu cultivo em outras regiões
	brasileiras se faz necessária. Com base nestes fatos o trabalho teve como objetivo
	testar diferentes fertilizantes orgânicos na produção de beterraba na região do Cone Sul de Rondônia. As mudas de beterraba foram produzidas em bandejas de isopor
	de 288 cédulas, e transplantadas 21 dias após semeadura em canteiros definitivos,
	no espaçamento de 25 X 10 cm. O delineamento experimental utilizado foi de blocos
	casualizados com quatro repetições. Utilizando seis tratamentos: T1 – Testemunha
	adubação química; T2 – Adubação química + Biofertilizante; T3 – Adubação química
	+ Fertipeixe; T4 – Adubação química + Composto Orgânico + Fertipeixe; T5 –
	Adubação química + Composto Orgânico+ Biofertilizante e T6 – Adubação química +
	Composto Orgânico + Biofertilizante + Fertipeixe. Aplicação do fertipeixe e
	biofertilizante foram realizadas via foliar, sendo o fertipeixe com as dose
	recomendada pelo fabricante e o biofertilizante a dose de 3\% a cada semana, após
	o transplantio das mudas. Todos os canteiros receberam uma adubação química de
	fundação, de acordo com análise do solo. A área total de cada canteiro foi de 2,64
	m$^2$, contendo oito fileiras espaçadas de 25 x 10 cm. A colheita foi realizada 90 dias
	após a germinação, e foram submetidos a analises. As variáveis analisadas foram,
	pH, diâmetro, matéria fresca e teor de sólidos solúveis. Não houve diferença
	estatisticamente para os parâmetros avaliados, todavia pode se perceber que os
	tratamentos que receberam a adubação orgânica apresentaram a maior
	produtividade, com destaque para a combinação de composto orgânico mais
	biofertilizante.
	
	\vspace{\onelineskip}
	
	\noindent
	\textbf{Palavras-chave}: Adubação Orgânica. Fertipeixe. Biofertilizante.	
	
\end{document}