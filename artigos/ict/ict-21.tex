\documentclass[article,12pt,onesidea,4paper,english,brazil]{abntex2}

\usepackage{lmodern, indentfirst, nomencl, color, graphicx, microtype, lipsum}			
\usepackage[T1]{fontenc}		
\usepackage[utf8]{inputenc}		

\setlrmarginsandblock{2cm}{2cm}{*}
\setulmarginsandblock{2cm}{2cm}{*}
\checkandfixthelayout

\setlength{\parindent}{1.3cm}
\setlength{\parskip}{0.2cm}

\SingleSpacing

\begin{document}
	
	\selectlanguage{brazil}
	
	\frenchspacing 
	
	\begin{center}
		\LARGE EFEITO DA INOCULAÇÃO DE SEMENTES SOBRE O DESENVOLVIMENTO E
		RENDIMENTO DO FEIJOEIRO\footnote{Trabalho realizado dentro da (área de Conhecimento CNPq: Ciências Agrárias) com financiamento do CNPq / IFRO.}
		
		\normalsize
		Mateus de Souza de Oliveira\footnote{Bolsista (PIBIC), mateus\_97so@yahoo.com, \textit{Campus} Ariquemes.} 
		Sarah Bruna Amontari Pinheiro\footnote{Bolsista (PIBIC EM), sarahbrun4@gmail.com, \textit{Campus} Ariquemes.} 
		Luciano dos Reis Venturoso\footnote{Orientador, luciano.venturoso@ifro.edu.br, \textit{Campus} Ariquemes.} 
		Lenita Aparecida Conus Venturoso\footnote{Co-orientadora, lenita.conus@ifro.edu.br, \textit{Campus} Ariquemes.} 
	\end{center}
	
	\noindent A inoculação é o processo pelo qual bactérias fixadoras de nitrogênio são
	adicionadas às sementes das plantas, com finalidade de substituir, total ou
	parcialmente, o uso de fertilizantes nitrogenados. Devido a necessidade de adoção
	de técnicas de manejo de baixo custo e menor impacto ambiental, objetivou-se
	avaliar o efeito da co-inoculação de sementes com \textit{Rhizobium tropici} e \textit{Azospirillum brasilense} sobre o desenvolvimento e rendimento do feijoeiro em semeadura direta.
	A pesquisa foi conduzida na área experimental do Instituto Federal de Rondônia,
	Campus Ariquemes, em Latossolo Vermelho Amarelo Distrófico. Foi utilizada a
	cultivar Carioca Precoce, visando população final de 260.000 plantas por hectare.
	Adotou-se o delineamento de blocos casualizados, com nove repetições. Os
	tratamentos constaram das inoculações de sementes: \textit{Rhizobium tropici},
	\textit{Azospirillum brasilense}, \textit{Rhizobium tropici} + \textit{Azospirillum brasilense} e um tratamento controle. Foi avaliada aos 25 dias após a semeadura (DAS) e no florescimento pleno do feijoeiro, a altura de plantas, comprimento de raiz, número de nódulos/planta, massa seca de nódulos, massa seca da parte aérea e raiz. Na colheita, número de vagens por planta, de grãos por vagem, massa de cem grãos e o rendimento, com
	13\% de umidade. Foi realizado ainda, análise foliar para verificação do teor de
	macronutrientes. Aos 25 DAS verificou-se maior número de nódulos nas sementes
	co-inoculadas e os menores valoras na testemunha. A co-inoculação e A. brasilense
	proporcionaram nódulos mais pesados. A constatação de nódulos na testemunha
	evidencia a presença efetiva de bactérias nativas. No florescimento, tanto o número
	de nódulos, quanto sua massa foi maior nas sementes inoculadas com \textit{R. tropici.} Na
	análise foliar, efeito significativo ocorreu apenas para o teor de nitrogênio, sendo
	observado os melhores resultados para a co-inoculação, e os menores resultados na
	testemunha. Todavia, os resultados favoráveis da inoculação não evidenciaram
	diferenças significativas no rendimento de grãos.
	
	\vspace{\onelineskip}
	
	\noindent
	\textbf{Palavras-chave}: \textit{Phaseolus vulgaris}. \textit{Rhizobium tropici}. \textit{Azospirillum brasilense}.
	
\end{document}