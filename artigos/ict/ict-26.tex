\documentclass[article,12pt,onesidea,4paper,english,brazil]{abntex2}

\usepackage{lmodern, indentfirst, nomencl, color, graphicx, microtype, lipsum,textcomp}			
\usepackage[T1]{fontenc}		
\usepackage[utf8]{inputenc}		

\setlrmarginsandblock{2cm}{2cm}{*}
\setulmarginsandblock{2cm}{2cm}{*}
\checkandfixthelayout

\setlength{\parindent}{1.3cm}
\setlength{\parskip}{0.2cm}

\SingleSpacing

\begin{document}
	
	\selectlanguage{brazil}
	
	\frenchspacing 
	
	\begin{center}
		\LARGE MANEJO E EXTRAÇÃO DO ÓLEO DE CASTANHA DO BRASIL\footnote{Trabalho realizado dentro de Recursos Florestais e Engenharia Florestal com financiamento do (edital 03 de 2016).}
		
		\normalsize
		Carlos Henrique Souza Costa\footnote{Bolsista de Iniciação Cientifica, carloshenriquejipa@gmail.com, \textit{Campus} Ji-Paraná.} 
		Andreza Pereira Mendonça\footnote{Orientadora, mendonca.andreza@gmail.com, \textit{Campus} Ji-Paraná.}
	\end{center}
	
	\noindent A produção de óleos vegetais nas comunidades rurais da Amazônia é uma
	alternativa de conservação e diversificação dos produtos da floresta, assim como de
	renda às famílias. Entre as espécies florestais com potencial de contribuir para o
	desenvolvimento econômico da região, se encontra a castanha do Brasil
	(\textit{Bertholletia excelsa} Bonpl.), por ser uma espécie de uso múltiplo. Sabendo-se que
	a quantidade de óleo extraível é afetada por parâmetros mecânicos da prensa e pelo
	tratamento prévio das sementes, o objetivo do trabalho foi identificar a influência do
	manejo das sementes de castanha do Brasil sob a quantidade de óleo extraível. As
	sementes de castanha foram obtidas de castanhas em áreas circunvizinhas ao
	município de Ji-Paraná, RO. As sementes com casca foram separadas em lotes de 1
	kg e colocadas para secar em estufa de ventilação forçada nas temperaturas de
	60,70 e 80°C até atingirem umidade de 8, 6 e 4\%. Ao final do processo de secagem,
	as cascas foram separadas das amêndoas com auxílio de um martelo. As amêndoas
	foram trituradas e prensadas em prensa hidráulica por 4 horas. O óleo extraído foi
	transferido para uma proveta graduada. Durante o processo de secagem, verificou-
	se que o tempo foi menor com aumento da temperatura, e que o aumento da
	temperatura de secagem faz com que haja uma maior taxa de remoção de água das
	sementes, devido a um maior gradiente de umidade entre a semente e o ar,
	decrescendo o tempo necessário para atingir o teor de água de equilíbrio. Verificou-se que as amêndoas de castanha submetidas à secagem a 70°C a 8\% de umidade	
	tiveram maior quantidade de óleo extraível (148,3 ml) em relação aos demais
	tratamentos. O tratamento com menor quantidade de óleo extraído foi o. As
	amostras de sementes de castanha submetidas a secagem a 80°C a 8\% foram
	perdidas durante o acompanhamento do controle de massa.
	
	\vspace{\onelineskip}
	
	\noindent
	\textbf{Palavras-chave}: Manejo de sementes. Uso múltiplo. \textit{Bertholletia excelsa}.
		
\end{document}