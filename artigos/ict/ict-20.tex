\documentclass[article,12pt,onesidea,4paper,english,brazil]{abntex2}

\usepackage{lmodern, indentfirst, nomencl, color, graphicx, microtype, lipsum}			
\usepackage[T1]{fontenc}		
\usepackage[utf8]{inputenc}		

\setlrmarginsandblock{2cm}{2cm}{*}
\setulmarginsandblock{2cm}{2cm}{*}
\checkandfixthelayout

\setlength{\parindent}{1.3cm}
\setlength{\parskip}{0.2cm}

\SingleSpacing

\begin{document}
	
	\selectlanguage{brazil}
	
	\frenchspacing 
	
	\begin{center}
		\LARGE EFEITO DA APLICAÇÃO DO EXTRATO DE \textit{CYPERUS ROTUNDUS} COMO
		FITOHORMÔNIO NO PREPARO DE MUDAS DE ALFACE E PEPINO\footnote{Trabalho realizado dentro da Ciências Agrárias com financiamento da PROPESP.}
		
		\normalsize
		Paulo Ariel de Paula\footnote{Bolsista Pibiti IFRO, pauloariel.ea@gmail.com, \textit{Campus} Colorado do Oeste.} 
		Vitório Macieske Neto\footnote{Colaborador(a), vitóriomacieske@gmail.com, \textit{Campus} Colorado do Oeste.} 
		Marcos Aurélio Anequine Macedo\footnote{Orientador(a), marcos.anequine@ifro.edu.br, \textit{Campus} Colorado do Oeste.} 
		Vanderley Antônio Chorobura Klein\footnote{Co-orientador(a), vanderley.klein@ifro.edu.br, \textit{Campus} Colorado do Oeste.} 
	\end{center}
	
	\noindent Toda planta precisa de um bom desenvolvimento radicular para então promover seu
	alto poder produtivo. Devido a esse fator, a Cyperus rotundus, conhecida como
	tiririca, é uma planta invasora que possui uma elevada concentração de ácido
	indolbutírico, que pode agir como promotor de enraizamento de plantas. O objetivo
	do trabalho foi avaliar a utilização do extrato de tiririca sobre variedades de alface e
	pepino, determinando a concentração que proporcionou maior incremento radicular
	e desenvolvimento (0\%, 2,5\%, 5\%, 10\%, 15\%). O experimento foi no Instituto
	Federal de Educação, Ciência e Tecnologia de Rondônia – Campus Colorado do
	Oeste, no setor de Produção Vegetal I, em casa de vegetação. Foi realizado a coleta
	dos bulbos de tiririca feito a lavagem, trituradas em liquidificador com água destilada
	na relação peso de tubérculo e volume de água 1 g/10 mL e deixada em repouso por
	48 horas. O plantio das sementes foi realizado em bandejas com substrato
	comercial. A aplicação do extrato 5 dias após a emergência e as avaliações aos 21
	dias após a emergência, nestas foram verificados: matéria seca da parte aérea,
	matéria seca da raiz, número de folhas, altura das mudas e comprimento da raiz.
	Nos resultados foi observado que para matéria seca da parte aérea, número de
	folhas e cumprimento de raiz que não houve diferenças significativas nos
	tratamentos. Para matéria seca da raiz e altura de planta os tratamentos foram
	significativos, mostrando que quando as doses são elevadas ocorre um decréscimo
	de matéria seca do sistema radicular. No entanto, para altura apresentou plantas de
	porte mais elevado quando a concentração aumentou, principalmente para o pepino.
	O tratamento 2 (concentração de 2,5\% de extrato de tiririca) apresentou maior
	aporte de matéria seca radicular para o pepino, isso promove um melhor
	aproveitamento do solo na absorção de nutrientes que pode resultar em maior
	rendimento produtivo. Portanto, está é uma alternativa de baixo custo para os
	produtores que possibilite incremento em produção a partir do melhor
	aproveitamento dos nutrientes presentes no solo oferecidos as hortaliças.
	
	\vspace{\onelineskip}
	
	\noindent
	\textbf{Palavras-chave}: Alternativa de produção. Enraizamento. Tiririca.
	
\end{document}