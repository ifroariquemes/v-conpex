\documentclass[article,12pt,onesidea,4paper,english,brazil]{abntex2}

\usepackage{lmodern, indentfirst, nomencl, color, graphicx, microtype, lipsum}			
\usepackage[T1]{fontenc}		
\usepackage[utf8]{inputenc}		

\setlrmarginsandblock{2cm}{2cm}{*}
\setulmarginsandblock{2cm}{2cm}{*}
\checkandfixthelayout

\setlength{\parindent}{1.3cm}
\setlength{\parskip}{0.2cm}

\SingleSpacing

\begin{document}
	
	\selectlanguage{brazil}
	
	\frenchspacing 
	
	\begin{center}
		\LARGE A RELAÇÃO HOMEM/RIO NO MUNICÍPIO DE JI-PARANÁ POR MEIO DE
		DOCUMENTOS FOTOGRÁFICOS\footnote{Trabalho realizado dentro da área de Ciências Humanas com financiamento do IFRO.}
		
		\normalsize
		Gustavo José Gregolin\footnote{Bolsista (Iniciação Científica): gustavossguto@gmail.com, \textit{campus} Ji-Paraná.} 
		Emanuel de Souza Alencar\footnote{Colaborador: emanuelalencarjipa@gmail.com, \textit{campus} Ji-Paraná.} 
		Lorelayne Evência da Silva\footnote{Colaboradora: loreevencia@gmail.com, \textit{campus} Ji-Paraná.} 
		Rogger Sidne Ribeiro\footnote{Colaborador: rogge.sidne@gmail.com, \textit{campus} Ji-Paraná.}
		Mônica do Carmo Apolinário de Oliveira\footnote{Orientadora: monica.oliveira@ifro.edu.br, \textit{campus} Ji-Paraná.} 
	\end{center}
	
	\noindent O regime das águas dos rios Machado e Urupá no município de Ji-Paraná/RO
	apresenta paisagens que, a partir da presença humana, tornam-se espaços
	antropizados. Vários registros fotográficos, encontrados nos álbuns de família dos
	moradores dessas localidades, apontaram os caminhos para o estudo sobre a
	dinâmica histórica e ambiental imposta pelo ritmo das águas, que em constante
	movimento, produzem e reproduzem o modo de vida e a organização social. Este
	estudo é uma soma das percepções visuais de documentos fotográficos, agregada a
	informações orais obtidas por meio de entrevistas semiestruturadas. A análise do
	acervo fotográfico permitiu a classificação das imagens que referenciam os Rios
	Machado e Urupá, seja em relação às catástrofes promovidas pelas enchentes ou
	pela relação direta do homem com os rios que circundam a cidade. As entrevistas
	permitiram elencar alguns indicadores sociais e econômicos dos grupos humanos
	que ocupam áreas comumente alagadas e que fazem uso dos rios para atividades
	de pesca e lazer. Há uma familiaridade com o espaço ocupado que são marcas
	identitárias que medeiam o ambiente urbano. Várias transformações na paisagem
	ocorrem em decorrência da presença humana e também pela ação das águas.
	Paisagens naturais e artificiais compõem o cenário das águas dos rios Machado e
	Urupá. O estudo de natureza qualitativa levou em consideração a análise de
	conteúdo, categorizando aspectos históricos, sociais e ambientais dos documentos
	fotográficos. Estudos sobre teoria da imagem, espaço, tempo, natureza e ocupação
	humana fomentaram a análise. A relevância desta pesquisa configura-se na
	preservação da memória, ao mesmo tempo em que interroga o presente como uma
	tentativa de alertar para os problemas sociais decorrentes das ações do homem
	sobre o espaço ocupado pelos rios. Como representações do real, as imagens	
	visuais constroem hierarquias, visões de mundo e, neste sentido, podem constituir-
	se em fontes preciosas para a compreensão dos fenômenos.
	
	\vspace{\onelineskip}
	
	\noindent
	\textbf{Palavras-chave}: Fotografia. Homem. Rio.
	
\end{document}