\documentclass[article,12pt,onesidea,4paper,english,brazil]{abntex2}

\usepackage{lmodern, indentfirst, nomencl, color, graphicx, microtype, lipsum}			
\usepackage[T1]{fontenc}		
\usepackage[utf8]{inputenc}		

\setlrmarginsandblock{2cm}{2cm}{*}
\setulmarginsandblock{2cm}{2cm}{*}
\checkandfixthelayout

\setlength{\parindent}{1.3cm}
\setlength{\parskip}{0.2cm}

\SingleSpacing

\begin{document}
	
	\selectlanguage{brazil}
	
	\frenchspacing 
	
	\begin{center}
		\LARGE AS IMPLICAÇÕES DAS ATIVIDADES EXTRACURRICULARES NO RENDIMENTO
		ESCOLAR DOS ALUNOS A PARTIR DO OLHAR DOS SERVIDORES\footnote{Trabalho realizado na área de Educação Física com financiamento do IFRO.}
		
		\normalsize
		Waleska Juracy Araújo de Lima\footnote{Bolsista, waleska.juracy@gmail.com, \textit{Campus} Porto Velho Calama.} 
		Iranira Geminiano de Melo\footnote{Orientadora, iranira.melo@ifro.edu.br, \textit{Campus} Porto Velho Calama.} 
		Elza Paula Silva Rocha\footnote{Co-orientadora, elza.rocha@ifro.edu.br, \textit{Campus} Porto Velho Calama.} 
		Nathály Carol\footnote{Colaboradora, nathalycarol22@gmail.com, \textit{Campus} Porto Velho Calama.}
		Emily Ingrid Bader de Souza Rómer\footnote{Colaboradora, emilyingridb@gmail.com, \textit{Campus} Porto Velho Calama}
		Lys Vitória Ribeiro de Almeida\footnote{Colaboradora, lys.vitoria.trabalho@gmail.com, \textit{Campus} Porto Velho Calama} 
	\end{center}
	
	\noindent A aprendizagem do conteúdo escolar dos alunos é um fenômeno complexo e de
	caráter multifatorial. A alimentação, o cansaço, o estresse, o acesso a certos recursos,
	são exemplos de fatores que podem impactar no rendimento escolar dos alunos.	
	Apesar disso, a literatura afirma que muitos dos envolvidos no processo de ensino-
	aprendizagem acabam por atribuir o baixo rendimento acadêmico à falta de dedicação	
	do aluno, ignorando outros fatores tão impactantes no processo quanto à falta de
	interesse. As atividades extracurriculares, como esporte, música e pesquisa, por
	serem consideradas mais atrativas aos olhos dos alunos, costumam ser julgadas
	como principais causadoras desse desinteresse. Com base nisso, teve-se por objetivo
	analisar, a partir do olhar dos servidores do Campus Porto Velho Calama, quais as
	implicações das atividades extracurriculares no rendimento escolar dos alunos. Como
	procedimento metodológico, utilizou-se de entrevistas semiestruturadas, gravadas e
	transcritas. A amostra foi composta de 13 pessoas do sexo feminino e 18 do sexo
	masculino, entre 20 e 60 anos, que trabalham no serviço público há no mínimo 5
	meses. Foram analisadas 31 entrevistas com diversos servidores do Campus, através
	das quais se pode averiguar que todos os entrevistados concordam com o lado
	multifatorial da aprendizagem. Cerca de 94\% dos servidores citaram ao menos um
	fator, derivado das atividades extracurriculares, que pode servir de estímulo ou auxiliar
	o aluno durante o processo de ensino-aprendizagem dentro da sala de aula. No
	entanto, cerca de 25,8\% dos servidores preocupam-se com a capacidade dos alunos
	de administrar o tempo, de forma que a atividade extracurricular não prejudique o
	processo dentro da sala de aula. As atividades extracurriculares podem ter diferentes
	implicações a depender do acompanhamento familiar e docente e do envolvimento
	dos alunos. A partir das evidências da pesquisa, a Instituição pode planejar estratégias
	que auxiliem os alunos e servidores de modo que a atividade extracurricular sirva ao
	seu propósito de auxiliar ou complementar o processo de ensino-aprendizagem.
	
	\vspace{\onelineskip}
	
	\noindent
	\textbf{Palavras-chave}: Aprendizagem. Atividade extracurricular. Servidores.
	
\end{document}