\documentclass[article,12pt,onesidea,4paper,english,brazil]{abntex2}

\usepackage{lmodern, indentfirst, nomencl, color, graphicx, microtype, lipsum}			
\usepackage[T1]{fontenc}		
\usepackage[utf8]{inputenc}		

\setlrmarginsandblock{2cm}{2cm}{*}
\setulmarginsandblock{2cm}{2cm}{*}
\checkandfixthelayout

\setlength{\parindent}{1.3cm}
\setlength{\parskip}{0.2cm}

\SingleSpacing

\begin{document}
	
	\selectlanguage{brazil}
	
	\frenchspacing 
	
	\begin{center}
		\LARGE AVALIAÇÃO DO PADRÃO FERMENTATIVO E PERDAS FERMENTATIVAS NA ENSILAGEM ORIUNDA CONSÓCIO ENTRE \textit{BRACHIARIA BRIZANTHA} CV. MARANDU E SORGO FORRAGEIRO EM UM SISTEMA DE INTEGRAÇÃO LAVOURA-PECUÁRIA-FLORESTA\footnote{Trabalho realizado dentro das ciências agrárias com financiamento do CNPq e IFRO.}
		
		\normalsize
		Dienice Oliveira Macedo\footnote{Bolsista do superior (PIBITI), domoliveira1@hotmail.com, \textit{Campus} Colorado do Oeste.} 
		Felype Francisco Oliveira\footnote{Bolsista do médio (PIBIC-EM), lypeusou@gmail.com, \textit{Campus} Colorado do Oeste.} 
		Jessica Pagung\footnote{Colaborador (a), jessicapagung18@gmail.com, \textit{Campus} Colorado do Oeste.} 
		Rafael Henrique Pereira dos Reis\footnote{Orientador (a), rafael.reis@ifro.edu.br, \textit{Campus} Colorado do Oeste.} 
	\end{center}
	
	\noindent O sorgo aparece como alternativa para produção de silagem, principalmente pela
	menor exigência nutricional e maior tolerância à pragas e doenças quando
	comparada ao milho, além de apresentar similaridade nos parâmetros produtivos e
	nutritivos. No entanto, existe certa resistência em reduzir o espaçamento de plantio
	entre linhas do sorgo devido a hipótese de redução de sua produtividade. Há ainda a
	possibilidade de cultivo de sorgo consorciado com capins em sistemas de integração
	lavoura pecuária. Diante do exposto, o trabalho teve por objetivo avaliar o padrão
	fermentativo e as perdas fermentativas oriundas do consórcio entre \textit{Brachiária
	brizantha} cv. Marandu e sorgo forrageiro BRS 655 em um sistema integração
	lavoura-pecuária-floresta. O experimento foi conduzido no Instituto Federal de
	Educação, Ciência e Tecnologia de Rondônia - Campus Colorado do Oeste
	utilizando-se o delineamento em blocos casualizados com 4 repetições adotando os
	seguintes tratamentos: Sorgo com 35 e 70 cm entre fileiras com semeadura do
	capim a lanço, Sorgo com 35 e 70 cm entre fileiras com semeadura do capim na
	linha do sorgo, sorgo a 70 cm com capim na fileira e entre fileiras do sorgo, capim a
	lanço, capim com 35 e 70 cm entre fileiras e, sorgo semeado solteiro com 35 e 70
	cm entre fileiras. Os dados coletados foram submetidos a análise de variância, e
	quando teste F foi significativo, efetuou-se o teste de comparação de médias de
	Tukey ao nível de 5\% de probabilidade de erro. As variáveis analisadas foram,
	perdas por gases e efluentes, recuperação da matéria seca e pH. As modalidades
	de plantio e espaçamento do consórcio entre sorgo forrageiro e capim Marandu não
	diferiram estatisticamente, apresentando os seguintes valores médios para as
	variáveis testadas: 22,12\% da matéria seca para variável perdas por gases, 4,266 kg
	t$^{-1}$ de silagem para perdas por efluentes, 94,78\% para recuperação da matéria seca,
	e 4,06 para a variável pH. Portanto, a redução no espaçamento do sorgo e as
	diferentes modalidades de consorciação com capim não influenciaram
	negativamente a recuperação da matéria seca, as perdas fermentativas, e o padrão
	fermentativo da silagem.
	
	\vspace{\onelineskip}
	
	\noindent
	\textbf{Palavras-chave}: Capim. Integração. \textit{Sorghum bicolor}.
	
\end{document}