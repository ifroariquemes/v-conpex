\documentclass[article,12pt,onesidea,4paper,english,brazil]{abntex2}

\usepackage{lmodern, indentfirst, nomencl, color, graphicx, microtype, lipsum}			
\usepackage[T1]{fontenc}		
\usepackage[utf8]{inputenc}		

\setlrmarginsandblock{2cm}{2cm}{*}
\setulmarginsandblock{2cm}{2cm}{*}
\checkandfixthelayout

\setlength{\parindent}{1.3cm}
\setlength{\parskip}{0.2cm}

\SingleSpacing

\begin{document}
	
	\selectlanguage{brazil}
	
	\frenchspacing 
	
	\begin{center}
		\LARGE AVALIAÇÃO DOS ESTÁGIOS DE MUNDANÇA DE COMPORTAMENTO DOS
		ESTUDANTES DOS CURSOS TÉCNICOS INTEGRADOS DO \\INSTITUTO
		FEDERAL DE RONDÔNIA\footnote{Pesquisa financiada pela Fundação Rondônia de Amparo ao Desenvolvimento das Ações Científicas e Tecnológicas e à Pesquisa do Estado de Rondônia (FAPERO) e pelo Conselho Nacional de Desenvolvimento Científico e Tecnológico (CNPq).}
		
		\normalsize
		Nícolas Costa Freitas\footnote{Bolsista, nicolasfreitas.0300@gmail.com, \textit{Campus} Porto Velho Calama.} 
		Iranira Geminiano de Melo\footnote{Orientadora, iranira.melo@ifro.edu.br, professora EBTT, IFRO, \textit{Campus} Porto Velho Calama.} 
		Célio José Borges\footnote{Colaborador, ceborges@gmail.com, professor do Departamento de Educação Física da UNIR.} 
		George Madson Dias Santos\footnote{Co-orientador, george.santos@ifro.edu.br, professor EBTT, IFRO, \textit{Campus} Porto Velho Calama.} 
	\end{center}
	
	\noindent A exposição a baixos níveis de atividade física se constitui em um fator de risco para
	o desenvolvimento de doenças crônicas não transmissíveis (obesidade, diabetes e
	doenças cardiovasculares, por exemplo). Pesquisas indicam que 39\% a 93,5\% dos
	adolescentes brasileiros não possuem o hábito de praticar atividades físicas, mesmo
	estando cientes dos benefícios que podem ser obtidos com um estilo de vida ativo.
	Nessa pesquisa foi desenvolvido um questionário denominado “Estágio de Mudança
	de Comportamento” (EMC), que tem como objetivo identificar em qual estágio de
	mudança de comportamento o indivíduo se encontra, indicando os processos de
	mudança em relação à prática de atividades físicas. A partir do problema social da
	atividade física relacionada à saúde foi realizada uma pesquisa, no Instituto Federal
	de Rondônia, com o objetivo de identificar quais os níveis do estágio de
	comportamento de estudantes do campus Porto Velho Calama, da modalidade de
	técnico integrado ao ensino médio. O EMC foi implementado como um questionário
	eletrônico disponível para ser respondido via \textit{Internet} (link:
	http://www.gesstec.org/estilodevida/estilodevida.html) pelos discentes, e a pesquisa
	contou com a participação de 122 pessoas, sendo 69 rapazes, e 53 moças. Quanto
	ao resultado da pesquisa, 10,66\% dos alunos estão no estágio de ação (pratica
	atividade física regularmente há menos de 6 meses), 36,07\% no estágio de
	manutenção (pratica atividade física regularmente há 6 meses ou mais), 16,39\% no
	estágio de contemplação (não há uma prática de atividade física, mas há intenções
	de iniciar uma vida ativa em até 6 meses), 35,25\% no estágio de preparação (há
	intenções de iniciar a prática de atividades físicas nos próximos 30 dias) e 1,64\% no
	estágio de pré-contemplação (não pratica nenhum tipo de atividade física
	regularmente e não há a intenção de mudar de comportamento). A pesquisa permitiu
	evidenciar que a maioria dos alunos pesquisados não praticam atividades físicas,
	mas desejam iniciar uma vida fisicamente ativa, identificando, assim, a necessidade
	de medidas para incentivo e apoio à prática de atividades físicas.
	
	\vspace{\onelineskip}
	
	\noindent
	\textbf{Palavras-chave}: Saúde. Atividade Física. Mudança de Comportamento.
	
\end{document}