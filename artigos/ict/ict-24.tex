\documentclass[article,12pt,onesidea,4paper,english,brazil]{abntex2}

\usepackage{lmodern, indentfirst, nomencl, color, graphicx, microtype, lipsum, textcomp}			
\usepackage[T1]{fontenc}		
\usepackage[utf8]{inputenc}		

\setlrmarginsandblock{2cm}{2cm}{*}
\setulmarginsandblock{2cm}{2cm}{*}
\checkandfixthelayout

\setlength{\parindent}{1.3cm}
\setlength{\parskip}{0.2cm}

\SingleSpacing

\begin{document}
	
	\selectlanguage{brazil}
	
	\frenchspacing 
	
	\begin{center}
		\LARGE ESTUDO DE PROPRIEDADES ANTIMICROBIANAS DO IPÊ VERDE \textit{Cibistax
		antisyphilitica}
		
		\normalsize
		Kelly Lourdiely Santos Lima\footnote{Bolsista, (Licenciatura em Química), kehellenmartins@gmail.com, \textit{Campus} Jí-Paraná.} 
		Vitor Hugo Baier de Oliveira\footnote{Bolsista, (Técnico em Química), \textit{Campus} Jí-Paraná.} 
		Renato André Zan\footnote{Coordenador, Professor EBTT de Química email: renato.zan@ifro.edu.br , \textit{Campus} Ji-Paraná.} 
	\end{center}
	
	\noindent \textbf{Introdução:} A descoberta de novos fármacos e produtos químicos, que atendam às
	necessidades que novas patologias têm imposto, deve passar por uma triagem
	inicial para identificação de tais substâncias. O uso popular de plantas no combate a
	doenças é milenar. O conhecimento científico dessas plantas tem sido um desafio
	para os pesquisadores, pois muitas permanecem desconhecidas, tanto do ponto de
	vista químico quanto farmacológico. O presente projeto visou à avaliação dos
	possíveis constituintes fenólicos da \textit{Cibistax antisyphilitica} como antimicrobianas.
	Portanto o conhecimento químico e do potencial biológico destes compostos irá
	valorizar essa planta da região do Norte, favorecendo assim seu uso popular.
	\textbf{Materiais e métodos:} De início foi coletado as amostras de folhas e cascas do ipê-
	verde, em seguida as amostras foram higienizadas e pesadas, adicionou álcool
	etílico o filtrado foi rotaevaporado para retira o álcool contido na amostra com a
	finalidade de obter o extrato puro. As amostras obtidas foram submetidas ao ensaio
	de atividade antimicrobiana utilizando a técnica de \textit{cup plate} (BRANDT-ROSE;
	MILLER, 1939). Para tanto, as bactérias patogênicas \textit{Staphylococcus aureus} ATCC
	25923, Streptococcus pneumoniae ATCC 11733, \textit{Klebsiella pneumoniae} ATCC
	4952, \textit{Escherichia coli} ATCC 25922, \textit{Enterococcus faecalis} ATCC 29212,
	\textit{Pseudomonas aeroginosa} ATCC 15442 e o fungo patogênico Candida albicans
	ATCC 24433 foram crescidas a 37°C por 4-6 h em meio Luria-Bertani sendo sua
	turbidez ajustada para escala 0,5 de McFarland para as bactérias e 1 para o fungo.
	Os microrganismos então foram inoculadas em placas de Petri contendo meio
	Muller-Hinton para o ensaio com bactérias e meio ágar Sabouraud-Dextrose para os
	fungos, realizado perfurações no meio com o inóculo de 5mm de diâmetro e dentro
	destes depositados 20 $\mu$l das amostras e armazenados a 4°C por 24 h para permitir
	a difusão no meio de cultura das amostras. Posteriormente, as amostras foram
	incubados a 37°C por 24 h. Foram considerados com atividade antimicrobiana, as
	amostras que não permitiram o crescimento microbiano ao redor do disco. Os halos
	de inibição produzidos foram medidos em milímetros. \textbf{Resultados:} Para análise de
	resultados deve ser considerada a média dos três ensaios ou dois resultados com
	atividade. O método é reprodutibilidade, executado em dias diferentes para
	confirmar o resultado. Não são repetições. Por isso, dando dois resultados com
	atividade, pode ser analisado como com atividade antimicrobiana, onde todas as
	amostras apresentaram atividade antimicrobiana, pois em nenhuma amostra
	apresentou crescimento microbiano nos halos. \textbf{Conclusão:} Portanto, podemos
	concluir que o ipê verde \textit{cibistax antisyphilitica},tem grande potencial como agente
	antimicrobiano.
	
	\vspace{\onelineskip}
	
	\noindent
	\textbf{Palavras-chave}: Timbó. Extrato Vegetal. CGMS.	
	
\end{document}