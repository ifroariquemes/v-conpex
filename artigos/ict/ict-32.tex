\documentclass[article,12pt,onesidea,4paper,english,brazil]{abntex2}

\usepackage{lmodern, indentfirst, nomencl, color, graphicx, microtype, lipsum,textcomp}			
\usepackage[T1]{fontenc}		
\usepackage[utf8]{inputenc}		

\setlrmarginsandblock{2cm}{2cm}{*}
\setulmarginsandblock{2cm}{2cm}{*}
\checkandfixthelayout

\setlength{\parindent}{1.3cm}
\setlength{\parskip}{0.2cm}

\SingleSpacing

\begin{document}
	
	\selectlanguage{brazil}
	
	\frenchspacing 
	
	\begin{center}
		\LARGE RELAÇÃO DE METABÓLITOS SECUNDÁRIOS COM A ATIVIDADE
		ANTIFÚNGICA DOS EXTRATOS VEGETAIS\footnote{Trabalho realizado dentro da (área de Conhecimento CNPq: Ciências Agrárias) com financiamento do CNPq / IFRO.}
		
		\normalsize
		Luana Jaguszevski\footnote{Bolsista (PIBIC), luanajaguszevski@gmail.com, \textit{Campus} Ariquemes.} 
		Isabela Pereira de Souza Schoaba\footnote{Bolsista (PIBIC EM), schoaba2@gmail.com, \textit{Campus} Ariquemes.} 
		Luciano dos Reis Venturoso\footnote{Orientador, luciano.venturoso@ifro.edu.br, \textit{Campus} Ariquemes.} 
		Lenita Aparecida Conus Venturoso\footnote{Co-orientadora, lenita.conus@ifro.edu.br, \textit{Campus} Ariquemes.} 
	\end{center}
	
	\noindent Em virtude da dificuldade no manuseio e toxicidade dos defensivos agrícolas, aliado
	a preocupação quanto à segurança alimentar, objetivou-se verificar o efeito inibitório
	de extratos vegetais utilizados individualmente e em misturas, no controle dos
	fungos \textit{Rhizoctonia solani, Colletotrichum gloeosporioides} e \textit{Moniliophthora perniciosa}. Foram implantados três bioensaios, um para cada fitopatógeno, em
	delineamento inteiramente casualizado com 18 tratamentos e 5 repetições. Os
	tratamentos foram compostos pelas plantas individuais (alho, arranha-gato,
	barbatimão, cravo-da-índia, eucalipto, macaé, erva de Santa Maria) e misturas (alho
	+ cravo-da-índia, alho + arranha-gato, alho + barbatimão, cravo-da-índia + arranha-
	gato, cravo-da-índia + macaé, eucalipto + arranha-gato, eucalipto + barbatimão,
	eucalipto + erva de Santa Maria, barbatimão + arranha-gato, macaé + erva de Santa
	Maria) e um tratamento controle, contendo apenas meio de cultura BDA. Os extratos
	foram obtidos a partir da trituração de 20 g do material vegetal de cada espécie em
	100 ml de água destilada. O material foi filtrado, e o extrato aquoso obtido,
	acondicionado em erlenmayers. Os extratos foram homogeneizados em meio BDA
	fundente, na concentração de 20\%, e vertidos em placas de Petri. Para as misturas,
	a concentração de 20\% foi obtida com 10\% de cada extrato vegetal. Após a
	solidificação do meio, foram transferidos discos de 0,5 cm de diâmetro do micélio
	dos patógenos, no centro das placas, e incubadas a 25°C. Foram analisados o
	crescimento micelial e a porcentagem de inibição do crescimento dos fitopatógenos.
	Os extratos contendo cravo-da-índia obtiveram maior ação sobre todos os fungos
	fitopatogênicos. O extrato de alho também apresentou resultados significativos, com
	destaque para a total inibição de \textit{M. perniciosa.} O barbatimão apresentou resultado
	sobre \textit{M. perniciosa. }O extrato de cravo-da-índia apresentou resultado positivo para
	a presença de triterpeno, saponina e tanino, enquanto o alho, triterpeno e esteroide.
	No extrato de barbatimão foi verificada a presença de saponina, tanino e flavonoide.
	Não foi observado relação entre a presença dos metabólitos secundários e a ação
	antifúngica dos extratos vegetais.
	
	\vspace{\onelineskip}
	
	\noindent
	\textbf{Palavras-chave}: Propriedades antifúngicas. Extratos aquosos. Fungos
	fitopatogênicos.	
	
\end{document}