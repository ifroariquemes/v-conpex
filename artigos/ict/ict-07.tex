\documentclass[article,12pt,onesidea,4paper,english,brazil]{abntex2}

\usepackage{lmodern, indentfirst, nomencl, color, graphicx, microtype, lipsum}			
\usepackage[T1]{fontenc}		
\usepackage[utf8]{inputenc}		

\setlrmarginsandblock{2cm}{2cm}{*}
\setulmarginsandblock{2cm}{2cm}{*}
\checkandfixthelayout

\setlength{\parindent}{1.3cm}
\setlength{\parskip}{0.2cm}

\SingleSpacing

\begin{document}
	
	\selectlanguage{brazil}
	
	\frenchspacing 
	
	\begin{center}
		\LARGE ASPECTOS NEGATIVOS E POSITIVOS ASSOCIADOS À PRÁTICA ESPORTIVA\footnote{Trabalho realizado na área de Educação Física com financiamento do IFRO, \textit{Campus} Calama.}
		
		\normalsize
		Náthaly Caroline de Souza Figueiredo\footnote{Bolsista, Nathalycarol22@gmail.com, \textit{Campus} Porto Velho Calama.} 
		Iranira Geminiano de Melo\footnote{Orientadora, Iranira.melo@ifro.edu.br, \textit{Campus} Porto Velho Calama.} 
		Elza Paula Silva Rocha\footnote{Co-orientadora, elza.rocha@ifro.edu.br, \textit{Campus} Porto Velho Calama.} 
	\end{center}
	
	\noindent A pesquisa visa analisar as contribuições da prática esportiva a partir da visão
	dos alunos. Sabemos que o esporte vem se tonando cada dia mais presente nas
	escolas públicas e privadas. Sendo contextualizado de diferentes formas, como em
	transmissões de jogos na televisão, jornais, em rádio, ou até mesmo em campos,
	praças, clubes, locais onde se tenha o ambiente adequado para a prática de
	diferentes modalidades. Metodologicamente a pesquisa foi realizada a partir de
	entrevista semiestruturada, sendo a amostra composta por alunos na faixa etária de
	15 a 20 anos, totalizando um grupo de 72 alunos, sendo 31 do sexo masculino e 41
	do sexo feminino, classificados como estudantes atletas e não atletas. Os resultados
	mostraram que 21 alunos participavam ou participaram de eventos esportivos, nas
	modalidades de: Futsal, Basquete, Jiu Jitsu, Handebol, Xadrez, Tênis de mesa,
	Atletismo e Artes maciais. Foram analisados diversos fatores que levam os alunos a
	praticar esportes, como estar bem consigo mesmo, a satisfação de estar
	representando o seu colégio, a autoconfiança, e também muitas vezes uma forma
	de amenizar os problemas da vida diária. Ao ver de estudantes o esporte é um
	elemento influenciador no desempenho escolar dos alunos, e que pode ter aspectos
	negativos e positivos associados à prática dele. Quanto a influência do esporte ser
	prejudicial ao rendimento escolar dos alunos, isso ocorre somente quando o aluno
	não possuir discernimento para conciliar o esporte com a vida acadêmica, levando-o
	a se dedicar excessivamente ao esporte e reduzir a atenção às atividades
	curriculares. Por outro lado, além dos benefícios já citados os participantes da
	pesquisa destacaram que a prática esportiva pode ser benéfica quando
	desempenha sua função na saúde: ajudando na concentração, raciocínio e podendo
	deixar a pessoa mais ativa. Portanto, afirma-se que na visão dos estudantes atletas
	os benefícios da prática esportiva extrapolam o campo educativo e são muito
	significativos tanto na vida acadêmica quanto na saúde pessoal. E de modo geral os
	participantes da pesquisa sugerem que se o aluno esportista tiver discernimento e
	organização do tempo dedicado ao esporte e ao estudo não terá nenhum prejuízo
	com a prática esportiva.
	
	\vspace{\onelineskip}
	
	\noindent
	\textbf{Palavras-chave}: Prática esportiva. Rendimento escolar. Saúde.
	
\end{document}