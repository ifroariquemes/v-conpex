\documentclass[article,12pt,onesidea,4paper,english,brazil]{abntex2}

\usepackage{lmodern, indentfirst, nomencl, color, graphicx, microtype, lipsum}			
\usepackage[T1]{fontenc}		
\usepackage[utf8]{inputenc}		

\setlrmarginsandblock{2cm}{2cm}{*}
\setulmarginsandblock{2cm}{2cm}{*}
\checkandfixthelayout

\setlength{\parindent}{1.3cm}
\setlength{\parskip}{0.2cm}

\SingleSpacing

\begin{document}
	
	\selectlanguage{brazil}
	
	\frenchspacing 
	
	\begin{center}
		\LARGE SUPERMARKET: DESENVOLVIMENTO DE UM APLICATIVO COMPARADOR DE PREÇOS\footnote{Trabalho realizado dentro da seguinte área de conhecimento: 1.03.00.00-7 Ciência da Computação - Administração de Empresas, com financiamento do IFRO - \textit{Campus} Cacoal.}
		
		\normalsize
		Gilson Duarte Rosa\footnote{Bolsista (IC-EM), gilsonduarte213@gmail.com, IFRO - \textit{Campus} Cacoal.} 
		Anita Martins Reinholz\footnote{Colaboradora, anitareinholz72@gmail.com, IFRO - \textit{Campus} Cacoal.} 
		Miguel Arcangelo Martins Pereira Junior\footnote{Colaborador, henboff@gmail.com, IFRO - \textit{Campus} Cacoal.} 
		Lucas Henrique Alves Borth\footnote{Colaborador, lucasescola13@gmail.com, IFRO - \textit{Campus} Cacoal.}
		Thiago José Sampaio Kaiser\footnote{Orientador: thiago.kaiser@ifro.edu.br – IFRO - \textit{Campus} Cacoal}
		Juliano Cristhian Silva\footnote{Co-orientador, juliano@ifro.edu.br, IFRO - \textit{Campus} Cacoal}
		Saiane de Barros Souza\footnote{Co-orientadora, saiane.souza@ifro.edu.br, IFRO - \textit{Campus} Cacoal} 
	\end{center}
	
	\noindent As guerras marcaram a história do mundo, não só pelos seus resultados óbvios,	mas também pelas inovações por elas proporcionadas. A Guerra Fria, por exemplo, foi fator primordial para o desenvolvimento do que se conhece, hoje, por Internet.
	Essa forma de comunicação avançou significativamente ao longo dos anos,
	deixando de ser uma rede totalmente privada do exército americano e universidades
	e se tornar acessível à sociedade. Desde então, a Internet vem tomando proporções
	cada vez maiores e, com isso, influenciando nosso cotidiano e modificando
	completamente o comportamento da sociedade, em especial na mobilidade que
	esses avanços proporcionam. Uma das grandes evoluções da tecnologia, que
	acarretaram tal mobilidade, foi o lançamento dos \textit{smartphones} e seus aplicativos, que possuem funcionalidades específicas e saciam as necessidades que o usuário
	encontra em determinado momento e, consequentemente, ganham cada vez mais
	utilidades e espaço no nosso cotidiano e, com intuito de fortalecer essa evolução, o presente projeto objetiva desenvolver um aplicativo para dispositivos móveis,
	plataforma Android, que promova a divulgação de promoções de supermercados
	aos consumidores, inicialmente do município de Cacoal/RO. O desenvolvimento do
	aplicativo será feito com base no paradigma de programação orientada a objetos em
	conjunto com uma IDE (\textit{Integrated Development Environment}) específica para 
	desenvolvimento para plataforma Android e um sistema de banco de dados. Almeja-
	se que o aplicativo possibilite, aos usuários, visualizar as ofertas de cada	
	estabelecimento em seus \textit{smartphones} e, proporcionar assim, uma ampla pesquisa
	de preço de forma rápida e simples. Espera-se, ainda, que a utilização do aplicativo,
	como meio de divulgação de ofertas contribua, significativamente, para redução da
	utilização de material impresso, comumente utilizada para tais divulgações.
	
	\vspace{\onelineskip}
	
	\noindent
	\textbf{Palavras-chave}: Supermercado. Acessibilidade. Divulgação de produtosSupermercado. Acessibilidade. Divulgação de produtos..	
	
\end{document}