\documentclass[article,12pt,onesidea,4paper,english,brazil]{abntex2}

\usepackage{lmodern, indentfirst, nomencl, color, graphicx, microtype, lipsum}			
\usepackage[T1]{fontenc}		
\usepackage[utf8]{inputenc}		

\setlrmarginsandblock{2cm}{2cm}{*}
\setulmarginsandblock{2cm}{2cm}{*}
\checkandfixthelayout

\setlength{\parindent}{1.3cm}
\setlength{\parskip}{0.2cm}

\SingleSpacing

\begin{document}
	
	\selectlanguage{brazil}
	
	\frenchspacing 
	
	\begin{center}
		\LARGE TOLERÂNCIA DA \textit{BRACHIARIA BRIZANTHA CV.} MARANDU A SUBDOSES DE GLIFOSATO NA CONSORCIAÇÃO COM MILHO EM SISTEMA DE INTEGRAÇÃO LAVOURA-PECUÁRIA\footnote{Trabalho realizado dentro da área de Ciências Agrárias, com financiamento do CNPq e do IFRO.}
		
		\normalsize
		Daniela Souza da Silva\footnote{Bolsista PIBIC-Af, danielaagro22@gmail.com, \textit{Campus} Colorado do Oeste.} 
		Welington Celestino Vieira\footnote{Bolsista PIBIC-EM, welington.vieira2000@gmail.com, \textit{Campus} Colorado do Oeste.} 
		Dienice Oliveira Macedo\footnote{Colaborador, graduando em Engenharia Agronômica, \textit{Campus} Colorado do Oeste.} 
		Ernando Balbinot\footnote{Orientador, ernando.balbinot@ifro.edu.br, \textit{Campus} Colorado do Oeste.} 
	\end{center}
	
	\noindent Uma alternativa para solucionar o problema de mais de 50\% das pastagens de
	Rondônia que se encontra em algum estágio de degradação seria a renovação da
	pastagem consorciando a forrageira com uma cultura anual, como o milho. Porém
	por se tratar de duas gramíneas, existe o risco de competição, que pode ser
	controlado pela aplicação de subdoses de herbicidas. A tecnologia Milho Roundup
	Ready$^{®}$ pode permitir a supressão da forrageira com uso de glifosato, facilitando o
	manejo e diminuindo os custos. O objetivo do estudo foi avaliar os efeitos do
	herbicida glifosato aplicado na \textit{Brachiaria brizantha cv.} Marandu em consórcio com
	o milho em diferentes doses e modalidades de semeadura, além da época de
	aplicação. O experimento foi conduzido no IFRO – \textit{Campus} Colorado do Oeste,
	utilizando o delineamento experimental de blocos casualizados, arranjado em
	parcelas sub-subdivididas (2x2x8), com três repetições, sendo duas modalidades
	de semeadura (na linha e à lanço), na parcela; duas épocas de aplicação (estádios
	V6 e V9 do milho), na subparcela; e, oito subdoses de glifosato (0 a 70\% da dose
	comercial), na sub-subparcela. Foram realizadas quatro avaliações visuais de
	fitotoxidade a cada 10 dias após a aplicação, utilizando-se escala percentual, onde
	0 representou ausência de sintomas e 100 representou morte de todas as plantas.
	Após análise de regressão dos valores obtidos, verificou-se que as modalidades de
	semeadura do capim não diferiram estatisticamente. Em ambos os estádios de
	aplicação as plantas sofreram com os efeitos das subdoses principalmente para as
	mais elevadas (60 e 70\%). Contudo, apesar do estresse provocado pelo herbicida e
	da competição com o milho a forrageira recuperou-se e retomou seu
	desenvolvimento, não sendo observada a morte total das plantas. Entretanto, foi
	observada de forma visual a redução da cobertura do solo pela forrageira quando
	utilizadas as subdoses mais elevadas do herbicida glifosato.
	
	\vspace{\onelineskip}
	
	\noindent
	\textbf{Palavras-chave}: Sistema ILP. Competição interespecífica. Fitotoxidade de
	herbicida.	
	
\end{document}