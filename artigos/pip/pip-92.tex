\documentclass[article,12pt,onesidea,4paper,english,brazil]{abntex2}

\usepackage{lmodern, indentfirst, nomencl, color, graphicx, microtype, lipsum}			
\usepackage[T1]{fontenc}		
\usepackage[utf8]{inputenc}		

\setlrmarginsandblock{2cm}{2cm}{*}
\setulmarginsandblock{2cm}{2cm}{*}
\checkandfixthelayout

\setlength{\parindent}{1.3cm}
\setlength{\parskip}{0.2cm}

\SingleSpacing

\begin{document}
	
	\selectlanguage{brazil}
	
	\frenchspacing 
	
	\begin{center}
		\LARGE PERMACULTURA: MÉTODOS SUSTENTÁVEIS DE ASSENTAMENTOS
		
		HUMANOS\footnote{Trabalho realizado dentro da (área de Conhecimento CNPq: Ciências Exatas e da Terra) com
			financiamento do IFRO/DEPESP Vilhena.}
		
		\normalsize
		Pedro Henrique Oliveira de Paula\footnote{Bolsista (auxílio pesquisador - DEPESP), pedroh03@hotmail.com, Campus Vilhena} 
		Kemilly Miranda Silveira\footnote{Colaboradora, kemilly1@outlook.com, Campus Vilhena} 
		Marisa Rodrigues de Lima\footnote{Orientadora, marisa.rodrigues@ifro.edu.br, Campus Vilhena} 
		Tatiana
		
		Abreu Curado Rezende\footnote{Co-orientadora, tatiana.rezende@ifro.edu.br, Campus Vilhena}
		 
	\end{center}
	
	\noindent Este presente trabalho teve como foco o estudo dos métodos convencionais de
	edificações, bem como seus malefícios à biodiversidade, a extração exacerbada de
	recursos naturais não renováveis pela indústria da construção civil e a má gestão do
	potencial energético no país. Os estudos dirigidos à Permacultura no âmbito da
	construção civil deixaram tangíveis as vertentes maneiras de se habitar no espaço
	de modo que a humanidade viva em comunhão com aquilo que a natureza lhe
	provém de melhor, fazendo-se de uma relação recíproca sem que a vida humana
	seja extremamente transformada. Essa proposta justifica-se pelo fato de a espécie
	humana estar longe de conceber transformações práticas em seus métodos
	construtivos, afim de não degradar o planeta Terra, assumindo, dessa forma, papel
	de agentes degradantes do espaço em que vivem. O trabalho foi realizado com base
	em uma revisão literária acerca do conceito de Permacultura, assim, foram
	necessários os integrantes do projeto de pesquisa e literatura o suficiente para que
	fosse feita a recolha e análise dos dados para que os resultados pudessem ser
	obtidos, além de inúmeros debates a respeito do tema. Após o término do trabalho,
	foram produzidas ao total quinze cartilhas que ficarão disponíveis na biblioteca do
	IFRO Campus Vilhena, afim de servir como suporte aos alunos do curso Técnico em
	Edificações além dos acadêmicos de Arquitetura e Urbanismo. Sendo assim, este
	trabalho procurou fomentar a utilização de métodos com base na Permacultura à
	comunidade acadêmica, desafiando a todos a serem responsáveis por sua própria
	existência baseados nos princípios da Permacultura. O objetivo geral deste trabalho
	foi alcançado por meio da disseminação da nova filosofia de vida, seja na
	construção civil ou, no modo de viver respeitando a terra em que vivemos. As
	diferentes formas de se edificar, deixando de lado as convencionais, foram
	suscitadas por meio deste trabalho, propiciando, portanto, maneiras da humanidade
	se assentar em terra sem que a mesma seja afetada significativamente de forma
	negativa.
	
	\vspace{\onelineskip}
	
	\noindent
	\textbf{Palavras-chave}:Permacultura. Construção civil. Sustentabilidade.
	
\end{document}
