\documentclass[article,12pt,onesidea,4paper,english,brazil]{abntex2}

\usepackage{lmodern, indentfirst, nomencl, color, graphicx, microtype, lipsum}			
\usepackage[T1]{fontenc}		
\usepackage[utf8]{inputenc}		

\setlrmarginsandblock{2cm}{2cm}{*}
\setulmarginsandblock{2cm}{2cm}{*}
\checkandfixthelayout

\setlength{\parindent}{1.3cm}
\setlength{\parskip}{0.2cm}

\SingleSpacing

\begin{document}
	
	\selectlanguage{brazil}
	
	\frenchspacing 
	
	\begin{center}
		\LARGE CONSTRUÇÃO DE UM CANTEIRO DE OBRAS MODELO E ESTUDO DOS
		IMPACTOS TRAZIDOS PELA ELABORAÇÃO DE UM LAYOUT PLANEJADO\footnote{Trabalho realizado dentro da área de Engenharia Civil com financiamento do Departamento de
			Pesquisa, Inovação e pós-graduação.}
		
		\normalsize
	Laura Rafaela da Silva Viana\footnote{Bolsista Modalidade PIBIC, laura.rafaela3@gmail.com, Campus Porto Velho Calama.} 
		Isadora Gonçalves Rodrigues\footnote{Colaboradora, isadora\_g@hotmail.com, Campus Porto Velho Calama.} 
	Mariana Dias de Andrade\footnote{Colaboradora, mariana.dandrade@hotmail.com, IFSP – Campus Avançado de Ilha Solteira.} 
	Leonardo
	Pereira Leocádio\footnote{Orientador, leonardo.leocadio@ifro.edu.br, Campus Porto Velho Calama.} 
		Valéria Costa de Oliveira\footnote{Co-orientadora, valeria.oliveira@ifro.edu.br, Campus Porto Velho Calama.}
	\end{center}
	
	\noindent 
	A indústria da construção civil é frequentemente citada como exemplo de setor
	atrasado, com baixos índices de produtividade e elevados percentuais de
	desperdício de recursos. Neste sentido, a discussão da qualidade e da produtividade
	tornou-se uma obrigação entre os profissionais e as empresas do setor. Tendo em
	vista que um dos fundamentais elementos no processo de obtenção e elevação da
	qualidade da indústria da Construção Civil é o Canteiro de Obras, ou seja, a “fábrica
	da edificação” o presente projeto de pesquisa e ensino tem como principal objetivo
	elaborar e construir o layout funcional de um Canteiro de Obras, verificando os
	resultados obtidos através de seu adequado planejamento. O projeto está sendo
	desenvolvido pelos alunos do terceiro ano do curso técnico em edificações integrado
	ao ensino médio em interface com a disciplina de gerenciamento de canteiro de
	obras. Como primeira etapa, o local de construção do futuro canteiro de obras foi
	utilizado como palco de estudos de medições para posterior criação de um layout
	planejado utilizando o software AutoCAD. Dessa forma, separados por times de
	aprendizado, os alunos produziram layouts diferentes que foram examinados por
	uma banca avaliadora. Depois da escolha e adequação pelos alunos do layout a ser
	utilizado iniciou-se o processo de orçamento, onde foi feito o levantamento e
	tabelamento de quantidades e preços dos materiais a serem utilizados para a
	construção. Com projeto ainda em andamento, o que se pode extrair como
	resultados é justamente o trabalho dos alunos para que grande parte do
	planejamento da construção do canteiro de obras fosse realizada. Após isso, o
	projeto inicia sua etapa de execução do layout planejado para logo em seguida a
	realização dos estudos dos impactos trazidos pela utilização de um canteiro de
	obras previamente planejado e já construído com as especificações de que
	necessita. A partir do englobamento de todos esses resultados, espera-se que mais
	uma grande contribuição seja feita a área da construção civil, enaltecendo a
	dependência e importância dos laços entre planejamento e execução.
	
	\vspace{\onelineskip}
	
	\noindent
	\textbf{Palavras-chave}: Edificações. Canteiro de Obras. Planejamento.
	
\end{document}
