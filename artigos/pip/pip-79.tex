\documentclass[article,12pt,onesidea,4paper,english,brazil]{abntex2}

\usepackage{lmodern, indentfirst, nomencl, color, graphicx, microtype, lipsum}			
\usepackage[T1]{fontenc}		
\usepackage[utf8]{inputenc}		

\setlrmarginsandblock{2cm}{2cm}{*}
\setulmarginsandblock{2cm}{2cm}{*}
\checkandfixthelayout

\setlength{\parindent}{1.3cm}
\setlength{\parskip}{0.2cm}

\SingleSpacing

\begin{document}
	
	\selectlanguage{brazil}
	
	\frenchspacing 
	
	\begin{center}
		\LARGE MAPEAMENTO DAS CULTURAS DE SOJA, SAFRA 2016/2017, NO MUNICÍPIO
		DE CABIXI (RO)\footnote{Trabalho realizado dentro da área de Conhecimento CNPq: Ciências Exatas e da Terra}
		
		\normalsize
		Marcelo Vinicius Assis de Brito\footnote{Acadêmico de Engenharia Agronômica, marcelobrito03@gmail.com, Campus Colorado do Oeste.} 
		Bárbara Laura Tavares\footnote{Acadêmica de Engenharia Agronômica, barbaralauratavares@gmail.com, Campus Colorado do
			Oeste} \\
		Ranieli dos Anjos de Souza Muler\footnote{Orientador(a), ranieli.muler@ifro.edu.br, Grupo de Pesquisas Espaciais (GREES), Campus Colorado
			do Oeste.} 
		Valdir
		Moura\footnote{Co-orientador, valdir.moura@ifro.edu.br, Grupo de Pesquisas Espaciais (GREES), Campus
			Colorado do Oeste.} 
	\end{center}
	
	\noindent Mapear áreas agrícolas de uma região é de grande relevância, dado que possibilita
	obter informações que podem ser base para a tomada de decisões político-
	administrativas sobre o território. Desta forma, o sensoriamento remoto subsidiado
	por uma gama de instrumentos, sistemas e sensores, orbitais ou não, são capazes
	de produzir informações a respeito da dinâmica da superfície terrestre em grandes
	extensões. Tais dados são úteis para avaliação da produção, produtividade,
	mudança do uso da terra entre outros, uma vez que, a agricultura condiciona grande
	parte dos arranjos econômicos e sócio-ambientais. Neste sentido, o objetivo deste
	trabalho foi mapear a cultura de soja no município de Cabixi (RO), safra 2016/2017,
	para obter estimativa de área plantada. Para obtenção de dados, utilizou-se imagens
	do sensor OLI/Landsat-8 (27/12/2016), órbita/ponto 230/069, bandas 6
	(infravermelho médio), 5 (infravermelho próximo) e 4 (visível vermelho), obtidas na
	plataforma da Divisão de Geração de Imagens (DGI) do Instituto Nacional de
	Pesquisas Espaciais (INPE). O processamento digital das imagens foi realizado com
	uso do Sistema de Informação Geográfica (SIG) Spring 5.5.1 e QGis 2.18. Neste
	estudo foi utilizado o método de classificação não-supervisionada, com uso do
	classificador Isoseg, similaridade e área 20/100. Após análise (pré e pós-
	processamento das imagens), foram estimados 19.278,52 hectares de área plantada
	com soja na área de estudo, considerando os estádios de V5 (vegetativo) a R6
	(reprodutivo). A estimativa de plantio de soja em Cabixi, safra 2016/2017, de acordo
	com os órgãos estaduais foi de 21.657 ha, desta forma, este estudo apresentou
	resultados aproximados. O valor de similaridade utilizado gerou confusão entre os
	alvos analisados, devido, principalmente, à pequena variação radiométrica existente
	entre os alvos. Contudo, acredita-se que a área de produção possa ser ainda maior,
	pois, durante o processamento foi possível observar por meio da interpretação das
	imagens em função da sua resposta espectral que, em dezembro, o município
	apresentava áreas na fase de semeadura e vegetativa inicial (V1 a V4), no entanto,
	esta fase não foi possível de ser discriminada com os métodos utilizados neste
	estudo, o que sugere o uso de outros classificadores e outros processos como o de
	classificação supervisionada.
	
	\vspace{\onelineskip}
	
	\noindent
	\textbf{Palavras-chave}: Soja. Mapeamento. Rondônia.
	
\end{document}
