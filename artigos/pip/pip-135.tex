\documentclass[article,12pt,onesidea,4paper,english,brazil]{abntex2}

\usepackage{lmodern, indentfirst, nomencl, color, graphicx, microtype, lipsum}			
\usepackage[T1]{fontenc}		
\usepackage[utf8]{inputenc}		

\setlrmarginsandblock{2cm}{2cm}{*}
\setulmarginsandblock{2cm}{2cm}{*}
\checkandfixthelayout

\setlength{\parindent}{1.3cm}
\setlength{\parskip}{0.2cm}

\SingleSpacing

\begin{document}
	
	\selectlanguage{brazil}
	
	\frenchspacing 
	
	\begin{center}
		\LARGE RESPOSTAS MORFOGÊNICAS DO CAPIM PANICUM MAXIMUM CV. MOMBAÇA
		ADUBADO COM SORO DE LEITE\footnote{Trabalho realizado dentro do Curso Técnico Agropecuária/Produção vegetal CNPq, com
			financiamento do CNPq e IFRO.}
		
		\normalsize
		Adilson Alexandre da Silva\footnote{Adilson Alexandre da Silva (Pibic EM) -adilsonfuturo@gmail.com- Campus Colorado do Oeste} 
		Angel Brenda Bueno dos Santos\footnote{Angel Brenda Bueno dos Santos -brendabueno8@gmail.com- Campus Colorado do Oeste} 
	Marcos Aurélio Anequine Macedo\footnote{Marcos Aurélio Anequine Macedo -marcos.anequine@ifro.edu.br- Campus Colorado do Oeste} 
	\end{center}
	
	\noindent  O projeto tem como estudo avaliar as respostas morfogênicas e produtividade de matéria
	seca de capim Panicum maximum cv. Mombaça adubado com soro de leite. Assim comparado a
	produtividade com resultados obtidos de produção do mesmo, submetido a adubação química. Nesse
	contexto, objetiva-se com este trabalho avaliar as respostas morfogênicas, biomassa verde e matéria
	seca de capim Panicum maximum cv. Mombaça adubado com soro de leite descartado pela indústria
	de laticínio da região. Serão estudadas quatro doses de soro de leite, correspondente aos volumes
	de: T1 = 50; T2 = 100; T3 = 150 e T4 = 200 m$^3$ ha-$^1$, comparados a adubação química (T5 = 100 kg
	de N ha-1, T6 = 200 kg de N ha-1, T7 = 300 kg de N ha-1) e o tratamento testemunha (T8 = sem
	adubação).
	O capim Mombaça necessita de quantidades ideais de fertilidade para seu bom
	desenvolvimento, no entanto existe um impasse devido ao alto custo dos fertilizantes. Contudo,
	existem formas alternativas (orgânicas) de adubação que podem ser empregadas visando
	produtividades semelhantes ou até superiores ao próprio adubo químico. O soro de leite é uma
	alternativa economicamente viável. É um resíduo que pode ser adquirido facilmente em laticínios da
	região, o projeto em questão conta com a parceria do laticínio Multibom, localizado a 4 km do local
	onde será o experimento.
	A área onde o experimento está sendo conduzido foi roçada com roçadeira manual costal e 7
	(sete) dias depois aplicou-se o herbicida Glifosato, com 356 g/L de equivalente Ácido de Glifosato,
	com bomba de aplicação costal, na dose de 6 L/ha, para a completa limpeza do local. O solo foi
	preparado de forma semi-mecanizada, com a utilização de microtrator Tobata a Diesel, de 14
	(quatorze) cv. A palhada resultante foi incorporada ao solo no momento do seu preparo. As parcelas
	foram delimitadas com estacas de madeira e utilizando o delineamento experimental em blocos
	casualisados (DBC) e constituindo-se de 8 (oito) tratamentos e 3 (três) repetições. A semeadura do
	capim ocorreu em linhas espaçadas 0,25 m.
	\vspace{\onelineskip}
	
	\noindent
	\textbf{Palavras-chave}: Adubação. pastagem. biomassa.
	
\end{document}
