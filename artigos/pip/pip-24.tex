\documentclass[article,12pt,onesidea,4paper,english,brazil]{abntex2}

\usepackage{lmodern, indentfirst, nomencl, color, graphicx, microtype, lipsum}			
\usepackage[T1]{fontenc}		
\usepackage[utf8]{inputenc}		

\setlrmarginsandblock{2cm}{2cm}{*}
\setulmarginsandblock{2cm}{2cm}{*}
\checkandfixthelayout

\setlength{\parindent}{1.3cm}
\setlength{\parskip}{0.2cm}

\SingleSpacing

\begin{document}
	
	\selectlanguage{brazil}
	
	\frenchspacing 
	
	\begin{center}
		\LARGE BIOQUÍMICA DO CHOCOLATE - BIOQUÍMICA NA COZINHA\footnote{Trabalho realizado dentro da (área de Conhecimento CNPq: Química.}
		
		\normalsize
		Bruna da Silva Mandu\footnote{Bolsista (voluntária), bruna.manduu@gmail.com, Ji-Paraná.} 
		Pedro Gabriel Silva e Silva\footnote{Colaborador, pedrobieljip@gmail.com, Ji-Paraná.} 
		Pâmela Siqueira Oliveira de Jesus\footnote{Orientadora, pamela.siqueira@ifro.edu.br, Ji-Paraná.} 
		Camila Budim Lopes\footnote{Co-orientadora, camila.lopes@ifro.edu.br, Colorado D’Oeste.} 
	\end{center}
	
	\noindent Os alimentos são resultado de combinações complexas de diversas espécies químicas, como carboidratos, lipídeos e proteínas. O chocolate, por apresentar-se em variadas composições, possui valores nutricionais distintos e, consequentemente, reações diversas no organismo humano. Com o objetivo de expor as características bioquímicas de diferentes alimentos e seus valores nutricionais assim como sua ação no organismo humano, ocorreu em 2016 a amostra Bioquímica na Cozinha, apresentada à comunidade interna e externa do IFRO - Campus Ji-Paraná, onde o grupo apresentou a bioquímica do chocolate, bem como verificou o conhecimento de aspectos bioquímicos do chocolate por parte do público. No decorrer da exposição, um formulário foi apresentado aos ouvintes, a fim de traçar o perfil do consumidor de chocolate e verificar os conhecimentos dos mesmos acerca do alimento. Em seguida, o grupo apresentou as diferenças entre os tipos de chocolates, os benefícios e malefícios, bem como a importância do consumo moderado. Ao avaliar as respostas do formulário, constatou-se que 40,2\% dos entrevistados consomem o produto com frequência, enquanto 38,1\% consomem ocasionalmente e 21,6\% consomem esporadicamente. Também constatou-se que 40,2\% não possuem conhecimento acerca da ação do chocolate no organismo, e 33\% afirmaram ter este conhecimento enquanto 26,8\% nunca procuraram saber sobre. Dessa forma, nota-se que a população entrevistada, não possui conhecimento a respeito do chocolate, embora o consuma com demasiada frequência. Ao explicar a respeito das reações bioquímicas, o público demonstrou-se interessado, questionando sobre termos explicados e comentando sobre experiências próprias. Além dos objetivos alcançados com o presente trabalho, ao grupo também foi concedida a oportunidade de aplicar o conhecimento adquirido em sala de aula e relacioná-lo ao cotidiano.
	
	\vspace{\onelineskip}
	
	\noindent
	\textbf{Palavras-chave}: Bioquímica. Conhecimento. Chocolate.
	
\end{document}
