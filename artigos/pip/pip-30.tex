\documentclass[article,12pt,onesidea,4paper,english,brazil]{abntex2}

\usepackage{lmodern, indentfirst, nomencl, color, graphicx, microtype, lipsum}			
\usepackage[T1]{fontenc}		
\usepackage[utf8]{inputenc}		

\setlrmarginsandblock{2cm}{2cm}{*}
\setulmarginsandblock{2cm}{2cm}{*}
\checkandfixthelayout

\setlength{\parindent}{1.3cm}
\setlength{\parskip}{0.2cm}

\SingleSpacing

\begin{document}
	
	\selectlanguage{brazil}
	
	\frenchspacing 
	
	\begin{center}
		\LARGE COMBATE AO AEDES AEGYPTI - JOGO EDUCACIONAL DESENVOLVIDO COM
		AS FERRAMENTAS UNITY E BLENDER\footnote{Informações sobre o resumo.}
		
		\normalsize
		Ana C. R. Miranda\footnote{Ana Carolina Ribeiro Miranda PIP, anacrm33@gmail.com, Campus Ji-Paraná.} 
	Paula G. M. Valadares\footnote{Paula Graziela Militão Valadares PIP, grazielamilitaov@gmail.com, Campus Ji-Paraná.} 
		Maiker H. M. de Oliveira\footnote{Jackson Henrique da Silva Bezerra, jackson.henrique@ifro.edu.br, Campus Ji-Paraná.} 
		Gabriel P. F. dos Santos
		Eisen G. da Foncesa
	\end{center}
	
	\noindent Atualmente, um dos principais ramos do mundo da informática é o
	desenvolvimento de games. Os jogos são ferramentas que contribuem na formação
	das crianças e, por ter função de entretenimento, se torna mais atrativo e eficiente
	para as mesmas. Uma ferramenta muito utilizada hoje em dia para o
	desenvolvimento de games é o \textit{Unity}, que possuem um estilo de programação e
	organização dos projetos muito didática, além de ser de fácil entendimento. Outra
	ferramenta muito utilizada nos desenvolvimentos de jogos para modelagem de
	personagens é o \textit{Blender}, que serve para modelagem, animação, texturização,
	composição, renderização, e criação de aplicações interativas em 3D, tais como
	jogos. O vírus transmitido pelo mosquito \textit{Aedes aegypti} é uma das preocupações da
	sociedade nos dias de hoje. Pensando nisso e, utilizando as ferramentas citadas, foi
	desenvolvido um jogo infantil que incentiva a eliminação de potenciais focos desses
	mosquitos causadores de diversos males. Para a construção do jogo foi necessário
	escolher a o software que viria a ser utilizado. A equipe, juntamente com o
	orientador, optou por utilizar as ferramentas desenvolvedoras \textit{Unity} e \textit{Blender},
	assim, houve a necessidade de se adaptar a ambos os softwares. O projeto foi
	desenvolvido por alunos do ensino médio e integrado do IFRO - Campus Ji-Paraná,
	sendo dois bolsistas e três colaboradores. Os alunos tiveram o auxílio das
	ferramentas citadas anteriormente, ambas disponibilizados pela coordenação do
	IFRO e, presente nos laboratórios de informática, onde, em um deles, a equipe
	desenvolveu o game que foi baseado no combate ao mosquito \textit{Aedes aegypti}, e tem
	como objetivo a conscientização das crianças. No início do projeto foi se adquirindo
	cada vez mais conhecimento das ferramentas que foram utilizadas, por intermédio
	de diversas pesquisas, onde aprendemos a desenvolver e modelar nas devidas
	ferramentas \textit{Unity}, e, posteriormente, \textit{Blender}. Com o conhecimento básico para
	desenvolvimento e modelagem adquirido, foi possível visualizar o objetivo final. Após
	isso, junto com a orientação do professor, definimos como seria o personagem e a
	interação com o foco do mosquito. O \textit{Blender} e o \textit{Unity} mostraram-se plataformas de
	extrema relevância para o desenvolvimento de games.
	
	\vspace{\onelineskip}
	
	\noindent
	\textbf{Palavras-chave}:Aedes aegypti, Ferramentas, jogo. \\
	\textbf{Fonte de financiamento}: IFRO
	
\end{document}
