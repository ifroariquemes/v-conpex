\documentclass[article,12pt,onesidea,4paper,english,brazil]{abntex2}

\usepackage{lmodern, indentfirst, nomencl, color, graphicx, microtype, lipsum}			
\usepackage[T1]{fontenc}		
\usepackage[utf8]{inputenc}		

\setlrmarginsandblock{2cm}{2cm}{*}
\setulmarginsandblock{2cm}{2cm}{*}
\checkandfixthelayout

\setlength{\parindent}{1.3cm}
\setlength{\parskip}{0.2cm}

\SingleSpacing

\begin{document}
	
	\selectlanguage{brazil}
	
	\frenchspacing 
	
	\begin{center}
		\LARGE MÉTODOS DE INOCULAÇÃO COM Azospirillum brasilense EM MILHO
		
		SAFRINHA\footnote{Trabalho realizado dentro da (área de Conhecimento CNPq: Ciências Agrárias) com financiamento
			do IFRO, Campus Ariquemes.}
		
		\normalsize
		Anderson Ferreira de Aquino\footnote{Bolsista (IC ET), anderson.aquino.1999@gmail.com, Campus Ariquemes} 
		Lucas Souza Markoviscz\footnote{Bolsista (IC ET), hiriaqthiariane@gmail.com, Campus Ariquemes} 
		Luciano dos Reis Venturoso\footnote{Orientador, luciano.venturoso@ifro.edu.br, Campus Ariquemes} 
		Lenita
		
		Aparecida Conus Venturoso\footnote{Co-orientadora, lenita.conus@ifro.edu.br, Campus Ariquemes} 
	\end{center}
	
	\noindent A cultura do milho é muito exigente em nutrientes, principalmente o nitrogênio. No
	entanto, este nutriente tem custo elevado e envolve-se em várias reações no solo, o
	que dificulta o seu manejo e disponibilidade para a planta. O uso de inoculantes
	contendo bactérias fixadoras de nitrogênio, como Azospirillum, pode resultar em
	uma economia importante para o agricultor. Diante do exposto, objetivou-se avaliar a
	utilização da bactéria diazotrófica Azospirillum brasilense associada a doses
	crescentes de adubação nitrogenada no desenvolvimento e produtividade de milho
	na safrinha, assim como o potencial de economia do fertilizante nitrogenado
	combinado a inoculação. O experimento foi conduzido, em Latossolo Vermelho
	Amarelo Distrófico, na área experimental do Instituto Federal de Rondônia, campus
	Ariquemes, em cultivo de safrinha. Foi adotado o delineamento experimental de
	blocos casualizados, em arranjo fatorial 4 x 5, com quatro repetições. Foi utilizado
	quatro métodos de inoculação da bactéria A. brasilense: via sementes, foliar, no
	sulco de plantio e uma testemunha; e cinco doses de fertilizante nitrogenado, 0, 40,
	80, 120 e 160 kg.ha-1 de N, na forma de ureia. A cultura foi semeada em parcelas
	contendo quatro linhas de 5 m de comprimento, espaçadas 0,8 m entre si. Foram
	avaliados os caracteres vegetativos e reprodutivos da cultura, como o percentual de
	emergência, ciclo fenológico, altura de plantas no florescimento e na maturação
	fisiológica, diâmetro do colmo, índice de área foliar, altura de inserção da espiga,
	percentual de acamamento + quebra, estande final, teor de nitrogênio foliar,
	comprimento e massa da espiga, número de fileira por espiga, números de grãos por
	fileira, massa de cem grãos e o rendimento de grãos. A inoculação via sementes
	proporcionou maior altura de plantas em relação à inoculação por sulco. Os métodos
	de inoculação adotados resultaram em espigas mais altas que o tratamento controle.
	Com relação aos componentes do rendimento, verificou-se que a inoculação via
	semente e sulco resultaram em aumento no número de grãos por fileira. Para o
	rendimento de grãos, foi observado que as sementes inoculadas apresentaram
	superioridade quando comparado à inoculação foliar.
	
	\vspace{\onelineskip}
	
	\noindent
	\textbf{Palavras-chave}:Zea mays. Bactérias fixadoras de nitrogênio. Formas de
	inoculação.
	
    \noindent 
    \textbf{Fonte de Financiamento}:Instituto Federal de Rondônia, Campus Ariquemes.
	
	
\end{document}
