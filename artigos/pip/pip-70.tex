\documentclass[article,12pt,onesidea,4paper,english,brazil]{abntex2}

\usepackage{lmodern, indentfirst, nomencl, color, graphicx, microtype, lipsum}			
\usepackage[T1]{fontenc}		
\usepackage[utf8]{inputenc}		

\setlrmarginsandblock{2cm}{2cm}{*}
\setulmarginsandblock{2cm}{2cm}{*}
\checkandfixthelayout

\setlength{\parindent}{1.3cm}
\setlength{\parskip}{0.2cm}

\SingleSpacing

\begin{document}
	
	\selectlanguage{brazil}
	
	\frenchspacing 
	
	\begin{center}
		\LARGE INTERAÇÃO DE BACTÉRIAS FIXADORAS DE NITROGÊNIO SOBRE O
		DESENVOLVIMENTO E RENDIMENTO DO FEIJOEIRO\footnote{Trabalho realizado dentro da (área de Conhecimento CNPq: Ciências Agrárias) com financiamento
			do CNPq / IFRO.}
		
		\normalsize
		Sarah Bruna Amontari Pinheiro\footnote{Bolsista (PIBIC EM), sarahbrun4@gmail.com, Campus Ariquemes} 
		Mateus de Souza de Oliveira\footnote{Bolsista (PIBIC), mateus\_97so@yahoo.com, Campus Ariquemes} \\
		Luciano dos Reis Venturoso\footnote{Orientador, luciano.venturoso@ifro.edu.br, Campus Ariquemes} 
		Lenita
		
		Aparecida Conus Venturoso\footnote{Co-orientadora, lenita.conus@ifro.edu.br, Campus Ariquemes} 
	
        
    \end{center}    
	
	\noindent A fixação biológica do nitrogênio é o processo através do qual o nitrogênio presente
	na atmosfera é convertido em formas que podem ser utilizadas pelas plantas. Esse
	processo é realizado por bactérias fixadoras de nitrogênio, às quais adicionadas as
	sementes do feijão podem substituir, total ou parcialmente, o uso de fertilizantes
	nitrogenados. Diante do exposto, objetivou-se avaliar o efeito da co-inoculação de
	sementes com Rhizobium tropici e Azospirillum brasilense sobre o desenvolvimento
	e rendimento do feijoeiro em semeadura direta. A pesquisa foi conduzida na área
	experimental do Instituto Federal de Rondônia, Campus Ariquemes, em Latossolo
	Vermelho Amarelo Distrófico. Foi utilizada a cultivar Carioca Precoce, visando
	população final de 260.000 plantas por hectare. Adotou-se o delineamento de blocos
	casualizados, com nove repetições. Os tratamentos constaram das inoculações de
	sementes: Rhizobium tropici, Azospirillum brasilense, Rhizobium tropici +
	Azospirillum brasilense e um tratamento controle. A semeadura foi realizada em
	parcelas de 1,8 x 5 m de comprimento. Foi avaliada aos 25 dias após a semeadura
	(DAS) e no florescimento pleno do feijoeiro, a altura de plantas, comprimento de raiz,
	número de nódulos/planta, massa seca de nódulos, massa seca da parte aérea e
	raiz. Na colheita, número de vagens por planta, de grãos por vagem, massa de cem
	grãos e o rendimento, com 13\% de umidade. Foi observado maior percentual de
	emergência nas sementes co-inoculadas quando comparado à testemunha. Por
	ocasião do florescimento do feijoeiro, constatou-se superioridade na massa seca,
	para as sementes inoculadas com R. tropici. Não foi verificado efeito significativo da
	inoculação de sementes sobre os componentes do rendimento, tampouco, sobre a
	produtividade de grãos da cultura. A produtividade média do feijoeiro foi de 526,4
	kg.ha$^{-1}$.
	
	\vspace{\onelineskip}
	
	\noindent
	\textbf{Palavras-chave}:Phaseolus vulgaris. Rhizobium tropici. Azospirillum brasilense.
	
	\noindent
	\textbf{Fonte de Financiamento}: CNPq e Instituto Federal de Rondônia.
	
\end{document}
