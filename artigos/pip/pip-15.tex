\documentclass[article,12pt,onesidea,4paper,english,brazil]{abntex2}

\usepackage{lmodern, indentfirst, nomencl, color, graphicx, microtype, lipsum}			
\usepackage[T1]{fontenc}		
\usepackage[utf8]{inputenc}		

\setlrmarginsandblock{2cm}{2cm}{*}
\setulmarginsandblock{2cm}{2cm}{*}
\checkandfixthelayout

\setlength{\parindent}{1.3cm}
\setlength{\parskip}{0.2cm}

\SingleSpacing

\begin{document}
	
	\selectlanguage{brazil}
	
	\frenchspacing 
	
	\begin{center}
		\LARGE USO DE RESÍDUOS ORGÂNICOS PARA PRODUÇÃO DE MUDAS DE CANAFÍSTULA
		(CASSIA FISTULA L.) NA REGIÃO CENTRAL DE RONDÔNIA\footnote{Engenharia Florestal
			Com financiamento do edital 35 de 2016}
		
		\normalsize
		Milene Queiroz Brunaldi Lima\footnote{Bolsista (acadêmico), milabrunaldi@gmail.com, Campus Ji-Paraná} 
		Polyana Barros Nascimento Carvalho\footnote{Bolsista (acadêmico), polyanabarrosnc@gmail.com, Campus Ji-Paraná} \\
		Andreza Mendonça\footnote{Orientador(a), andreza.mendonca@ifro.edu.br, Campus Ji-Paraná} 
		Maria Elessandra Rodrigues Araujo\footnote{Co-orientador(a), maria.elessandra@ifro.edu.br, Campus Ji-Paraná} 
	\end{center}
	
	\noindent O êxito na formação de florestas de alta produção depende, em grande parte, da qualidade
	das mudas plantadas, que além de terem que resistir às condições adversas encontradas no
	campo após o plantio deverá sobreviver e, por fim, produzir árvores com crescimento
	desejável. Portanto, o objetivo do trabalho foi avaliar o desenvolvimento de mudas de
	canafistula a partir do reaproveitamento de resíduos orgânicos como substrato. As sementes
	foram coletadas em áreas circunvizinhas ao município de Ji-Paraná, Rondônia, as quais
	foram beneficiadas e semeadas em areia lavada. Os resíduos foram coletados em áreas
	circunvizinhas a Ji-Paraná e formada pilhas para compostagem do material, os materiais
	utilizados nas pilhas de compostagem por tratamento foram: casca da mandioca, folha da
	leucina (Leucaena leucocephala), palha de café, bagaço de cana-de-açúcar. Foram
	instalados seis tratamentos, distribuídos inteiramente ao acaso com quatro repetições, as
	pilhas de compostagem foram revolvidas para oxigenação, e umedecidos quando
	necessário, diariamente. Além disso, foi verificada a temperatura das pilhas, durante todo
	processo de compostagem de 120 dias. Desta forma, espera-se apontar um substrato que
	atenda as exigências nutricionais da espécie avaliada. O desenvolvimento das mudas de
	canafístula foi no 50\% de sombreamento sob diferentes misturas de substratos resultado da
	compostagem. Houve diferenças significativas nas médias da altura das mudas entre os
	substratos testados, sendo que valores mais altos foram constatados quando se utilizou o
	substrato proveniente da compostagem de casca de mandioca + leucina, Altura (cm) de
	24,840, provavelmente este substrato permitiu um maior acumulo de reservas. Provando
	que as propriedades físicas e químicas do substrato são fundamentais para o
	desenvolvimento da muda.
	
	\vspace{\onelineskip}
	
	\noindent
	\textbf{Palavras-chave}: Compostos orgânicos. Mudas florestais. Nutrição florestal.
	
\end{document}
