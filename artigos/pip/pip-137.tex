\documentclass[article,12pt,onesidea,4paper,english,brazil]{abntex2}

\usepackage{lmodern, indentfirst, nomencl, color, graphicx, microtype, lipsum}			
\usepackage[T1]{fontenc}		
\usepackage[utf8]{inputenc}		

\setlrmarginsandblock{2cm}{2cm}{*}
\setulmarginsandblock{2cm}{2cm}{*}
\checkandfixthelayout

\setlength{\parindent}{1.3cm}
\setlength{\parskip}{0.2cm}

\SingleSpacing

\begin{document}
	
	\selectlanguage{brazil}
	
	\frenchspacing 
	
	\begin{center}
		\LARGE \MakeUppercase{Respostas morfogênicas do capim Panicum maximum cv. Mombaça com soro de leite}\footnote{Trabalho realizado dentro do Curso Técnico Agropecuária/Forragicultura, com financiamento do CNPq e IFRO.}
		
		\normalsize
	Adilson Alexandre da Silva	\footnote{Bolsista (PIBIC EM), adilsonfuturo@gmail.com, Campus Colorado do Oeste.} 
		Angel Brenda Bueno dos Santos\footnote{Bolsista (PIBIC), brendabueno8@gmail.com, Campus Colorado do Oeste.} \\
		Guilherme Peiter Pires\footnote{Colaborador, peiterpires@gmail.com, Campus Colorado do Oeste.} 
	Marcos Aurélio Anequine de Macedo\footnote{Orientador, marcos.anequine@ifro.edu.br, Campus Colorado do Oeste.} 
	\end{center}
	
	\noindent Para preservar as melhores respostas morfogênicas no capim Mombaça é necessário manter a fertilidade do solo em níveis favoráveis ao desenvolvimento da planta. A determinação precisa da quantidade de forragem disponível é importante porque a partir desta pode-se calcular a capacidade de suporte da pastagem para evitar desperdícios ou superlotação. Nesse contexto, objetivou-se com este trabalho avaliar a relação folha/colmo e folha/material morto, do capim Panicum maximum cv. Mombaça adubado com soro de leite descartado pela indústria de laticínio da região. O experimento foi conduzido em uma propriedade rural de Colorado do Oeste em pasto já estabelecido, durante o período de maio a agosto de 2017. O delineamento experimental utilizado foi de blocos casualizados, em esquema de parcelas subdivididas, com três repetições. Foram estudadas quatro doses de soro de leite, correspondente aos volumes de (T1 = 66; T2 = 133; T3 = 200 e T4 = 267 m$^3$ ha-$^1$), aplicado no capim com regador e comparados a adubação com ureia (T5 = 66; T6 = 133, T7 = 200 kg de ureia ha-1) e um tratamento testemunha (T8 = sem adubação). As doses de soro de leite foram divididas em quatro aplicações, uma a cada mês, e duas de ureia. O capim foi cortado a 45 cm do solo em 1m$^2$ da região central da parcela e levados ao laboratório de bromatologia para a realização da separação morfológica (folha, colmo e material morto). As amostras foram secadas a temperatura de 65ºC por 72 horas em estufa de circulação de ar, sendo pesadas antes e após a secagem. Os dados obtidos foram analisados utilizando o teste de Scott- Knott para comparação das médias. Não houve resultados significativos para as variáveis avaliadas. Não houve presença de colmos nas amostras coletadas.
	
	\vspace{\onelineskip}
	
	\noindent
	\textbf{Palavras-chave}: Adubação. Pastagem. Biomassa.
	
\end{document}
