\documentclass[article,12pt,onesidea,4paper,english,brazil]{abntex2}

\usepackage{lmodern, indentfirst, nomencl, color, graphicx, microtype, lipsum}			
\usepackage[T1]{fontenc}		
\usepackage[utf8]{inputenc}		

\setlrmarginsandblock{2cm}{2cm}{*}
\setulmarginsandblock{2cm}{2cm}{*}
\checkandfixthelayout

\setlength{\parindent}{1.3cm}
\setlength{\parskip}{0.2cm}

\SingleSpacing

\begin{document}
	
	\selectlanguage{brazil}
	
	\frenchspacing 
	
	\begin{center}
		\LARGE MORCEGO-VAMPIRO (Desmodus rotundus, E. Geoffroy, 1810) EM
		REMANESCENTE FLORESTAL, ARIQUEMES, RONDÔNIA, BRASIL
		\footnote{Trabalho realizado dentro das Ciências Biológicas sem fomento de Instituição de Pesquisa}
		
		\normalsize
		Alysson Rossi dos Santos\footnote{Acadêmico do curso de Ciências Biológicas, alyssonr@hotmail.com, Instituto Federal de Educação,
			Ciência e Tecnologia de Rondônia - IFRO campus Ariquemes} 
		Elaine Oliveira Costa de Carvalho\footnote{Orientador, elaine.carvalho@ifro.edu.br.} 
		
	\end{center}
	
	\noindent Chiroptera é a ordem dos mamíferos que abrange todas as espécies de morcegos.
	É uma das mais amplas, representando cerca de 22\% do total de mamíferos
	conhecidos do planeta. Existe um quiróptero da Família Phyllostomidae que é
	conhecido popularmente como o morcego-vampiro, Desmodus rotundus, um animal
	hematófago que pesa entre 25 e 40 gramas, sendo sua dieta composta estritamente
	do consumo de sangue. Ocorre somente na América Latina, desde o norte da
	Argentina ao norte do México, tendo como principal habitat áreas florestadas ou
	desérticas. Segundo a lista da IUCN e dados do IBAMA não é uma espécie
	ameaçada. O estudo teve como objetivo realizar inventário da mastofauna em
	remanescente florestal do IFRO Campus Ariquemes, durante 2015 a 2017, a fim de
	elaborar um guia de identificação das possíveis espécies amostradas. O método
	adotado foi o registro de imagem em vídeo com equipamento sensível ao movimento
	(armadilha fotográfica), funcionando a pilhas e memória com capacidade para
	armazenar até 470 vídeos com duração de 30 segundos/cada. O equipamento foi
	afixado, na altura de 40 cm distante do solo, em espécie vegetal arborícola de
	pequeno porte que é utilizada como marcador de trilha por animais silvestres. A
	leitura da memória e a manutenção do equipamento foram realizadas a cada sete
	dias. No inventário registrou-se 11 vídeos com o D. rotundus, todos no período
	noturno. Desse total, em dez vídeos (91\%) foram no período correspondente à
	estação de chuvas na Amazônia e em apenas um vídeo na estação seca. As
	ocorrências foram em ato de predação (27\%) ou sobrevoando próximo (73\%) a
	espécie Tapirus terrestris, conhecida popularmente como anta, que é um mamífero
	herbívoro de grande porte, pesando entre 130 e 300 kg quando adulto. Um dos
	registros do ato de predação ocorreu em individuo filhote da anta enquanto
	acompanhava a mãe, sendo todos os demais registros somente com o indivíduo
	adulto fêmea. Os registros sugerem a anta como um mamífero importante na dieta
	alimentar do morcego-vampiro e que o remanescente florestal pode proporcionar
	habitat favorável à existência de uma população da espécie D. rotundus.
	
	\vspace{\onelineskip}
	
	\noindent
	\textbf{Palavras-chave}: Quirópteros, Tapirus terrestris, IFRO.
	
\end{document}
