\documentclass[article,12pt,onesidea,4paper,english,brazil]{abntex2}

\usepackage{lmodern, indentfirst, nomencl, color, graphicx, microtype, lipsum}			
\usepackage[T1]{fontenc}		
\usepackage[utf8]{inputenc}		

\setlrmarginsandblock{2cm}{2cm}{*}
\setulmarginsandblock{2cm}{2cm}{*}
\checkandfixthelayout

\setlength{\parindent}{1.3cm}
\setlength{\parskip}{0.2cm}

\SingleSpacing

\begin{document}
	
	\selectlanguage{brazil}
	
	\frenchspacing 
	
	\begin{center}
		\LARGE BACIA HIDROGRÁFICA COMO UNIDADE TERRITORIAL PARA A ÁREA DE AGROPECUÁRIA NO MUNICIPIO DE CACOAL/RO\footnote{Trabalho realizado dentro da área de Conhecimento CNPq: Geografia. Não houve financiamento}
		
		\normalsize
	Amanda Rates Pereira\footnote{Aluna do Curso de Técnico em Agropecuária, no Campus Cacoal, do Instituto Federal de Educação, Ciência e Tecnologia de Rondônia – IFRO. E-mail: amanda-ratis@hotmail.com} 
		Isabela Pereira Laurenço\footnote{Aluna do Curso de Técnico em Agropecuária, no Campus Cacoal, do Instituto Federal de Educação, Ciência e Tecnologia de Rondônia – IFRO. E-mail: isabelapereira.ifro@hotmail.com} 
	Livia Pereira Laurenço\footnote{Aluna do Curso de Técnico em Agropecuária, no Campus Cacoal, do Instituto Federal de Educação, Ciência e Tecnologia de Rondônia – IFRO. E-mail: liviapereira7@outlook.com} 
	Nathalia Ratis Truiz\footnote{Aluna do Curso de Técnico em Agropecuária, no Campus Cacoal, do Instituto Federal de Educação, Ciência e Tecnologia de Rondônia – IFRO. E-mail: nathaliaratis02@gmail.com}
	Ayrton Schupp Pinheiro Oliveira\footnote{Professor/Orientador de Geografia no Campus Cacoal, do Instituto Federal de Educação, Ciência e Tecnologia de Rondônia – IFRO. E-mail: ayrtonspoliveira@gmail.com} 
	\end{center}
	
	\noindent Devido à água ser um recurso natural, único, escasso, essencial à vida e ser distribuída de forma desigual no planeta ganha-se uma relevância o tema do manejo e preservação das bacias hidrográficas. A presente pesquisa teve como objetivo o levantamento do crescimento de cabeças de gado do município de Cacoal e o mapeamento das unidades hidrográficas do mencionado município. A pesquisa contou com a participação das discentes do segundo ano do Curso Integrado em Agropecuária do Instituto Federal de Rondônia/Campus Cacoal. O presente trabalho se utilizou inicialmente de leituras bibliográficas do acervo da Biblioteca do Instituto Federal de Rondônia/ Campus Cacoal, assim como de artigos científicos publicados na rede mundial de computadores. Após as leituras, foi feito um levantamento dos dados analíticos de produção anual da agropecuária do município de Cacoal, disponibilizados pelo Instituto Brasileiro de Geografia e Estatística (IBGE), sendo estes dados, compilados em um único arquivo. Juntamente com os dados do IBGE, foi produzido um mapa das principais bacias hidrográficas do município de Cacoal. Para a confecção dos mapas, foram utilizados dados disponíveis em formato digital shapefile da Secretária de Estado de Desenvolvimento Ambiental (SEDAM), onde foram confeccionados por um software livre, sendo posteriormente confeccionado um mapa contendo as unidades hidrográficas que compõem a região do município de Cacoal. Após a produção dos dados, foram organizadas as relações da produção agropecuária com a utilização dos corpos hídricos do município de Cacoal. Os resultados da pesquisa apontam para um crescimento exponencial do numero de cabeças de gado e que podem gerar problemas quanto à compactação e aceleração dos processos erosivos, prejudicando na deterioração dos recursos hídricos do município e consequentemente na queda da produção econômica na agropecuária do município.
	
	\vspace{\onelineskip}
	
	\noindent
	\textbf{Palavras-chave}: Cacoal. Agropecuária. Bacia Hidrográfica.
	
\end{document}
