\documentclass[article,12pt,onesidea,4paper,english,brazil]{abntex2}

\usepackage{lmodern, indentfirst, nomencl, color, graphicx, microtype, lipsum}			
\usepackage[T1]{fontenc}		
\usepackage[utf8]{inputenc}		

\setlrmarginsandblock{2cm}{2cm}{*}
\setulmarginsandblock{2cm}{2cm}{*}
\checkandfixthelayout

\setlength{\parindent}{1.3cm}
\setlength{\parskip}{0.2cm}

\SingleSpacing

\begin{document}
	
	\selectlanguage{brazil}
	
	\frenchspacing 
	
	\begin{center}
		\LARGE ESTUDO DO HISTÓRICO DA VARIABILIDADE DOS PREÇOS DE VENDA DE CAPSICUM ANNUUM NOS PRINCIPAIS CENTROS DE DISTRIBUIÇÃO DO PAÍS\footnote{Trabalho realizado dentro da área de Conhecimento CNPq: Agronomia com financiamento do Instituto Federal de Educação, Ciência e Tecnologia de Rondônia.}
		
		\normalsize
		Sâmilla Emilly de Oliveira Gouveia,\footnote{Bolsista IFRO, emillygoouveia@gmail.com, Campus Colorado do Oeste.} 
		Marcos Antonio Eliziario de Oliveira,\footnote{Colaborador, marcosoliveira.mo210@gmail.com, Campus Colorado do Oeste.} 
		Willian Mota\footnote{Orientador, willianmota@ifro.edu.br Campus Colorado do Oeste.} 
	
	\end{center}
	
	\noindent O estado de Rondônia ainda é muito dependente da importação de alguns produtos hortifrutigranjeiros, entre eles o pimentão verde. O cultivo da cultura se torna uma oportunidade de significativa lucratividade ao produtor rural rondoniense. Dessa forma é importante que o produtor realize um estudo prévio do preço de mercado visando um planejamento de cultivo adequado. Por ser um produto considerado elástico, seu consumo varia principalmente em função do seu preço. A partir do cálculo do custo da produção do pimentão e analisando o valor nos principais centros de distribuição do país, o produtor poderá oferecer seu produto a um preço inferior ao praticado no mercado quando possível, e assim garantir destino a sua produção. Assim, o presente trabalho tem como objetivo auxiliar o produtor rondoniense a planejar sua colheita de pimentão para comercialização identificando por meio desse trabalho os meses com melhor cotação dos preços. Para o efetivo estudo foram coletados os valores médios do Kg do pimentão verde das centrais de abastecimento de Campinas, Belo Horizonte e Curitiba, e estes foram retificados pelo índice IGP-M e utilizados subsequentemente para a produção de médias mensais. Assim, foi realizada a elaboração de gráficos e em seguida estudo do histórico da variabilidade dos preços entre os anos de 2012 e 2016. Posterior ao estudo dos gráficos foi verificado que os maiores preços de mercado no pimentão verde são praticados entre os meses de janeiro e abril, e de julho a outubro. Ao projetar a sua colheita para os períodos de maior valor de comercialização do produto, poderá oferecer seu produto a um preço inferior ao praticado no mercado quando possível, e assim garantir destino à sua produção.
	
	\vspace{\onelineskip}
	
	\noindent
	\textbf{Palavras-chave}: Agronegócio. Comercialização. Pimentão. \\
	\textbf{Fonte de financiamento}: IFRO. \\
	\textbf{Área do conhecimento do CNPQ}: Agronomia.
	
\end{document}
