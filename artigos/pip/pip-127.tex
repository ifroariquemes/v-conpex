\documentclass[article,12pt,onesidea,4paper,english,brazil]{abntex2}

\usepackage{lmodern, indentfirst, nomencl, color, graphicx, microtype, lipsum}			
\usepackage[T1]{fontenc}		
\usepackage[utf8]{inputenc}		

\setlrmarginsandblock{2cm}{2cm}{*}
\setulmarginsandblock{2cm}{2cm}{*}
\checkandfixthelayout

\setlength{\parindent}{1.3cm}
\setlength{\parskip}{0.2cm}

\SingleSpacing

\begin{document}
	
	\selectlanguage{brazil}
	
	\frenchspacing 
	
	\begin{center}
		\LARGE PROJETO DE ROBÓTICA LEGO E MATEMÁTICA\footnote{Projeto realizado dentro da Ciência da Computação.}
		
		\normalsize
		Wanderson S. de Oliveira\footnote{Bolsista, wandersons577@gmail.com, Ji-Paraná} 
		Matheus Fernandes da Luz\footnote{Colaborador, matheusfernandesluz@gmail.com, Ji-Paraná} 
		Felipe R. de Andrade\footnote{Colaborador, feliperufini01@gmail.com, Ji-Paraná} 
		João G. dos Santos\footnote{Colaborador, jgmoreir2@gmail.com, Ji-Paraná} 
		Jackson H. da Silva Bezerra\footnote{Orientador, jackson.ifro@gmail.com, Ji-Paraná}
	\end{center}
	
	\noindent O projeto Lego e Matemática buscou facilitar o aprendizado de matemática dos
	alunos de alunos do primeiro ano do curso de informática no campus Ji-Paraná,
	integrando a matemática com os conceitos de robótica, utilizando a linha de robôs
	LEGO Mindstorms. Primeiramente, os estudantes foram introduzidos aos principais
	conceitos de lógica de programação. Nessa etapa foi utilizado a linguagem Logo,
	que é uma antiga linguagem para programação de robôs simples, muito utilizado nos
	meios educacionais. Após a introdução dos alunos à linguagem Logo, buscamos
	integrar esta com o conteúdo de matemática que os estudantes estavam
	aprendendo em suas aulas de matemática, que nessa época era teoria dos
	conjuntos. Depois de todas essas etapas, os estudantes já possuíam noção clara da
	estrutura de um programa, além de conseguir criar e resolver os mais diversos
	problemas envolvendo lógica. Então, introduzimos aos mesmos o LEGO Mindstorms
	EV3, primeiramente ensinando-os a montar os robôs e depois ensinamos como
	programá-los. Com conhecimento básico em mente, com todos os materiais a
	disposição, além de uma equipe dedicada, foi possível obter êxito nos objetivos do
	projeto. O aprendizado do conteúdo de matemática dos alunos que participaram de
	projeto se mostrou muito satisfatório, mas além disso, os mesmos também tiveram
	bom desempenho nas matérias técnicas que envolvem lógica. Além de mostrar aos
	participantes novas oportunidades na área de informática.
	
	\vspace{\onelineskip}
	
	\noindent
	\textbf{Palavras-chave}:Robótica. Matemática. LEGO. Lógica.
	
\end{document}
