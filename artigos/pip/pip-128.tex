\documentclass[article,12pt,onesidea,4paper,english,brazil]{abntex2}

\usepackage{lmodern, indentfirst, nomencl, color, graphicx, microtype, lipsum}			
\usepackage[T1]{fontenc}		
\usepackage[utf8]{inputenc}		

\setlrmarginsandblock{2cm}{2cm}{*}
\setulmarginsandblock{2cm}{2cm}{*}
\checkandfixthelayout

\setlength{\parindent}{1.3cm}
\setlength{\parskip}{0.2cm}

\SingleSpacing

\begin{document}
	
	\selectlanguage{brazil}
	
	\frenchspacing 
	
	\begin{center}
		\LARGE PRONTIDÃO FÍSICA DOS MORADORES DE PURUZINHO\footnote{Trabalho realizado na área de Educação Física com financiamento da Fundação Rondônia de Amparo ao
			Desenvolvimento das Ações Científicas e Tecnológicas e à Pesquisa do Estado de Rondônia (FAPERO) e pelo
			Conselho Nacional de Desenvolvimento Científico e Tecnológico (CNPq)}
		
		\normalsize
		Mel Naomí da Silva Borges\footnote{Voluntária, melborges85@gmail.com, Campus Porto Velho Calama} 
		Amanda Carolina Candido Silva\footnote{Voluntária, amandaccandidosilva@gmail.com, Campus Porto Velho Calama} 
		Maria Enísia Soares de Souza\footnote{Colaborador, enisiasoares@gmail.com, Campus Porto Velho Calama} 
		Matheus Magalhães Paulino Cruz\footnote{Colaborador, matheus.cruz@ifro.edu.br, Campus Porto Velho Calama}
		Olakson Pinto Pedrosa\footnote{Colaborador, olakson.pedrosa@ifro.edu.br, Campus Porto Velho Calama}
		Tiago Lins de Lima\footnote{Colaborador, tiago.lins@ifro.edu.br, Campus Porto Velho Calama}
		Xênia de Castro Barbos\footnote{Colaborador, xenia.castro@ifro.edu.br, Campus Porto Velho Calama}
        Iranira Geminiano de Melo\footnote{Orientadora, iranira.melo@ifro.edu.br, Campus Porto Velho Calama}		
		 
	\end{center}
	
	\noindent A comunidade do Puruzinho, localizada em Humaitá – AM apresenta uma falta de
	estudo relacionada ao nível de prontidão física de seus moradores e um crescimento
	gradativo mundial de enfermidades relacionadas aos hábitos individuais, requerendo
	atitudes para o combate às doenças e à promoção de hábitos de vida saudáveis.
	Entretanto, os profissionais em atividades físicas (AF) devem definir procedimentos
	de triagem pré-participação, visando reconhecer as condições físicas do indivíduo e
	prevenindo riscos à sua saúde. Assim, o objetivo foi avaliar o nível de prontidão
	física dos moradores locais de acordo com cada faixa etária. No universo
	pesquisado – que apresenta uma série de dificuldades relacionadas à falta de
	políticas públicas efetivas – foi adotado como critério de inclusão a idade, isto é, ter
	a partir de 12 anos, sendo excluídos esses e os que se recusaram a participar da
	pesquisa. A pesquisa teve o total de 31 homens e 18 mulheres de 13 a 73 anos.
	Antes de proceder a coleta de dados foi solicitado que o colaborador assinasse o
	Termo de Consentimento Livre e Esclarecido (TCLE), maiores de 18 anos, e o
	Termo de Assentimento Livre Esclarecido (TALE) aqueles com idade entre 12 a 18
	anos. Após isso, o Q-PAF foi desenvolvido, os dados fossem tabulados no Software
	Microsoft Excel 2010 e analisados a partir de estatística descritiva com o uso do
	Suplemento do Excel Xlstat 2014. O resultado mostra que 83,30\% dos jovens que
	tem entre 13 e 19 anos possuem prontidão física, sendo que metade desses são
	mulheres e a outra metade homens. Dos adultos entre 20 a 59 anos, 30.50\% estão
	aptos a praticar AF, todavia, 63.60\% do total dessa faixa etária são do sexo
	masculino. E, somente um idoso maior de 60 anos foi entrevistado, este não possui
	prontidão física, mostrando que, de acordo com o aumento gradativo da idade dos
	moradores, o nível de prontidão dos mesmos vai decaindo concomitantemente e os
	homens são os mais ativos, principalmente pelo trabalho físico que realizam.
	
	\vspace{\onelineskip}
	
	\noindent
	\textbf{Palavras-chave}:Saúde. Q-PAF. Atividade física.
	
\end{document}
