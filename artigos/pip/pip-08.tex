\documentclass[article,12pt,onesidea,4paper,english,brazil]{abntex2}

\usepackage{lmodern, indentfirst, nomencl, color, graphicx, microtype, lipsum, textcomp}			
\usepackage[T1]{fontenc}		
\usepackage[utf8]{inputenc}		

\setlrmarginsandblock{2cm}{2cm}{*}
\setulmarginsandblock{2cm}{2cm}{*}
\checkandfixthelayout

\setlength{\parindent}{1.3cm}
\setlength{\parskip}{0.2cm}

\SingleSpacing

\begin{document}
	
	\selectlanguage{brazil}
	
	\frenchspacing 
	
	\begin{center}
		\LARGE ANÁLISE DO DESENVOLVIMENTO DA ALTURA DOS INDIVÍDUOS DE ILHAS DE DIVERSIDADE NA RECUPERAÇÃO FLORESTAL NO ENTORNO DE UMA NASCENTE DEGRADADA NO MUNICÍPIO DE JI-PARANÁ/RONDÔNIA\footnote{Trabalho realizado dentro da área de Conhecimento CNPq: Ciências Agrárias com financiamento do IFRO}
		
		\normalsize
	Thaís Xavier da Silva\footnote{Bolsista, thaisxsilva99@outlook.com, Campus Ji-paraná} 
		Raimundo Gomes da Silva Junior\footnote{Orientador, raimundo.junior@ifro.edu.br, Campus Ji-paraná} 
	 
	\end{center}
	
	\noindent A restauração florestal por meio de plantio convencional em uma área total necessita de um alto investimento e acaba gerando baixa diversidade biológica a longo período. Sendo assim, a implantação de mudas dispostas em ilhas de diversidade tornou-se uma forma de atrair maior diversidade biológica para as áreas degradadas. O presente trabalho objetiva avaliar o desenvolvimento em altura dos indivíduos que compõem ilhas de diversidade na restauração de uma nascente degradada no município de Ji-Paraná/Rondônia. Para realização do experimento, a nascente degradada foi cercada em um raio de 400 m², onde instalou-se 14 ilhas de diversidade contendo, cada uma, 5 espécies (4 pioneiras e 1 não pioneira) distantes 1,0 m x 1,0m uma da outra, sendo assim, foram plantadas 70 mudas, totalizando 56 espécies pioneiras e 14 não pioneiras. A ilha que apresentou melhor média de desenvolvimento em altura ao término do experimento foi a ilha número 9 (47,97\%), e a ilha em que houve o menor desenvolvimento em altura dos indivíduos foi a de número 12 (21,3\%). A média de indivíduos sobreviventes em cada núcleo de diversidade foi de 2,64 indivíduos. A utilização de ilhas de diversidade na recuperação da área degradada evidenciou esse processo como um acelerador do processo sucessional, restabelecendo a integração entre o ecossistema.
	
	\vspace{\onelineskip}
	
	\noindent
	\textbf{Palavras-chave}: Núcleos de diversidade. Recuperação Florestal. Incremento em altura.
\end{document}
