\documentclass[article,12pt,onesidea,4paper,english,brazil]{abntex2}

\usepackage{lmodern, indentfirst, nomencl, color, graphicx, microtype, lipsum, textcomp}			
\usepackage[T1]{fontenc}		
\usepackage[utf8]{inputenc}		

\setlrmarginsandblock{2cm}{2cm}{*}
\setulmarginsandblock{2cm}{2cm}{*}
\checkandfixthelayout

\setlength{\parindent}{1.3cm}
\setlength{\parskip}{0.2cm}

\SingleSpacing

\begin{document}
	
	\selectlanguage{brazil}
	
	\frenchspacing 
	
	\begin{center}
		\LARGE ESTUDO DA DOSAGEM DO CONCRETO COM MATERIAIS DISPONÍVEIS NA CIDADE DE VILHENA\footnote{Trabalho realizado dentro da área de Materiais e Componentes de Construção (CNPq).}
		
		\normalsize
		Hadriely Moreira Diniz,\footnote{Colaboradora, hadrielydiniz@gmail.com, campus Vilhena.} 
		Isabella Cristina Sabino da Silva,\footnote{Colaboradora, isaah.silvaah310@gmail.com, campus Vilhena.} 
		Sulivan Silva e Silva,\footnote{Servidor pesquisador, sulivan.silva@ifro.edu.br l, campus Vilhena.} 
		Junior Batista Duarte\footnote{Orientador(a), junior.duarte@ifro.edu.br, campus Vilhena.} 
	\end{center}
	
	\noindent
	O estudo de dosagem tem a utilidade de encontrar as proporções ideais para o concreto. Esse material tem várias aplicações na indústria da construção civil, pois é muito versátil variando desde o concreto convencional, com propriedades úteis para a maioria das aplicações, até os concretos especiais, como concreto de alto desempenho. Sobre esse tema só existe um trabalho desenvolvido no Cone Sul de Rondônia, e pertence ao mesmo autor, porém foi feito com algumas limitações que foram dirimidas nessa pesquisa. Portanto esse trabalho é a continuação da pesquisa “Estudo da dosagem do concreto com materiais provenientes do estado de Rondônia”. O objetivo desse trabalho foi investigar qual é o devido proporcionamento do concreto com agregados disponíveis na cidade de Vilhena. O método usado foi o do IBRACON em que a partir da escolha do abatimento e da relação água-cimento inicial e dimensionado três traços, foi feito os ensaios de massa específica, abatimento de tronco de cone e resistência a compressão, todos conforme as especificações normativas brasileiras. Em relação a primeira pesquisa houveram diferenças significativas, foram desenvolvidos aparatos de ensaio para a medida de resistência e massa específica, foi consultadas normas que não haviam sido consultadas e foi modelado a equação e traçado a curva de Lyse. O teor de argamassa alcançado foi de 55\%. A massa específica de todos os traços ficaram levemente a baixo dos 2400kg/m³, demonstrando que o agregado da região proporciona um concreto mais leve. Os valores de abatimento tiveram variação de 10mm, dentro da tolerância consultada na bibliografia, e entre 60 e 80mm, valor referente a maioria dos concretos estruturais. Apesar de ter sido feito o ensaio de resistência a tração por compressão diametral, usando as correlações de norma, os valores alcançaram praticamente o mesmo que os apresentados em bibliografia. A curva da lei de Lyse obteve correlação excelente, 99,2\%. A equação de Prizskulnik e Kirilos obteve uma correlação de 99,6\%. Ou seja, todas as curvas foram modeladas e corroboradas conforme bibliografia consultada.
	
	\vspace{\onelineskip}
	
	\noindent
	\textbf{Palavras-chave}: Dosagem. IBRACON. Lei.
	
\end{document}
