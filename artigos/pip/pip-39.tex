\documentclass[article,12pt,onesidea,4paper,english,brazil]{abntex2}

\usepackage{lmodern, indentfirst, nomencl, color, graphicx, microtype, lipsum}			
\usepackage[T1]{fontenc}		
\usepackage[utf8]{inputenc}		

\setlrmarginsandblock{2cm}{2cm}{*}
\setulmarginsandblock{2cm}{2cm}{*}
\checkandfixthelayout

\setlength{\parindent}{1.3cm}
\setlength{\parskip}{0.2cm}

\SingleSpacing

\begin{document}
	
	\selectlanguage{brazil}
	
	\frenchspacing 
	
	\begin{center}
		\LARGE DESENVOLVIMENTO DE METODOLOGIA DE INVESTIGAÇÃO TOXICOLÓGICA AGUDA DE PRODUTOS VETERINÁRIOS A BASE DE DELTAMETRINA E CIPERMETRINA USANDO \textit{DAPHNIA SIMILIS} COMO BIOINDICADOR\footnote{Trabalho realizado dentro da Química com financiamento do Instituto Federal de Rondônia – IFRO.}
		
		\normalsize
	Pamella Albino Bentes\footnote{Bolsista (IT-ES), pamella.a.bentes@gmail.com, Campus Ji-Paraná.} 
		Pedro Gabriel Silva e Silva\footnote{Bolsista (IT-EM), pedrobieljip@gmail.com, Campus Ji-Paraná.} 
	Miqueias Ferreira da Silva\footnote{Colaborador, mikkeias007@gmail.com, Campus Ji-Paraná.} 
	Alecsandra Oliveira de Souza\footnote{Orientadora, alecsandra.souza@ifro.edu.br, Campus Ji-Paraná.} 
	\end{center}
	
	\noindent Dentre os vários produtos utilizados no controle de pragas que acometem a lavoura, rebanhos e os animais domésticos, encontram-se os que são formulados à base de cipermetrina e deltametrina: dois piretróides sintéticos que estão entre os principais princípios ativos utilizados em ambiente domésticos, no combate a pragas, como os carrapatos. Embora sua toxicidade, em mamíferos, seja menor que a dos compostos organofosforados e organoclorados, sua lipofilicidade é preocupante devido à alta toxicidade para peixes e a possibilidade de bioacumulação na cadeia alimentar. Nesse contexto, buscou-se investigar o potencial toxicológico agudo das dosagens indicadas para utilização doméstica dos produtos veterinários formulados a base de cipermetrina e deltametrina utilizando o microcrustáceo de água doce, \textit{Daphnia similis} como bioindicador de toxicidade. Os ensaios toxicológicos foram realizados no laboratório do IFRO – campus Ji-Paraná, de acordo com adaptações feitas à norma NBR 12713, sendo analisado o efeito da exposição de sete concentrações de cada produto (18 mmol/L à 1,8x10$^{-6}$ mmol/L – cipermetrina) e (2,5 mmol/L a 2,5x10$^{-6}$ mmol/L – deltametrina). A exposição foi realizada por período de 48 horas sendo quantificados organismos imobilizados/mortos. Ambos os compostos apresentaram alta taxa de mortalidade dos organismos expostos, sendo que todas as concentrações apresentaram efeito significativo (p < 0,05), o que indica a toxicidade aguda desses compostos, para o organismo-teste \textit{D. similis}. Com a realização desses ensaios evidenciou-se que os piretróides cipermetrina e deltametrina apresentam efeito toxicológico agudo à espécie \textit{Daphinia similis}. Visto que este microcrustáceo faz parte da cadeia alimentar de varias espécies de peixes, e que, devido ao seu caráter lipofílico, uma biomagnificação pode ocorrer é sugerido que a utilização de produtos formulados a partir destes compostos, pode trazer risco de contaminação à espécies não alvo que sejam expostas a tais formulações.
	
	\vspace{\onelineskip}
	
	\noindent
	\textbf{Palavras-chave}: Piretróides. Toxicidade aguda. \textit{Daphnia similis}. \\
	\textbf{Fonte de financiamento}: Instituto Federal de Educação, Ciência e Tecnologia de Rondônia – IFRO. 
	
\end{document}
