\documentclass[article,12pt,onesidea,4paper,english,brazil]{abntex2}

\usepackage{lmodern, indentfirst, nomencl, color, graphicx, microtype, lipsum}			
\usepackage[T1]{fontenc}		
\usepackage[utf8]{inputenc}		

\setlrmarginsandblock{2cm}{2cm}{*}
\setulmarginsandblock{2cm}{2cm}{*}
\checkandfixthelayout

\setlength{\parindent}{1.3cm}
\setlength{\parskip}{0.2cm}

\SingleSpacing

\begin{document}
	
	\selectlanguage{brazil}
	
	\frenchspacing 
	
	\begin{center}
		\LARGE CLIMA ORGANIZACIONAL: ESTUDO DE CASO DE UMA INSTITUIÇÃO DE ENSINO EM PORTUGAL\footnote{Trabalho realizado dentro da área de Conhecimento CNPq: Administração Pública e de Empresas, Ciências Contábeis e Turismo.}
		
		\normalsize
		Kelly Cristiane Catafesta\footnote{Aluna do curso de graduação de Tecnologia em Gestão Pública, kelly.catafesta@gmail.com, Campus Porto Velho Zona Norte.} 
		Higor Cordeiro de Souza\footnote{Professor (a) orientador (a), higor.souza@ifro.edu.br, vivianameirinhos@iscap.ipp.pt Campus Porto Velho Zona Norte.} 
		Viviana Meirinhos
		
	\end{center}
	
	\noindent O clima organizacional é um fenômeno perceptual duradouro, construído com base na experiência, multidimensional e compartilhado pelos membros de uma unidade de organização, cuja função principal é orientar e regular os comportamentos individuais de acordo com os padrões determinados por ela. As pesquisas de clima organizacional são amplamente utilizadas na área de Gestão de Pessoas e destacamos os trabalhos de CHIAVENATO (2005), KOYS e DECOTIIS (1991), PUENTE-PALACIOS \& FREITAS (2006), MENDES (2013), SILVA (2003), PASCHOAL (2008) e COUTO (2012). Os elevados números de escalas produzidas nas pesquisas de clima mostra falta de uniformidade e não tratam de um conjunto igualitário de elementos que constituem o construto, dificultando a unidade e sintetização do objeto. O presente estudo objetiva apresentar um questionário de avaliação de clima organizacional composto pelas dez dimensões mais comumente usadas na literatura: autonomia, confiança, pressão, valorização, justiça, inovação, relações interpessoais, condições de trabalho, bem-estar e satisfação; demonstrar os resultados da aplicação em uma organização e realizar a avaliação dos resultados alcançados. O questionário foi produzido observando o objetivo proposto para cada dimensão e aplicado a 34 profissionais que atuam na área administrativa e pedagógica de uma instituição de ensino portuguesa. Os dados foram agrupados em planilhas para verificar a constância das respostas apresentadas e analisado qualitativamente cada dimensão pesquisada. Os resultados alcançados com a pesquisa bibliográfica evidenciam que apesar da convergência nas delimitações do objeto, não há consenso e padrão quanto aos parâmetros de mensuração do construto o que torna o assunto um desafio no meio acadêmico. Os resultados obtidos com a aplicação do questionário na instituição de ensino portuguesa demonstram que é uma ferramenta eficiente para diagnósticos de clima organizacional sendo que os resultados da instituição apontam um clima favorável na mesma, evidenciando seus pontos fortes e os pontos que necessitam de atenção por parte da gestão, sendo possível visualizar intervenções específicas que poderão ser realizadas com vistas à sua melhoria. Percebeu-se que há possibilidade de aprofundamento sobre o objeto de estudo, pois as pessoas nas organizações estão em constante mudança e na mesma velocidade alteram-se os ambientes de trabalho sendo necessário estar atento a novos paradigmas.
	
	\vspace{\onelineskip}
	
	\noindent
	\textbf{Palavras-chave}: Clima Organizacional. Diagnóstico. Multidimensional
	
\end{document}
