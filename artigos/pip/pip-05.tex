\documentclass[article,12pt,onesidea,4paper,english,brazil]{abntex2}

\usepackage{lmodern, indentfirst, nomencl, color, graphicx, microtype, lipsum}			
\usepackage[T1]{fontenc}		
\usepackage[utf8]{inputenc}		

\setlrmarginsandblock{2cm}{2cm}{*}
\setulmarginsandblock{2cm}{2cm}{*}
\checkandfixthelayout

\setlength{\parindent}{1.3cm}
\setlength{\parskip}{0.2cm}

\SingleSpacing

\begin{document}
	
	\selectlanguage{brazil}
	
	\frenchspacing 
	
	\begin{center}
		\LARGE AÇAÍ SUÍÇO
		
		\normalsize
		Linconl Maia Pinheiro\footnote{Bolsista (Acadêmico do curso técnico em Alimentos) linconl\_maia14@hotmail.com, Instituto Federal de Educação, Ciência e Tecnologia de Rondônia - IFRO campus Ariquemes} 
		Eduardo Silva Taveira\footnote{Bolsista (Acadêmico do curso técnico em Alimentos) eduardo.silva15217@gmail.com, Instituto Federal de Educação, Ciência e Tecnologia de Rondônia - IFRO campus Ariquemes} 
		Rafael Conti Alves\footnote{Bolsista (Acadêmico do curso técnico em Alimentos) hirorafa@gmail.com, Instituto Federal de Educação, Ciência e Tecnologia de Rondônia - IFRO campus Ariquemes} 
		Jacob Costa Silva\footnote{Bolsista (Acadêmico do curso técnico de Manutenção e Suporte em Informática) jacobcosta.costasilva@gmail.com, Instituto Federal de Educação, Ciência e Tecnologia de Rondônia - IFRO campus Ariquemes}
		Adriano Marcos Dantas\footnote{Orientador(a)adriano.dantas@ifro.edu.br, Instituto Federal de Educação, Ciência e Tecnologia de Rondônia - IFRO campus Ariquemes} 
	\end{center}
	
	\noindent O projeto tem como principal objetivo ser inovador, a principal finalidade do mesmo é apresentar novos sabores ao mercado, expandir um produto que é muito comum na região amazônica, e que tem grande potencial de tornar-se ainda mais comum em outras regiões do brasil, ou até mesmo do mundo. Trata-se de um produto feito à base de leite bovino e açaí, muito comum na região amazônica. O principal objetivo é mostrar ao cliente a versatilidade que o açaí pode ter, sendo imensa a gama de produtos que podem ser feitos à base do mesmo, além de fornecer ao cliente um produto incomum e que surpreenda e agrade-o. Para chegar a um resultado final, foram várias as experiências praticadas. A primeira das tentativas ocorreu com todos os integrantes do projeto. O projeto foi primeiramente feito em uma cozinha, onde alguns instrumentos da mesma foram utilizados, como processadores, talheres, copos. Os ingredientes eram adicionados conforme os participantes decidiam. Duas tentativas foram feitas nesse dia e local, sendo a segunda tentativa sendo mais satisfatória entre as feitas. Outros experimentos foram feitos em locais e dias diferentes, sempre utilizando uma cozinha como espaço, porém, utilizando equipamentos diferentes - batedeiras, liquidificadores - e ingredientes diferentes, fazendo com que resultados ainda mais satisfatórios fossem obtidos. Os resultados obtidos foram satisfatórios. O açaí, por ser a matéria-prima desse projeto, proporcionou um sabor diferente, que é facilmente agradável. Espera-se que o produto agrade o público, pois proporcionou a sensação de novidade. Tornar o açaí um alimento popular nas demais regiões do Brasil é o que se espera desse produto.
	
	\vspace{\onelineskip}
	
	\noindent
	\textbf{Palavras-chave}: Produto. Açaí. Alimento.
	
\end{document}
