\documentclass[article,12pt,onesidea,4paper,english,brazil]{abntex2}

\usepackage{lmodern, indentfirst, nomencl, color, graphicx, microtype, lipsum}			
\usepackage[T1]{fontenc}		
\usepackage[utf8]{inputenc}		

\setlrmarginsandblock{2cm}{2cm}{*}
\setulmarginsandblock{2cm}{2cm}{*}
\checkandfixthelayout

\setlength{\parindent}{1.3cm}
\setlength{\parskip}{0.2cm}

\SingleSpacing

\begin{document}
	
	\selectlanguage{brazil}
	
	\frenchspacing 
	
	\begin{center}
		\LARGE \textbf{DESENVOLVIMENTO INICIAL DE PORTA-ENXERTOS CÍTRICOS EM RONDÔNIA}\footnote{Trabalho realizado dentro da Fitotecnia com financiamento do Programa Institucional de Pesquisa — PIP (PROPESP).}
		
		\normalsize
	Jhenifer Fernanda Nascimento,\footnote{Bolsista do Programa Institucional de Pesquisa (PIP-PROPESP), jhenidiol2013@gmail.com, Campus Cacoal.} 
		Edvania Neres Lino\footnote{Colaborador(a), edvania.neres12@gmail.com, Campus Cacoal.} 
		Kamylla Saldanha Pittelkow\footnote{Colaborador(a), kamyllapittelkow@gmail.com, Campus Cacoal.} 
		Karolayne Natali de Oliveira\footnote{Colaborador(a), karolayne.natali@gmail.com, Campus Cacoal.}
		Dierlei dos Santos \footnote{Orientador(a),dierlei.santros@gmail.com, Campus Cacoal.}
		Arnaldo Libório dos Santos Filho \footnote{Co-orientador(a), arnaldo.filho@ifro.edu.br, Campus Cacoal.} 
	\end{center}
	
	\noindent A citricultura nacional se caracteriza como uma das atividades econômicas mais importantes da agricultura. Adicionalmente, vários problemas de ordem fitossanitária também estão associados, alguns sendo superados com utilização de porta-enxertos tolerantes ou resistentes. Objetivou-se analisar o desenvolvimento inicial de doze variedades de porta-enxertos em Cacoal/RO. Essas variedades foram doadas pela Universidade Federal de Viçosa e implantadas no IFRO Campus Cacoal, no dia 04 de maio de 2017. Sendo elas: Limão Cravo (\textit{Citrus limonia}), Tangerina Cleópatra (\textit{Citrus Reshni}), Tangerina Sunki (\textit{Citrus Sunki}), Limão Volkameriano (\textit{Citrus Vulkameriana}), Citrange Troyer (\textit{Citrus Sinensis} X \textit{Poncirus trifoliata}), Citrange Carrizo (\textit{Citrus sinensis} X \textit{Poncirus Trifoliata}), Rangpur X Swingle 1707 (\textit{Citrus paradisi} X ( \textit{Poncirus trifoliata} X \textit{Citrus Limonia})), Citrandarin 1710 (\textit{Citrus reticulada} X \textit{Poncirus trifoliata}), Citrandarin 1697 (\textit{Citrus sunki} X \textit{Poncirus trifoliata}), Citradia 1708 (\textit{Citrus aurantium} X \textit{Poncirus trifoliata}), Citrumelo swingle (\textit{Citrus paradisi} X \textit{Poncirus trifoliatas}) e Flying dragon (\textit{Poncirus trifoliata var. monstruosa}). A semeadura de 200 sementes por variedade foi realizada em canteiro no solo, a uma profundidade de aproximadamente três cm, em três fileiras de um metro espaçadas de 10 cm. A partir de 18 dias houve emergência de todas as variedades. Após 90 dias da semeadura, com auxílio de enxadão, foram arrancadas em blocos de solos, sem prejuízo às raízes, metade das plantas semeadas, e dessas, foram isoladas 25 plantas aleatoriamente. Essas plantas foram divididas em 5 amostras com 5 plantas cada para composição das repetições. Avaliou-se a altura da parte aérea, comprimento do sistema radicular, número de folhas, massa verde das raízes e partes aérea. As amostras foram acondicionadas em sacos de papel e levadas à estufa de circulação forçada de ar, para obtenção de massa seca da raiz e parte aérea. Os dados foram submetidos a Análise de variância e as médias comparadas pelo Teste de Tukey a 5\% de probabilidade. Para todas as variáveis analisadas, o Limão Volkameriano e Citrange Carrizo se destacaram, ao passo que Citradia 1708 apresentou o menor vigor inicial. Apenas com essas informações, ainda não há possibilidade de inferências sobre quais variedades melhores de adaptarão as condições edafoclimáticas de Rondônia.
	
	\vspace{\onelineskip}
	
	\noindent
	\textbf{Palavras-chave}: Citros. Porta-enxerto. Adaptação.
	
\end{document}
