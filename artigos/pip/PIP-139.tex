\documentclass[article,12pt,onesidea,4paper,english,brazil]{abntex2}

\usepackage{lmodern, indentfirst, nomencl, color, graphicx, microtype, lipsum,textcomp}			
\usepackage[T1]{fontenc}		
\usepackage[utf8]{inputenc}		

\setlrmarginsandblock{2cm}{2cm}{*}
\setulmarginsandblock{2cm}{2cm}{*}
\checkandfixthelayout

\setlength{\parindent}{1.3cm}
\setlength{\parskip}{0.2cm}

\SingleSpacing

\begin{document}
	
	\selectlanguage{brazil}
	
	\frenchspacing 
	
	\begin{center}
		\LARGE SUSTENTABILIDADE E ROBÓTICA EDUCACIONAL: A UTILIZAÇÃO DE
		SUCATAS ELETRÔNICAS NA CONSTRUÇÃO DE ROBÔS PROGRAMÁVEIS\footnote{Trabalho realizado dentro da (área de Conhecimento CNPq: Ciência da Computação).}
		
		\normalsize
	Victor H. M. Mota\footnote{Bolsista (modalidade), ark.hugo@gmail.com, Campus Ji-Paraná.} 
	Emanuel M. de Almeida\footnote{Colaborador(a), emanuel.logado@gmail.com, Campus Ji-Paraná.} 
	Bruno A. N. de Oliveira\footnote{Colaborador(a), neires.bruno@gmail.com, Campus Ji-Paraná.} 
	Matheus H. B. de Lima\footnote{Colaborador(a), matheushbl999@gmail.com, Campus Ji-Paraná.} 
	Jackson H. S. Bezerra\footnote{jackson.henrique@ifro.edu.br, Campus Ji-Paraná.}
	João E. Teixeira Junior\footnote{Co-orientador(a), joao.teixeira@ifro.edu.br, Campus Ji-Paraná.}
	\end{center}
	
	\noindent O seguinte projeto de pesquisa reflete a preocupação com a
	sustentabilidade, partindo da utilização de sucatas eletrônicas para a produção de
	robôs, pois muitos materiais são jogados fora mesmo ainda podendo ser
	reutilizados, principalmente motores de impressoras, brinquedos ou outros objetos,
	pensando nisso e aproveitando os conhecimentos de programação que o curso
	Técnico em Informática oferece aos seus alunos, foi desenvolvido o projeto de
	criação de robôs programáveis por meio da reutilização de materiais como placas
	mãe, madeira, latas, entre outros para a sua estruturação, junto a placas Arduíno
	para realizarem a parte lógica do robô. Após dias estudando sobre a linguagem de
	programação da placa que é basicamente uma junção de C e C++, junto a conceitos
	de eletrônica e aos componentes do Arduíno, também realizamos compras e
	desmontagem de aparelhos eletrônicos descartados para assim obtermos diversos
	componentes de suma importância para a criação do robô e realização de testes do
	mesmo. A partir disso junto ao apoio e orientação dos professores foi possível
	darmos inicio na construção estrutural e lógica do robô, principalmente na estrutura,
	visto que foi a parte com mais adversidades. Assim foi construído um robô do tipo
	sumô para competições de robótica. É necessário admitir que a flexibilidade de
	materiais do Arduíno contribui imensamente para a construção de robôs com sucata,
	o que é algo excelente, devido ao fato que diminui muito o custo do desenvolvimento
	e torna possível a todos. Visto a simplicidade da criação de robôs com Arduíno,
	aplicamos um minicurso compartilhando nossos conhecimentos adquiridos com os
	alunos do 1°ano para eles sejam capazes de dar continuidade ao projeto.
	
	\vspace{\onelineskip}
	
	\noindent
	\textbf{Palavras-chave}: Robótica. Sucata. Arduíno.
	
\end{document}
