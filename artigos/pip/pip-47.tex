\documentclass[article,12pt,onesidea,4paper,english,brazil]{abntex2}

\usepackage{lmodern, indentfirst, nomencl, color, graphicx, microtype, lipsum}			
\usepackage[T1]{fontenc}		
\usepackage[utf8]{inputenc}		

\setlrmarginsandblock{2cm}{2cm}{*}
\setulmarginsandblock{2cm}{2cm}{*}
\checkandfixthelayout

\setlength{\parindent}{1.3cm}
\setlength{\parskip}{0.2cm}

\SingleSpacing

\begin{document}
	
	\selectlanguage{brazil}
	
	\frenchspacing 
	
	\begin{center}
		\LARGE EDUCAÇÃO AMBIENTAL: Ictiologia como ferramenta a conscientização ambiental, no IFRO Campus Cacoal\footnote{Trabalho realizado dentro da área de Ciências Biológicas, com financiamento do EDITAL Nº 26/2017/CAC - CGAB/IFRO, DE 28 DE JULHO DE 2017.}
		
		\normalsize
		Paulo Vitor Moreira Miranda de Almeida,
		Mahmoud Mehanna\footnote{Colaborador, email: mahmoud.mehanna@ifro.edu.br.} 
	
	\end{center}
	
	\noindent “Só se pode conservar o que se conhece”, com essa máxima pode-se compreender que o isolamento de fragmentos florestais e a destruição dos ambientes naturais têm contribuído para o desaparecimento de inúmeras espécies da fauna silvestre brasileira. No ensino de Ciências a metodologia didática pedagógica que complementa as aulas teóricas, com aulas práticas no laboratório de Ciências Biológicas, a aprendizagem dos alunos é solidificada. Com o intuito de estruturar o conhecimento sobre a fauna e educação ambiental, é realizado o levantamento da fauna do Instituto Federal de Rondônia, Campus Cacoal. No período de agosto e setembro de 2017, foi realizado uma amostragem em virtude de o nível de água estar mais baixo, facilitando assim a captura dos peixes. Nesta primeira etapa, foram amostrados a composição parcial da ictiofauna presente no Campus. Os peixes amostrados foram anestesiados e preservados em formol a 10\%, após o período de 48 horas foram transferidos e conservados em álcool a 70\%. Foram identificados e verificado a sua menor categoria taxonômica. O resultado obtido foi de 4 ordens, 7 famílias, 8 gêneros e 10 espécies. Foram capturados 192 indivíduos, sendo 129 individuos de \textit{Astyanax bimaculatus}, 7 de \textit{Serrapinnus} sp., 7 de \textit{Characidium cf. zebra}, 24 de \textit{Hoplias malabaricus}, 10 de \textit{Hoplosternum littorale}, 2 \textit{Megalechis cf. picta}, 6 \textit{Crenicichla lepidota}, 6 de \textit{Gymnotus carapo} e um exemplar de \textit{Rhamdia quelen} e um de \textit{Cichalasoma cf boliviense}. Com esse resultado preliminar, pode se assim desenvolver uma melhor visão ao ensino do Instituto Federal de Rondônia e a magnitude da consciência ambiental no ensino médio.
	
	\vspace{\onelineskip}
	
	\noindent
	\textbf{Palavras-chave}: Preservação. Didática. Ensino.
	
\end{document}
