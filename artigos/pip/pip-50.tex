\documentclass[article,12pt,onesidea,4paper,english,brazil]{abntex2}

\usepackage{lmodern, indentfirst, nomencl, color, graphicx, microtype, lipsum}			
\usepackage[T1]{fontenc}		
\usepackage[utf8]{inputenc}		

\setlrmarginsandblock{3cm}{3cm}{*}
\setulmarginsandblock{3cm}{3cm}{*}
\checkandfixthelayout

\setlength{\parindent}{1.3cm}
\setlength{\parskip}{0.2cm}

\SingleSpacing

\begin{document}
	
	\selectlanguage{brazil}
	
	\frenchspacing 
	
	\begin{center}
		\LARGE EFEITO DE DIFERENTES SUBSTRATOS NO ENRAIZAMENTO DE CLONES DE TOMATEIRO\footnote{Trabalho realizado dentro da (área de Conhecimento CNPq: Fitotecnia) com financiamento do (Instituto Federal de Educação, Ciência e Tecnologia de Rondônia – campus Colorado do Oeste).}
		
		\normalsize
		Romário Batista de Lima Oliveira\footnote{romariolima1935@gmail.com, Campus Colorado do Oeste.} 
	Juliana Pereira Mendes\footnote{juliiana.mendes25@gmail.com, Campus Colorado do Oeste.} 
	Aldo Max Custódio\footnote{aldo.custodio@ifro.edu.br, Campus Colorado do Oeste.} 
	
	\end{center}
	
	\noindent A propagação vegetativa do tomateiro é uma das alternativas para se reduzir os custos de produção de mudas, e a estaquia é a técnica de maior viabilidade econômica para o estabelecimento de plantios clonais, por permitir a multiplicação de genótipos selecionados em curto período de tempo. O uso de mudas de boa qualidade promove ganhos de rendimento e, sobretudo, melhora a qualidade dos frutos colhidos. O substrato é um dos principais fatores que afetam a qualidade das mudas produzidas. A formação das mudas é apontada como uma das fases mais importantes para o ciclo da cultura, influenciando diretamente no crescimento e desempenho produtivo da planta. O trabalho foi realizado com objetivo avaliar a produção de mudas por estaquia de três genótipos de tomateiro em quatro substratos diferentes. Os tratamentos constaram de três genótipos de tomateiro (Yoshimatsu, Perseo e Vênus) e quatro substratos diferentes (húmus de minhoca, composto orgânico, palha de arroz carbonizada + solo de mata e comercial Carolina®), arranjados como fatorial 3x4 no delineamento experimental inteiramente casualizado, com 4 repetições. As estacas foram adquiridas de plantas “matrizes”, obtidas de sementes, e que foram plantadas no Setor de Produção Vegetal I do Instituto Federal de Educação, Ciência e Tecnologia de Rondônia – campus Colorado do Oeste, a produção e avaliação das mudas ocorreu em casa de vegetação. Após trinta dias do plantio as estacas/mudas foram retiradas do substrato e avaliadas. Foram consideradas na avaliação o percentual de estacas/mudas enraizadas, obtida por meio de contagem de todas mudas contidas nas bandejas. Não houve efeito significativo da interação substrato e variedade com relação a porcentagem de estacas enraizadas. A maior porcentagem de estacas enraizadas, foram quando se utilizaram o genótipo Yoshimatsu (100\%) de enraizamento. O genótipo Vênus apresentou (96,25\%) de enraizamento. Em relação aos genótipos Yoshimatsu e Vênus o Perseo apresentou resultado estatisticamente inferior (93,75\%) para variável percentual de mudas enraizadas. Para os substratos não houve diferença estatística (média = 95,50\%). Assim, conclui-se que independente da cultivar e do substrato as estacas apresentaram bom índice de enraizamento.
	
	\vspace{\onelineskip}
	
	\noindent
	\textbf{Palavras-chave}: Substrato. Propagação. \textit{Lycopersicum esculentum}.
	
\end{document}
