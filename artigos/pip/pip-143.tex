\documentclass[article,12pt,onesidea,4paper,english,brazil]{abntex2}

\usepackage{lmodern, indentfirst, nomencl, color, graphicx, microtype, lipsum}			
\usepackage[T1]{fontenc}		
\usepackage[utf8]{inputenc}		

\setlrmarginsandblock{2cm}{2cm}{*}
\setulmarginsandblock{2cm}{2cm}{*}
\checkandfixthelayout

\setlength{\parindent}{1.3cm}
\setlength{\parskip}{0.2cm}

\SingleSpacing

\begin{document}
	
	\selectlanguage{brazil}
	
	\frenchspacing 
	
	\begin{center}
		\LARGE UMA ABORDAGEM FARMACOLÓGICA E TERAPÊUTICA DA TURNERA
		ULMIFOLIA\footnote{Trabalho realizado dentro da (área de Conhecimento CNPq:
			Química Orgânica) com financiamento do Instituto Federal de Ciência e Tecnologia de Rondônia
			(PROPESP).}
		
		\normalsize
	Rafaela Rodrigues Pinheiro\footnote{Bolsista (Iniciação Científica no Ensino Médio (IC-EM).) rafaelarodriguespinheiro2@gmail.com,
		Campus Calama.} 
	Julia Lize Alves da Silva\footnote{Colaborador(a): julialize2014@gmail.com.br, Campus Calama.} 
	Márcia Bay\footnote{Orientador(a): marcia.bay@ifro.edu.br, Campus Calama.} 
	\end{center}
	
	\noindent Os produtos naturais são importantes fontes de informações para o desenvolvimento
	de novos fármacos, para isso, eles devem passar por etapas de descobrimento,
	isolamento e estudos do seu mecanismo de ação, antes de seguir para os estudos
	clínicos. Essa pesquisa tem como objetivo revisar a literatura da Turnera Ulmifolia,
	uma planta utilizada na medicina tradicional em diversas condições patológicas e
	orgânicas com o propósito de levantar os dados já estudados para dar
	prosseguimento na pesquisa a qual estamos desenvolvendo. A pesquisa foi
	realizada em diferentes bases de dados, periódicos e livros especializados sobre o
	tema. A Turnera Ulmifolia conhecida no Brasil como Chanana, Damiana ou Flor-do-
	Guarujá, tem amplo uso, como as espécies do gênero Turnera que têm uso
	ornamental e as raízes e folhas são empregadas na medicina popular como
	expectorante contra tosse, gripe, bronquite, inflamações, problemas de próstata e
	câncer. Algumas atividades farmacológicas foram comprovadas em espécies desse
	gênero, como anti-inflamatória, antiulcerogênica, antioxidante e antimalárica. As
	pesquisas de isolamentos nessa espécie vegetal demonstram maior presença de
	flavonoides, que são descritos como excelentes agentes antiflamatórios. Em síntese,
	esta família possui diferentes espécies com potencial terapêutico, sendo urgentes
	estudos que validem os usos medicinais descritos, o que levou aos métodos de
	isolamento de substâncias em cromatografia em coluna, cromatografia em camada
	delgada e testes com o extrato da planta que estão sendo executados, como de
	atividade de bactérias resistentes a antibióticos, em parceria com a plataforma da
	Fundação Osvaldo Cruz (Fio Cruz) de Porto Velho.
	
	\vspace{\onelineskip}
	
	\noindent
	\textbf{Palavras-chave}: Plantas Medicinais. Estudos Fitoquímicos. Turnera Ulmifolia.
	
	\noindent
	\textbf{Fonte Financiadora}: Instituto Federal de Ciência e Tecnologia de Rondônia-
	IFRO
	
\end{document}
