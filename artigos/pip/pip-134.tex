\documentclass[article,12pt,onesidea,4paper,english,brazil]{abntex2}

\usepackage{lmodern, indentfirst, nomencl, color, graphicx, microtype, lipsum}			
\usepackage[T1]{fontenc}		
\usepackage[utf8]{inputenc}		

\setlrmarginsandblock{2cm}{2cm}{*}
\setulmarginsandblock{2cm}{2cm}{*}
\checkandfixthelayout

\setlength{\parindent}{1.3cm}
\setlength{\parskip}{0.2cm}

\SingleSpacing

\begin{document}
	
	\selectlanguage{brazil}
	
	\frenchspacing 
	
	\begin{center}
		\LARGE REDES SOCIAIS: CONTRATAÇÃO ATRAVÉS DE ANÁLISE DE PERFIS EM
		REDES SOCIAIS\footnote{Trabalho realizado dentro das ciências exatas e da terra.}
		
		\normalsize
		Amilton Victor Togno Menezes \,\, Bruna de Jesus Vieira da Silva\\Emanuel Otokovieski Trevisan\footnote{Aluno(s) colaborar(es), avmb25@gmail.com, emanuel.otokovieski@gmail.com,
			brunadejesusv26@gmail.com, Campus Ji-Paraná.} 
	Ilma Rodrigues de Sousa Fausto\footnote{Orientador (a), ipfausto@gmail.com, Campus Ji-Paraná.} 
	\end{center}
	
	\noindent O seguinte projeto apresenta os diversos tipos de redes sociais, como: Facebook,
	Whatsapp, Youtube, Instagram e entre outras, sendo elas muito utilizadas na vida
	dos jovens. É visto também a cidadania digital, na qual consiste agir com um
	cidadão, respeitando direitos e deveres, individuais e coletivos, abordando normas
	de comportamentos adequados e responsáveis em relação ao uso da tecnologia.
	Outro fator é a segurança na rede, mencionando formas e dicas de segurança para
	não se tornar um alvo de ataques cibernéticos. Os objetivos detalhados visam
	cidadania digital, identificar as atitudes que prejudicam a imagem do usuário,
	apontar comportamentos inadequados e informar a seleção feita por empresas
	através do perfil das redes sociais usadas. A pesquisa efetuada pode ser
	classificada como exploratória, apresentando o método hipotético-dedutivo, que
	consiste na realização de testes das hipóteses. Como base para a realização do
	projeto, utilizou-se conteúdos de autores como: Ribble (2010), Hamann (2014), Crespo
	(2013) entre outros. A pesquisa de campo teve como foco, identificar as atitudes
	praticadas pelos educandos, sendo utilizado o instrumento de Formulário do Google
	Drive. A pesquisa contém sete perguntas, com exceção dos dados pessoais.
	Dividido em quatro etapas, o questionário apresentará na primeira, a coleta de
	dados para uma base concreta, na segunda, a disponibilização do questionário por
	link para os alunos, já na terceira, coleta dos dados e levantamento de resultados, e
	por fim na quarta, um arquivo com dicas para que os alunos tenham uma orientação
	a seguir como base, para aprimorar suas atitudes. Os resultados obtidos indicam
	que os alunos entrevistados, possuem um contato consideravelmente alto com
	redes sociais, tornando-se uma atitude comum em seu dia-a-dia, posteriormente, um
	gráfico indica que a rede que é mais utilizada por eles, é o Facebook, pois há
	maiores possibilidades de entretenimento, como fotos de amigos, contato com
	pessoas distantes entre outros benefícios, logo atrás desta rede, encontra-se,
	Youtube e Whatsapp, sendo esse último uma rede que vem tomando cada vez mais
	força entre os usuários.
	
	\vspace{\onelineskip}
	
	\noindent
	\textbf{Palavras-chave}: Redes sociais. Comportamento. IFRO.
	
\end{document}
