\documentclass[article,12pt,onesidea,4paper,english,brazil]{abntex2}

\usepackage{lmodern, indentfirst, nomencl, color, graphicx, microtype, lipsum}			
\usepackage[T1]{fontenc}		
\usepackage[utf8]{inputenc}		

\setlrmarginsandblock{2cm}{2cm}{*}
\setulmarginsandblock{2cm}{2cm}{*}
\checkandfixthelayout

\setlength{\parindent}{1.3cm}
\setlength{\parskip}{0.2cm}

\SingleSpacing

\begin{document}
	
	\selectlanguage{brazil}
	
	\frenchspacing 
	
	\begin{center}
		\Large POLÍTICAS DE INTERVENÇÃO URBANA E CULTURAL: 		
\\		\LARGE O CASO DA
		“REVITALIZAÇÃO” DO CAIS MAUÁ \footnote{Trabalho realizado dentro da área das Ciências Sociais com financiamento do Edital no 29/2016 do
			Campus Guajará-Mirim.}
		
		\normalsize
		João Victor Rech Ruiz\footnote{Bolsista (IC - EM), email: jvruiz2015@hotmail.com, Campus Guajará-Mirim} 
		Amábily Massai Ribeiro\footnote{Colaboradora, email: massairibeiro12@gmail.com, Campus Guajará-Mirim} \\
		Flávia Caroline F. Ferreira\footnote{Colaborador, email: flavia.caroline2001@gmail.com, Campus Guajará-Mirim} 
		Carla Betânia Reiher\footnote{Orientadora, email: carla.reiher@ifro.edu.br, Campus Guajará-Mirim} 
	\end{center}
	
	\noindent O presente projeto teve como interesse o estudo sobre as políticas de revitalização e
	enobrecimento urbano, tendo como objetivo geral compreender como e de que
	forma se efetiva o processo de tais intervenções urbanas. Apresenta como recorte a
	realização de um estudo sobre o projeto de “Revitalização” do Cais Mauá, área
	portuária localizada no centro da capital do Rio Grande do Sul, a cidade de Porto
	Alegre. Os estudos sobre as intervenções urbanas e culturais, ou seja, as políticas
	de enobrecimento encontram-se em crescente interesse pelas Ciências Sociais nas
	últimas décadas, adotando a concepção que há uma necessidade de analisar as
	cidades para que possamos compreender melhor o tecido social, tendo em vista que
	tais políticas implicam construções ideológicas ou representações de determinados
	segmentos da sociedade, sendo que as dinâmicas das cidades, por suas
	complexidades, podem equivaler-se a sociedade, pois nelas encontramos diferentes
	forças que mantém, reproduzem e complexificam a sociedade. O projeto teve como
	objetivo analisar se a intervenção urbana e cultural que abrange a área portuária
	Cais Mauá se trata de um processo de enobrecimento urbano. Esta pesquisa
	compreendeu, simultaneamente, um levantamento histórico-documental e
	iconográfico. O arrolamento histórico-documental utilizou de métodos convencionais
	de fichamento e catalogação de fontes primárias, secundárias e iconográficas. Na
	conjuntura das fontes primárias utilizamos a análise histórica a partir de bibliografia
	vasta existente sobre a cidade de Porto Alegre, desde seu surgimento enquanto
	povoado até os dias atuais. No contexto de fontes secundárias centrou-se na análise
	de documentos que permeiam a “Revitalização” do Cais. A proposta do projeto de
	“revitalização” é criar um espaço moderno, mesclando aspectos da cultura local
	assentados na tradição, como os bens patrimoniados, com aspectos de uma
	urbanização globalizadora. Observou-se na análise dos dados que esta política de
	intervenção urbana e cultural alocada na área portuária do Centro Histórico de Porto
	Alegre se trata de enobrecimento urbano. Desta forma o projeto de “revitalização”
	não segue o discurso de que o Cais é para todos, pois há uma segmentação social,
	onde parcelas da sociedade são excluídas do espaço em questão.
	
	\vspace{\onelineskip}
	
	\noindent
	\textbf{Palavras-chave}: Políticas de intervenção; Enobrecimento; Revitalização.
	
	\noindent
	\textbf{Fonte de financiamento}: Edital n$^{\circ}$29/IFRO/Campus Guajará-Mirim/2016.
	
\end{document}
