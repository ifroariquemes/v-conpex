\documentclass[article,12pt,onesidea,4paper,english,brazil]{abntex2}

\usepackage{lmodern, indentfirst, nomencl, color, graphicx, microtype, lipsum}			
\usepackage[T1]{fontenc}		
\usepackage[utf8]{inputenc}		

\setlrmarginsandblock{2cm}{2cm}{*}
\setulmarginsandblock{2cm}{2cm}{*}
\checkandfixthelayout

\setlength{\parindent}{1.3cm}
\setlength{\parskip}{0.2cm}

\SingleSpacing

\begin{document}
	
	\selectlanguage{brazil}
	
	\frenchspacing 
	
	\begin{center}
		\LARGE JOGOS ELETRÔNICOS: \\INFLUÊNCIAS NA VIDA E NO ÂMBITO ESCOLAR\footnote{Trabalho realizado dentro da Ciências exatas e da terra.}
		
		\normalsize
		Brenda Gonçalves Rocha\footnote{Aluno(a) Colaborador(a), brendagon.rocha@gmail.com, Campus Ji-Paraná} 
		Felipe de Oliveira Andrade\footnote{Aluno(a) Colaborador(a), feliperufini01@gmail.com, Campus Ji-Paraná} 
	    Ilma Rodrigues de Souza Fausto\footnote{Orientador(a), ipfausto@gmail.com, Campus Ji-Paraná} 
		 
	\end{center}
	
	\noindent O presente trabalho teve sua elaboração e desenvolvimento no Instituto Federal de
	Educação, Ciência e Tecnologia de Rondônia (IFRO), Campus Ji-Paraná,
	proporcionado por um projeto de iniciação científica aplicado à matéria de Orientação
	para pesquisa e prática profissional (OPPP), ministrado pela professora Ilma
	Rodrigues de Souza Fausto no decorrer do ano de 2017. Possui como objetivo
	principal identificar a influência dos jogos eletrônicos na vida dos estudantes e no
	âmbito escolar colhendo resultados a partir de um questionário desenvolvido na
	plataforma Formulários do Google Drive. Os alunos colaboradores realizaram uma
	pesquisa bibliográfica e uma aplicação de questionário com os estudantes do turno
	matutino e vespertino, abrangendo todos os cursos integrados ao ensino médio
	disponibilizados pela instituição, estes responderam ao questionário onde tornou-se
	possível a análise dos resultados. Com os dados obtidos concluiu-se que grande parte
	dos estudantes do Campus Ji-Paraná que fazem uso dos jogos eletrônicos abrangem
	a faixa etária entre 15 e 17 anos, os quais afirmam que os jogos não influenciam de
	forma negativa em seu comportamento, e que através deles pode-se obter uma maior
	socialização e interação entre os jogadores, bem como troca de conhecimentos.
	Embora os estudantes alegarem que os jogos não prejudicam o rendimento escolar
	as pesquisas bibliográficas contradizem este pensamento. Para a apresentação dos
	resultados analisados, serão ministrados palestras destinadas a todos os alunos da
	instituição, abrangendo todos os cursos com o objetivo de informar o quanto os jogos
	eletrônicos podem influenciar na vida de um estudante e os seus respectivos
	benefícios, para que todos possam estar cientes das consequências, e utilizarem os
	jogos eletrônicos de maneira consciente.
	
	\vspace{\onelineskip}
	
	\noindent
	\textbf{Palavras-chave}:Jogos Eletrônicos. Estudantes. Influências.
	
\end{document}
