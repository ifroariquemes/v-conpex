\documentclass[article,12pt,onesidea,4paper,english,brazil]{abntex2}

\usepackage{lmodern, indentfirst, nomencl, color, graphicx, microtype, lipsum}			
\usepackage[T1]{fontenc}		
\usepackage[utf8]{inputenc}		

\setlrmarginsandblock{2cm}{2cm}{*}
\setulmarginsandblock{2cm}{2cm}{*}
\checkandfixthelayout

\setlength{\parindent}{1.3cm}
\setlength{\parskip}{0.2cm}

\SingleSpacing

\begin{document}
	
	\selectlanguage{brazil}
	
	\frenchspacing 
	
	\begin{center}
		\LARGE DESENVOLVIMENTO DE UM PROTÓTIPO PARA O ENSINO DO FUNCIONAMENTO BÁSICO DE UM COMPUTADOR COM UNIDADE LÓGICO ARITMÉTICA DE 4 BITS\footnote{Trabalho realizado dentro da área de Ciências Exatas e da Terra com financiamento do DEPESP (Departamento de Pesquisa, Inovação e Pós-graduação) Campus Vilhena.}
		
		\normalsize
		KOVALSIKOSKI, Dhaniela\footnote{Bolsista (Iniciação científica – Ensino Médio), dhany\_bliz@hotmail.com, Campus Vilhena.} 
	NETO, Nelson Tremea\footnote{Bolsista (Iniciação científica – Ensino Médio), whiteaxefun@gmail.com, Campus Vilhena.} 
	ANDRADE, Marco A. A.\footnote{Orientador(a), marco.andrade@ifro.edu.br, Campus Vilhena.} 
	SANTOS, Cleyton\footnote{Co-orientador(a), cleyton.santos@ifro.edu.br, Campus Calama.} 
	\end{center}
	
	\noindent Um computador é constituído de uma série de componentes, como processador, unidades de memória, barramentos de comunicação, dispositivos de entrada e saída, dentre outros. O processador é composto por registradores, UC (Unidade de Controle) e ULA (Unidade Lógico Aritmética). A ULA é responsável pelas operações lógicas, como as definidas pelos operadores E e/ou OU, e aritméticas, como a soma e a subtração. Desta forma, esta pesquisa desenvolveu um protótipo do funcionamento básico de um computador, utilizando uma ULA de 4 bits, eletrônica digital e os conceitos portas lógicas. Foi realizada uma pesquisa com a abordagem de caráter misto, iniciando-se com o levantamento bibliográfico acerca do tema, definição dos parâmetros e desenho do circuito do projeto e em sequência foi construído um protótipo, utilizando portas lógicas e protoboard, de uma ULA de 4 bits, contendo as operações lógicas, limitadas a E, Não E, OU, não OU, OU exclusivo e não OU exclusivo, e as operações aritméticas, limitadas a soma e subtração, bem como, a seleção dessas operações. O protótipo proposto foi desenvolvido com base apenas na montagem de componentes eletrônicos em protoboards, entretanto, percebeu-se que o tempo para montagem seria demasiado ao seu propósito, bem como, a organização de cabos trabalhosa, sendo assim, pretende-se a implementação de uma versão do protótipo com PCI (Placa de Circuito Impresso). Por fim, propõe-se que este protótipo seja utilizado nas disciplinas voltadas a temática de organização de computadores, onde seja importante demonstrar, mesmo que de forma limitada, o funcionamento básico de um computador, como entrada, processamento e saída de dados.
	
	\vspace{\onelineskip}
	
	\noindent
	\textbf{Palavras-chave}: ULA. Computador. Lógica.
	
\end{document}
