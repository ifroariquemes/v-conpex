\documentclass[article,12pt,onesidea,4paper,english,brazil]{abntex2}

\usepackage{lmodern, indentfirst, nomencl, color, graphicx, microtype, lipsum}			
\usepackage[T1]{fontenc}		
\usepackage[utf8]{inputenc}		

\setlrmarginsandblock{2cm}{2cm}{*}
\setulmarginsandblock{2cm}{2cm}{*}
\checkandfixthelayout

\setlength{\parindent}{1.3cm}
\setlength{\parskip}{0.2cm}

\SingleSpacing

\begin{document}
	
	\selectlanguage{brazil}
	
	\frenchspacing 
	
	\begin{center}
		\LARGE AVALIAÇÃO DOS MÉTODOS DE QUEBRA DE DORMÊNCIA EM SEMENTES DE
		ENTEROLOBIUM TIMBOUVA\footnote{Trabalho realizado dentro da engenharia florestal com financiamento do (IFRO)}
		
		\normalsize
		Rachel Maria Machado Ferreira Franco\footnote{Rachel Maria Machado Ferreira Franco,rachelffranco@gmail.com, Ji-Paraná} 
		Maria Elessandra Rodrigues Araújo\footnote{Maria Elessandra Rodrigues Araújo, elessandra.cg@gmail.com, Ji-Paraná} 
		Andreza Pereira
		Mendonça\footnote{Andreza Pereira Mendonça, Mendonça.andreza@gmail.com, Ji-Paraná} 
		
	\end{center}
	
	\noindent A Enterolobium timbouva, conhecido como "orelhão de macaco", é uma alternativa
	econômica por sua madeira, além de ser uma das espécies florestais com potencial
	para uso em reflorestamento de áreas degradadas e plantações mistas,
	principalmente pelo seu rápido crescimento inicial Entretanto as sementes desta
	espécie apresentam dormência devido à impermeabilidade do tegumento à água,
	causando um baixo índice de germinação, dificultando a produção de mudas.
	Objetivou-se avaliar diferentes métodos de quebra de dormência . O trabalho foi
	conduzido no laboratório de análises de sementes do IFRO - Campus, Ji-Paraná.
	Utilizou-se diferentes métodos de quebra de dormência das sementes de
	Enterolobium timbouva, para avaliação fisiológica das sementes foram utilizadas
	200 sementes por tratamento, distribuídas em quatro repetições de 50 sementes. O
	delineamento experimental utilizado nas diferentes etapas foi o inteiramente
	casualizado com quatro repetições. O software utilizado na análise foi o ASSISTAT,
	e as médias, após análise de variância, comparadas pelo teste de Tukey a 5\% de
	probabilidade. A dormência física das sementes de orelhão foi superada com maior
	taxa de germinação após o desponte e imersão em água por 24 horas (96\%). Os
	tratamentos para superação da dormência: desponte, desponte + $H_2$O 24h e lixa+
	$H_2$O 24h não diferiram estatisticamente. Ao avaliar o comprimento total das
	plântulas verificou-se que os tratamentos desponte, desponte + $H_2$O 24h e ácido
	sulfúrico 5$'$ não diferiram estatisticamente. Dentre os métodos testado para quebra
	de dormência das sementes de \textit{Enterolobium contortisiliquum} o desponte mais
	imersão em água por 24 horas foi o que apresentou os melhores resultados em
	germinação e vigor.
	
	\vspace{\onelineskip}
	
	\noindent
	\textbf{Palavras-chave}: Dormência, orelhão, métodos.
	
\end{document}
