\documentclass[article,12pt,onesidea,4paper,english,brazil]{abntex2}

\usepackage{lmodern, indentfirst, nomencl, color, graphicx, microtype, lipsum}			
\usepackage[T1]{fontenc}		
\usepackage[utf8]{inputenc}		

\setlrmarginsandblock{2cm}{2cm}{*}
\setulmarginsandblock{2cm}{2cm}{*}
\checkandfixthelayout

\setlength{\parindent}{1.3cm}
\setlength{\parskip}{0.2cm}

\SingleSpacing

\begin{document}
	
	\selectlanguage{brazil}
	
	\frenchspacing 
	
	\begin{center}
		\LARGE GESTÃO DE ORGANIZAÇÕES: A RELAÇÃO ENTRE ESTILO DE LIDERANÇA E
		GÊNERO\footnote{Trabalho realizado dentro da área de Conhecimento CNPq: 60200006 ADMINISTRAÇÃO}
		
		\normalsize
		Victória Giovanna Mundim Barros\footnote{Bolsista (modalidade de Ensino Médio), vitoriamundim@hotmail.com, Campus Porto Velho Zona
			Norte} 
	   David Lucas da Silva Ferreira\footnote{
	   	Bolsista (modalidade de Ensino Superior), davidlucas1988@hotmail.com, Campus Porto Velho Zona
	   	Norte} 
		Esiomar Andrade S. Filho\footnote{Co-orientador(a), esiomar.silva@ifro.edu.br, Campus Porto Velho Zona Norte} 
		Clotilde Tânia
		Rwrsilany Silva\footnote{ Orientador(a), rwrsilany.silva@ifro.edu.br, Campus Porto Velho Zona Norte}
		Gustavo Melazi Girardi\footnote{Colaborador(a), gustavo.girardi@ifro.edu.br, Campus Guajará Mirim}  
	\end{center}
	
	\noindent Estudos realizados mostram que o estilo de liderança interfere no comprometimento
	dos colaboradores nas instituições. Além disso, a globalização e o sucessivo
	interesse das corporações em buscar novos mercados têm despertado a atenção da
	comunidade corporativa e acadêmica sobre os impactos da diversidade nas
	organizações como um todo. Isto porque a força de trabalho apresenta-se cada vez
	mais heterogênea, em termos de raça, etnia, gênero e outros grupos culturalmente
	diversos. Contudo, ainda são observadas diferenças e discriminações nos meios
	corporativos no que se refere ao gênero e estilos de liderança. Diante disso, este
	trabalho objetiva analisar a relação entre o estilo de liderança e gênero dos líderes
	das organizações privadas, como forma de disponibilizar ao meio acadêmico e
	corporativo uma referência de aprendizagem organizacional e impulsionar uma
	mudança estratégica nas empresas, tendo em vista a limitação de publicações
	bibliográficas no mercado. Para tanto, realizou-se uma pesquisa com uma
	abordagem quantitativa com uma amostra de executivos, homens e mulheres,
	utilizando o questionário MLQ de Bass, corroborado amplamente em diversos países
	e populações. Os dados foram tratados por meio de estatística descritiva inferencial
	e correlacional. Os resultados indicam que não há diferenças relevantes entre
	homens e mulheres em nenhum dos estilos e seus respectivos fatores, o que
	contraria os resultados encontrados por Bass (1996) e confirma a pesquisa de Basto
	(2013). Diante disso, recomenda-se a produção de novos estudos em outras
	amostras para validar os resultados encontrados bem como estudos que ampliem a
	compreensão dos fatores que interferem diretamente no comprometimento dos
	colaboradores a depender do gênero e/ou estilo de liderança.
	
	\vspace{\onelineskip}
	
	\noindent
	\textbf{Palavras-chave}:Gestão de Organizações. Estilo de Liderança. Gênero.
	
\end{document}
