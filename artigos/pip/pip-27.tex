\documentclass[article,12pt,onesidea,4paper,english,brazil]{abntex2}

\usepackage{lmodern, indentfirst, nomencl, color, graphicx, microtype, lipsum}			
\usepackage[T1]{fontenc}		
\usepackage[utf8]{inputenc}		

\setlrmarginsandblock{3cm}{3cm}{*}
\setulmarginsandblock{3cm}{3cm}{*}
\checkandfixthelayout

\setlength{\parindent}{1.3cm}
\setlength{\parskip}{0.2cm}

\SingleSpacing

\begin{document}
	
	\selectlanguage{brazil}
	
	\frenchspacing 
	
	\begin{center}
		\LARGE CARACTERIZAÇÃO PRELIMINAR DA QUALIDADE DAS ÁGUAS SUBTERRÂNEAS EM ÁREA URBANA NO MUNICIPIO DE CACOAL-RO\footnote{Trabalho realizado dentro das Ciências Exatas e da Terra com financiamento do DEPESP/IFRO Campus Cacoal.}
		
		\normalsize
		Paulo Fernando Costa Teles\footnote{Paulo Fernando Costa Teles (Bolsista), epffernando1115@gmail.com, Campus Cacoal.} 
	Vinicius Spanhol Ferrari\footnote{Vinicius Spanhol Ferrari (Colaborador), viniciusferrari008@gmail.com, Campus Cacoal.} 
		Helen Oliveira Costa\footnote{Helen Oliveira Costa (Colaboradora) helencostasf@gmail.com Campus Cacoal.} 
		Anthony Muniz Prado de Oliveirai\footnote{Anthony Muniz Oliveira Prado (Colaborador), anthonymuniz12@gmail.com Campus Cacoal.}
		Isael Minzon Gomes\footnote{Isael Minzon Gomes (Orientador), isael.minzon@ifro.edu.br, Campus Cacoal.} 
	\end{center}
	
	\noindent As águas subterrâneas têm tido uma grande demanda mundial. Muitas cidades e países dependem parcial ou exclusivamente desse recurso natural. O município de Cacoal capta água do rio Machado (ou Ji-Paraná) para seu abastecimento público, o que até o momento tem suprido as necessidades dos munícipes, com estimativa para o ano de 2016, de 87.877 habitantes. Porém, alguns órgãos públicos, igrejas, postos de gasolina e outras instituições tem recorrido às fontes subterrâneas hídricas para consumo de água. A qualidade das águas subterrâneas geralmente é boa, no entanto temos uma lacuna quanto ao conhecimento de seus constituintes iônicos. Diante desta situação, esse trabalho busca realizar um estudo hidroquímico destas águas a partir de campanhas de campo e de laboratório. Os parâmetros hidroquímicos determinados são: cálcio, potássio, sódio, ferro, cloreto, nitrato, sulfato, alcalinidade total, pH, condutividade, temperatura, e sólidos totais dissolvidos, de acordo com metodologias do Standard Methods for Examination of Water and Wastwater (2012) como espectrometria ótica de emissão atômica com plasma acoplado (ICP-OES) e cromatografia iônica (IC), seguindo os princípios básico de amostragem e analises em Química Analítica e Ambiental. Os resultados mostram as classificações das águas subterrâneas em Cacoal e suas melhores indicações para usos de acordo com sua qualidade química, expressando os resultados em métodos gráficos, como o Diagrama de Piper, e possíveis correlações entre os íons analisados, indicando a relação entre a geologia da região e sua influência na composição química dos recursos hídricos subterrâneos em Cacoal, bem como possíveis ações antrópicas. Os resultados serão comparados com a legislação vigente sobre qualidade de água, expedidas pelo Ministério da Saúde e Ministério do Meio Ambiente.
	
	\vspace{\onelineskip}
	
	\noindent
	\textbf{Palavras-chave}: águas subterrâneas, Cacoal, hidroquímica.
	
\end{document}
