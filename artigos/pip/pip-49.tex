\documentclass[article,12pt,onesidea,4paper,english,brazil]{abntex2}

\usepackage{lmodern, indentfirst, nomencl, color, graphicx, microtype, lipsum}			
\usepackage[T1]{fontenc}		
\usepackage[utf8]{inputenc}		

\setlrmarginsandblock{2cm}{2cm}{*}
\setulmarginsandblock{2cm}{2cm}{*}
\checkandfixthelayout

\setlength{\parindent}{1.3cm}
\setlength{\parskip}{0.2cm}

\SingleSpacing

\begin{document}
	
	\selectlanguage{brazil}
	
	\frenchspacing 
	
	\begin{center}
		\LARGE EFEITO DE DIFERENTES SUBSTRATOS NA PRODUÇÃO DE MASSA SECA DA RAIZ EM CLONES DE TOMATEIRO\footnote{Trabalho realizado dentro da (área de Conhecimento CNPq: Fitotecnia) com financiamento do (Instituto Federal de Educação, Ciência e Tecnologia de Rondônia – campus Colorado do Oeste).}
		
		\normalsize
		Juliana Pereira Mendes\footnote{juliiana.mendes25@gmail.com, Campus Colorado do Oeste.} 
	Romário Batista de Lima Oliveira\footnote{romariolima1935@gmail.com, Campus Colorado do Oeste.} 
		Aldo Max Custódio\footnote{aldo.custodio@ifro.edu.br, Campus Colorado do Oeste.} 

	\end{center}
	
	\noindent A introdução de novas tecnologias de manejo e produção de mudas é estratégica para se alcançar maiores produtividades e rentabilidade na cultura do tomateiro. A propagação vegetativa do tomateiro é uma das alternativas para se reduzir os custos de produção de mudas, e a estaquia é a técnica de maior viabilidade econômica para o estabelecimento de plantios clonais, por permitir a multiplicação de genótipos selecionados em curto período de tempo. O uso da estaquia para multiplicação do tomateiro permite ao produtor multiplicar materiais híbridos, cujas sementes são de custos elevados e/ou difíceis de serem encontradas no mercado local. O substrato é um dos principais fatores que afetam a qualidade das mudas produzidas. O trabalho foi desenvolvido com objetivo avaliar a produção de mudas por estaquia de três genótipos de tomateiro em quatro substratos diferentes. Os tratamentos constaram de três genótipos de tomateiro (Yoshimatsu, Perseo e Vênus) e quatro substratos diferentes (húmus de minhoca (HM), composto orgânico (CO), palha de arroz carbonizada + solo de mata (CA+S) e comercial Carolina® (CC), arranjados como fatorial 3x4 no delineamento experimental inteiramente casualizado, com 4 repetições. As estacas foram adquiridas de plantas “matrizes” obtidas de sementes, que foram plantadas no Setor de Produção Vegetal I do Instituto Federal de Educação, Ciência e Tecnologia de Rondônia – campus Colorado do Oeste. Após trinta dias do plantio as estacas/mudas foram retiradas do substrato e avaliadas. Foram consideradas na avaliação a massa seca da raiz em gramas, obtida por meio da secagem do material em estufa com circulação de ar a 75$^\circ$C, durante 48 horas, a pesagem foi realizada em balança analítica digital de alta precisão. Houve efeito significativo da interação substrato e variedade com relação a produção de massa seca da raiz. A cultivar Yoshimatsu produziu mais massa de raízes com o substrato HM (1,15 g) e CC (1,21 g). Enquanto o híbrido Vênus apresentou maior produção com os substratos CO (0,71 g) e CA+S (0,6 g). Para a cultivar Perseo não houve diferença entre os substratos (média = 0,34g). Assim, conclui-se que o melhor substrato para produção de raiz depende da cultivar.
	
	\vspace{\onelineskip}
	
	\noindent
	\textbf{Palavras-chave}: Estaquia. Mudas. \textit{Lycopersicum esculentum}.
	
\end{document}
