\documentclass[article,12pt,onesidea,4paper,english,brazil]{abntex2}

\usepackage{lmodern, indentfirst, nomencl, color, graphicx, microtype, lipsum}			
\usepackage[T1]{fontenc}		
\usepackage[utf8]{inputenc}		

\setlrmarginsandblock{2cm}{2cm}{*}
\setulmarginsandblock{2cm}{2cm}{*}
\checkandfixthelayout

\setlength{\parindent}{1.3cm}
\setlength{\parskip}{0.2cm}

\SingleSpacing

\begin{document}
	
	\selectlanguage{brazil}
	
	\frenchspacing 
	
	\begin{center}
		\LARGE PRINCIPAIS CAUSAS DA EVASÃO NO CURSO SUPERIOR DE GESTÃO PÚBLICA
		
		DO IFRO – CAMPUS PORTO VELHO ZONA NORTE\footnote{
			Trabalho realizado dentro da área de Conhecimento CNPq: Ciências Sociais Aplicadas, financiado
			com recursos do Edital no 15/2016/IFRO/DEPESP/ZONA NORTE, de 31 de maio de 2016.}
		
		\normalsize
		Danielly Eponina Santos Gamenha\footnote{Bolsista, Danielly Eponina Santos Gamenha, daniellysantos70@gmail.com, Porto Velho Zona Norte} 
		Maria Beatriz Souza Pereira\footnote{Bolsista, Maria Beatriz Souza Pereira, mbe.pereira@gmail.com, Porto Velho Zona Norte} 
		Danielli Vacari de Brum\footnote{Coordenador(a), Danielli Vacari de Brum, danielli.brum@ifro.edu.br, Porto Velho Zona Norte} 
		 
	\end{center}
	
	\noindent Apesar de vastas pesquisas relacionadas ao tema, a evasão escolar no Brasil é uma
	problemática que está presente da Educação Básica à Universidade, e apesar de
	convivermos com ela, a mesma vem crescendo a cada ano e até hoje não se
	descobriu uma medida eficaz para erradicá-la. O presente artigo apresenta uma
	análise das razões de evasão dos alunos do curso Superior em Gestão Pública do
	Instituto Federal de Educação, Ciência e Tecnologia de Rondônia – Campus Porto
	Velho Zona Norte. A coleta de dados foi obtida através de questionário com amostra
	mínima em função do erro constituída por 80 alunos. O software utilizado para a
	apuração e análise estatística foi o Sphinx Léxica. Matricularam-se no curso, de
	2013 a 2016, 374 alunos, dentre os quais 26 trancaram o curso, 90 desistiram e 10
	retornaram, segundo dados coletados junto a Coordenação de Registros
	Acadêmicos. Durante a análise podemos verificar que mais da metade dos
	respondentes são do sexo masculino (51,2\%), não possuem filhos (51,2\%), são
	solteiros (51\%), com idade entre 18 e 41 anos (87\%), que trabalham e são
	responsáveis pela renda da família (75\%). As principais causas de desistência foram
	problemas pessoais (51,2\%), seguidos de problemas de saúde (46,3\%).
	Desinteresse e desmotivação do curso (43,8\%) e dificuldade financeira com
	transporte, alimentação e xerox, representaram 31,3\% dos fatores apontados.
	Alguns dados apontam que os mesmos permanecem na instituição pela qualidade
	de ensino do corpo docente e principalmente por estabelecerem laços com
	professores, amigos e servidores, pela inserção no mercado de trabalho e
	possibilidades salariais, para concursos públicos em outras áreas, para terem um
	curso superior ou simplesmente por não ter outra opção. Porém estes não são
	fatores suficientes que os mantêm em sala de aula até a conclusão de sua
	graduação. Portanto, é necessário realizar um trabalho especial, efetivando um
	acompanhamento mensal, onde a equipe pedagógica e o corpo docente trabalhem
	em conjunto, observando o comportamento do aluno, realizando um levantamento
	com o intuito de reunir informações para encontrar soluções práticas, e, voltando
	uma atenção especial quando houver queda na assiduidade nas aulas, objetivando
	erradicar este crescente problema.
	
	\vspace{\onelineskip}
	
	\noindent
	\textbf{Palavras-chave}: Desistência Escolar. Cursos Superiores. IFRO.
	
\end{document}
