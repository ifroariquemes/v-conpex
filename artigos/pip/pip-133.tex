\documentclass[article,12pt,onesidea,4paper,english,brazil]{abntex2}

\usepackage{lmodern, indentfirst, nomencl, color, graphicx, microtype, lipsum}			
\usepackage[T1]{fontenc}		
\usepackage[utf8]{inputenc}		

\setlrmarginsandblock{2cm}{2cm}{*}
\setulmarginsandblock{2cm}{2cm}{*}
\checkandfixthelayout

\setlength{\parindent}{1.3cm}
\setlength{\parskip}{0.2cm}

\SingleSpacing

\begin{document}
	
	\selectlanguage{brazil}
	
	\frenchspacing 
	
	\begin{center}
		\LARGE REDES SOCIAIS: A VULNERABILIDADE\\DAS INFORMAÇÕES\footnote{Trabalho realizado dentro da Ciências exatas e da terra.}
		
		\normalsize
		Nathalya Sousa Oliveira \, Gabrielle Gonçalves Lopes \, Isabely Silva Casagrande\footnote{Aluno(s) Colaborador(es), nathalyasullivan@gmail.com, issahcasagrande@gmail.com,\\gabrielleloipes2001@gmail.com, Campus Ji-Paraná.} 
	Ilma Rodrigues\footnote{Orientadora, ipfausto@gmail.com, Campus Ji-Paraná.} 

	\end{center}
	
	\noindent A pesquisa em questão teve seu desenvolvimento no IFRO – Campus Ji-Paraná,
	visando abordar um projeto de iniciação científica aplicado à matéria de Orientação
	para pesquisa e prática profissional (OPPP). O foco principal do projeto é identificar
	o comportamento e a vulnerabilidade no âmbito das redes sociais coletando
	resultados através de questionários na plataforma Formulários do Google Drive, em
	conjunto com a aplicação Facebook. O projeto teve seu desenvolvimento nos
	laboratórios de informática do IFRO – Campus Ji-Paraná ao decorrer das aulas de
	OPPP, ministradas pela docente e orientadora Ilma Rodrigues de Souza Fausto ao
	longo do ano de 2017, tendo sua atividade no turno matutino às terças-feiras. As
	alunas colaboradoras realizaram uma pesquisa com discentes do período diurno do
	curso de informática, onde estes responderam a um questionário comportamental
	sobre o uso das redes, junto a isto, houve a criação de um perfil falso no Facebook
	por parte das colaboradoras, para observação das atitudes comportamentais dos
	usuários. Com os resultados obtidos, gera-se a estatística que comprova a utilização
	em massa destes meios, todavia, inclui-se a experiência obtida a partir dos dados
	levantados pela conta, onde usuários apresentaram um alto nível de confiança
	mesmo não sabendo quem estava por trás do perfil em questão. Para apresentação
	dos resultados, serão ministradas palestras aos alunos do primeiro ano do curso de
	informática com o intuito de mostrar o quanto se está exposto no ambiente virtual, de
	modo que se gere conscientização por parte destes.
	
	\vspace{\onelineskip}
	
	\noindent
	\textbf{Palavras-chave}: Segurança. Redes sociais. Informação.
	
\end{document}
