\documentclass[article,12pt,onesidea,4paper,english,brazil]{abntex2}

\usepackage{lmodern, indentfirst, nomencl, color, graphicx, microtype, lipsum}			
\usepackage[T1]{fontenc}		
\usepackage[utf8]{inputenc}		

\setlrmarginsandblock{2cm}{2cm}{*}
\setulmarginsandblock{2cm}{2cm}{*}
\checkandfixthelayout

\setlength{\parindent}{1.3cm}
\setlength{\parskip}{0.2cm}

\SingleSpacing

\begin{document}
	
	\selectlanguage{brazil}
	
	\frenchspacing 
	
	\begin{center}
		\LARGE PROPRIEDADES QUÍMICAS DO SOLO SOB DIFERENTES SISTEMAS DE
		
		MANEJOS EM AMBIENTE AMAZÔNICO\footnote{Trabalho realizado dentro da área de Ciências Agrárias com financiamento do CNPq/ IFRO.}
		
		\normalsize
		Thainá Alves Andrade\footnote{Colaboradora, thaina.taa@hotmail.com, campus Ariquemes} 
		Edson Souza Martins Júnior\footnote{Bolsista (PIBIC EM), edson.souza.ifro@gmail.com, barttoxa@gmail.com, campus Ariquemes} 
		Matheus Lorrah Costa de Oliveira 
		Lenita Aparecida Conus Venturoso\footnote{Orientadora, lenita.conus@ifro.edu.br, campus Ariquemes} 
		Luciano dos Reis Venturoso\footnote{Co-orientador, Luciano.venturoso@ifro.edu.br, campus Ariquemes}
	\end{center}
	
	\noindent A implantação de atividades agrícolas e pecuárias em Rondônia vem modificando as
	propriedades químicas do solo, devido ao diversos sistema de produção e manejos
	adotados nas áreas. Nesse cenário, o grande desafio tem sido desenvolver sistemas
	capazes de recuperar áreas degradada, conciliando conservação ambiental com
	sustentabilidade econômica. Assim, o objetivo do estudo foi avaliar os atributos
	químicos de um Latossolo Vermelho-Amarelo distrófico em diferentes sistemas de
	manejos do solo em Ariquemes, RO. A pesquisa foi realizada no Instituto Federal de
	Rondônia, campus Ariquemes, e conduzida em delineamento inteiramente
	casualizado, com três repetições. Os tratamentos foram constituídos pelos sistemas
	de manejo do solo: preparo convencional (milho), plantio direto (soja), fruticultura
	solteira (goiaba), fruticultura consorciada (cupuaçu e seringueira), floresta plantada
	(teca), pastagem (braquiária) e mata nativa. As amostras para as análises químicas
	foram coletadas nas entrelinhas das culturas, nas profundidades 0,00-0,10, 0,10-
	0,20 e 0,20-0,40 m, para determinação de pH em CaCl2, alumínio trocável (Al+3),
	fósforo disponível (P), cátions trocáveis (K+
	
	, Ca+2 e Mg+2), acidez trocável (H++Al+3),
	soma de bases (SB), capacidade de troca de cátions (T) e saturação de bases (V).
	Verificou-se efeito significativo dos sistemas de manejos sobre os atributos químicos
	do solo, nas profundidades avaliadas. Resultados superiores para Ca+2
	, Mg+2
	, SB e
	V, foram obtidos na área de fruticultura solteira na camada de 0,00-0,10 m,
	influenciando positivamente os atributos químicos dessa área em todas as camadas.
	Observou-se baixas saturações por bases nas áreas, sendo inferior a considerada
	ideal para as culturas, demonstrando que a calagem e adubação são práticas
	necessárias para manter a fertilidade dos solos. As áreas floresta plantada e mata
	nativa apresentaram menores resultados para os atributos químicos em todas as
	profundidades, indicando uma elevada acidez nestes sistemas e menores teores de
	nutrientes. Nas condições do presente estudo, os sistemas de manejos alteram a
	fertilidade, sendo a fruticultura solteira o sistema que proporciona melhores atributos
	químicos do solo.
	
	\vspace{\onelineskip}
	
	\noindent
	\textbf{Palavras-chave}:Fertilidade. Atributos químicos. Uso do solo.
	
\end{document}
