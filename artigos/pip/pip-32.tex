\documentclass[article,12pt,onesidea,4paper,english,brazil]{abntex2}

\usepackage{lmodern, indentfirst, nomencl, color, graphicx, microtype, lipsum}			
\usepackage[T1]{fontenc}		
\usepackage[utf8]{inputenc}		

\setlrmarginsandblock{2cm}{2cm}{*}
\setulmarginsandblock{2cm}{2cm}{*}
\checkandfixthelayout

\setlength{\parindent}{1.3cm}
\setlength{\parskip}{0.2cm}

\SingleSpacing

\begin{document}
	
	\selectlanguage{brazil}
	
	\frenchspacing 
	
	\begin{center}
		\LARGE COMPONENTES DO RENDIMENTO E RENDIMENTO DE CULTIVARES DE SOJA
		COM TRANSGENIA RR E INTACTA\footnote{Trabalho realizado dentro da (área de Conhecimento CNPq: Ciências Agrárias) com financiamento
			do CNPq / IFRO.}
		
		\normalsize
	Lucas Belarmino dos Santos Almeida\footnote{Bolsista (PIBIC EM), lucas261198.lb@gmail.com, Campus Ariquemes.} 
		Gabriel da Silva Stoinski\footnote{Bolsista (PIBIC EM), gabrielstoinsk@gmail.com, Campus Ariquemes.} 
		Gabriela Alves Corrêa\footnote{Colaboradora, bicacorrd@gmail.com, Campus Ariquemes.} 
	Lenita
	Aparecida Conus Venturoso\footnote{Orientadora, lenita.conus@ifro.edu.br, Campus Ariquemes.}
	Luciano dos Reis Venturoso\footnote{Co-orientador, luciano.venturoso@ifro.edu.br, Campus Ariquemes.}
	 
	\end{center}
	
	\noindent A cultura da soja tem sido cultivada em quase todas as regiões do mundo, e os
	materiais geneticamente modificados tem ganhado espaço, passando a ser mais
	utilizados do que a soja convencional, tornando necessário estudos que possam
	determinar cultivares com melhor adaptação às condições do estado de Rondônia.
	O trabalho teve por objetivo avaliar o desempenho de dezesseis cultivares de soja
	transgênica, no município de Ariquemes. O ensaio experimental foi conduzido na
	área experimental do Instituto Federal de Rondônia, Campus Ariquemes. Adotou-se
	o delineamento experimental de blocos casualizados, com quatro repetições. O
	preparo do solo foi executado de modo convencional. Foram cultivados os materiais,
	BRBMG 12-008, BRB 11-15660, BRB 11-16058, BRBMG 12-0019, BRi 12-20669,
	BRi 12-20675, BRB 12-20587, BRASRR 12-13375, BRGO 11-3814-3, BRS 8890RR,
	BRRY 45-16349, P98Y11, M 8210 IPRO, P98Y30, TMG 132 RR e BRSMG 850GRR.
	As cultivares foram semeadas em parcelas contendo quatro linhas de 5 m de
	comprimento, espaçadas entre si por 0,45 m e densidade seguindo as
	recomendações de cada material. Foram avaliados o número de vagens por planta,
	número de grãos por vagem, massa de cem grãos e o rendimento. Foi verificado
	efeito significativo para os componentes do rendimento, número de vagens e grãos
	por vagem. Destaca-se a cultivar BRS 8890RR, com 162 vagens em média por
	planta. Com relação ao número de grãos, observou-se em média 2,18 grãos por
	vagem, com destaque para BRi 12-20675 com 2,92 vagens. Para o rendimento de
	grãos, a superioridade foi observada para os materiais BRi 12-20669, BRSMG
	850GRR, BRB 11-15660, TMG 132 RR e BRB 11-16058, com valores de 4118,4,
	3483,4, 3224,7, 3000,5 e 2953,1 kg.ha-1, respectivamente.
	
	\vspace{\onelineskip}
	
	\noindent
	\textbf{Palavras-chave}: \textit{Glycine max}. Cultivares transgênicas. Soja em Rondônia. \\
	\textbf{Fonte de Financiamento}: CNPq e Instituto Federal de Rondônia.
	
\end{document}
