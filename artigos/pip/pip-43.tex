\documentclass[article,12pt,onesidea,4paper,english,brazil]{abntex2}

\usepackage{lmodern, indentfirst, nomencl, color, graphicx, microtype, lipsum}			
\usepackage[T1]{fontenc}		
\usepackage[utf8]{inputenc}		

\setlrmarginsandblock{2cm}{2cm}{*}
\setulmarginsandblock{2cm}{2cm}{*}
\checkandfixthelayout

\setlength{\parindent}{1.3cm}
\setlength{\parskip}{0.2cm}

\SingleSpacing

\begin{document}
	
	\selectlanguage{brazil}
	
	\frenchspacing 
	
	\begin{center}
		\LARGE Desenvolvimento do OEL, um objeto de aprendizagem para auxílio à
		alfabetização infantil\footnote{Trabalho realizado dentro da área da Ciência da Computação, com financiamento do Departamento de
			Pesquisa do Instituto Federal de Educação, Ciência e Educação – Campus Porto Velho Calama.}
		
		\normalsize
	Luis Filipe de Castro Sampaio\footnote{Bolsista de iniciação à pesquisa, luizw12345@gmail.com, Porto Velho Calama.} 
		Heloisy Pimenta Queiroz\footnote{Bolsista de iniciação à pesquisa, heloisy.pimenta1@gmail.com, Porto Velho Calama.} 
	Jonas Wellington Andrade\footnote{Bolsista de iniciação à pesquisa, jonasandrade.pvh26@gmail.com, Porto Velho Calama.} 
	Ana Cláudia
	Oliveira da Silva\footnote{Colaborador(a), ana.silva@ifro.edu.br, Porto Velho Calama.} 
	Silvio Luiz de Freitas\footnote {Co-orientador, silvio.freitas@ifro.edu.br, Porto Velho Calama.}
	Marcel Leite Rios\footnote{Co-orientador, marcel.rios@ifro.edu.br, Porto Velho Calama.}
	Kaio Alexandre da Silva\footnote{Orientador, kaio.silva@ifro.edu.br, Porto Velho Calama.}
	
	\end{center}
	
	\noindent Na última década, com a rápida difusão de dispositivos móveis na sociedade, criouse
	um ambiente com oportunidades de inovações no processo educacional.
	\textit{Softwares} que possuam o objetivo de transmitir conhecimento sobre determinado
	assunto aos seus usuários são considerados objetos de aprendizagem. Com a
	evolução da tecnologia e o surgimento dos \textit{smartphones} e \textit{tablets}, nasceu um novo
	estilo de aprendizagem chamado de \textit{Mobile Learning}, que possibilita ao usuário
	aprender em qualquer lugar e a qualquer hora. Utilizar tais tecnologias como aliadas
	ao ensino, auxilia na abertura de oportunidades para o aluno trabalhar a sua
	criatividade, ao mesmo tempo em que se torna um elemento de motivação e
	colaboração, tornando o processo de aprendizagem atraente, divertido e
	significativo. Neste sentido, foi desenvolvido o Objeto de Estudo de Letramento
	(OEL), composto por um sistema \textit{Web}, uma base de dados, um \textit{Web Service} e um
	aplicativo móvel. A ideia principal do OEL é possibilitar ao professor, o
	acompanhamento do uso do aplicativo pelos seus alunos, pois é através deste uso
	que o sistema poderá identificar as dificuldades dos alunos e recomendar ao
	professor realizar o acompanhamento de forma mais personalizada. O aplicativo
	móvel foi desenvolvido para a plataforma \textit{Android}, a base de dados escolhida foi o
\textit{MySQL} e o \textit{Web Service} foi desenvolvido com a linguagem de programação \textit{PHP}.
	Como resultados até o momento foram publicados os trabalhos: “Desenvolvimento
	de um Objeto de Aprendizagem baseado em Mobile Learning e sistemas de
	recomendações para o auxílio ao processo de letramento infantil na educação
	básica” na XIII Escola Regional de Banco de Dados, “Desenvolvimento de um Web
	Service baseado em REST para a interligação de dados entre uma aplicação mobile
	e um portal WEB” no VIII Computer on the Beach e “Objeto de Estudo de
	Letramento (OEL), um objeto de aprendizagem para auxílio à alfabetização infantil”
	no IV Encontro Nacional de Computação dos Institutos Federais. Além destes
	trabalhos publicados ressalta-se que foram aprovados sete trabalhos em congressos
	que não puderam ser apresentados por falta de recurso na instituição para a
	realização da viagem e que a pesquisa continua sendo executada através do
	programa internacionalização estudantil.
	
	\vspace{\onelineskip}
	
	\noindent
	\textbf{Palavras-chave}: Mobile Learning. Objeto de Aprendizagem. Desenvolvimento de
	sistemas.
	
\end{document}
