\documentclass[article,12pt,onesidea,4paper,english,brazil]{abntex2}

\usepackage{lmodern, indentfirst, nomencl, color, graphicx, microtype, lipsum}			
\usepackage[T1]{fontenc}		
\usepackage[utf8]{inputenc}		

\setlrmarginsandblock{2cm}{2cm}{*}
\setulmarginsandblock{2cm}{2cm}{*}
\checkandfixthelayout

\setlength{\parindent}{1.3cm}
\setlength{\parskip}{0.2cm}

\SingleSpacing

\begin{document}
	
	\selectlanguage{brazil}
	
	\frenchspacing 
	
	\begin{center}
		\LARGE QUALIDADE DA COLHEITA EM FUNÇÃO DA PLATAFORMA DE CORTE DA
		
		COLHEDORA\footnote{Trabalho realizado dentro da (área de Conhecimento CNPq: Ciências Agrárias)}
		
		\normalsize
		Rodrigo de Aguiar Gonçalves\footnote{Bolsista (IT - ES), ro.aguiiar688@gmail.com, Campus Colorado do Oeste} 
		Ronicley Souza da Silva\footnote{Colaborador, ronicley.ifro@gmail.com, Campus Colorado do Oeste} 
		Patricia Candida de Menezes\footnote{Orientadora, patricia.menezes@ifro.edu.br, Campus Colorado do Oeste} 
		Rafael
		Henrique Pereira dos Reis\footnote{Co-orientador, rafael.reis@ifro.edu.br, Campus Colorado do Oeste} 
	\end{center}
	
	\noindent A produção de soja apresenta grande importância para o cenário agrícola de
	Rondônia. A cultura é um dos principais produtos agrícolas do Estado. Para garantir
	o retorno financeiro para os investimentos realizados durante o processo de
	produção, a colheita, que é a última operação realizada no campo e onde podem
	ocorrer muitas perdas, precisa ser monitorada adequadamente. Considerando que a
	plataforma de corte utilizada, assim como a velocidade de colheita podem influenciar
	na quantidade de perdas na operação, objetivou-se avaliar a qualidade da colheita
	mecanizada da soja realizada com dois tipos de plataforma de corte e duas
	velocidades de deslocamento. O experimento foi realizado no município de Cabixi -
	RO em área agrícola com a cultivar P98Y12 instalada. Foram utilizadas duas
	colhedoras modelo MF 9790 equipadas com plataformas dos tipos esteira
	transportadora (Draper) e condutor helicoidal (Caracol). A colheita foi realizada em
	duas velocidades, V1: 6 km h-1 e V2: 8 km h
	-1
	. O delineamento experimental foi
	estabelecido de acordo com as premissas básicas do controle estatístico de
	processo, utilizando-se as cartas de controle como ferramenta. Foram amostrados
	80 pontos distanciados a cada 100 metros, sendo dois pontos em cada passada da
	máquina. Os indicadores de qualidade avaliados no campo foram perdas na
	plataforma, perdas nos mecanismos internos e perdas totais. As perdas médias na
	plataforma de condutor helicoidal foram 35\% superiores à esteira transportadora na
	velocidade de 6 km h-1
	
	(V1) e 51\% na velocidade de 8 km h-1 (V2). Para perdas nos
	mecanismos internos a plataforma de esteira transportadora apresentou menores
	médias e variabilidade do processo nas duas velocidades de deslocamento em
	relação à de condutor helicoidal. As perdas totais variaram de 1,8 a 4,4\% excedendo
	o limite específico superior de controle que foi estipulado baseado no nível tolerado
	de perdas. A plataforma de esteira transportadora apresentou melhor qualidade na
	colheita com menores médias de perdas e menor variabilidade do processo. A
	velocidade de deslocamento teve pouca influência na qualidade da operação para
	os indicadores de qualidade avaliados. O nível de perdas na colheita apresentou
	valores acima do limite aceitável indicando necessidade de melhorias do processo.
	
	\vspace{\onelineskip}
	
	\noindent
	\textbf{Palavras-chave}:Caracol. Draper. Perdas.
	
	\noindent
	\textbf{Fonte de financiamento}:Instituto Federal de Educação Ciência e Tecnologia de
	Rondônia.
	
\end{document}
