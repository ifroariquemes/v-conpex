\documentclass[article,12pt,onesidea,4paper,english,brazil]{abntex2}

\usepackage{lmodern, indentfirst, nomencl, color, graphicx, microtype, lipsum}			
\usepackage[T1]{fontenc}		
\usepackage[utf8]{inputenc}		

\setlrmarginsandblock{2cm}{2cm}{*}
\setulmarginsandblock{2cm}{2cm}{*}
\checkandfixthelayout

\setlength{\parindent}{1.3cm}
\setlength{\parskip}{0.2cm}

\SingleSpacing

\begin{document}
	
	\selectlanguage{brazil}
	
	\frenchspacing 
	
	\begin{center}
		\LARGE \MakeUppercase{Abelhas sem ferrão (Meliponíneos) no Instituto Federal de Educação, Ciência e Tecnologia de Rondônia}\\\MakeUppercase{campus Ariquemes}\footnote{Trabalho realizado dentro das Ciências Biológicas com financiamento do Instituto Federal de Educação, Ciência e Tecnologia de Rondônia – IFRO campus Ariquemes (2016/2017).}
		
		\normalsize
	Alysson Rossi dos Santos\footnote{Bolsista (acadêmico do curso de Ciências Biológicas), alyssonr@hotmail.com, Instituto Federal de Educação, Ciência e Tecnologia de Rondônia - IFRO campus Ariquemes} 
		Ana Paula Schneider\footnote{Bolsista (acadêmica do curso Técnico em Agropecuária), ana\_agropecuaria@outlook.com, Instituto Federal de Educação, Ciência e Tecnologia de Rondônia - IFRO campus Ariquemes} \\
	Francisco Tarcísio Lisboa\footnote{Colaborador, ftlisboa@hotmail.com, Campus} 
		Clotilde Tânia Rodrigues Luz\footnote{Orientadora, tania.luz@ifro.edu.br, Instituto Federal de Educação, Ciência e Tecnologia de Rondônia - IFRO campus Ariquemes} 
	\end{center}
	
	\noindent Abelhas são insetos voadores com atividades imprescindíveis para o ser humano e à preservação da natureza. Abelhas sem ferrão (meliponíneos) possuem ferrão atrofiado que não pode ser usado como meio de defesa. Esses insetos meliponíneos, no continente americano, são encontrados desde o Rio Grande do Sul até o México, estabelecendo importante relação com as plantas na produção de frutos, sementes e mel. O estudo, conduzido por dois acadêmicos bolsistas e uma professora do Instituto Federal de Educação, Ciência e Tecnologia de Rondônia-IFRO campus Ariquemes, ocorre em um remanescente florestal do mesmo campus com o objetivo de coletar, por meio de redes entomológicas, as espécies de meliponíneos existentes no local, e promover o manejo e o aproveitamento de ninhos das espécies encontradas em caixas entomológicas a serem disponibilizadas para estudos no campus. A primeira etapa do estudo constituiu-se de visita técnica a dois meliponários, estabelecidos na zona rural no município de Alto Paraíso/RO, para adquirir conhecimento sobre os processos de identificação de locais com ninhos na natureza, transferência de colmeias para caixas entomológicas, manejo, preferências botânicas e hábitos das espécies. Na ocasião foi possível identificar quatro espécies de meliponíneos, sendo: Melipona melanoventer, Melipona flavolineata, Melipona seminigra abunensis, Melipona fuliginosa, assim como, a estrutura das respectivas colmeias para produção comercial de mel. Na segunda etapa, realizada durante o dia no remanescente florestal após forte chuva na madrugada anterior, localizou-se uma colmeia arborícola da espécie Melipona compressipes e a captura de duas espécies de meliponíneos, sendo uma de hábitos terrestres e a outra ainda não identificada. Mesmo com a incidência de alta pluviosidade no local do estudo na noite anterior à realização da segunda etapa, o registro de três espécies diferentes permite propor a existência de mais espécies, visto o local possuir características botânicas e climáticas favoráveis à existência de colmeias de abelhas sem ferrão.
	
	\vspace{\onelineskip}
	
	\noindent
	\textbf{Palavras-chave}: Abelhas sem ferrão. Rondônia. Conpex.
	
\end{document}
