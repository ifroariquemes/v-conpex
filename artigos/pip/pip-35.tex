\documentclass[article,12pt,onesidea,4paper,english,brazil]{abntex2}

\usepackage{lmodern, indentfirst, nomencl, color, graphicx, microtype, lipsum, textcomp}			
\usepackage[T1]{fontenc}		
\usepackage[utf8]{inputenc}		

\setlrmarginsandblock{2cm}{2cm}{*}
\setulmarginsandblock{2cm}{2cm}{*}
\checkandfixthelayout

\setlength{\parindent}{1.3cm}
\setlength{\parskip}{0.2cm}

\SingleSpacing

\begin{document}
	
	\selectlanguage{brazil}
	
	\frenchspacing 
	
	\begin{center}
		\LARGE CRESCIMENTO DE MUDAS DE COPAÍBA SOB OMISSÃO DE NUTRIENTES\footnote{Trabalho realizado dentro da área de Conhecimento CNPq: Ciências Agrárias com financiamento do
			Instituto Federal de Rondônia – Campus Ji-Paraná.}
		
		\normalsize
	Josiane da Silva Barboza\footnote{Bolsista, josiane\_dasilva2013@hotmail.com, Campus Ji-Paraná.} 
		Waldelaine Rodrigues Hoffman\footnote{Bolsista, waldelaine-hoffmann@hotmail.com, Campus Ji-Paraná.} 
		Adalberto Alves da Silva\footnote{Orientador, adalberto.alves@ifro.edu.br , Campus Ji-Paraná.} 
		Deilton
		Wellington Ribeiro Nogueira5\footnote{Co-orientador, deilton.nogueira@ifro.edu.br, Campus Ji-Paraná.} 
	\end{center}
	
	\noindent A região amazônica nas últimas décadas vem perdendo a cobertura vegetal de
	grandes áreas devido a pressões antrópicas para expansão do agronegócio,
	mineração e exploração de madeira, o que põe em risco espécies florestais nativas
	de inestimável valor ambiental e econômico para a região. Portanto, para que haja
	sucesso na recuperação das áreas degradadas faz-se necessário que se efetivem
	estudos sistemáticos sobre as condições e exigências nutricionais de espécies
	florestais nativas, a exemplo da copaíba (\textit{Copaifera langsdorffii}). O presente estudo
	objetivou caracterizar os aspectos nutricionais e os efeitos causados pela omissão
	de nutrientes no crescimento inicial de mudas de copaíba, empregando a técnica da
	omissão de nutrientes. O presente estudo foi conduzido em casa de vegetação no
	Instituto Federal de Rondônia (IFRO), Câmpus Ji-Paraná, por um período de 90
	dias, empregando como substrato areia e vermiculita (2:1), ambos esterilizados em
	autoclave.O delineamento experimental foi estabelecido em blocos casualizados
	(DBC) com 12 tratamentos dispostos em 4 blocos perfazendo 48 parcelas
	experimentais em recipientes de cano de PVC, os quais abrigaram uma planta em
	cada vaso. Para o acompanhamento do desenvolvimento de cada amostra foram
	realizadas biometrias quinzenalmente. Em relação as soluções nutritivas, foi
	utilizado o método de diagnose por subtração, no qual adota-se o tratamento
	completo como testemunha e omite-se um elemento por vez em cada tratamento. O
	pH das soluções foi controlado entre 5,5 e 6,5, afim de simular condições naturais
	em campo. Diariamente cada planta foi regada com 50 ml de solução. Após a etapa
	de crescimento, as partes aéreas e radiculares foram separadas, medidas e secas
	até peso constante. Ao aplicar a análise estatística do Teste de Tukey, pode-se
	afirmar, com 95\% de confiança, que a omissão de nutrientes, em todos os
	tratamentos, resultaram em diferença significativa quando comparados ao
	tratamento completo, que atingiu 17,23 cm. Destaque para o nutriente limitante ferro,
	naturalmente o componente mais abundante em solos brasileiros, mas ausente no
	substrato utilizado, que atingiu somente 6,18(±2,41)cm. Por fim, pode-se concluir
	que a \textit{Copaifera langsdorffii} é uma espécie clímax de crescimento lento, que exige
	uma grande demanda nutricional e condições específicas para seu crescimento.
	
	\vspace{\onelineskip}
	
	\noindent
	\textbf{Palavras-chave}: \textit{Copaifera langsdorffii}. produção de mudas. Amazônia.
	
\end{document}
