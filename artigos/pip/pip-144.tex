\documentclass[article,12pt,onesidea,4paper,english,brazil]{abntex2}

\usepackage{lmodern, indentfirst, nomencl, color, graphicx, microtype, lipsum,textcomp}			
\usepackage[T1]{fontenc}		
\usepackage[utf8]{inputenc}		

\setlrmarginsandblock{2cm}{2cm}{*}
\setulmarginsandblock{2cm}{2cm}{*}
\checkandfixthelayout

\setlength{\parindent}{1.3cm}
\setlength{\parskip}{0.2cm}

\SingleSpacing

\begin{document}
	
	\selectlanguage{brazil}
	
	\frenchspacing 
	
	\begin{center}
		\LARGE UTILIZAÇÃO DE HERBICIDAS PRÉ-EMERGENTES PARA CONTROLE DE
		PLANTAS DANINHAS NA CULTURA DA SOJA1\footnote{Trabalho realizado dentro da área de Conhecimento CNPq: Ciências Agrárias com financiamento do
			Conselho Nacional de Desenvolvimento Científico e Tecnológico (CNPq).}
		
		\normalsize
		Diefferson Campos da Fonseca\footnote{Bolsista (Ensino Médio), diefferson.ifro@gmail.com, Campus Colorado do Oeste.} 
	Rafael dos Santo Oliveira\footnote{Colaborador(a), rafaelsantosoliveira16@gmail.com, Campus Colorado do Oeste} 
	Marcos Aurélio Anequine de Macedo\footnote{Orientador(a), marcos.anequine@ifro.edu.br, Campus Colorado do Oeste.} 
		Hugo de Almeida Dan\footnote{Co-orientador(a), halmeidadan@gmail.com,.} 
	\end{center}
	
	\noindent A necessidade de se manter a cultura de interesse econômico no limpo em seus
	estádios iniciais mostra-se como relevante estratégia para aumento de
	produtividade. Assim, objetivou-se com esse trabalho realizar a avaliação de
	diferentes tipos de produtos com eficácia residual fazendo-se uso de aplicação em
	pós-emergência para o controle de plantas daninhas. O experimento foi conduzido
	em propriedade pertencente ao município de Cerejeiras-RO. O delineamento
	experimental utilizado foi em blocos casualizados com três repetições, arranjado em
	esquema de parcelas subdivididas 2x5+1 correspondendo a dois modos de
	aplicação, pré e pós-emergência, 5 tipos de herbicidas: imazethapyr, carfentrazoneethyl,
	carfentrazone-ethyl+sulfentrazone, diclosulam, imazethapyr+flumioxazin e uma
	1 testemunha. A princípio aplicou-se diferentes tipos de herbicidas pré-emergentes
	em área total das parcelas, ao decorrer cerca de vinte e cinco dias realizou-se a
	divisão das parcelas ao meio e realizado aplicação de glyphosate em pósemergência,
	caracterizando-se como um tratamento sequencial ou dobrado. Os
	tratamentos em pré e em pós foram avaliados aos 45 dias após emergência (DAE).
	A avaliação de pré-emergentes aos 45 DAE, apresentou níveis de controle de 30 a
	60\%, observou-se maiores médias de controle para o tratamento de
	imazethapyr+flumioxazin. A avaliação de pós-emergente aos 45 DAE, apresentou
	um nível de controle de 80 a 100\%, observou-se maiores médias de controle para o
	tratamento de imazethapyr+flumioxazin. Em avaliação de pós-emergente o
	tratamento com imazethapyr isolado com o uso de sequencial apresentou percentual
	de controle de cerca de 94\%. O alto índice de controle deve-se ao fato que o produto
	em questão possui alto efeito residual e por existir uma baixa porcentagem de
	plantas tolerantes ou resistentes à molécula. Para os tratamentos em pósemergência
	destaca-se que todos os tratamentos atingiram percentuais acima de
	90\% de controle não diferenciando-se estatisticamente. De acordo com a discussão,
	conclui-se que, o uso da mistura de imazethapyr+flumioxazin em pré-emergência até
	os 45 DAE foi o tratamento no qual apresentou maior eficiência para o controle na
	área. Destaca-se que o uso do glyphosate em pós-emergência mostrou-se como
	importante ferramenta para o manejo igualando contribuindo para um controle
	eficiente.
	
	\vspace{\onelineskip}
	
	\noindent
	\textbf{Palavras-chave}: Controle químico. Glycine max. planta daninha.
	
\end{document}
