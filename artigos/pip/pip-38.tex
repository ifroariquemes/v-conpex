\documentclass[article,12pt,onesidea,4paper,english,brazil]{abntex2}

\usepackage{lmodern, indentfirst, nomencl, color, graphicx, microtype, lipsum}			
\usepackage[T1]{fontenc}		
\usepackage[utf8]{inputenc}		

\setlrmarginsandblock{2cm}{2cm}{*}
\setulmarginsandblock{2cm}{2cm}{*}
\checkandfixthelayout

\setlength{\parindent}{1.3cm}
\setlength{\parskip}{0.2cm}

\SingleSpacing

\begin{document}
	
	\selectlanguage{brazil}
	
	\frenchspacing 
	
	\begin{center}
		\LARGE DESEMPENHO AGRONÔMICO DO PEPINO HÍBRIDO KATANÁ EM DIFERENTES
		SISTEMAS DE CONDUÇÃO EM AMBIENTE PROTEGIDO\footnote{Trabalho realizado dentro da área de Conhecimento CNPq: Manejo e Tratos Culturais.}
		
		\normalsize
	Marcielly Mayara Gomes\footnote{Marcielly Mayara Gomes (graduanda), agromarciellygomes@gmail.com, Campus Colorado do
		Oeste.} 
		Rafael Santos Oliveira\footnote{Rafael Santos Oliveira (graduando), rafaelsantosoliveira16@gmail.com , Campus Colorado do
			Oeste.} 
		Marcos Aurelio Anequine Macedo\footnote{Marcos Aurelio Anequine Macedo, marcos.anequine@ifro.edu.br , Campus Colorado do Oeste.} 
		Fábio José das Dores\footnote{Fábio José das Dores, fabio.jose@ifro.edu.br, Campus Colorado do Oeste.} 
	\end{center}
	
	\noindent
	 O pepino \textit{(Cucumes sativus)} é uma cucurbitaceae que se destaca em função da sua
	grande aceitação, sendo uma hortaliça-fruto que pode ser consumida na forma de
	salada, crua ou mesmo processada, como conserva. Em ambiente protegido podese
	empregar práticas fitotécnicas na cultura do pepino como tutoramento, técnica
	que permite ao produtor um grande aumento de produtividade, melhorando a
	qualidade dos pepinos esteticamente. Em geral, a cerca cruzada e a cerca vertical
	são os sistemas de tutoramento mais utilizados no cultivo de hortaliças de hábito
	trepador, onde são empregados varas de madeira, bambus, arame ou fitilho de
	polietileno. O sistema de rede de tutoramento artesanal é utilizado como suporte
	vertical para a haste do pepino. Esta pesquisa teve por objetivo avaliar o
	desempenho agronômico do pepino híbrido kataná em diferentes sistemas de
	condução: estaca de bambu, arame/fitilho e rede, os pepineiros foram plantados em
	vasos, utilizou-se o delineamento inteiramente casualizado (DIC), sendo realizado
	em ambiente protegido no Instituto Federal de Educação, Ciência e tecnologia de
	Rondônia – Campus Colorado do Oeste, localizado na BR 435 Km 63, zona rural no
	município Colorado do Oeste/RO, visando melhores resultados de produtividade
	através da condução da rama do pepino. Os dados foram coletados semanalmente,
	acompanhando o desenvolvimento da cultura, medindo a altura da haste com auxilio
	de fita métrica e contabilizando o número de frutos. Os resultados obtidos foram
	submetidos a análise de variância, utilizando o método Scott \& Knott. Verificou-se
	que os tratamentos se diferiram estatisticamente, sendo que a condução rede de
	tutoramento obteve maior produtividade por planta e as conduções estaca de bambu
	e fitilho/arame não houve diferença estatística entre os tratamentos, obtendo
	rendimento inferior a condução de rede, tendo-se o comprimento da haste do pepino
	como fator relevante na produção, possibilitando no sistema de rede maior
	crescimento. Conclui-se assim que a rede de tutoramento possibilita o melhor
	desenvolvimento da haste do pepino e consequentemente maior produção de frutos,
	aumentando a produtividade dessa olerícola e auxiliando os produtores na condução
	dos pepineiros.
	
	\vspace{\onelineskip}
	
	\noindent
	\textbf{Palavras-chave}: Condução. Pepino. Desempenho.
	
\end{document}
