\documentclass[article,12pt,onesidea,4paper,english,brazil]{abntex2}

\usepackage{lmodern, indentfirst, nomencl, color, graphicx, microtype, lipsum}			
\usepackage[T1]{fontenc}		
\usepackage[utf8]{inputenc}		

\setlrmarginsandblock{2cm}{2cm}{*}
\setulmarginsandblock{2cm}{2cm}{*}
\checkandfixthelayout

\setlength{\parindent}{1.3cm}
\setlength{\parskip}{0.2cm}

\SingleSpacing

\begin{document}
	
	\selectlanguage{brazil}
	
	\frenchspacing 
	
	\begin{center}
		\LARGE MANEJO E EXTRAÇÃO DO ÓLEO DA CASTANHA DO BRASIL\footnote{Ciências Agrárias com financiamento do IFRO.}
		
		\normalsize
		Aline Vieira da Silva\footnote{Aline Vieira da Silva, alinevieira548@gmail.com, Campus Ji-Paraná} 
		Andreza Pereira Mendonça\footnote{Andreza Pereira Mendonça, mendonca.andreza@gmail.com, Campus Ji-Paraná} 
		Maria Elessandra R. Araújo\footnote{Maria Elessandra R. Araújo, maria.elessandra@gmail.com, Campus Ji-Paraná} 
		Matheus Favaro
		Moreira\footnote{Matheus Favaro Moreira, favarom38@gmail.com, Campus Ji-Paraná} 
	\end{center}
	
	\noindent A amêndoa da castanha-do-Brasil é constituída de 60 a 70\% de lipídios e de 15 a
	20\% de proteína, além de vitaminas e minerais. O óleo típico apresenta 13,8\% de
	ácido palmítico, 8,7\% de ácido esteárico, 31,4\% de ácido oléico e 45,2\% de ácido
	linoléico, além de pequenas quantidades dos ácidos mirístico e palmitoléico. A
	característica físico-química do óleo é o fator limitante para seu uso necessitando,
	portanto, de procedimentos adequados de secagem e extração do óleo que
	assegurem sua qualidade. O objetivo do trabalho foi avaliar a qualidade do óleo
	extraível de castanha do Brasil. O presente trabalho foi conduzido no laboratório de
	sementes e viveiros do Instituto Federal de Rondônia - Campus, Ji-Paraná. Foram
	utilizadas sementes de Bertholletia excelsa. As castanhas foram coletadas de
	castanhais circunvizinhos a Ji-Paraná, RO. As sementes foram separadas em lotes
	de 1 kg e secas em estufa de ventilação forçada sob diferentes temperaturas (60,70
	e 80oC) e umidade (4, 6 e 8\%). Após a secagem, as amêndoas foram trituradas e
	prensadas em prensa hidráulica por 4 horas. A análise de acidez foi desenvolvida
	através de uma metodologia já existente. Os óleos de castanha extraídos tiveram o
	índice de acidez dentro do padrão estabelecido pela Resolução 270 de 2005 da
	ANVISA para óleos brutos prensados a frio (4 mg KOH g-1) independente da temperatura e umidade das amêndoas. Sendo que os menores valores foram identificados a 60$^{\circ}$C (0,51 a 0,53 mg KOH g-1). Notou-se ainda que os óleos
	extraídos mantiveram o mesmo padrão de cor, indicando que a temperatura não
	alterou as propriedades físicas do óleo e nem acidez.
	
	\vspace{\onelineskip}
	
	\noindent
	\textbf{Palavras-chave}: Bertholletia excelsa. Uso múltiplo. Óleos vegetais.
	
\end{document}
