\documentclass[article,12pt,onesidea,4paper,english,brazil]{abntex2}

\usepackage{lmodern, indentfirst, nomencl, color, graphicx, microtype, lipsum}			
\usepackage[T1]{fontenc}		
\usepackage[utf8]{inputenc}		

\setlrmarginsandblock{2cm}{2cm}{*}
\setulmarginsandblock{2cm}{2cm}{*}
\checkandfixthelayout

\setlength{\parindent}{1.3cm}
\setlength{\parskip}{0.2cm}

\SingleSpacing

\begin{document}
	
	\selectlanguage{brazil}
	
	\frenchspacing 
	
	\begin{center}
		\LARGE ASPECTOS MICROBIOLÓGICOS DE POLPAS DE CUPUAÇU E CAJÁ
		COMERCIALIZADAS NAS FEIRAS LIVRES DE PORTO VELHO -- RONDÔNIA\footnote{Trabalho realizado dentro da área de Ciências Biológicas}
		
		\normalsize
		Edjane Paz da Silva\footnote{Colaboradora, edjanejack@hotmail.com, Campus Porto Velho Calama} 
		Gabriel Henrique Abrantes Holanda\footnote{Colaborador, gabrielhenrique2802@gmail.com, Campus Porto Velho Calama} \\
		Karollayne Fernandes dos Santos\footnote{Colaboradora, karollaynefernandes16@gmail.com, Campus Porto Velho Calama} 
		Marcia Bay\footnote{Orientadora, márcia.bay@ifro.edu.br, Campus Porto Velho Calama} 
	\end{center}
	
	\noindent As constantes transformações tecnológicas no setor de alimentos visam atender a
	crescente demanda populacional e dentro desse contexto, ressalta-se a
	preocupação tanto por parte do consumidor quanto do poder público em garantir o
	controle de qualidade dos alimentos, em especial o controle referente às condições
	microbiológicas as quais os alimentos estão sujeitos. A microbiota de um alimento é
	fortemente influenciada pela composição química do produto em questão, tal como o
	teor de água, pois quanto maior for a atividade de água, melhores serão as
	condições de crescimento microbiano. Alimentos com alto de teor de água livre, tais
	como frutas e seus derivados “in natura”, como polpas, estão sujeitos a uma maior
	contaminação durante a etapa de processamento, seja por indicadores de
	patogenicidade como coliformes fecais ou por bactérias de alto risco, como a
	salmonella ssp. Nesse contexto o objetivo desse trabalho é avaliar a existência do
	problema de contaminação bacteriológica em polpas de Cajá e Cupuaçu
	comercializadas em duas distintas feiras da cidade de Porto Velho. O estudo foi
	realizado entre alunos e professores do curso técnico em química do Instituto
	Federal de Rondônia, Campus Porto Velho Calama, nas dependências do
	laboratório de microbiologia. A metodologia executada foi adaptada a partir de
	metodologia utilizada pela EMBRAPA, consistindo na avaliação qualitativa do
	crescimento microbiano em placas de petri por meio dos meios de cultura Caldo
	Verde Brilhante e Ágar nutriente, bem como também através do teste de
	fermentação em tubo de ensaio para a confirmação da presença de coliformes. Os
	testes realizados evidenciaram que das quatro amostras analisadas, 75\%
	apresentaram contaminação por coliformes e bactérias que indicam a possível
	presença de patógenos, e Salmonella ssp, bactéria patógena que representa
	potencial risco a saúde, porém em apenas uma única amostra foi confirmada a
	presença da Enterococcus faecalis, também patógena. Os resultados encontrados
	evidenciam o perigo que o consumidor está exposto ao consumir produtos sem
	certificação de qualidade reconhecida, além de que apontam para a necessidade de
	políticas públicas que instruam o micro e pequeno empreendedor quanto às boas
	práticas de produção, manipulação e armazenamento de alimentos.
	
	\vspace{\onelineskip}
	
	\noindent
	\textbf{Palavras-chave}: Avaliação microbiológica. Controle de qualidade. Polpas de fruta.
	
\end{document}
