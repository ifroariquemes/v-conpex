\documentclass[article,12pt,onesidea,4paper,english,brazil]{abntex2}

\usepackage{lmodern, indentfirst, nomencl, color, graphicx, microtype, lipsum}			
\usepackage[T1]{fontenc}		
\usepackage[utf8]{inputenc}		

\setlrmarginsandblock{2cm}{2cm}{*}
\setulmarginsandblock{2cm}{2cm}{*}
\checkandfixthelayout

\setlength{\parindent}{1.3cm}
\setlength{\parskip}{0.2cm}

\SingleSpacing

\begin{document}
	
	\selectlanguage{brazil}
	
	\frenchspacing 
	
	\begin{center}
		\LARGE COMPUTAÇÃO COGNITIVA NO DESENVOLVIMENTO MOBILE
		
		\normalsize
	Aline Monise Santos Lima\footnote{Estudante do curso Técnico em Informática, Campus Ji-Paraná.} 
		Nicolas Schíavon\footnote{Estudante do curso Técnico em Informática, Campus Ji-Paraná.} 
		Bruno Aparecido Neires de Oliveira\footnote{Estudante do curso Técnico em Informática, Campus Ji-Paraná.} 
		Ilma Rodrigues de Souza Fausto\footnote{Professor(a) Orientador.} 
	\end{center}
	
	\noindent
	Este trabalho foi realizado para a disciplina de Orientação para Pesquisa e Prática
	Profissional, no Campus Ji-Paraná. A computação cognitiva também chamada de
	Inteligência Artificial é uma tecnologia que permite o computador “pensar” como o
	ser humano, capacidade de interagir com o ser humano, até mesmo aprender e de
	identificar várias coisas. Uma das maiores capacidades deste sistema é aprender a
	linguagem natural, percebendo até mesmo as diferenças idiomáticas e sotaques.
	Sabendo da importância dessa tecnologia para o desenvolvimento e futuro da
	sociedade atual, foi feito um levantamento sobre a utilização desse sistema na
	cidade de Ji-Paraná, no âmbito das escolas, empresas e no dia a dia da população;
	Foi identificado o grau de utilização dessa tecnologia e os benefícios de ter esse
	sistema. Foi realizada uma pesquisa exploratória para ter mais familiaridade com o
	tema nas áreas de Ciência da Computação, Computação Cognitiva e Tecnologia
	Mobile. Através de questionário descritivo usando a plataforma de Formulários do
	Google, foi feita a análise estatística dos dados colhidos, pesquisa quantitativa,
	exploratória e de campo. Ao fazer a análise quantitativa dos dados, percebeu-se a
	pouca utilização dessa tecnologia na cidade de Ji-Paraná tanto no âmbito da
	educação e nas empresas. No dia a dia, essa tecnologia já tem sido mais utilizada
	nas redes sociais e nos afazeres. Em relação aos objetivos gerais do projeto, obteve
	sucesso em ambos os resultados da pesquisa e foi de suma importância a
	participação dos alunos da instituição na qual comprovou que os alunos da
	instituição tem pouquíssimo conhecimento sobre.
	
	\vspace{\onelineskip}
	
	\noindent
	\textbf{Palavras-chave}: Cognitiva. Mobile. Computação. \\
	\textbf{Fonte de financiamento}: Cnpq: Ciências exatas e da Terra.
	
\end{document}
