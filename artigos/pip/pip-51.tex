\documentclass[article,12pt,onesidea,4paper,english,brazil]{abntex2}

\usepackage{lmodern, indentfirst, nomencl, color, graphicx, microtype, lipsum}			
\usepackage[T1]{fontenc}		
\usepackage[utf8]{inputenc}		

\setlrmarginsandblock{2cm}{2cm}{*}
\setulmarginsandblock{2cm}{2cm}{*}
\checkandfixthelayout

\setlength{\parindent}{1.3cm}
\setlength{\parskip}{0.2cm}

\SingleSpacing

\begin{document}
	
	\selectlanguage{brazil}
	
	\frenchspacing 
	
	\begin{center}
		\LARGE EFEITO DE DIFERENTES SUBSTRATOS ORGÂNICOS SOB O CRESCIMENTO
		DO SISTEMA RADICULAR DE MUDAS DE COUVE\footnote{Trabalho realizado dentro da Grande Área: Agronomia com financiamento do IFRO.}
		
		\normalsize
		SILVEIRA.L.P.,\footnote{Bolsista Ensino Médio, lucaspereirasilveira18@gmail.com, Campus Colorado do Oeste.} 
	PRADO.L.S.,\footnote{Colaborador, lucas.agronomiaifro@gmail.com, Campus Colorado do Oeste.} 
		NUNES, J.D.K.,\footnote{Orientadora, jessica.krugel@ifro.edu.br, Campus Colorado do Oeste.} 
	 
	\end{center}
	
	\noindent O substrato para a produção de mudas tem por finalidade garantir o
	desenvolvimento de uma planta com qualidade, em curto período de tempo, e baixo
	custo. Objetivou-se avaliar o uso de compostos orgânicos na produção de mudas de
	couve-de-folha (\textit{Brassica oleracea var. acéfala}) é uma brássica típica de outonoinverno
	e apresenta certa tolerância ao calor podendo ser plantada ao longo de todo
	o ano, comparando-os a um substrato comercial (testemunha). Os substratos
	testados foram feitos com areia, farinha de osso, palha de arroz incinerada e
	vermicomposto em quatro combinações distintas acrescidos de adubação foliar via
	biofertilizante suíno. O vermicomposto foi obtido pela decomposição de esterco
	bovino e restos alimentares oriundos do refeitório da instituição, a farinha de osso foi
	adquirida em casas agropecuárias da região e as demais matérias primas para os
	outros substratos foram encontrados na área da própria instituição. Os parâmetros
	avaliados foram: comprimento da raiz (CR), altura de plantas (AP), massa seca da
	parte aérea (MSPA) e massa seca da raiz (MSR). Para todos os parâmetros
	analisados houve diferença significativa onde o composto orgânico apresentou
	resultados melhores ou similares ao substrato comercial (excetuando-se no
	parâmetro de comprimento da raiz que não diferiu), indicando a possibilidade de sua
	utilização no cultivo de hortaliças. O bom resultado se deve também a aplicação do
	biofertilizante, que, por ser um composto rico em nitrogênio, proporcionou um melhor
	desenvolvimento das mudas, conferindo-lhes o aspecto de folhas com tons verde
	intenso (plantas mais sadias). Para a formulação de um bom substrato é necessário
	a utilização de compostos ricos em nutrientes de maneira a proporcionar um
	balanceamento dos teores químicos e físicos requeridos para um bom
	desenvolvimento das mudas. A mistura de 70\% vermicomposto, 10\% areia, 10\%
	casca de arroz incinerada e 10\% de farinha de osso apresentou os melhores
	resultados com desenvolvimento igual ou superior ao desenvolvimento das plantas
	cultivadas com substrato comercial. Proporcionando mudas excelentes ao nível de
	aquisição de pequenos e grandes produtores e formulações de substratos de
	grandes potencias.
	
	\vspace{\onelineskip}
	
	\noindent
	\textbf{Palavras-chave}:Horticultura. Adubação orgânica. Substrato. \\
	\textbf{Fonte de financiamento}: Pró-Reitoria de Pesquisa, Inovação e Pós-Graduação
	(PROPESP) - IFRO.
	
\end{document}
