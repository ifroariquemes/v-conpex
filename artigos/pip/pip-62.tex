\documentclass[article,12pt,onesidea,4paper,english,brazil]{abntex2}

\usepackage{lmodern, indentfirst, nomencl, color, graphicx, microtype, lipsum}			
\usepackage[T1]{fontenc}		
\usepackage[utf8]{inputenc}		

\setlrmarginsandblock{2cm}{2cm}{*}
\setulmarginsandblock{2cm}{2cm}{*}
\checkandfixthelayout

\setlength{\parindent}{1.3cm}
\setlength{\parskip}{0.2cm}

\SingleSpacing

\begin{document}
	
	\selectlanguage{brazil}
	
	\frenchspacing 
	
	\begin{center}
		\LARGE EXTENSÃO RURAL A PRODUTORES DA APROCIS AVALIANDO A PROGÊNIE YOSHIMATSU COMO PORTA-ENXERTO AOS HÍBRIDOS DE TOMATEIROS SERTÃO F1, CARDYNA E FASCÍNIO CULTIVADOS EM ÁREA COM HISTÓRICO DE MURCHA BACTERIANA\footnote{Trabalho realizado dentro da Ciências Agrárias com financiamento do Cnpq.}
		
		\normalsize
		Jean Carlos da Silva Ribeiro,\footnote{Bolsista Pibic-af, jeanribeiro.ifro@gmail.com, Campus Colorado do Oeste.} 
	Jiovane Anderson da Silva Ribeiro,\footnote{Bolsita, jiovaneribeiro.ifro@gmail.com, Campus Colorado do Oeste.} 
		Marcos Aurélio Anequine de Macedo,\footnote{Orientador Marcos Aurélio Anequine de Macedo,marcos.anequine@ifro.edu.br, Campus Colorado do Oeste.} 
		Valdique Gilberto de Lima\footnote{Co-orientador Valdique Gilberto de Lima, valdique.lima@ifro.edu.br, Campus Colorado do Oeste.} 
	\end{center}
	
	\noindent A murcha bacteriana se apresenta como um fator limitante de produção no cultivo de oleráceas, na qual as plantas suscetíveis a doença apresentam mortalidade instantânea após a infecção com o inóculo. A progênie de tomate Yoshimatsu apresenta resistência genética à bactéria causadora da R. solanacearum sendo alternativa de cultivo utilizada como porta enxerto aos híbridos comerciais suscetíveis a doença. A técnica de enxertia consistiu na junção de tecidos de duas plantas diferentes, para propagação com a finalidade de se obter as características desejáveis de cada uma. O projeto foi desenvolvido na Associação dos Pequenos Produtores Rurais do Planalto Parecis, município de Vilhena. As mudas da progênie Yoshimatsu e dos híbridos comerciais foram produzidas em bandejas plásticas de 288 células com substrato comercial e posteriormente realizou-se a enxertia dos híbridos no porta-enxerto Yoshimatsu. A enxertia foi realizada aos 20-25 dias após o semeio utilizando a técnica de fenda cheia, após a realização da enxertia e do manejo de cicatrização da planta, a mudas foram transplantadas para o campo na propriedade de um produtor da APROCIS com histórico de ocorrência de murcha bacteriana. O delineamento experimental foi em blocos casualizados, com 10 tratamentos e 4 repetições e cada parcela experimental contendo 10 plantas. As avaliações foram realizadas em épocas adequadas em conformidade ao desenvolvimento da planta, onde foi analisado a resposta dos híbridos à técnica de enxertia e consequente taxa epidemiológica da doença. Os híbridos comerciais apresentaram elevada taxa de pegamento e cicatrização ao porta enxerto Yoshimatsu após o procedimento de enxertia, e bom desenvolvimento da planta após o transplantio à campo, a avaliação da taxa epidemiológica encontra-se em andamento.
	
	\vspace{\onelineskip}
	
	\noindent
	\textbf{Palavras-chave}: Resistência. Enxertia. Tomate.\\
	\textbf{Fonte financiadora}: Cnpq. 
\end{document}
