\documentclass[article,12pt,onesidea,4paper,english,brazil]{abntex2}

\usepackage{lmodern, indentfirst, nomencl, color, graphicx, microtype, lipsum}			
\usepackage[T1]{fontenc}		
\usepackage[utf8]{inputenc}		

\setlrmarginsandblock{2cm}{2cm}{*}
\setulmarginsandblock{2cm}{2cm}{*}
\checkandfixthelayout

\setlength{\parindent}{1.3cm}
\setlength{\parskip}{0.2cm}

\SingleSpacing

\begin{document}
	
	\selectlanguage{brazil}
	
	\frenchspacing 
	
	\begin{center}
		\LARGE A MUSICOTERAPIA NO TRATAMENTO DE DOENÇAS MENTAIS: DA GÊNESE AO\\TRATAMENTO PSICOTERÁPICO\footnote{Trabalho realizado dentro da área de Conhecimento CNPq: Música, com financiamento do(a) GP -- PDA/IFRO.}
		
		\normalsize
		Rhélrison Bragança Carneiro\footnote{Pesquisador afiliado ao GP -- PDA/IFRO, rhelrisonibn@gmail.com, Campus Cacoal - IFRO} 
		Sérgio Nunes de Jesus\footnote{Orientador(a), sergio30canibal@gmail.com, Campus Cacoal - IFRO.} 
	
	\end{center}
	
	\noindent A relação entre música e cura perpetua desde as primeiras civilizações no planeta, tal era utilizada como um meio de comunicação religiosa, a fim de que, com essa, o indivíduo atraísse para si a cura, como também, em alguns contextos, tal era utilizada como linguagem emotiva. Assim sendo, é incontestável que a música, desde as eras mais remotas, vem sendo utilizada como ferramenta para obtenção de cura para doenças, mesmo à luz dos preceitos não-científicos. 
	Desde as civilizações egípcias e gregas a música já possuía caráter terapêutico, uma vez que, relacionava-se a obtenção de cura aos enfermos e, até mesmo, favores divinos.
	Somente no século XVII, após o rompimento dos paradigmas possibilitados pelo Renascimento Cultural, surge uma atitude racional a face do tratamento psiquiátrico por meio da música, começando, então, os estudos relacionados a esse, ainda não reconhecido, campo da ciência, dessa forma, estabelecendo-se um paralelo entre música e terapia. 
	Foi apenas durante a Primeira Guerra Mundial, no contexto pós- traumatológico dos hospitais americanos, que é atribuída a música o caráter de método terapêutico pelos estudos que assim seguiram-se e foram constatados como verdadeiros. 
	A pesquisa teve como objetivos traçar um paralelo entre o histórico dos usos terapêuticos da música no cuidado à saúde mental até o surgimento da musicoterapia como método terapêutico e ciência sistemática. Para a realização deste trabalho, foram realizadas revisões bibliográficas, pesquisa em artigos relacionados ao tema e fundamentação teórica segundo a visão histórica da musicoterapia, que se objetou na busca de paralelos históricos entre música e terapia até o surgimento da musicoterapia como ciência. 
	Os dados demonstram a relação histórica, entre a história da música e a terapia musical até o surgimento da musicoterapia que constitui um dos principais métodos de tratamento terapêutico da atualidade, sendo reconhecida como linguagem comunicativa, ligada a área afetivo-emocional do ser humano, em meio ao musicoterapeuta e o paciente.
	
	\vspace{\onelineskip}
	
	\noindent
	\textbf{Palavras-chave}: História. Música. Terapia.
	
\end{document}
