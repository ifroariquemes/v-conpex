\documentclass[article,12pt,onesidea,4paper,english,brazil]{abntex2}

\usepackage{lmodern, indentfirst, nomencl, color, graphicx, microtype, lipsum}			
\usepackage[T1]{fontenc}		
\usepackage[utf8]{inputenc}		

\setlrmarginsandblock{2cm}{2cm}{*}
\setulmarginsandblock{2cm}{2cm}{*}
\checkandfixthelayout

\setlength{\parindent}{1.3cm}
\setlength{\parskip}{0.2cm}

\SingleSpacing

\begin{document}
	
	\selectlanguage{brazil}
	
	\frenchspacing 
	
	\begin{center}
		\LARGE DESENVOLVIMENTO DE UM SISTEMA COMPUTACIONAL PARA
		MONITORAMENTO E DENÚNCIA DE CRIMES AMBIENTAIS.\footnote{Trabalho realizado dentro da Ciência da Computação.}
		
		\normalsize
		Arthur Bruno P. Pires\footnote{Colaborador, espertaobruno98@gmail.com, Campus Guajará-Mirim.} 
	Lucas L. B. Firmino\footnote{Colaborador, firminononfim@gmail.com, Campus Guajará-Mirim.} 
		Ronald Vitor C. Medine\footnote{Colaborador, ronaldvitor90@gmail.com, Campus Guajará-Mirim.} 
		Erick Rian de F. Pimentel\footnote{Colaborador, nicepimentel2013@hotmail.com, Campus Guajará-Mirim.} 
		Vitor M. da Silva \footnote{Colaborador, moquedace2015@gmail.com, Campus Guajará-Mirim.}
		Thalisson da C. Araujo\footnote{Colaborador, thallisson.araujo@hotmail.com, Campus Guajará-Mirim.}
		Danilo S. Souza\footnote{Orientador(a), danilo.souza@ifro.edu.br, Campus Guajará-Mirim.}
		Graziela T. Tejas\footnote{Co-orientador(a), graziela.tejas@ifro.edu.br, Campus Guajará-Mirim.}
		
		
		
	\end{center}
	
	\noindent O processo de industrialização inserido no modelo capitalista aproveita os
	recursos da natureza de maneira imediata considerando assim uma degradação
	ao meio ambiente, entendendo como a perda da qualidade de vida, ocasionada
	então por uma ação antrópica. A explosão demográfica e o uso dos elementos
	naturais de modo incontrolável trouxeram uma preocupação já no pós-guerra de
	modo a atentar-se que esses recursos eram e são finitos, diante disso muitos
	estudiosos, pesquisadores, ecologistas e ambientalistas começaram a
	desenvolver diretrizes e propostas ao tratamento de questões ambientais. O
	gerenciamento do crimes ambientais no Brasil é de responsabilidade dos órgãos
	ambientais integrantes do Sistema Nacional de Meio Ambiente - SISNAMA
	instituído pela Lei nº 9.605, de 12 de fevereiro de 1998. Considerando o quadro
	de gravidade dos eventos e de suas consequências, torna-se vital para o
	planejamento de ações a estruturação de sistemas de informação e vigilância
	sobre crimes ambientas. Assim, o seguinte projeto tem como principal foco o
	desenvolvimento de um sistema computacional para auxiliar a população de
	Guajará-Mirim e os órgãos governamentais e não governamentais no combate
	aos crimes ambientais na região Amazônica. Onde propõe-se o desenvolvimento
	de uma aplicação móvel integrada com uma plataforma web para realização de
	denúncias de crimes ambientais a partir de dispositivos móveis. Pretende-se
	identificar os órgãos ambientais e organizações não governamentais presentes
	na área de estudo para propor a utilização dos sistema em fase experimental. Ao
	final do projeto espera-se obter dados gerados em período de tempo para
	elaboração de mapas temáticos e relatórios de modo apresentar a situação da
	região sudoeste da amazônia legal.
	
	\vspace{\onelineskip}
	
	\noindent
	\textbf{Palavras-chave}: Sistemas de Informação Geográficos, Crimes Ambientas,
	Dispositivos Móveis.
	
\end{document}
