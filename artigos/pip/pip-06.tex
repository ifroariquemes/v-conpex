\documentclass[article,12pt,onesidea,4paper,english,brazil]{abntex2}

\usepackage{lmodern, indentfirst, nomencl, color, graphicx, microtype, lipsum}			
\usepackage[T1]{fontenc}		
\usepackage[utf8]{inputenc}		

\setlrmarginsandblock{2cm}{2cm}{*}
\setulmarginsandblock{2cm}{2cm}{*}
\checkandfixthelayout

\setlength{\parindent}{1.3cm}
\setlength{\parskip}{0.2cm}

\SingleSpacing

\begin{document}
	
	\selectlanguage{brazil}
	
	\frenchspacing 
	
	\begin{center}
		\LARGE ACCOUNTABILITY GOVERNAMENTAL: APLICABILIDADE DAS LEIS E O CONTROLE SOCIAL PELA COMUNIDADE DO MUNICÍPIO DE\\PORTO VELHO -- RO\footnote{Trabalho realizado dentro da área de Conhecimento CNPq: 60200006 ADMINISTRAÇÃO.}
		
		\normalsize
		Paulo Elias dos Santos\footnote{Bolsista (modalidade de Ensino Superior), paulo.santos@ifro.edu.br, Campus Porto Velho Zona Norte} 
		Artur Virgílio Simpson Martins\footnote{Colaborador(a), artur.martins@ifro.edu.br, Campus Porto Velho Zona Norte} 
		Rwrsilany Silva\footnote{Orientador(a), rwrsilany.silva, Campus Porto Velho Zona Norte} 
		Gustavo Melazi Girardi\footnote{Co-orientador(a), gustavo.girardi@ifro.edu.br, Campus Guajará Mirim} 
	\end{center}
	
	\noindent A transparência tem sido eminentemente discutida como necessária para propiciar um acompanhamento dos gastos públicos por parte dos cidadãos. Este direito tem sido assegurado por meio de leis que preveem a participação da comunidade na gestão das políticas públicas por meio do controle social. Este estudo objetiva analisar as Leis que asseguram a participação da comunidade na gestão das políticas públicas no âmbito do município de Porto Velho e verificar a praticidade do portal de transparência. A Lei Complementar Federal nº 131/2009 determinou a liberação ao pleno conhecimento e acompanhamento da sociedade, em tempo real, de informações pormenorizadas sobre a execução orçamentária e financeira, em meios eletrônicos de acesso ao público. No entanto, não é suficiente apenas disponibilizar as informações sobre as receitas e as despesas públicas se os dados não tiverem linguagem clara, objetiva, direta, transparente e acessível para que o cidadão possa compreender e assim exercer o controle social. O estudo justifica-se pela contribuição aos cidadãos que poderão utilizar os dados que serão apresentados para conhecer a legislação, induzindo a administração à transparência da execução orçamentária e financeira. Para tanto, realizou-se uma pesquisa de natureza qualitativa, com objetivo principal de analisar minuciosamente o portal da transparência do município de Porto Velho – RO. A posteriori, foram entrevistadas pessoas que utilizam o portal de transparência e assim mensurou a percepção destas quanto à disponibilização das informações administradas e custodiadas pela Prefeitura no Portal da Transparência. Os resultados indicam que apesar do Município divulgar na internet as informações da gestão pública, estas não são franqueadas mediante procedimentos objetivos e ágeis, de forma transparente e em linguagem de fácil compreensão. Argumentamos assim que no Portal da gestão pública, as informações não são franqueadas mediante procedimentos objetivos e ágeis, de forma transparente e em linguagem de fácil compreensão. Logo, o Portal da Transparência como instrumento de transparência, accountability e controle social não é eficaz, vez que o cidadão comum não consegue compreender o que é informado e por sua vez não acompanha de forma concomitante a execução do orçamento público e, consequentemente não consegue exercer o controle social.
	
	\vspace{\onelineskip}
	
	\noindent
	\textbf{Palavras-chave}: Portal da Transparência. Accountability. Controle Social.
	
\end{document}
