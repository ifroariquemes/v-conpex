\documentclass[article,12pt,onesidea,4paper,english,brazil]{abntex2}

\usepackage{lmodern, indentfirst, nomencl, color, graphicx, microtype, lipsum,}			
\usepackage[T1]{fontenc}		
\usepackage[utf8]{inputenc}		

\setlrmarginsandblock{2cm}{2cm}{*}
\setulmarginsandblock{2cm}{2cm}{*}
\checkandfixthelayout

\setlength{\parindent}{1.3cm}
\setlength{\parskip}{0.2cm}

\SingleSpacing

\begin{document}
	
	\selectlanguage{brazil}
	
	\frenchspacing 
	
	\begin{center}
		\LARGE AVALIAÇÃO DO PNAES NO IFRO CAMPUS PORTO VELHO CALAMA ENTRE OS ANOS DE 2013 A 2016\footnote{Trabalho realizado dentro da área de Conhecimento CNPq: Ciências Sociais Aplicadas com financiamento do Instituto Federal de Rondônia IFRO}
		
		\normalsize
		Maria Rita Medeiros Flôr\footnote{Bolsista PIBIC-EM, maritamflor@gmail.com, Campus Porto Velho Calama} 
		Thiago Soares Teixeira\footnote{Colaborador, thiagosuarezi@gmail.com, Campus Porto Velho Calama} 
		Rodrigo Lopes da Silva\footnote{Colaborador, rodrigoslopes.sl@gmail.com, Campus Porto Velho Zona Norte} 
		Thiago Pacife de Lima\footnote{Orientador, thiago.lima@ifro.edu.br, Campus Porto Velho Calama} 
	\end{center}
	
	\noindent A Educação é um dos mais importantes direitos de uma sociedade democrática e tem como suporte as legislações, que estabelecem como responsabilidade do Estado a garantia de igualdade e a responsabilidade de intervir e combater desigualdades. A prescrição legal da educação como um direito a coloca como ponto prioritário nas políticas públicas, que por seu caráter obrigatório a diferencia de outros direitos sociais. A ampliação da Rede Federal de Educação Profissional, juntamente com a ampliação no ensino superior federal, por meio do Programa de Reestruturação e Expansão das Universidades Federais (Reuni), tiveram como um de seus objetivos ampliar as oportunidades de acesso à educação (técnica e superior), que somadas às recentes políticas de democratização no acesso ao ensino público federal, trouxeram à tona a questão da assistência estudantil, tirando-a da posição secundária que ocupava nas instituições públicas. Nesse contexto, este estudo buscou avaliar a política de Assistência Estudantil desenvolvida no âmbito do IFRO, Campus Porto Velho Calama entre os anos de 2013 a 2016. Os principais objetivos foram caracterizar a eficiência na execução orçamentária dos recursos do Programa Nacional de Assistência Estudantil – PNAES e qual sua efetividade no percentual de conclusão dos cursos pelos estudantes beneficiados. Os dados foram coletados a partir de pesquisa em documentos institucionais como editais e relatórios gerenciais publicados ou fornecidos pelos setores competentes. Para caracterização da efetividade foi verificado o status de matrícula para aproximadamente 1467 estudantes matriculados no período. Os resultados evidenciaram que, no período, houve melhoria na eficiência da execução orçamentária, passando de 92\% em 2013 para 99\% em 2016, neste mesmo período houve crescimento no orçamento anual, com exceção de 2016 quando foi registrada uma redução de 27,86\%, o que influenciou no número de estudantes atendidos que reduziu de aproximadamente 1400 alunos atendidos por ano para 910. Dentre os matriculados no período 210 concluíram os cursos e destes 30\% foram beneficiados, 762 estudantes estão em curso e 495 foram desligados da Instituição, destes 17\% haviam sido beneficiados pela assistência estudantil. A partir dos resultados preliminares foi possível verificar que a implementação do PNAES no Campus tem se aprimorado no decorrer dos anos.
	
	\vspace{\onelineskip}
	
	\noindent
	\textbf{Palavras-chave}: Avaliação. Política pública. Assistência estudantil. \\
	\textbf{Fonte de Financiamento}: Instituto Federal de Rondônia – IFRO, Edital n$^\circ$ 38/2016/PROPESP/IFRO.
	
\end{document}
