\documentclass[article,12pt,onesidea,4paper,english,brazil]{abntex2}

\usepackage{lmodern, indentfirst, nomencl, color, graphicx, microtype, lipsum}			
\usepackage[T1]{fontenc}		
\usepackage[utf8]{inputenc}		

\setlrmarginsandblock{2cm}{2cm}{*}
\setulmarginsandblock{2cm}{2cm}{*}
\checkandfixthelayout

\setlength{\parindent}{1.3cm}
\setlength{\parskip}{0.2cm}

\SingleSpacing

\begin{document}
	
	\selectlanguage{brazil}
	
	\frenchspacing 
	
	\begin{center}
		\LARGE JOBMARKET: FACILITANDO A CONTRATAÇÃO DE SERVIÇOS1\footnote{Trabalho realizado dentro das seguintes áreas de conhecimento: 1.03.00.00-7 Ciência da
			Computação e 6.02.01.00-2 Administração de Empresas - com financiamento do IFRO -
			Campus Cacoal.}
		
		\normalsize
		Pedro Henrique L. Albuquerque\footnote{Bolsista: Pedro Henrique L. Albuquerque, pedro.albuqu3rque@gmail.com, Campus Cacoal.} 
		Maria Eduarda A. de Moura\footnote{Colaboradora: Maria Eduarda A. de Moura, eduardaaragao18@gmail.com, Campus Cacoal.} 
		Juliano Cristhian Silva\footnote{Orientador: Juliano Christian, juliano@ifro.edu.br, Campus Cacoal.}
		Thiago José S. Kaiser\footnote{Coorientador: Thiago Kaiser, thiago.kaiser@ifro.edu.br, Campus Cacoal.}
		Saiane Barros de Souza\footnote{Coorientadora: Saiane Barros de Souza, saiane.souza@ifro.edu.br, Campus Cacoal.}
		 
	\end{center}
	
	\noindent Com o propósito de contribuir na melhoria da atual conjuntura socioeconômica do
	país, é desenvolvido no campus Cacoal, com o auxílio de docentes, o projeto com
	caráter tecnológico que visa o desenvolvimento de uma aplicação que possa ser
	utilizada por usuários e que promova a contratação de serviços autônomos. O
	sistema servirá como uma via de mão dupla, que auxiliará tanto o prestador de
	serviço que deseja aumentar sua carteira de clientes quanto o contratante que,
	devido as inserções tecnológicas em nossa sociedade, está sempre à procura de
	uma maneira mais eficiente de atendimento de suas necessidades. Trabalhadores
	que exercem funções sem qualquer vínculo empregatício poderão utilizar a
	ferramenta para anunciar seus serviços, possibilitando a ampliação de sua carteira
	de clientes. Sendo assim, a proposta ora apresentada, permite a recolocação ao
	mercado de empreendedores, que possam atender as demandas vindas de uma
	postura diferenciada de clientes aos quais há a necessidade por serviços de
	qualidade e ao mesmo tempo da confiabilidade oferecida pelo contratado, pois a
	solução permitirá, além da disponibilização dos serviços, uma avaliação feita pelos
	contratantes anteriores, possibilitando maior garantia no relacionamento entre
	contratante e contratado. Para que seja desenvolvido o aplicativo, além de da vasta
	pesquisa bibliográfica e toda a documentação necessária para a construção da
	solução, também será utilizada ferramenta de desenvolvimento integrado. O
	armazenamento de dados dos usuários será feito a partir da ferramenta MySQL.
	Após tais processos, o aplicativo será disponibilizado à sociedade por meio de um
	plano de marketing que destaque a relevância de sua utilização. Com o
	desenvolvimento dessa aplicação espera-se ampliar a formação do acadêmico, com
	foco na integração entre ensino-pesquisa-extensão, bem como fomentar a
	contratação de prestadores de serviço, o que facilitará o trâmite entre os dois atores
	principais desse mercado o que proporcionará aumento na demanda e oferta de
	prestação de serviço.
	
	\vspace{\onelineskip}
	
	\noindent
	\textbf{Palavras-chave}: Contratação de serviços; Contratação via Internet; Aplicativo.
	
	\noindent
	\textbf{Financiamento}: IFRO – Campus Cacoal.
	
\end{document}
