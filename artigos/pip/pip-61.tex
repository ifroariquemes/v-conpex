\documentclass[article,12pt,onesidea,4paper,english,brazil]{abntex2}

\usepackage{lmodern, indentfirst, nomencl, color, graphicx, microtype, lipsum}			
\usepackage[T1]{fontenc}		
\usepackage[utf8]{inputenc}		

\setlrmarginsandblock{2cm}{2cm}{*}
\setulmarginsandblock{2cm}{2cm}{*}
\checkandfixthelayout

\setlength{\parindent}{1.3cm}
\setlength{\parskip}{0.2cm}

\SingleSpacing

\begin{document}
	
	\selectlanguage{brazil}
	
	\frenchspacing 
	
	\begin{center}
		\LARGE ÉTICA NO SERVIÇO PÚBLICO FEDERAL: CONTEXTO E REFLEXÕES\footnote{Trabalho realizado dentro da área de Conhecimento CNPq: Administração Pública e de Empresas, Ciências Contábeis e Turismo.}
		
		\normalsize
		Kelly Cristiane Catafesta\footnote{Aluna do curso de graduação de Tecnologia em Gestão Pública, kelly.catafesta@gmail.com, Campus Porto Velho Zona Norte.} 
		
	\end{center}
	
	\noindent A ética provoca discussões conceituais tendo em filósofos consagrados como Sócrates e Aristóteles seu berço filosófico. A ética deriva do consciente e está intrinsecamente relacionada com a moral e os traços de caráter do indivíduo. As práticas morais aceitas pelo grupo social fazem da ética o conjunto de princípios da nossa vida em sociedade. As classes profissionais elaboram o código de ética que contêm os padrões morais observáveis e regulam as relações entre os membros dessa categoria, bem como com a sociedade. No serviço público federal o Decreto n$^\circ$ 1.171/94 institui os parâmetros que devem observar os servidores. Porém perdura encravada no senso comum a visão da ausência de moral e ética no serviço público. Este estudo delimita-se a pesquisar os princípios de condutas éticas, contextualizar o código de ética do Decreto n$^\circ$ 1.171/94 e instigar hipóteses que refletem na desconfiança junto à sociedade. Utilizou-se como metodologia a pesquisa bibliográfica, onde autores propõem um debate sobre ética e moral, a importância da elaboração de códigos de ética para as categorias profissionais e dos conceitos de ética dos servidores federais. Pode-se inferir que a ética é parte integrante da convivência em sociedade e tem como finalidade regular as relações humanas e profissionais. A ética não existe sem as regras morais originadas do consciente e caráter dos indivíduos. O Decreto n$^\circ$ 1.171/94 orienta e institui os parâmetros de ética observados pelos servidores federais. Ele instiga a confiança da sociedade nas instituições públicas uma vez que um servidor materializa em si a representação da instituição que trabalha. Concluiu-se que o Decreto n$^\circ$ 1.171/94 abrange questões relativas ao comportamento, conceitos e deveres que os servidores devem observar. A questão latente é como garantir comprometimento ético perante a enraizada cultura brasileira do “jeitinho” e das pequenas corrupções que por vezes não são percebidas como erradas, trazendo a ideia de quem não faz não é esperto. No âmbito do serviço público o trabalho pedagógico de disseminação do código de ética, com vistas a ampliar sua compreensão, instituí-lo na cultura organizacional e culminar com o julgamento ético na ação do servidor, são ações a serem trilhadas incessantemente.
	
	\vspace{\onelineskip}
	
	\noindent
	\textbf{Palavras-chave}: Ética. Moral. Decreto n$^\circ$ 1.171/94.
	
\end{document}
