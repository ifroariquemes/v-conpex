\documentclass[article,12pt,onesidea,4paper,english,brazil]{abntex2}

\usepackage{lmodern, indentfirst, nomencl, color, graphicx, microtype, lipsum}			
\usepackage[T1]{fontenc}		
\usepackage[utf8]{inputenc}		

\setlrmarginsandblock{2cm}{2cm}{*}
\setulmarginsandblock{2cm}{2cm}{*}
\checkandfixthelayout

\setlength{\parindent}{1.3cm}
\setlength{\parskip}{0.2cm}

\SingleSpacing

\begin{document}
	
	\selectlanguage{brazil}
	
	\frenchspacing 
	
	\begin{center}
		\LARGE MÚSICA E LINGUAGEM: FERRAMENTA COMO MEIO DE INTERAÇÃO SOCIAL\footnote{Trabalho realizado dentro da área de Conhecimento CNPq: Música, com financiamento do(a) GP-
			PDA/IFRO.}
		
		\normalsize
		Rhélrison Bragança Carneiro\footnote{Pesquisador afiliado ao GP-PDA/IFRO, rhelrisonibn@gmail.com, Campus Cacoal - IFRO} 
		Sérgio Nunes de Jesus\footnote{Orientador(a), sergio30canibal@gmail.com, Campus Cacoal - IFRO} 
		 
	\end{center}
	
	\noindent A música constitui-se de um emaranhado de estruturas que, quando combinadas de
	diferentes formas, resultam na chamada “linguagem musical”. Desde os primórdios o
	homem manipula os elementos que dispunha, interagindo, dessa forma, com o
	ambiente em que vive, por consequência, cria a linguagem como representação
	abstrata do seu interior. As terapias utilizam da linguagem como ferramenta
	elementar no processo de desconstrução de situações conflituosas cujas
	empobrecem a representação interna do indivíduo, tal processo só é possível pela
	mediação estabelecida por um elemento intermediador, nesse caso, a linguagem.
	Assim sendo, as perspectivas de estudo da linguagem são cada vez mais aguçadas
	pelos terapeutas, objetando-se na interpretação minuciosa do que está vinculado ao
	paciente. A partir dessa importância dada à linguagem no contexto terapêutico, é
	imposto à ciência a responsabilidade de atribuir à música caráter de procedimento
	terapêutico, assim sendo, caracteriza-la como linguagem, no que se diz ao contexto
	da musicoterapia. A pesquisa teve como objetivo buscar a fundamentação da
	linguagem musical no âmbito da musicoterapia, ou seja, as características que
	conferem à musical o papel de linguagem, de forma a ser utilizada como meio de
	intervenção terapêutica e quais as resoluções desse método para a psiquiatria. Para
	a realização deste trabalho, foram realizadas revisões bibliográficas, pesquisa em
	artigos relacionados ao tema e fundamentação teórica segundo a visão da
	psicanálise, objetando-se na fundamentação sistemática da linguagem musical no
	contexto musicoterápico e sua utilização como elemento interativo. Por meio dos
	dados obtidos, pode-se observar que, a música revela-se como linguagem
	estruturada, permitindo, assim, a atribuição de conotações amplas, porém não
	infinitas, ligadas à área afetivo-emocional do ser humano, dessarte, constituindo-se
	como ferramenta fundamental na intermediação terapêutica e, consequentemente,
	possibilitando interação e comunicação, sendo suas resoluções para o campo da
	psiquiatria de extrema riqueza, uma vez que, essa estabelece contato direto com o
	inconsciente, expressando-se pelo pré-consciente, o que possibilita a interação no
	mais alto grau de expressividade comunicativa.
	
	\vspace{\onelineskip}
	
	\noindent
	\textbf{Palavras-chave}: Comunicação. Linguagem. Musicoterapia.
	
\end{document}
