\documentclass[article,12pt,onesidea,4paper,english,brazil]{abntex2}

\usepackage{lmodern, indentfirst, nomencl, color, graphicx, microtype, lipsum, textcomp}			
\usepackage[T1]{fontenc}		
\usepackage[utf8]{inputenc}		

\setlrmarginsandblock{2cm}{2cm}{*}
\setulmarginsandblock{2cm}{2cm}{*}
\checkandfixthelayout

\setlength{\parindent}{1.3cm}
\setlength{\parskip}{0.2cm}

\SingleSpacing

\begin{document}
	
	\selectlanguage{brazil}
	
	\frenchspacing 
	
	\begin{center}
		\LARGE DESENVOLVIMENTO DE MUDAS DE MARACUJAZEIRO – AMARELO CULTIVADAS EM DIFERENTES RECIPIENTES E INTENSIDADES LUMINOSAS\footnote{Trabalho realizado dentro da Grande Área: Agronomia com financiamento do IFRO.}
		
		\normalsize
	BARBOSA, D.R.N.\footnote{Autor principal, dadivarenata@gmail.com, Campus Colorado do Oeste.} 
		BORGES, B.R.S.\footnote{Colaborador, brunoborges0615@gmail.com, Campus Colorado do Oeste.} 
		DORES, F.J.\footnote{Orientador, fabio.jose@ifro.edu.br, Campus Colorado do Oeste.} 
		
	\end{center}
	
	\noindent A cultura do maracujá vem apresentando nos últimos anos uma grande expansão no Brasil, o sucesso do seu cultivo está em implantá-la com mudas de qualidade e em condições adequadas; informações estas que devem ser difundidas para os produtores objetivando alavancar a fruticultura no Estado. Objetivou-se avaliar a interação exercida pela utilização de diferentes recipientes na formação de mudas de maracujazeiro, formadas em ambientes com diferentes índices de luminosidade. O experimento foi conduzido, no Instituto Federal de Rondônia - Campus Colorado do Oeste. Utilizou-se o delineamento inteiramente casualizado, em esquema de parcelas subdivididas, (4 ambientes de sombreamento (A1 (0\% (pleno sol)); A2 (30\%); A3 (50\%) e A4 (80\%)) x 3 tipos de recipientes (R1 (saco de polietileno), R2 (tubete de polietileno) e R3 (copos plásticos descartáveis), com 5 repetições (plantas), sendo as parcelas principais os ambientes sombreados e as subparcelas os recipientes. Semeou-se três sementes por recipiente e realizou-se um desbaste aos 15 dias após a semedura (DAS). Aos 40 DAS avaliou-se a altura das plantas (realizada com uma régua graduada em centímetros, medindo-se a distância entre o colo e a gema apical), diâmetro do colo (foi utilizado um paquímetro digital com valores expresso em mm, medido a 2 cm do colo da planta) e número de folhas (considerando-se a contagem da folha mais basal até a última aberta e os resultados mensurados em unidades (unid.)). As mudas de maracujazeiro cultivadas em tubetes (R2) e acondicionadas em ambiente com 30\% de sombreamento (A2) apresentaram altura, diâmetro do colo e número de folhas superior ao das mudas cultivadas nos demais recipientes e ambientes. O melhor desenvolvimento das mudas cultivadas em tubetes se deve ao fato de que as sementes ficam envoltas por uma quantidade maior de substrato, proporcionando melhor germinação e desenvolvimento. O sombreamento de 30\% se apresentou ideal em virtude de que o maracujá é uma cultura muito sensível a luminosidade e o sombrite nessa proporção não permite excesso de sombra que retarda o processo fotossintético da planta e nem excesso de radiação solar que possa vir a causar queimadura nas folhas jovens.
	
	\vspace{\onelineskip}
	
	\noindent
	\textbf{Palavras-chave}: \textit{Passiflora edulis}. Sombreamento. Recipiente. \\
	\textbf{Fonte de financiamento}: Pró-Reitoria de Pesquisa, Inovação e Pós-Graduação (PROPESP) - IFRO.
	
\end{document}
