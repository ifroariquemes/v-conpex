\documentclass[article,12pt,onesidea,4paper,english,brazil]{abntex2}

\usepackage{lmodern, indentfirst, nomencl, color, graphicx, microtype, lipsum,textcomp}			
\usepackage[T1]{fontenc}		
\usepackage[utf8]{inputenc}		

\setlrmarginsandblock{2cm}{2cm}{*}
\setulmarginsandblock{2cm}{2cm}{*}
\checkandfixthelayout

\setlength{\parindent}{1.3cm}
\setlength{\parskip}{0.2cm}

\SingleSpacing

\begin{document}
	
	\selectlanguage{brazil}
	
	\frenchspacing 
	
	\begin{center}
		\LARGE AVALIAÇÃO DE TEOR DE NITRATO EM CULTIVARES DE ALFACE \MakeUppercase{(Lactuca sativa L.)}\\NO SISTEMA HIDROPÔNICO NFT,\\NO CONE SUL DE RONDÔNIA.\footnote{Trabalho realizado dentro da área de Ciências Agrárias com financiamento do CNPq.}
		
		\normalsize
	Défner Junior Mielke\footnote{Défner Junior Mielke (PIBIC), junior.agronomia2014@gmail.com, Campus Colorado do Oeste.} 
		Rodrigo de Aguiar Gonçalves\footnote{Rodrigo de Aguiar Gonçalves, ro.aguiiar688@gmail.com, Campus Colorado do Oeste.} 
		Marcos Aurélio Anequine de Macedo\footnote{Marcos Aurélio Anequine de Macedo, marcos.anequine@ifro.edu.br, Campus Colorado do Oeste.} 
	\end{center}
	
	\noindent O cultivo de alface em sistemas hidropônico, principalmente o NFT (nutrient film technique), tem – se cada vez aumentado mais uso dessa técnica no Brasil, devido ao fato de proporciona renda em um curto período tempo, pois as culturas utilizadas nesse sistema serem de ciclo curto (alface, rúcula) e fácil manejo. Entretanto, esta técnica possibilita um alto acúmulo de nitrato na região foliar, sendo prejudicial à saúde quando consumida em excesso. O horário de coleta da matéria fresca foliar merece atenção, pois plantas colhidas nos períodos diurnos apresentam maiores teores de nitrato que as colhidas ao entardecer. O trabalho objetivou-se em avaliar a concentração de nitrato na matéria fresca foliar de duas cultivares de alface (Robusta e Pira Verde) do grupo lisas em sistema hidropônico NFT (Técnica de Filme de Nutrientes), com solução nutritiva proposta por HOAGLAND \& ARNON (1950); nas condições edafoclimáticas do Cone sul de Rondônia. O delineamento utilizado foi inteiramente casualizado (DIC) em esquema fatorial 2x4 correspondendo a 2 cultivares de alface em 4 períodos de colheitas diferentes) com quatro repetições, totalizando 32 parcelas. As parcelas foram compostas por seis plantas com espaçamento de 0,25m x 0,25m entre plantas e entre linhas medindo 1,5 metros de comprimento. A colheita foi realizada manualmente em quatro horários diferentes, sendo as 06h00min, 12h:00min, 18h:00min, e 24h:00min, em seguida as plantas foram encaminhadas para o laboratório de química do campus para as devidas avaliações. Para os parâmetros avaliados, pode-se observar que, não houve interação entre cultivares x horários de colheita, os resultados não apresentaram diferença estatisticamente para concentração de nitrato nas folhas, entretanto, a coleta realizada nos períodos de 00:00 horas para a cultivar Robusta e 06:00 horas para a cultivar Pira Verde apresentaram maiores teores.
	
	\vspace{\onelineskip}
	
	\noindent
	\textbf{Palavras-chave}: Hidropônia. Solução nutritiva. Horários de colheita.
	
\end{document}
