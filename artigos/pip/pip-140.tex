\documentclass[article,12pt,onesidea,4paper,english,brazil]{abntex2}

\usepackage{lmodern, indentfirst, nomencl, color, graphicx, microtype, lipsum,textcomp}			
\usepackage[T1]{fontenc}		
\usepackage[utf8]{inputenc}		

\setlrmarginsandblock{2cm}{2cm}{*}
\setulmarginsandblock{2cm}{2cm}{*}
\checkandfixthelayout

\setlength{\parindent}{1.3cm}
\setlength{\parskip}{0.2cm}

\SingleSpacing

\begin{document}
	
	\selectlanguage{brazil}
	
	\frenchspacing 
	
	\begin{center}
		\LARGE TESTE DE GERMINAÇÃO EM SEMENTES DE FEIJÃO-MACUCO\footnote{Trabalho realizado dentro da área de Ciências Agrárias, com financiamento do CNPq e do IFRO.}
		
		\normalsize
	Marcelo Resende da Silva\footnote{Bolsista PIBIC-EM, marcelo.resende.s2901@gmail.com, Campus Colorado do Oeste.} 
	Jessica Fernandes Dias\footnote{Bolsista PIBIC-Af, jessiagro12@gmail.com, Campus Colorado do Oeste.} 
	Ernando Balbinot\footnote{Orientador ernando.balbinot@ifro.edu.br, Campus Colorado do Oeste.} 
	Fabio Batista de Lima\footnote{Co-Orientador fabio.lima@ifro.edu.br Campus Colorado do Oeste.} 
	\end{center}
	
	\noindent O feijão-macuco (Pachyrhizus tuberosus (Lam.) Spreng.) é uma hortaliça não convencional pertencente à família Fabaceae, originária das cabeceiras do rio Amazonas. Por ser uma espécie pouco conhecida e estudada, não existe muita informação científica, é importante uma importante fonte de alimentação suplementar, diversificada no mercado. O objetivo do trabalho foi avaliar o potencial fisiológico de sementes de feijão-macuco por meio do teste de germinação. Esse teste em laboratório revela o desenvolvimento das estruturas essências do embrião, para formação de uma planta normal sob condições favoráveis de campo. O trabalho foi conduzido no Instituto Federal de Educação Ciência e Tecnologia de Rondônia, Campus Colorado do Oeste, utilizando-se sementes colhidas em uma área de plantio da própria instituição. As sementes colhidas passaram por beneficiamento manual para uniformização do lote quanto ao tamanho e coloração das sementes. Para o teste de germinação foram utilizadas 200 sementes, divididas em 4 repetições de 50 sementes, distribuídas em três folhas de papel germitest, umedecidas com água destilada na proporção de 2,5 vezes o peso do papel, com temperatura constante de 25°C e fotoperíodo de 24 horas em câmera de germinação tipo BOD. As avaliações foram realizadas no 10° dia, sendo considerado para a porcentagem de germinação o número de plântulas normais. As plântulas consideradas normais foram avaliadas quanto ao comprimento e massa de matéria seca do hipocótilo e da raiz primária. O lote de sementes avaliado apresentou germinação de 82,67\%, sendo considerada adequada, pois as sementes de tamanho menor geralmente apresentam dormência física. O comprimento médio do hipocótilo foi de 3,86 cm, e para a raiz primária foi de 6,17 cm. A massa de matéria seca da raiz foi de 1,46 g e do hipocótilo de 1,49 g. De forma geral pode-se concluir que as sementes de feijão-macuco de tamanho maior apresentam bom potencial fisiológico após a colheita.
	
	\vspace{\onelineskip}
	
	\noindent
	\textbf{Palavras-chave}: Pachyrhizus tuberosus. Germinação. Massa seca de plântula.
	
\end{document}
