\documentclass[article,12pt,onesidea,4paper,english,brazil]{abntex2}

\usepackage{lmodern, indentfirst, nomencl, color, graphicx, microtype, lipsum}			
\usepackage[T1]{fontenc}		
\usepackage[utf8]{inputenc}		

\setlrmarginsandblock{3cm}{3cm}{*}
\setulmarginsandblock{3cm}{3cm}{*}
\checkandfixthelayout

\setlength{\parindent}{1.3cm}
\setlength{\parskip}{0.2cm}

\SingleSpacing

\begin{document}
	
	\selectlanguage{brazil}
	
	\frenchspacing 
	
	\begin{center}
		\LARGE A PERCEPÇÃO SOBRE A ÉTICA EM TORNO DO IMPERATIVO CATEGÓRICO
		DE KANT E SUA RELAÇÃO COM O DIREITO E A PSICANÁLISE\footnote{Trabalho realizado dentro da Filosofia e sem financiamento orçamentário.}
		
		\normalsize
	Ellen Cristina\footnote{Colaborador(a), ellensetubal.14@gmail.com, Campus Guajará-mirim.} 
		Johanna Trindade\footnote{Colaborador(a), johannatrindade93@gmail.com, Campus Guajará-mirim.} 
		Laura de Paula\footnote{Colaborador(a), laurie.paulie@gmail.com, Campus Guajará-mirim.} 
		Décio Keher Marques\footnote{Orientador(a), decio.marques@ifro.edu.br, Campus Guajará-Mirim.} 
	\end{center}
	
	\noindent Esta pesquisa tem como objetivo interligar a psicanálise com o Direito como objetos
	da filosofia. Há uma enorme literatura em torno das temáticas que perpassam pelos
	filósofos Emanuel Kant, Freud e Lacan. Os conceitos interligados sobressaem-se
	desde o imperativo categórico kantiano, o superego de Freud e a sua materialização
	na linguagem em Lacan. A psicanálise, descoberta por Sigmund Freud é um método
	para a compreensão e análise do homem enquanto sujeito psíquico. Este método foi
	responsável pela descoberta do inconsciente, que segundo Jacques Lacan, está
	relacionado com a linguagem que se estrutura como o centro de suas preocupações
	e de seu trabalho clínico e teórico. A consciência do ser humano de pertença do
	mundo sensível e ao mundo inteligível leva-o a obrigação moral, na compreensão
	kantiana. Se fosse somente pertencente ao mundo sensível estaria exclusivamente
	sob o judice da lei natural e obedecendo apenas as suas inclinações, não havendo a
	possibilidade de se estabelecer relações morais. Ao contrário, se se pertencesse
	apenas ao mundo inteligível, estaria o ser humano exclusivamente sob o mandato
	da razão e as relações seriam evidentemente morais baseada na autonomia
	(liberdade do sujeito de escolher sem dependência do mundo empírico). A
	submissão da vontade à razão nasce a obrigação por parte do sujeito e o dever é
	entendido como necessidade de ação em respeito à lei para Kant, o que, este
	mesmo objeto é também pesquisado por Lacan. Na ótica kantiana, o agir
	racionalizado implica a introspecção do sujeito da representação das leis. O que é
	formalmente conceituado de “imperativo categórico”, o qual, é formal e impõe-se que
	o sujeito faça o que prevê a lei. O resultado do imperativo categórico obriga o sujeito
	a agir conforme um mundo racional e inteligível. Esta obrigação formal imposta ao
	sujeito equipara-se ao superego descoberto por Freud. Para isso recorremos ao
	método da pesquisa bibliográfica através de artigos prontos e dados obtidos com
	auxílio da internet e livros. O resultado consiste na utilização desses escritos como
	subsídios nas aulas da disciplina de Filosofia e de Ética Profissional e Cidadania.
	
	\vspace{\onelineskip}
	
	\noindent
	\textbf{Palavras-chave}: Sujeito psíquico. Imperativo categórico. Filosofia.
	
\end{document}
