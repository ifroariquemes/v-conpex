\documentclass[article,12pt,onesidea,4paper,english,brazil]{abntex2}

\usepackage{lmodern, indentfirst, nomencl, color, graphicx, microtype, lipsum}			
\usepackage[T1]{fontenc}		
\usepackage[utf8]{inputenc}		

\setlrmarginsandblock{2cm}{2cm}{*}
\setulmarginsandblock{2cm}{2cm}{*}
\checkandfixthelayout

\setlength{\parindent}{1.3cm}
\setlength{\parskip}{0.2cm}

\SingleSpacing

\begin{document}
	
	\selectlanguage{brazil}
	
	\frenchspacing 
	
	\begin{center}
		\LARGE O PODER IDEOLÓGICO DA MÍDIA NA PERSPECTIVA DA OBRA DE LOUIS
		ALTHUSSER\footnote{Área de Conhecimento CAPES/CNPq: 7.02.03 - Sociologia / Sociologia do
			Desenvolvimento}
		
		\normalsize
		Wdmila Gabriela Borges Romanini\footnote{Colaboradora Wdmila Gabriela Borges Romanini, wdmilagbr15@gmail.com, Campus Cacoal.} 
		Sabrina Vales Vieira\footnote{Colaboradora Sabrina Vales Vieira, sabrinasvvs24@gmail.com, Campus Cacoal.} 
		Davys Sleman de Negreiros\footnote{Orientador Davys Sleman de Negreiros, davys.negreiros@ifro.edu.br, Campus Cacoal.} 
		
	\end{center}
	
	\noindent O poder ideológico se caracteriza como um padrão comportamental de influência
	que uma pessoa detém autoridade, ou seja, a opinião que determinado individuo
	segue. Esta pesquisa tem como objetivo apresentar ideias de Althusser referentes
	aos aparelhos ideológicos do estado, em especial da Mass media (AIE de
	informação) que possui maior influência na sociedade. Na obra de Louis Althusser
	"Aparelhos ideológicos do estado" apresentam diferentes ideologias que influenciam
	a sociedade, sendo elas: AIE familiar, cultural, jurídica, sindical e a escolar, sendo
	esta, uma ideologia que forma um indivíduo e faz ele ter as suas próprias
	convicções, além do autor apresentar a informação em outra forma de exercer poder
	na formação da opinião pública e suas concepções, já que, atualmente a população
	em geral está ligada em informações passadas pela internet e mídia televisiva.
	Quem tem acesso aos meios de comunicação, naturalmente possui influência e
	poder. Infelizmente, a maioria das pessoas não possui a capacidade crítica diante
	das notícias, aceitando como verdade tudo que lhe é transmitido. A metodologia
	aplicada foi pesquisas bibliográficas, realizado por intermédio das capas, páginas,
	colunas, seções e editoriais. Os dados coletados foram objeto de análise de
	conteúdo e informação da mídia. Diante dos fatos apresentados, conclui-se que a
	mídia age selecionando os assuntos e cobrindo outros, gerando na sociedade
	apenas um assunto preponderante, pois a comunicação consiste no instrumento
	mais importante de manipulação para impedir qualquer mudança no processo de
	exploração e dominação da classe dominante. Até mesmo a cultura de uma
	sociedade pode ser influenciada pelos meios de comunicação (Internet, jornal,
	rádio).
	
	\vspace{\onelineskip}
	
	\noindent
	\textbf{Palavras-chave}: Poder ideológico, Mídia, Sociedade.

	\noindent
	\textbf{Fonte de Financiamento}: IFRO
	
\end{document}
