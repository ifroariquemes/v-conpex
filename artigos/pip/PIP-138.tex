\documentclass[article,12pt,onesidea,4paper,english,brazil]{abntex2}

\usepackage{lmodern, indentfirst, nomencl, color, graphicx, microtype, lipsum}			
\usepackage[T1]{fontenc}		
\usepackage[utf8]{inputenc}		

\setlrmarginsandblock{2cm}{2cm}{*}
\setulmarginsandblock{2cm}{2cm}{*}
\checkandfixthelayout

\setlength{\parindent}{1.3cm}
\setlength{\parskip}{0.2cm}

\SingleSpacing

\begin{document}
	
	\selectlanguage{brazil}
	
	\frenchspacing 
	
	\begin{center}
		\LARGE SKINHEADS E CARECAS: DA SIMBOLOGIA AOS SENTIDOS SOCIAIS\footnote{Trabalho realizado dentro da área de Ciências Humanas (CNPq)}
		
		\normalsize
		Rayssa Rossatt de Souza Xavier\footnote{Bolsista (PIP-EM), rayssarossatt@gmail.com, Campus Cacoal} 
	Elisa Camargo Gomes\footnote{Colaborador(a), elisacg21@gmail.com, Campus Cacoal} 
		Shayenny Dias Felicio de Almeida\footnote{Colaborador(a), shayennydias@hotmail.com, Campus Cacoal} 
	Sérgio Nunes de Jesus\footnote{Orientador(a), sergio.nunes@ifro.edu.br, Campus Cacoal} 
	\end{center}
	
	\noindent O presente trabalho tem como pressuposto abordar a simbologia e os sentidos sociais com base em grupos que possuem grande impacto social como sociedade marginal (entende-se à margem da sociedade), dando enfoque aos Skinheads e suas variações como, por exemplo, os Skinheads White Power, além dos “Skinheads” brasileiros: os Carecas do ABC, Carecas do Brasil e Carecas do Subúrbio; esses que surgiram da mesma vertente ideológica e se diferenciaram conforme cada uma das necessidades da época, difundindo assim características como patriotismo, simplicidade, agressividade e ódio por meio de músicas (canções) que instigam atitudes e represálias ao governo – essas, em determinados seguimentos, causam/causaram tendências de intolerância no cotidiano social. Para tanto, o trabalho foi desenvolvido nos encontros dos grupos de pesquisa Língua(gem), cultura e sociedade: saberes e práticas discursivas na Amazônia, sob a orientação do professor Sérgio Nunes de Jesus, com as alunas do 3º ano, do curso Técnico em Agroecologia Integrado ao Ensino Médio, do Campus Cacoal, do Instituto Federal de Educação, Ciência e Tecnologia de Rondônia/IFRO, com o projeto PDA 2017, com base em pesquisas bibliográficas e de campo. Ademais, oriundos da personalidade de determinadas culturas, os símbolos abrangem a interdependência entre linguagens sociais e psicológicas de um sistema, tornando-as “físicas”; passíveis de identificação. Com a insurgência de um grupo, surge também a necessidade de uma linguagem para representá-lo. Além dos dialetos e gírias internos, é tencionada a adoção e/ou criação de um elemento que o represente externamente: um símbolo. A adoção do supracitado, muitas vezes, está relacionada à adoção da própria ideologia seguida por seus criadores, assim difundindo ideias e revelando intenções diante da sociedade; nazismo, fascismo, homofobia, violência, entre outros, que podem ou não provocar intolerância. Sendo assim, pode-se dizer que, a relação entre os grupos (Skinheads e Carecas) e a simbologia adotada dar-se-á na alusão pensada entre um e outro, pelas diferenciações dos elementos estudados, das simbologias e ideologias pautadas no desenvolvimento da contracultura no cenário mundial.
	
	\vspace{\onelineskip}
	
	\noindent
	\textbf{Palavras-chave}: Carecas. Símbolos. Skinheads.
	
\end{document}
