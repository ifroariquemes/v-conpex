\documentclass[article,12pt,onesidea,4paper,english,brazil]{abntex2}

\usepackage{lmodern, indentfirst, nomencl, color, graphicx, microtype, lipsum}			
\usepackage[T1]{fontenc}		
\usepackage[utf8]{inputenc}		

\setlrmarginsandblock{2cm}{2cm}{*}
\setulmarginsandblock{2cm}{2cm}{*}
\checkandfixthelayout

\setlength{\parindent}{1.3cm}
\setlength{\parskip}{0.2cm}

\SingleSpacing

\begin{document}
	
	\selectlanguage{brazil}
	
	\frenchspacing 
	
	\begin{center}
		\LARGE INVENTÁRIO DE MASTOFAUNA EM REMANESCENTE FLORESTAL,
		ARIQUEMES, RONDÔNIA, BRASIL\footnote{Trabalho realizado dentro das Ciências Biológicas sem fomento de Instituição de Pesquisa}
	
		\normalsize
	Alysson Rossi dos Santos\footnote{Acadêmico do curso de Ciências Biológicas, alyssonr@hotmail.com, Instituto Federal de Educação,
		Ciência e Tecnologia de Rondônia - IFRO campus Ariquemes} 
	Elaine Oliveira Costa de Carvalho\footnote{Orientador, elaine.carvalho@ifro.edu.br.}
	
	\end{center}
	
	\noindent Os mamíferos são encontrados em praticamente todos os biomas e tipos de habitas
	do planeta Terra. Existem mamíferos desde as grandes altitudes, como abaixo do
	nível do mar. Geralmente os mamíferos apresentam pelos no corpo, o coração tem
	quatro cavidades como nas aves, são endotérmicos, as fêmeas tem glândulas
	mamárias produzindo leite para nutrição dos filhotes e a respiração é pulmonar. O
	Brasil é considerado o país com maior número de espécies desta classe, onde, pela
	Lista Anotada de Mamíferos do Brasil, existem 701 espécies, distribuídas em 243
	gêneros, 50 famílias e 12 ordens, tendo a Amazônia 56\% de espécies endêmicas. O
	presente trabalho descreve um inventário de espécies da mastofauna em
	remanescente florestal do IFRO-campus Ariquemes/RO. As metodologias adotadas
	foram o registro de imagens de pegadas e vestígios, avistamento e o uso de uma
	armadilha fotográfica posicionada ao centro de um remanescente florestal, tendo
	este área total de 1.850 hectares (212 hectares parte do campus do IFRO),
	resultando em 16.368 horas de esforço amostral, somando as três metodologias. A
	armadilha fotográfica foi afixada em espécie vegetal arborícola que serve como
	marcador de trilha utilizada por animais silvestres. Os registros de pegadas e
	avistamento foram realizados a cada sete dias em excursões dentro do
	remanescente florestal, durante seis horas/dia, utilizando máquina fotográfica digital
	para registro das imagens. Em levantamento prévio de dados dos últimos 22 meses,
	foram registradas imagens de 31 espécies da mastofauna, identificadas em 09
	ordens, com as respectivas quantidades distribuídas em: Artiodactyla (03), Carnívora
	(07), Chiropthera (01), Cingulata (04), Perissodactyla (01), Pilosa (02), Rodentia (04),
	Didelphimorphia (04), Primates (05). Observando o qualitativo das espécies
	registradas, sendo de pequeno, médio e grande porte, aérea, arborícola, aquático,
	escansorial e terrestre, e a distribuição quase universal em todas as ordens de
	mamíferos existentes, fortalece o continuísmo do trabalho, sendo possível que, com
	um esforço amostral maior e mais distribuído, outras espécies também sejam
	registradas.
	
	\vspace{\onelineskip}
	
	\noindent
	\textbf{Palavras-chave}: Inventário de espécies, mamíferos, IFRO.
	
\end{document}
