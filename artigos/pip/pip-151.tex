\documentclass[article,12pt,onesidea,4paper,english,brazil]{abntex2}

\usepackage{lmodern, indentfirst, nomencl, color, graphicx, microtype, lipsum,textcomp}			
\usepackage[T1]{fontenc}		
\usepackage[utf8]{inputenc}		

\setlrmarginsandblock{2cm}{2cm}{*}
\setulmarginsandblock{2cm}{2cm}{*}
\checkandfixthelayout

\setlength{\parindent}{1.3cm}
\setlength{\parskip}{0.2cm}

\SingleSpacing

\begin{document}
	
	\selectlanguage{brazil}
	
	\frenchspacing 
	
	\begin{center}
		\LARGE PADRONIZAÇÃO DO TESTE DE TETRAZÓLIO PARA SEMENTES DE CASSIA FISTULA\footnote{Trabalho realizado dentro da (área de Conhecimento CNPq: Recursos florestais) com financiamento do (a) (CNPQ/ IFRO).}
		
		\normalsize
		Matheus Favaro Moreira\footnote{Bolsista (Iniciação cientifica - EM), email favarom38@gmail.com, Campus Ji-Paraná.} 
		Maria Elessandra Rodrigues Araujo\footnote{Orientador (a), email maria.elessandra@ifro.edu.br, Campus Ji-Paraná.} 
		Andreza Pereira Mendonça\footnote{Co-orientador(a), email andreza.mendonca@ifro.edu.br, Campus Ji-Paraná.}  
	\end{center}
	
	\noindent A espécie Cassia fistula L é uma espécie exótica, que foi introduzida no Brasil há muitos anos, está apresenta grande importância ecológica, uma vez que pode ser utilizada para recuperação de áreas degradadas pela capacidade de promover acúmulo de matéria orgânica no solo. No entanto a determinação da viabilidade de sementes de Cassia fistula pelo teste de germinação é relativamente demorada, especialmente se for necessário superar a dormência. O teste de tetrazólio é promissor nesses casos, uma vez que é um teste rápido, que não sofre influência da dormência das sementes, por se basear apenas no processo respiratório das células, e que permite a avaliação tanto da viabilidade quanto do vigor das sementes. Diante do exposto, este trabalho teve por objetivo avaliar diferentes concentrações do sal de tetrazólio e períodos de coloração de sementes, para determinação da viabilidade de sementes de Cassia fistula. O presente trabalho foi conduzido no laboratório de análises de sementes e Laboratório de Biologia do Instituto Federal de Rondônia - Campus, Ji-Paraná. Foram utilizadas sementes de Cassia fistula coletadas em áreas circunvizinhas ao município de Ji-Paraná. Foram avaliadas diferentes condições de preparo e coloração das sementes, analisando os aspectos dos tecidos, bem como a intensidade e uniformidade de coloração. A eficiência das várias condições empregadas para o teste de tetrazólio foi avaliada comparando os resultados destas com os do teste de germinação. O delineamento experimental utilizado nas diferentes etapas foi o inteiramente casualizado com quatro repetições. O software utilizado na análise foi o ASSISTAT, e as médias, após análise de variância, comparadas pelo teste de Tukey a 5\% de probabilidade. Comparando os diferentes tratamentos (concentração e períodos) com a germinação verificou-se diferença significativa, exceto quando as sementes foram imersas na solução de 0,05\% de tetrazólio no período de 4 horas, mostrando-se eficiente para avaliar a viabilidade de sementes de Cassia fistula.
	
	\vspace{\onelineskip}
	
	\noindent
	\textbf{Palavras-chave}: Cassia fistula. Tetrazólio. Dormência.
	
\end{document}
