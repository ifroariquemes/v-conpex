\documentclass[article,12pt,onesidea,4paper,english,brazil]{abntex2}

\usepackage{lmodern, indentfirst, nomencl, color, graphicx, microtype, lipsum}			
\usepackage[T1]{fontenc}		
\usepackage[utf8]{inputenc}	


\setlrmarginsandblock{2cm}{2cm}{*}
\setulmarginsandblock{2cm}{2cm}{*}
\checkandfixthelayout

\setlength{\parindent}{1.3cm}
\setlength{\parskip}{0.2cm}

\SingleSpacing

\begin{document}
	
	\selectlanguage{brazil}
	
	\frenchspacing 
	
	\begin{center}
		\LARGE AVALIAÇÃO DO PODER INSETICIDA DO ÓLEO ESSENCIAL DE CANFORA (Cinnamomum camphora), PARA O CONTROLE DO INSETO-PRAGA DE GRÃO ARMAZENADOS, Sitophilus zeamais, EM MILHO (Zea mays)\footnote{Trabalho realizado dentro da (área de Conhecimento CNPq: Entomologia Agrícola) com
			financiamento do (a): Instituto federal de educação, ciência e tecnologia de Rondônia. Campus
			Colorado do Oeste.}
		
		\normalsize
		Caroline Alves Lima\footnote{Bolsista (PIP), alveslima.caroline@gmail.com. Campus Colorado do Oeste} 
		Eliane Mazzorana de Campos\footnote{Colaborador(a), campmazorana@gmail.com. Campus Colorado do Oeste} 
		Aline Fonseca do Nascimento\footnote{Orientador(a), aline.fonseca@ifro.edu.br. Campus Colorado do Oeste} 
		
	\end{center}
	 
	\noindent 
	As pragas de grão armazenado são um desafio para a produção de grãos, uma vez que causam danos e influenciam na qualidade do produto. Dentre estas espécies destaca-se o Sithophilus zeamais, praga chave do grão de milho armazenado, deste modo, é de extrema necessidade a busca de técnicas e métodos para que estas possam ser controladas, tanto curativa como preventivamente. Uma das principais fontes de novas moléculas para controle de insetos-pragas são os óleos essenciais de plantas. Deste modo, o objetivo deste estudo foi buscar em óleos essenciais moléculas que possam ser eficientes no controle do gorgulho do milho S. zeamais. A metodologia utilizada foi através do método de fumigação do óleo essencial da espécie canfora (Cinnamomum camphora). O experimento foi conduzido no laboratório de biologia, no Instituto Federal de ciências e tecnologia, IFRO, Campus Colorado do Oeste. Foram utilizados insetos criados em sementes de milho, acondicionados em recipientes de vidro. O teste de fumigação foi realizado em câmaras de vidro, onde foram confinados 20 adultos de S. zeamais, não sexados, com 0 a 15 dias de idade. A avaliação da mortalidade foi realizada nos períodos de 24, 48 e 72 horas. Utilizou-se as concentrações de 5, 10, 25, 35 e 40 $\mu$L L-1 de ar. O delineamento experimental utilizado foi o delineamento inteiramente casualizado (DIC). Os dados de toxicidade obtidos dos bioensaios de concentração-mortalidade foram submetidos à análise de Probit a P>0,05 através do programa estatístico SAS. A concentração letal $CL_{50}$ e $CL_{90}$ do óleo essencial de C. camphora no período de 24 horas foram respectivamente 13,84$\mu$L/L de ar e 29,66 $\mu$L/L.. PAULIQUEVIS, C. F. \& FAVERO, S. (2015) em teste sobre S. zemais testaram o óleo essencial da pariparoba que apresentou efeito fumigante em 24 horas de exposição foi obtido os valores da $CL_{50}$ e $CL_{99}$, respectivamente, 0,95 e 6,73 $\mu$L g-1. As $CL_{50}$ de C. camphora obtidas neste estudo foi eficiente, porém em concentração maior. O óleo essencial de canfora apresenta resultados promissores, contudo, são necessários maiores estudos, especialmente quanto ao conhecimento das interações e toxicidade dos constituintes do óleo essencial testado.
	\vspace{\onelineskip}
	
	\noindent
	\textbf{Palavras-chave}: Controle biológico. Inseticida botânico. Sustentabilidade
	
\end{document}
