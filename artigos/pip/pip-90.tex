\documentclass[article,12pt,onesidea,4paper,english,brazil]{abntex2}

\usepackage{lmodern, indentfirst, nomencl, color, graphicx, microtype, lipsum}			
\usepackage[T1]{fontenc}		
\usepackage[utf8]{inputenc}		

\setlrmarginsandblock{2cm}{2cm}{*}
\setulmarginsandblock{2cm}{2cm}{*}
\checkandfixthelayout

\setlength{\parindent}{1.3cm}
\setlength{\parskip}{0.2cm}

\SingleSpacing

\begin{document}
	
	\selectlanguage{brazil}
	
	\frenchspacing 
	
	\begin{center}
		\LARGE PERFIL DA INDÚSTRIA DE BEBIDAS NA CIDADE DE PORTO VELHO – RO\footnote{Trabalho realizado dentro da área de conhecimento de Ciências Sociais Aplicadas sem
			financiamento.}
		
		\normalsize
		Bruno Henrique Silva Rodrigues\footnote{Discentes do Curso Superior de Tecnologia em Agronegócio e pesquisadores voluntários,
			bruno.rodrigues@ifro.edu.br e hattosjackson.hbs@gmail.com, Campus Cacoal.} 
		Hattos Barros Araújo
		Jamile Mariano Macedo\footnote{Orientadora: Mestra em Desenvolvimento Regional pela UNIR e Docente EBTT do IFRO,
			jamile.macedo@ifro.edu.br, Campus Porto Velho Calama} 
		
	\end{center}
	
	\noindent Segundo o Banco Nacional do Desenvolvimento, ocrescimento do setor de bebidas
	no Brasil foi expressivo nos últimos dez anos, sendo os principais produtos os
	refrigerantes e as cervejas, que ultrapassam 75\% do total de bebidas produzidasna
	indústria. Este trabalho tevecomo objetivo fazer um levantamento de dados sobre os
	segmentos de bebidas que existem em Porto Velho, verificando os segmentos
	majoritáriose suas taxas de crescimento, a fim deavaliar os mais promissores.Foram
	realizados estudos com dados secundários, oriundos das fontes Empresas do
	Brasil®
	e Cadastro Brasil®
	
	, sendo efetuada a diferenciação dos segmentos a partir da
	Classificação Nacional de Atividades Econômicas (CNAE). A pesquisa realizada
	resultouem três grupos: água envasada; bebidas não alcoólicas e bebidas
	alcoólicas. As indústrias foram classificadas de acordo com o grupo e especificadas
	de acordo com a CNAE, levando em conta sua atividade econômica
	principal.Utilizou-seo Microsoft Excel 2010®
	
	para tabulação e tratamento dos dados.
	Foi verificado quehá 39 indústrias no ramo de bebidas, sendo 74,35\% do grupo das
	bebidas não alcoólicas, 15,38\% do grupo das bebidas alcoólicas e 10,26\% de água
	envasada. Os principais segmentos são os sucos concentrados de frutas 55,17\% e o
	de fabricação de refrigerantes 24,14\%.Observou-seque nos últimos 16 anos, o
	segmento dos refrigerantes aumentou 40\%, e o de sucos concentrados 60\%.
	Observou-se, que além destes, já consta uma indústria na área de isotônicos. Diante
	
	do exposto, é possível que a maior proporção de indústrias na área de sucos, deva-
	seprincipalmente à abundância de matéria-prima e o menor custo para manutenção
	
	da atividade. Já o menor crescimento do segmento dos refrigerantespossivelmente
	seja reflexo da concorrência com marca multinacional situada no município. Diante
	dessas informações, o mercado para sucos e refrigerantes está bem estabelecido, o
	que pode contribuir emoportunidades para a geração de novos produtos, como
	cervejas e isotônicos.
	
	\vspace{\onelineskip}
	
	\noindent
	\textbf{Palavras-chave}:Perfil. Indústrias de Bebidas. Sucos Concentrados.
	
\end{document}
