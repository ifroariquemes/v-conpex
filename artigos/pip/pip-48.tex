\documentclass[article,12pt,onesidea,4paper,english,brazil]{abntex2}

\usepackage{lmodern, indentfirst, nomencl, color, graphicx, microtype, lipsum}			
\usepackage[T1]{fontenc}		
\usepackage[utf8]{inputenc}		

\setlrmarginsandblock{2cm}{2cm}{*}
\setulmarginsandblock{2cm}{2cm}{*}
\checkandfixthelayout

\setlength{\parindent}{1.3cm}
\setlength{\parskip}{0.2cm}

\SingleSpacing

\begin{document}
	
	\selectlanguage{brazil}
	
	\frenchspacing 
	
	\begin{center}
		\LARGE EFEITO DA CONCENTRAÇÃO DE BLENDAS DE ÓLEOS ESSENCIAIS NA ATIVIDADE ANTIMICROBIANA EM QUEIJOS MUSSARELA NO CONTROLE DE \textit{STAPHYLOCOCCUS AUREUS}\footnote{Trabalho realizado dentro da Ciências agrárias, fonte de financiamento: IFRO.}
		
		\normalsize
	Rebekah Anne Freese,\footnote{Bolsista (PIBITI), becky4400@hotmail.com, Campus Colorado do Oeste.} 
		Caroline de Oliveira Ribeiro,\footnote{Bolsista (PIBITI), caroline\_assisoliveira@hotmail.com, Campus Colorado do Oeste.} 
	Lizianne de Matos Emerick,\footnote{Colaborador, lizianneemerick@gmail.com, Campus Colorado do Oeste.} 
		Nélio Ranieli Ferreira de Paula\footnote{Orientador, nelio.ferreira@ifro.edu.br, Campus Colorado do Oeste.} 
	\end{center}
	
	\noindent Produtos alimentícios derivados de leite são alimentos susceptíveis a diversos tipos de contaminações, dentre estes o queijo apresenta grande potencial. Devido a este fato, a utilização de compostos aromáticos voláteis óleos essenciais como antimicrobiano em alimentos processados visam aumentar a vida útil de prateleira dos alimentos de forma eficiente e mais saudáveis frente ao uso de conservantes artificiais. Os objetivos deste trabalho foram estudar o efeito antimicrobiano de blendas de óleos essenciais extraídos de plantas condimentares como: hortelã pimenta, cardamomo, capim limão e cravo buscando proporcionar benefícios aos produtores e consumidores, quanto a maior conservação, armazenamento e vida de prateleira, sem alterar seu sabor e qualidade. O método utilizado para a extração de óleos foi através dos processos de hidrodestilação em aparelho de Clevenger. A atividade antimicrobiana foi analisada nos meios de cultura ágar TSA e Baird Parker. O delineamento foi inteiramente casualizado, em esquema fatorial (4x6), realizado em duplicata. As blendas testadas foram caracterizadas como B1 com 31,62\% de cardamomo, 15,62\% capim-limão, 39,28\% de cravo e 13,28\% hortelã do percentual de óleo essencial e B2 com 31,62\% de cardamomo, 39,28\% capim-limão, 15,62\% de cravo e 13,28\% hortelã do percentual de óleo essencial. Para tanto, as Blendas foram compostas por 1\% de óleo essencial, com 990 $\mu$L de água peptonada e 10 $\mu$L do óleo essencial. De acordo com os resultados verificou se o efeito na concentração de óleos essenciais utilizados tanto na Blenda 1, quanto na Blenda 2, com ação bacteriostática pois houve redução do crescimento bacteriano de \textit{Staphylococcus aureus} inoculado em queijo mussasrela, no período de 48 horas. Para o tratamento blenda 1, houve diminuição de 0,29 ciclos log e para a blenda 2 houve um crescimento 0,2 ciclos log mostrando desta forma a existência de sinergismos nas diferentes concentrações de óleos avaliadas.
	
	\vspace{\onelineskip}
	
	\noindent
	\textbf{Palavras-chave}: Óleos essenciais. Blendas. Mussarela.\\
	\textbf{Fonte de Financiamento}: IFRO.
	
\end{document}
