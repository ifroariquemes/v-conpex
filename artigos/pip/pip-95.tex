\documentclass[article,12pt,onesidea,4paper,english,brazil]{abntex2}

\usepackage{lmodern, indentfirst, nomencl, color, graphicx, microtype, lipsum}			
\usepackage[T1]{fontenc}		
\usepackage[utf8]{inputenc}		

\setlrmarginsandblock{2cm}{2cm}{*}
\setulmarginsandblock{2cm}{2cm}{*}
\checkandfixthelayout

\setlength{\parindent}{1.3cm}
\setlength{\parskip}{0.2cm}

\SingleSpacing

\begin{document}
	
	\selectlanguage{brazil}
	
	\frenchspacing 
	
	\begin{center}
		\LARGE PRODUÇÃO DE MUDAS DE MAMOEIRO EM FUNÇÃO DE DIFERENTES SUBSTRATOS E RECIPIENTES\footnote{Trabalho realizado dentro da (área de Conhecimento CNPq: Ciências Agrárias) com financiamento
			do IFRO, Campus Ariquemes.}
		
		\normalsize
		Lucas Bravim Furlan\footnote{Pesquisador Iniciante, furlanlucas466@gmail.com, Campus Ariquemes} 
		Wendell Douglas de Oliveira Silva\footnote{Colaborador, wendelldouglasdeoliveira1999@gmail.com, Campus Ariquemes} 
		Luciano dos Reis Venturoso\footnote{Orientador, luciano.venturoso@ifro.edu.br, Campus Ariquemes} \\
		Lenita Aparecida Conus Venturoso\footnote{Co-orientadora, lenita.conus@ifro.edu.br, Campus Ariquemes} 
		Lucas Pedro Cipriani\footnote{Co-orientador, lucas.cipriani@ifro.edu.br, Campus Ariquemes}
	\end{center}
	
	\noindent A produtividade e a qualidade dos frutos de mamoeiro dependem dos tratos culturais
	dispensados às plantas, desde a obtenção de sementes até a formação de mudas.
	O sistema produtivo requer constante renovação dos pomares, utilizando-se
	substrato comercial, de elevado custo, e sem aproveitar materiais regionais. A
	necessidade de pesquisas envolvendo o mamoeiro é de extrema importância, no
	estado de Rondônia, visto que são incipientes estudos relacionados à produção de
	mudas. Objetivou-se avaliar a utilização de diferentes substratos e tipos de
	recipientes sobre a qualidade de mudas de mamoeiro, cultivar BS. O experimento foi
	conduzido no viveiro do Instituto Federal de Rondônia, campus Ariquemes, no
	período de janeiro a maio de 2017. Foi adotado o delineamento inteiramente
	casualizados, em arranjo fatorial 2x5, com 10 repetições. Foram plantadas três
	sementes por recipiente, tubete e sacola de polietileno, utilizando-se os substratos,
	T1: solo, T2: solo + esterco caprino (1:1), T3: solo + esterco bovino (1:1), T4:
	substrato comercial, T5: solo + Ferti-peixe. No T5, o Ferti\~{}Peixe® foi utilizado nas
	sacolinhas na dose de 1 ml.dm-3 de solo. Foi avaliado o índice de velocidade de
	emergência, percentual de emergência, comprimento da parte área e de raiz, massa
	verde e seca da parte aérea e de raiz. Com exceção à emergência, verificou-se
	interação significativa para todas as variáveis analisadas. De modo geral, o
	desenvolvimento das plantas nas sacolas de polietileno foi superior ao observado
	nos tubetes. O produto Ferti\~{}Peixe em ambos os recipientes e o substrato comercial
	no tubete provocaram redução na velocidade de emergência. O substrato contendo
	esterco caprino proporcionou maior crescimento da parte aérea às plantas, em
	ambos os recipientes. Enquanto que o comprimento do sistema radicular nas
	sacolas, apresentou similaridade entre T2, T3 e T5. Nos tubetes não houve
	diferença entre os substratos, fato que provavelmente, está relacionado a limitação
	do tamanho dos mesmos. Com relação à massa vegetal verde e seca, novamente o
	substrato contendo esterco caprino superou os demais substratos, independente do
	recipiente utilizado. O esterco caprino demonstrou potencial para ser incorporado ao
	sistema de produção de mudas de mamoeiro.
	
	\vspace{\onelineskip}
	
	\noindent
	\textbf{Palavras-chave}: Carica papaya. Esterco caprino. Esterco bovino.
	
	\noindent
	\textbf{Fonte de Financiamento}:Instituto Federal de Rondônia, Campus Ariquemes.
	
\end{document}
