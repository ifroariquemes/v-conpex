\documentclass[article,12pt,onesidea,4paper,english,brazil]{abntex2}

\usepackage{lmodern, indentfirst, nomencl, color, graphicx, microtype, lipsum}			
\usepackage[T1]{fontenc}		
\usepackage[utf8]{inputenc}		

\setlrmarginsandblock{3cm}{3cm}{*}
\setulmarginsandblock{3cm}{3cm}{*}
\checkandfixthelayout

\setlength{\parindent}{1.3cm}
\setlength{\parskip}{0.2cm}

\SingleSpacing

\begin{document}
	
	\selectlanguage{brazil}
	
	\frenchspacing 
	
	\begin{center}
		\LARGE AVALIAÇÕES AGRONÔMICAS DE CULTIVARES DE ALHO SUBMETIDAS A
		VERNALIZAÇÃO EM DIFERENTES ÉPOCAS DO ANO NO\\CONE SUL DE
		RONDÔNIA
		
		\normalsize
		Guilherme Machado Guedes\footnote{Bolsista (PIBiC - Em), email, guilhermeguedes555@gmail.com, Campus Colorado do Oeste
			Rondônia} 
		Vitor Vicente Klein\footnote{Colaborador(a), email, vitorvklein@gmail.com, Campus Colorado do Oeste Rondônia} \\
		Marcos Aurélio Anequine Macedo\footnote{Orientador(a), email, marcos.anequine@ifro.edu.br, Campus Colorado do Oeste Rondônia} 
	Valdique
	Gilberto de Lima\footnote{Co-orientador(a), email, valdique.lima@ifro.edu.br Campus Colorado do Oeste Rondônia} 
	\end{center}
	
	\noindent A produção de alho em Rondônia, não diferentemente nos demais estados da região
	norte, é inexistente. Isso se deve ao fato da região apresentar clima não propício ao
	cultivo desta cultura, que necessita de temperaturas amenas e fotoperíodo longo
	para que ocorra a diferenciação do bulbo (formação dos dentes). Esse déficit de
	produção supervaloriza o valor de mercado para os consumidores da região, devido
	a distância dos polos produtivos. Alternativas foram criadas para que a cultura fosse
	produzida em regiões de clima quente, sendo uma delas a vernalização, que
	consiste em colocar os bulbilhos (dentes) do alho em câmaras frias sob
	temperaturas em torno de 5$^\circ$C durante alguns dias antes do plantio. Além disso,
	existe a cultivar BRS Hozan, melhorada para ser produzida em regiões de clima
	quente sem necessitar da vernalização. Sendo assim, o objetivo do trabalho foi
	avaliar as cultivares de alho semeadas durante os meses de março a agosto em
	diferentes períodos de vernalização. A pesquisa foi realizada a campo no setor de
	Olericultura do Campus Colorado do Oeste, com 7 tratamentos, que consistiam em
	T1: BRS Hozan; T2: Ito sem vernalização; T3: Ito com 20 dias de vernalização; T4:
	Ito com 40 dias de vernalização; T5: Caçador sem vernalização; T6: Caçador com
	20 dias de vernalização; T7: Caçador com 40 dias de vernalização. O delineamento
	utilizado foi o de blocos casualizados, com quatro repetições. Nos resultados obtidos
	percebeu-se que o tratamento 06, correspondente a cultivar caçador vernalizado por
	20 dias, apresentou melhores resultados no quesito produtividade, ficando em torno
	de 3600 kg por hectare. Além disso, a vernalização apresentou-se satisfatória, pois
	aqueles tratamentos que não passaram por este processo (as testemunhas) não
	apresentaram diferenciação dos bulbos, mostrando-se necessária a produção de
	alho na região.
	
	\vspace{\onelineskip}
	
	\noindent
	\textbf{Palavras-chave}:Alho. Vernalização. Épocas do ano.
	
	\noindent
	\textbf{Fonte de Financiamento}: Conselho Nacional de Desenvolvimento Científico e
	Tecnológico – CNPq.
	
\end{document}
