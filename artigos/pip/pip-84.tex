\documentclass[article,12pt,onesidea,4paper,english,brazil]{abntex2}

\usepackage{lmodern, indentfirst, nomencl, color, graphicx, microtype, lipsum}			
\usepackage[T1]{fontenc}		
\usepackage[utf8]{inputenc}		

\setlrmarginsandblock{2cm}{2cm}{*}
\setulmarginsandblock{2cm}{2cm}{*}
\checkandfixthelayout

\setlength{\parindent}{1.3cm}
\setlength{\parskip}{0.2cm}

\SingleSpacing

\begin{document}
	
	\selectlanguage{brazil}
	
	\frenchspacing 
	
	\begin{center}
		\LARGE MONITORAMENTO DAS PERDAS NA COLHEITA MECANIZADA DE SOJA\footnote{Trabalho realizado dentro da (área de Conhecimento CNPq: Ciências Agrárias)}
		
		\normalsize
		Daniele Jesus Venturim\footnote{Bolsista (IT - EM), daniventurim123@gmail.com, Campus Colorado do Oeste} 
		Ronicley Souza da Silva\footnote{Colaborador, ronicley.ifro@gmail.com, Campus Colorado do Oeste} \\
		Patricia Candida de Menezes\footnote{Orientadora, patricia.menezes@ifro.edu.br, Campus Colorado do Oeste} 
		Rafael
		Henrique Pereira dos Reis\footnote{Co-orientador, rafael.reis@ifro.edu.br, Campus Colorado do Oeste} 
	\end{center}
	
	\noindent A qualidade da colheita mecanizada de soja está diretamente relacionada com a
	redução das perdas durante essa operação. Considerando a necessidade do
	monitoramento das perdas para melhorar a qualidade da colheita mecanizada de
	soja, objetivou-se avaliar o método de coleta de perdas com armações circulares na
	colheita realizada com diferentes plataformas de corte e velocidades de
	deslocamento. O experimento foi realizado em área agrícola no município de Cabixi
	– RO. Foram utilizadas duas colhedoras modelo MF 9790 com plataformas de
	esteira transportadora e condutor helicoidal sendo operadas nas velocidades de 6
	km h-1 e 8 km h
	-1
	. O delineamento experimental foi de acordo com o controle
	estatístico de processo. Foram amostrados 80 pontos distanciados a cada 100
	metros, sendo dois pontos em cada passada da máquina. Em cada ponto foram
	coletadas quatro amostras com armações circulares, com área de 0,25 m2 cada, as
	quais foram lançadas após a passagem da plataforma ficando posicionadas ao
	longo da largura da máquina. Os grãos e vagens encontrados abaixo da armação
	foram considerados como perdas na plataforma, enquanto que os encontrados
	acima da armação como perdas nos mecanismos internos. Para a plataforma com
	condutor helicoidal, as perdas na plataforma apresentaram maiores médias e
	variabilidade na armação C e as perdas nos mecanismos internos se concentraram
	nas armações B e C, na parte central da largura da máquina. Na plataforma de
	esteira transportadora as perdas estavam distribuídas mais uniformemente na
	largura da máquina, ou seja, nas quatro armações. O método de coleta de perdas,
	com utilização de armações circulares, permite verificar a distribuição das perdas de
	grãos ao longo da largura da máquina e, de forma indireta, saber também como está
	a distribuição de palha, uma vez que o comportamento de ambas será semelhante.
	Desta forma é possível não só monitorar as perdas mas também verificar o
	funcionamento e atuar nas regulagens adequadas para melhorar a qualidade da
	operação.
	
	\vspace{\onelineskip}
	
	\noindent
	\textbf{Palavras-chave}: Armações circulares. Colheita. Coleta de perdas.
	
	\noindent
	\textbf{Fonte de financiamento}:Instituto Federal de Educação, Ciência e Tecnologia de
	Rondônia.
	
\end{document}
