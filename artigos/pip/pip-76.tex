\documentclass[article,12pt,onesidea,4paper,english,brazil]{abntex2}

\usepackage{lmodern, indentfirst, nomencl, color, graphicx, microtype, lipsum}			
\usepackage[T1]{fontenc}		
\usepackage[utf8]{inputenc}		

\setlrmarginsandblock{2cm}{2cm}{*}
\setulmarginsandblock{2cm}{2cm}{*}
\checkandfixthelayout

\setlength{\parindent}{1.3cm}
\setlength{\parskip}{0.2cm}

\SingleSpacing

\begin{document}
	
	\selectlanguage{brazil}
	
	\frenchspacing 
	
	\begin{center}
		\LARGE LEVANTAMENTO FLORÍSTICO DO BAIRRO JARDIM TROPICAL EM OURO PRETO DO OESTE – RO
		
		\normalsize
		Ygor F. T. Fonseca\footnote{Colaborador, ygorfernandoopo@gmail.com, Campus Ji-Paraná.} 
		William S. Neimog\footnote{Colaborador, william.neimog@gmail.com, Campus Ji-Paraná.} 
		Janice F. do Nascimento\footnote{Orientadora, janice.nascimento@ifro.edu.br, Campus Ji-Paraná.} 
		 
	\end{center}
	
	\noindent As árvores são de suma importância e indispensáveis para o desenvolvimento
	urbano, sendo considerado um fator de salubridade ambiental. No entanto, a
	arborização é essencial para o conforto humano, no qual se beneficia de alguns
	aspectos, tais como; reduzir os efeitos sonoros, melhorar a qualidade do ar,
	proporcionar abrigo para avefauna local, etc. Diante do exposto, o objetivo deste
	trabalho foi realizar um levantamento da arborização urbana no Bairro Jardim
	Tropical, Ouro Preto do Oeste – RO, e analisar a diversidade de espécies utilizadas.
	Para o levantamento florístico foi utilizada uma tabela abrangendo nome popular,
	cientifico e a origem das espécies. Visto que, o Bairro apresenta fácil acesso
	possibilitando a caminhada pela área, na qual os indivíduos foram fotografados e
	identificados, assim utilizando-se a internet para auxiliar na identificação através de
	guias de espécies. Foram inventariados 848 indivíduos, distribuídas em 16 famílias
	botânicas e 23 espécies. As espécies mais ocorrentes foram Veitchia merrillii
	(Palmeira) (143), Licania tomentosa (Oiti) (134), Ficcus benjamina (Benjamina) (63),
	Tuia ocidentalis (Pinheiro) (50) e Tabebuia serratifolia (Ipê-amarelo) (48), que juntas
	representam 68,96\% do total inventariado. Apesar de que, a arborização da área de
	estudo, é composta por 46,91\% de espécies nativas e 53,05\% de exóticas. A
	utilização de espécies exóticas é muito comum na arborização urbana brasileira,
	principalmente por serem de caráter ornamental, de fácil produção e rápido
	crescimento. A família Arecaceae foi a mais representativa, com 223 indivíduos e
	outra espécie com grande quantidade de indivíduos foi o Oiti (Licania tomentosa).
	Por fim, pode-se observar uma grande quantidade de indivíduos distribuídos em
	poucas espécies.
	
	\vspace{\onelineskip}
	
	\noindent
	\textbf{Palavras-chave}: Arborização urbana; Diversidade; Paisagismo.
	
\end{document}
