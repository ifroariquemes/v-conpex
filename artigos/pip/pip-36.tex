\documentclass[article,12pt,onesidea,4paper,english,brazil]{abntex2}

\usepackage{lmodern, indentfirst, nomencl, color, graphicx, microtype, lipsum}			
\usepackage[T1]{fontenc}		
\usepackage[utf8]{inputenc}		

\setlrmarginsandblock{2cm}{2cm}{*}
\setulmarginsandblock{2cm}{2cm}{*}
\checkandfixthelayout

\setlength{\parindent}{1.3cm}
\setlength{\parskip}{0.2cm}

\SingleSpacing

\begin{document}
	
	\selectlanguage{brazil}
	
	\frenchspacing 
	
	\begin{center}
		\LARGE CURVA DE EMBEBIÇÃO DE ÁGUA EM SEMENTES DE PINHO CUIABANO
		\textit{(Parkia multijuga benth)}\footnote{Trabalho realizado dentro da (área de Conhecimento CNPq: Botânica.}
		
		\normalsize
	Adriana Cristina Turmina\footnote{Bolsista drickaro@hotmail.com, Campus Ariquemes – RO.} 
	Irizon Da Cunha Santana\footnote{Colaborador riu.santana@hotmail.com, Campus Ariquemes - RO.} 
		Ady Correa da Costa Oliveira\footnote{Orientadora ady.oliveira@ifro.edu.br, Campus Ariquemes - RO.} 
		José Fabio
		Xavier\footnote{Co-orientador fabio.xavier@ifro.edu.br, Campus Ariquemes - RO.} 
	\end{center}
	
	\noindent A presente pesquisa teve como o objetivo avaliar a influência do peso de sementes
	de pinho cuiabano (\textit{Parkia multijuga} (Benth), para curva de embebição em água,
	onde se avaliou a quantidade de água necessária para sua quebra de dormência. As sementes foram separadas aleatoriamente, obtendo-se dois tratamentos: T1 -
	sementes testemunha, T2 – sementes escarificadas. Para a confecção da curva de embebição de água foram utilizadas 6 repetições de 10 sementes cada, que foram colocadas em copos plásticos descartáveis e embebidas em um volume de aproximadamente 250 mL de água destilada. Os períodos de tempo em que as
	sementes ficaram embebidas em água foram: 0, 1, 2, 3, 4, 5, 6, 7, 8, 9, 10, 12, 24, 48, 72 e 96 horas. Após cada período de embebição, as sementes foram retiradas dos copos, secas em papel toalha, para retirada do excesso de água, e pesadas novamente. A partir dos dados obtidos com as pesagens foram traçadas as curvas
	de embebição sendo colocadas novamente nos copos com água e realocadas no germinador, sob temperatura constante de 25$^\circ$C. A partir dos dados obtidos foram
	confeccionadas as curvas de embebição de água, para cada tratamento de sementes. De acordo com resultados obtidos, as curvas de embebição de água para sementes de pinho cuiabano seguem o padrão trifásico de embebição, onde as
	sementes do tratamento testemunha embeberam menos quantidade de agua, enquanto as sementes escarificadas obtiveram maior volume, maior índice de absorção e ganho de peso após a imersão em água em relação às demais classes de tamanho.
	
	\vspace{\onelineskip}
	
	\noindent
	\textbf{Palavras-chave}:Pinho cuiabano. Sementes florestais. Absorção de água.
	
\end{document}
