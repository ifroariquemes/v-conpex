\documentclass[article,12pt,onesidea,4paper,english,brazil]{abntex2}

\usepackage{lmodern, indentfirst, nomencl, color, graphicx, microtype, lipsum}			
\usepackage[T1]{fontenc}		
\usepackage[utf8]{inputenc}		

\setlrmarginsandblock{2cm}{2cm}{*}
\setulmarginsandblock{2cm}{2cm}{*}
\checkandfixthelayout

\setlength{\parindent}{1.3cm}
\setlength{\parskip}{0.2cm}

\SingleSpacing

\begin{document}
	
	\selectlanguage{brazil}
	
	\frenchspacing 
	
	\begin{center}
		\LARGE ISENÇÃO DO ICMS NA IMPORTAÇÃO DE EQUIPAMENTOS MÉDICO-
		HOSPITALAR SEM SIMILAR NACIONAL À LUZ DA TEORIA DOS 4 ES\footnote{Trabalho realizado dentro da área de Conhecimento CNPq: 6.01.00.00-1 DIREITO}
		
		\normalsize
	David Lucas da Silva Ferreira\footnote{Bolsista (modalidade de Ensino Superior), davidlucas1988@hotmail.com, Campus Porto Velho Zona
		Norte} 
	Tony Marcel Lima\footnote{Colaborador (a), tonymarcellima@gmail.com, Universidade Federal de Rondônia} 
	Esiomar Andrade S. Filho\footnote{Colaborador(a), esiomar.silva@ifro.edu.br, Campus Porto Velho Zona Norte} 
	Rwrsilany Silva5\footnote{Orientador(a), rwrsilany.silva@ifro.edu.br, Campus Porto Velho Zona Norte}
	Gustavo Melazi Girardi\footnote{Co-orientador(a), gustavo.girardi@ifro.edu.br, Campus Guajará Mirim}  
	\end{center}
	
	\noindent Com o conhecimento sobre Direito Tributário e suas ligações com a Constituição
	Federal, Código Tributário Nacional e a Lei Complementar n. 24/75 pode-se obter
	uma base quanto à isenção tributária e sua aplicação na importação de equipamento
	médico-hospitalar no Estado de Rondônia, a partir do convênio firmado por meio do
	Conselho Nacional de Política Fazendária. O objetivo deste trabalho é verificar se a
	isenção do ICMS na importação de equipamentos médico-hospitalar no Estado de
	Rondônia obedece a todos os ditames impostos pela legislação tributária para a sua
	concessão. Assim, tem-se como escopo reunir informações acerca do tema a fim de
	produzir conhecimento, além de realizar análise quanto à eficácia, eficiência,
	efetividade e economicidade na Administração Pública Estadual. Nessa perspectiva,
	objetivaremos verificar o impacto das normas que regem a isenção e sua influência
	quanto a Teoria dos 4 Es. A metodologia da pesquisa ponderou em ser qualitativa,
	por não utilizar dados estatísticos; descritiva, com informações não manipuladas;
	com procedimentos da pesquisa classificada como documental, visto a utilização de
	assuntos que não receberam tratamentos, sem deixar de levar em consideração a
	informação de autores, de modo geral. Como resultado verificou-se que a isenção
	atende aos ditames legais para sua obtenção, implicando a resolução da
	problemática da pesquisa, bem como o atendimento a efetividade e economicidade
	em sua integralidade. Todavia, no que diz respeito à eficácia, a norma, quanto ao
	seu entendimento, traz alguns equívocos de conceitos aos leitores, quando faz
	menção à suspensão e compensação (extinção), que são modalidades diferentes da
	isenção (exclusão). Por fim, implicando temerariamente na eficiência dos atos e
	fatos que permeiam a isenção tributária condicionada à prestação de serviços em
	
	\vspace{\onelineskip}
	
	\noindent
	\textbf{Palavras-chave}:Isenção do ICMS. Equipamentos. Teoria dos 4 ES.
	
\end{document}
