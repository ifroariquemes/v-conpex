\documentclass[article,12pt,onesidea,4paper,english,brazil]{abntex2}

\usepackage{lmodern, indentfirst, nomencl, color, graphicx, microtype, lipsum}			
\usepackage[T1]{fontenc}		
\usepackage[utf8]{inputenc}		

\setlrmarginsandblock{2cm}{2cm}{*}
\setulmarginsandblock{2cm}{2cm}{*}
\checkandfixthelayout

\setlength{\parindent}{1.3cm}
\setlength{\parskip}{0.2cm}

\SingleSpacing

\begin{document}
	
	\selectlanguage{brazil}
	
	\frenchspacing 
	
	\begin{center}
		\LARGE AVALIAÇÃO DA EFICÁCIA DE EXTRATOS DE PLANTAS CONSIDERADAS
		PRAGAS NO CONTROLE DE ARACNÍDEOS NO CONE SUL DE RONDÔNIA.\footnote{Trabalho realizado dentro da (área de Conhecimento CNPq: Química Orgânica e Ambiental) com
			financiamento do PROPESP – CICLO 2016-2017 Edital nº 37 - PIP-EM}
		
		\normalsize
	Neiva MOREIRA\footnote{Bolsista Ensino Superior discentes de graduação em agronomia –, e-mail:
		vanessacastromenezes@gmail.comemail, Campus Colorado do Oeste} 
		Vanessa Ferreira de MENEZES\footnote{Bolsista Ensino Médio discentes do curso Técnico em Agropecuário –, e-mail:
			sc.marcelo@hotmail.com,Campus Colorado do Oeste} 
		Paulo Henrique de O. FELIPPE\footnote{Colaborador Paulo Henrique de O. FELIPPE, phof18@gmail.com, Campus Colorado do Oeste} 
	Marcelo
	Calixto\footnote{Orientador(a) Neiva Moreira,neiva.moreira@gmail.com, Campus Colorado do Oeste}
		Rodrigo FRAGUAS\footnote{Co-orientador(a) Rodrigo Fraguas, rodrigofraguas2@gmail.com, Campus Colorado do Oeste} 
	\end{center}
	
	\noindent Objetivou-se neste estudo realizar avaliação da eficácia de extratos de
	plantas consideradas pragas no controle de aracnídeos no cone sul de Rondônia, os
	trabalhos que avaliam as atividades de plantas brasileiras com ação inseticida, ainda
	são considerados escassos, principalmente na avaliação contra artrópodes, mas há
	o reconhecimento pela população do poder de extratos de plantas na diminuição ou
	mesmo eliminação desses parasitas. Assim a partir desses conhecimentos foram
	determinar as espécies de vegetais com potencial inseticida e ou acaricida; realizar
	testes buscando “in vitro” a atividade sobre aracnídeos. Identificadas plantas
	consideradas pragas e selecionados aqueles tem potencial que estão situados em
	diferentes pontos, na região de Colorado do Oeste-RO, foram realizados bioensaios
	para testa eficácia dos vegetais sobre os Amblyomma cajennense. Observou-se que
	os resultados encontrados mostram que os bioensaios mostraram, que os extratos
	se comportam diferentes ao longo do tempo. O que permite, identificar que os
	extratos vegetais com ação contra artrópodes mais eficientes foram: Piper aduncum,
	Morinda citrifolia, que apresentaram eficiência acima de 80\%, mas faz necessário
	novos estudos sob diferentes abordagens, entretanto, é necessária a
	complementação dos dados com análises clínicas, toxicológicas e fitoquímicas para
	a validação de experimentos “in vivo”. Os dados compilados neste trabalho poderão
	nortear a elaboração e o registro de novos produtos anti-carrapaticidas para animais
	nas entidades competentes. Espera-se que estes dados científicos alcancem o
	produtor rural, levando consigo os benefícios econômicos, ambientais e de saúde
	pública, advindos da utilização de métodos de controle anti-carrapaticidas
	convencionais.
	
	\vspace{\onelineskip}
	
	\noindent
	\textbf{Palavras-chave}: Morinda citrifolia, Amblyomma cajennense, in vivo.
	
\end{document}
