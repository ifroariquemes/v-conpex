\documentclass[article,12pt,onesidea,4paper,english,brazil]{abntex2}

\usepackage{lmodern, indentfirst, nomencl, color, graphicx, microtype, lipsum,textcomp}			
\usepackage[T1]{fontenc}		
\usepackage[utf8]{inputenc}		

\setlrmarginsandblock{2cm}{2cm}{*}
\setulmarginsandblock{2cm}{2cm}{*}
\checkandfixthelayout

\setlength{\parindent}{1.3cm}
\setlength{\parskip}{0.2cm}

\SingleSpacing

\begin{document}
	
	\selectlanguage{brazil}
	
	\frenchspacing 
	
	\begin{center}
		\LARGE TRANCAMENTO DE MATRÍCULA: A (IM)POSSIBLIDADE DE CONTENÇÃO DA EVASÃO\footnote{Trabalho realizado dentro da área: 7.08.02.00-9 Administração Educacional com financiamento do IFRO – Campus Ariquemes.}
		
		\normalsize
		Vitória Santos\footnote{Bolsista (PIP), vivisantos1664@gmail.com, Campus Ariquemes.} 
	Danilo Pereira Escudero\footnote{Colaborador, danilo.escudero@ifro.edu.br, Campus Ariquemes.} 
	Quezia da Silva Rosa\footnote{Orientadora, quezia.rosa@ifro.edu.br, Campus Ariquemes.} 
	Francyelle Ruana Farias da Silva5\footnote{Co-orientadora, franruana@gmail.com, Campus Ariquemes.} 
	\end{center}
	
	\noindent Entende-se por evasão relação entre os alunos que entram e os alunos que concluem com êxito um determinado curso. As principais causas desta diferença são condições socioeconômicas, a inadequação do horário de trabalho e do estudo. Na Rede Federal de Educação esse é um problema incomodo, uma vez que alto investimento é feito em estrutura e pessoal e existe uma grande dificuldade em estancar as altas taxas de evasão apresentadas no ensino profissionalizante. Quando se trata de cursos subsequentes, as causas acima elencadas se agravam ainda mais. No IFRO, um dos cursos que tem apresentado evasão elevada é o curso Técnico em Aquicultura subsequente do Campus Ariquemes. Uma oportunidade para contornar a questão da evasão acontece no trancamento da matrícula, uma vez que nesse momento é possível conhecer a motivação do aluno, e tentar sanar suas necessidades. Esta pesquisa espera quantificar as taxas de evasão escolar das turmas ofertadas e identificar o número de alunos que trancaram o curso. Com isso será possível avaliar se havia ou não possibilidade de atuar contra a evasão nesse momento da vida escolar do aluno. Foi adotada a pesquisa de campo do tipo exploratória. Em relação à amostra, foi realizado um senso. Como técnica para coleta de dados utilizou-se a análise nos documentos que foram solicitados junto à Coordenação de Registros Acadêmicos (CRA). Os dados foram processados e analisados pelos próprios pesquisadores, utilizando apenas planilha eletrônica. A turma 2010/2 – vespertino, teve a menor taxa de evasão (22\%) e nenhuma matrícula foi trancada. A turma 2011/2 – noturno teve taxa de evasão de 33\% e uma matrícula trancada. A turma 2012/2 – vespertino teve 34\% de evasão registrada e quatro matriculas trancadas. A turma 2013/1 – matutino e a turma 2014/1 – matutino, obtiveram taxas de evasão, com 80\% e 85\% respectivamente e não houve nenhum trancamento de matrícula. Conclui-se atuar apenas quando o aluno tiver intensão de trancar a matrícula não é efetivo, uma vez que ele raramente toma essa decisão. O aluno do subsequente, ao contrário do ensino integrado, simplesmente deixa de frequentar as aulas. É necessário um trabalho de prevenção, e não de reação.
	
	\vspace{\onelineskip}
	
	\noindent
	\textbf{Palavras-chave}: Evasão. Ensino subsequente . Aquicultura.
	
\end{document}
