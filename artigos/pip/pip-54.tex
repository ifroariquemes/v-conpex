\documentclass[article,12pt,onesidea,4paper,english,brazil]{abntex2}

\usepackage{lmodern, indentfirst, nomencl, color, graphicx, microtype, lipsum}			
\usepackage[T1]{fontenc}		
\usepackage[utf8]{inputenc}		

\setlrmarginsandblock{2cm}{2cm}{*}
\setulmarginsandblock{2cm}{2cm}{*}
\checkandfixthelayout

\setlength{\parindent}{1.3cm}
\setlength{\parskip}{0.2cm}

\SingleSpacing

\begin{document}
	
	\selectlanguage{brazil}
	
	\frenchspacing 
	
	\begin{center}
		\LARGE EMERGÊNCIA DE PLÂNTULAS DE JATOBÁ (\textit{Hymenea courbaril}) EM
		DIFERENTES SUBSTRATOS\footnote{Trabalho realizado dentro da área de Conhecimento CNPq: Botânica.}
		
		\normalsize
		Adriana Cristina Turmina,\footnote{Bolsista drickaro@hotmail.com, Campus Ariquemes - RO.} 
		Ady Correa da Costa Oliveira,\footnote{Orientadora ady.oliveira@ifro.edu.br, Campus Ariquemes – RO.} 
	José Fabio Xavier\footnote{Co-orientador fabio.xavier@ifro.edu.br, Campus Ariquemes - RO.} 
	
	\end{center}
	
	\noindent A demanda por madeira está aumentando cada vez mais e as reservas de florestas
	nativas diminuindo preocupantemente, a demora na germinação também contribui para o aumento dos custos de produção de mudas no viveiro, necessitando de mão de obra, irrigação e cuidados por mais tempo. O que nos leva a buscar técnicas mais eficazes para a produção de mudas visando abastecer o mercado de reflorestamento. A espécie estudada foi o jatobá (\textit{Hymenea courbaril}) que apresenta distribuição por todo o país, principalmente na região norte e possui madeira de excelente qualidade, e uma espécie de crescimento rápido para reflorestamento, além de seu fruto ter excelente fonte de nutrientes para o corpo humano. O trabalho foi realizado com objetivo de comparar por meio de teste de emergência, a capacidade de desenvolvimento de plântulas de jatobá, em diferentes tipos de substratos. O experimento foi representado por 3 tratamentos, que foram com 3 repetições cada: Tratamento 1 = Terra; Tratamento 2 = Areia; Tratamento 3 = Substrato (terra, areia e serragem). Para montagem do experimento foram utilizadas 3 repetições de 50 sementes, para cada tratamento, totalizando 150 sementes.
	Foram avaliados: primeira contagem de emergência, aos 9 dias após a montagem do experimento; emergência total, aos 30 dias após a semeadura; tempo médio de
	emergência e índice de velocidade de emergência. A partir dos resultados obtidos nesta pesquisa, constata-se que a melhor opção para a produção de mudas de \textit{Hymenea courbaril} é areia, que proporcionou 90\% de emergência de plântulas e maior índice de velocidade de emergência.
	
	\vspace{\onelineskip}
	
	\noindent
	\textbf{Palavras-chave}: Jatobá. Sementes florestais. Emergência de plântulas.
	
\end{document}
