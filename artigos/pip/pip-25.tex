\documentclass[article,12pt,onesidea,4paper,english,brazil]{abntex2}

\usepackage{lmodern, indentfirst, nomencl, color, graphicx, microtype, lipsum}			
\usepackage[T1]{fontenc}		
\usepackage[utf8]{inputenc}		

\setlrmarginsandblock{2cm}{2cm}{*}
\setulmarginsandblock{2cm}{2cm}{*}
\checkandfixthelayout

\setlength{\parindent}{1.3cm}
\setlength{\parskip}{0.2cm}

\SingleSpacing

\begin{document}
	
	\selectlanguage{brazil}
	
	\frenchspacing 
	
	\begin{center}
		\LARGE CACHORRO-VINAGRE (\textit{Speothos venaticus,} Lund, 1842) EM REMANESCENTE FLORESTAL, ARIQUEMES, RONDÔNIA, BRASIL\footnote{Trabalho realizado dentro das Ciências Biológicas sem fomento de Instituição de Pesquisa.}
		
		\normalsize
		Alysson Rossi dos Santos\footnote{Acadêmico do curso de Ciências Biológicas, alyssonr@hotmail.com, Instituto Federal de Educação, Ciência e Tecnologia de Rondônia - IFRO campus Ariquemes.} 
		Elaine Oliveira Costa de Carvalho\footnote{Orientador, elaine.carvalho@ifro.edu.br.} 
		
	\end{center}
	
	\noindent Um dos animais selvagens de vida livre naturalmente raro e difícil de ser observado em campo é a espécie de canídeo neotropical de nome popular cachorro-vinagre, Speothos venaticus. A antropização de áreas de floresta tem grande impacto sobre a conservação dessa espécie. É considerado como espécie de categoria vulnerável (VU) pelo Ministério do Meio Ambiente, sendo inclusive apontada como extinta em alguns Estados brasileiros. Pode ser encontrado na América Central e América do Sul, onde é o único canídeo a apresentar formação de matilhas nessas regiões. O estudo teve como objetivo realizar inventário da mastofauna em remanescente florestal do IFRO Campus Ariquemes, durante 2015 a 2017. O método adotado foi o registro de imagem em vídeo com equipamento sensível ao movimento (armadilha fotográfica), funcionando a pilhas e memória com capacidade para armazenar até 470 vídeos com duração de 30 segundos/cada. O equipamento foi afixado, na altura de 40 cm distante do solo, em espécie vegetal arborícola de pequeno porte que é utilizada como marcador de trilha por animais silvestres. A leitura da memória e a manutenção do equipamento foram realizadas a cada sete dias. O único registro do cachorro-vinagre ocorreu em horário diurno após 182 dias desde o início do estudo, totalizando um esforço amostral de 4.368 horas-armadilha. A espécie aparece em formação de um grupo composto por quatro indivíduos, sendo um casal adulto junto a um casal de filhotes, com todos aparentando excelente sanidade, onde a época do ano em que ocorreu o registro é a estação das chuvas na Amazônia. Mesmo o fragmento florestal inventariado apresentar ambiente favorável e diversidade de espécies pertencentes à dieta alimentar do cachorro-vinagre, o registro não caracteriza a sua existência permanente no local, enquanto grupo ou parte de uma suposta população, visto a possibilidade de estar utilizando o fragmento apenas como meio de dispersão, o que não diminui a importância da área podendo ser parte considerável na dinâmica populacional desta espécie na região.
	
	\vspace{\onelineskip}
	
	\noindent
	\textbf{Palavras-chave}: Canídeos, Amazônia, IFRO.
	
\end{document}
