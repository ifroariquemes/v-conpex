\documentclass[article,12pt,onesidea,4paper,english,brazil]{abntex2}

\usepackage{lmodern, indentfirst, nomencl, color, graphicx, microtype, lipsum}			
\usepackage[T1]{fontenc}		
\usepackage[utf8]{inputenc}		

\setlrmarginsandblock{2cm}{2cm}{*}
\setulmarginsandblock{2cm}{2cm}{*}
\checkandfixthelayout

\setlength{\parindent}{1.3cm}
\setlength{\parskip}{0.2cm}

\SingleSpacing

\begin{document}
	
	\selectlanguage{brazil}
	
	\frenchspacing 
	
	\begin{center}
		\LARGE PROJETO CARTOGRAFIA EDUCACIONAL: ANÁLISE DE INDICADORES DE CURSOS TÉCNICOS DO IFRO\footnote{Trabalho realizado dentro da área Ciências Humanas/CNPq com financiamento do IFRO/CNPq
			(Edital no 35 PIP/IFRO/CNPq).}
		
		\normalsize
		Reuria da Silva Moreira\footnote{Bolsista (PIBIC/2017), reuria.moreira@ifro.edu.br Campus Porto Velho Zona Norte.} 
		Pedro Paulo Maronari Barros\footnote{Bolsista (PIBIC/2016), pedropaulomaronari@hotmail.com, Campus Porto Velho Calama.} 
		Álvaro Victor de Oliveira Aguiar\footnote{Colaborador, alvarovctoliveira@gmail.com, Campus Porto Velho Calama.} \\
		Rosa Martins Costa Pereira\footnote{Orientadora, rosa.martins@ifro.edu.br, Pró-Reitoria de Desenvolvimento Institucional/Reitoria.}
		Gilberto Paulino da Silva\footnote{Coorientador, Gilberto.paulino@ifro.edu.br, Pró-Reitoria de Desenvolvimento Institucional/Reitoria.} 
	\end{center}
	
	\noindent A pesquisa teve como objetivos analisar indicadores de desempenho
	acadêmico do IFRO no período de 2009 a 2015, sistematizar indicadores de
	eficácia, eficiência, evasão e retenção dos cursos técnicos e de graduação
	(modalidade presencial) do IFRO, elaborar tabelas a partir de dados primários
	extraídos de sistemas, elaborar relatório por indicador, produzir relatório final com
	análise dos indicadores definidos para a pesquisa e calculados pelas instituições da
	Rede Federal de EPCT em cumprimento aos Acórdãos do Tribunal de Contas da
	União. As atividades desenvolvidas, neste Plano de Trabalho, consistiram na
	realização da extração de dados de todos os cursos técnicos do IFRO com ciclos
	concluídos e ofertados no período estudado, elaboração de planilhas modelo,
	realização de cálculos por indicador para cada curso e campus, finalização das
	planilhas, sistematização de dados e elaboração de relatório final. Os resultados
	demonstram tanto a situação institucional geral quanto a especificidade de cada
	campus com relação aos seguintes indicadores: a) Relação Candidato por Vaga
	(RCV), Relação de Ingresso por Matrícula Atendida (RIM), Relação de Concluintes
	por Matrícula Atendida (RCM), Eficiência Acadêmica de Concluintes (EAC) e
	Retenção do Fluxo Escolar (RFE). Em síntese, os resultados indicam que: 1. Na
	maioria das unidades, o quantitativo de matrícula efetivada é maior que o de vagas
	ofertadas, o que, em primeira análise, demonstra que os cursos ofertados nestas
	unidades possuem adequada aderência com os interesses das comunidades locais;
	2. A instituição efetivou mais matrículas do que as vagas ofertadas em todos os
	cursos técnicos no período estudado; há um bom índice de eficiência acadêmica dos
	cursos técnicos da instituição; 3. Há um alto índice de retenção do fluxo escolar em
	alguns cursos. 4. A relação concluintes – matrículas atendidas também deve ser
	melhorada, isto é, a instituição precisa rever processos e práticas que assegurem a
	conclusão dos estudantes e 5. Há uma baixa eficiência acadêmica da maioria das
	unidades no que diz respeito ao número expressivo de desligados + evadidos +
	transferidos no contexto em que o número de finalizações sem êxito é maior que o
	número de concluintes.
	
	\vspace{\onelineskip}
	
	\noindent
	\textbf{Palavras-chave}: IFRO. Indicadores. Cursos Técnicos.
	
\end{document}
