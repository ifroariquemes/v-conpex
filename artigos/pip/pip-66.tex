\documentclass[article,12pt,onesidea,4paper,english,brazil]{abntex2}

\usepackage{lmodern, indentfirst, nomencl, color, graphicx, microtype, lipsum}			
\usepackage[T1]{fontenc}		
\usepackage[utf8]{inputenc}		

\setlrmarginsandblock{2cm}{2cm}{*}
\setulmarginsandblock{2cm}{2cm}{*}
\checkandfixthelayout

\setlength{\parindent}{1.3cm}
\setlength{\parskip}{0.2cm}

\SingleSpacing

\begin{document}
	
	\selectlanguage{brazil}
	
	\frenchspacing 
	
	\begin{center}
		\LARGE EXTRATOS VEGETAIS NO CONTROLE DE FUNGOS FITOPATOGÊNICOS\footnote{
			Trabalho realizado dentro da (área de Conhecimento CNPq: Ciências Agrárias) com financiamento
			do CNPq / IFRO.}
		
		\normalsize
		Isabela Pereira de Souza Schoaba\footnote{Bolsista (PIBIC EM), schoaba2@gmail.com, Campus Ariquemes} 
	Luana Jaguszevski\footnote{Bolsista (PIBIC), luanajaguszevski@gmail.com, Campus Ariquemes} 
	Luciano dos Reis Venturoso\footnote{Orientador, luciano.venturoso@ifro.edu.br, Campus Ariquemes} 
	Lenita Aparecida Conus Venturoso\footnote{
			Co-orientadora, lenita.conus@ifro.edu.br, Campus Ariquemes} 
	\end{center}
	
	\noindent O Brasil tem se destacado como o maior consumidor de agrotóxicos do mundo,
	sendo que o uso contínuo e exagerado dos mesmos, pode gerar malefícios para a
	saúde humana e ineficácia nas safras seguintes. O uso de extratos de plantas vêm
	demonstrando resultados satisfatórios no controle de doenças. Diante do exposto,
	objetivou avaliar o potencial antifúngico de extratos vegetais, individualmente e em
	misturas, no controle dos fungos Rhizoctonia solani, Colletotrichum gloeosporioides
	e Moniliophthora perniciosa. Foram implantados três bioensaios, um para cada
	fitopatógeno, em delineamento inteiramente casualizado com 18 tratamentos e 5
	repetições. Os tratamentos foram compostos pelas plantas individuais (alho,
	arranha-gato, barbatimão, cravo-da-índia, eucalipto, macaé, erva de Santa Maria) e
	
	misturas (alho + cravo-da-índia, alho + arranha-gato, alho + barbatimão, cravo-da-
	índia + arranha-gato, cravo-da-índia + macaé, eucalipto + arranha-gato, eucalipto +
	
	barbatimão, eucalipto + erva de Santa Maria, barbatimão + arranha-gato, macaé +
	erva de Santa Maria) e um tratamento controle, contendo apenas meio de cultura
	BDA. Os extratos foram obtidos a partir da trituração de 20 g do material vegetal de
	cada espécie em 100 ml de água destilada. O material foi filtrado, e o extrato aquoso
	obtido, acondicionado em erlenmayers. Os extratos foram homogeneizados em meio
	BDA fundente, na concentração de 20\%, e vertidos em placas de Petri. Para as
	misturas, a concentração de 20\% foi obtida com 10\% de cada extrato vegetal. Após
	a solidificação do meio, foram transferidos discos de 0,5 cm de diâmetro do micélio
	dos patógenos, no centro das placas, e incubadas a 25oC. Foram analisados o
	crescimento micelial e a porcentagem de inibição do crescimento dos fitopatógenos.
	Os extratos que continham cravo-da-índia obtiveram maior ação sobre todos os
	fungos fitopatogênicos. O extrato de alho também apresentou resultados
	significativos, principalmente para M. perniciosa. O barbatimão utilizado de forma
	individual, obteve resultados satisfatório sobre M. perniciosa. Não houve resultados
	promissores com a utilização das misturas de extratos, pois a inibição verificada no
	crescimento dos fungos pode ser resultado exclusivo do alho e cravo-da-índia.
	
	\vspace{\onelineskip}
	
	\noindent
	\textbf{Palavras-chave}:Colletotrichum gloeosporioides. Rhizoctonia solani. Moniliophthora
	perniciosa.
	
	\noindent
	\textbf{Fonte de Financiamento}:CNPq e Instituto Federal de Rondônia.
	
\end{document}
