\documentclass[article,12pt,onesidea,4paper,english,brazil]{abntex2}

\usepackage{lmodern, indentfirst, nomencl, color, graphicx, microtype, lipsum}			
\usepackage[T1]{fontenc}		
\usepackage[utf8]{inputenc}		

\setlrmarginsandblock{2cm}{2cm}{*}
\setulmarginsandblock{2cm}{2cm}{*}
\checkandfixthelayout

\setlength{\parindent}{1.3cm}
\setlength{\parskip}{0.2cm}

\SingleSpacing

\begin{document}
	
	\selectlanguage{brazil}
	
	\frenchspacing 
	
	\begin{center}
		\LARGE TÍTULO: ADUBAÇÃO E CONDUÇÃO DA LINHAGEM YOSHIMATSU NO CONESUL DE RONDONIA
		
		\normalsize
		DE SANTANA, J. O. C.\footnote{Bolsista, joaoothavio11@gmail.com, Campus Colorado do Oeste} 
		GARCIA, L. Y. K.\footnote{Colaborador, lucasyutaka@hotmail.com, Campus Colorado do Oeste} 
		MACEDO, M. A. A.\footnote{Orientador, marcos.anequine@ifro.edu.br, Campus Colorado do Oeste} 
	DE LIMA, V. G.\footnote{Co-orientador, valdique.lima@ifro.edu.br, Campus Colorado do Oeste} 
	\end{center}
	
	\noindent A cultura do tomate possui grande influência na economia brasileira e também considerada uma atividade de grande risco, além de alta exigência em adubação. A deficiência de cálcio em tomateiros é causadora de necroses na parte inferior do fruto denominadas fundo preto ou podridão apical que inviabilizam o fruto de serem comercializados. A linhagem Yoshimatsu apresenta uma alta suscetibilidade ao aparecimento de fundo preto. O cálcio apresenta interação com outros íons no solo, principalmente o potássio, que pode gerar deficiência de cálcio em proporções desequilibradas. Com a utilização de diferentes doses de adubação de potássio, junto com diferentes manejos de condução foram analisadas suas influências na incidência de fundo preto e produtividades dos tomates na região do cone sul de Rondônia. Com 16 tratamentos, o experimento contava com quatro diferentes dosagens de adubações e quatro tipos de conduções. Foram feitos preparos de mudas de forma convencional em bandejas, pesagem dos adubos de forma separada e adubação em sulcos com incorporação, e adubação com potássio de forma separada para não haver misturas entre os tratamentos. O experimento conduzido com espaçamento de 0,50m x 1,25m, com quatro blocos apresentando 3 linhas por bloco. Houve o transplantio das mudas com 25 dias após a semeadura. Irrigação utilizada foi por sistema de gotejo com tripas nas linhas e tutoramento dos tomates no sistema mexicano passando fitilhos ao lado dos tomateiros seguindo a linha de plantio. Foram realizados monitoramentos dos frutos a fim de identificar frutos com fundo preto. Não houve presença de frutos com presença de deficiência independente dos tratamentos.
	
	\vspace{\onelineskip}
	
	\noindent
	\textbf{Palavras-chave}: Tomate. Podridão apical. Deficiência
	
\end{document}
