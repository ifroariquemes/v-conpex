\documentclass[article,12pt,onesidea,4paper,english,brazil]{abntex2}

\usepackage{lmodern, indentfirst, nomencl, color, graphicx, microtype, lipsum}			
\usepackage[T1]{fontenc}		
\usepackage[utf8]{inputenc}		

\setlrmarginsandblock{2cm}{2cm}{*}
\setulmarginsandblock{2cm}{2cm}{*}
\checkandfixthelayout

\setlength{\parindent}{1.3cm}
\setlength{\parskip}{0.2cm}

\SingleSpacing

\begin{document}
	
	\selectlanguage{brazil}
	
	\frenchspacing 
	
	\begin{center}
		\LARGE AVALIAÇÃO DA ATIVIDADE ANTIMICROBIANA DE EXTRATOS NATURAIS FRENTE A PATÓGENOS DE ORIGEM BOVINA\footnote{Trabalho realizado dentro da Ciências Biológicas}
		
		\normalsize
		Lizianne de Matos Emerick\footnote{Colaborador, email lizianne.emerick@ifro.edu.br, Campus Colorado do Oeste} 
		Ronaldo Julio Silva Rufino\footnote{Colaborador, email ronaldojuliorufino@gmail.com, Campus Colorado do Oeste} 
		Natália Conceição\footnote{Orientadora, email natalia.conceicao@ifro.edu.br, Campus Colorado do Oeste} 
	Camila Budim Lopes\footnote{Co-orientadora, email Camila.lopes@ifro.edu.br, Campus Colorado do Oeste} 
	\end{center}
	
	\noindent A mastite bovina é caracterizada pelo processo inflamatório na glândula mamária e pode ser causada por diferentes tipos de microrganismos, sendo mais frequente às bactérias dos gêneros estreptococos e estafilococos. Esta doença resulta em uma grande perda na produção do rebanho bovino além da qualidade do leite ser afetada negativamente. Dentre os agentes causadores, destaca-se o Staphylococcus aureus que geralmente é tratado com antimicrobianos convencionais aplicados via local ou intramuscular. No entanto os microrganismos estão cada vez mais resistentes aos antimicrobianos de uso corrente, aumentando assim a necessidade de estudos sobre substâncias naturais, extratos obtidos de plantas, como alternativa no tratamento dessa doença infecciosa. Dessa forma o objetivo deste trabalho foi avaliar a atividade de três extratos naturais obtidos pelo método de fervura, contra S. aureus isolados de animais com mastite bovina. Para avaliação da atividade antimicrobiana das plantas foi utilizado o teste de difusão em agar (técnica do poço). Foram avaliadas duas cepas do microrganismo identificadas de acordo com testes bioquímicos convencionais. Os extratos de Dysphania ambrosioides (Erva Santa Maria), Piper aduncun e Piper medium foram preparados a partir das partes aéreas das plantas, pelo processo de maceração e em seguida a decocção. O antimicrobiano comercial enrofloxacina (10\%) e solução salina (0,9\%) foram utilizados como controle positivo e negativo, respectivamente. De acordo com os testes realizados foi possível observar que a Piper mostrou uma boa atividade inibitória com halo de inibição de 18 a 21 mm. Desta forma pode-se dizer que a família Piperaceae apresentou uma boa atividade antimicrobiana podendo assim ser uma possibilidade para o tratamento de mastite bovina, no entanto é válido ressaltar que ainda são necessários estudos mais aprofundados a respeito das propriedades e compostos antimicrobianos presentes nessas plantas.
	
	\vspace{\onelineskip}
	
	\noindent
	\textbf{Palavras-chave}: Mastite. Piperaceae. Staphylococcus aureus. \\
	\textbf{Fonte de Financiamento}: IFRO 
	
\end{document}
