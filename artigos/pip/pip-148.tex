\documentclass[article,12pt,onesidea,4paper,english,brazil]{abntex2}

\usepackage{lmodern, indentfirst, nomencl, color, graphicx, microtype, lipsum,textcomp}			
\usepackage[T1]{fontenc}		
\usepackage[utf8]{inputenc}		

\setlrmarginsandblock{2cm}{2cm}{*}
\setulmarginsandblock{2cm}{2cm}{*}
\checkandfixthelayout

\setlength{\parindent}{1.3cm}
\setlength{\parskip}{0.2cm}

\SingleSpacing

\begin{document}
	
	\selectlanguage{brazil}
	
	\frenchspacing 
	
	\begin{center}
		\LARGE DESIDRATAÇÃO DE PLANTAS FORRAGEIRAS TROPICAIS PARA PRODUÇÃO DE FENO SOB INTERVALOS DE REVOLVIMENTO\footnote{Trabalho realizado dentro da área de Conhecimento CNPq: Ciências Agrárias com financiamento do CNPq.}
		
		\normalsize
		Almir Gabriel Fernandes Vicente\footnote{Bolsista (PIBIC - Af), almir.fernandesvicente@gmail.coml, Campus Colorado do Oeste.} 
	Clebson Valeriano Alcange\footnote{Bolsista (PIBIC - EM), clebson.ifro@gmail.com, Campus Colorado do Oeste.} 
		Wender Mateus Peixoto\footnote{Colaborador, wendermatthew@gmail.com, Campus Colorado do Oeste.} 
		Henrique Pereira dos Reis\footnote{Orientador, rafael.reis@ifro.edu.br, Campus Colorado do Oeste.} 
	\end{center}
	
	\noindent A fenação é o processo de desidratação da forrageira, que parte da umidade inicial de 75 – 80\% até atingir 15 - 20\%, ponto em que as características nutricionais da planta forrageira tendem a ser conservadas. Alguns fatores como as características morfológicas da planta fenada (diâmetro de colmo e relação folha/colmo) que varia entre diferentes forrageiras e as condições climáticas durante a fenação limitam a taxa de desidratação da forrageira. Frente as problemáticas na produção de feno, o uso de revolvimento é recomendado por elevar a perda de água durante a desidratação da forrageira. Neste sentido o presente trabalho objetivou-se avaliar a dinâmica de desidratação e o potencial para produção de feno de diferentes gramíneas dos gêneros Brachiaria, Panicum e Cynodon sob diferentes intervalos de revolvimento. O delineamento experimental foi em blocos casualizados arranjado em esquema fatorial duplo (4x4) com quatro repetições, sendo quatro forrageiras (Marandu, Massai, Mombaça e Tifton-85) e quatro intervalos de revolvimento (sem revolvimento e revolvimento a cada 2, 4 e 6 horas). Para verificar a taxa de desidratação, as amostras foram coletadas a cada duas horas, iniciando a primeira coleta as 06:00 horas da manhã e findando as 18:00 horas do mesmo dia. Foi avaliada a taxa de desidratação, produtividade e porcentagem de matéria seca (MS) e a relação folha/colmo das forrageiras. Não houve efeito significativo (P<0,05) da interação entre o tempo e intervalo de revolvimento e o intervalo de revolvimento avaliado de forma isolada para o Marandu. Para o Mombaça e o Massai o intervalo de revolvimento a cada duas horas foi o que proporcionou melhor taxa da desidratação. O Tifton-85 é a forrageira mais indicada para fenação, obtendo teor de matéria seca superior as demais. O Tifton-85 necessita de menor tempo de desidratação, podendo ser fenada com 8 horas de desidratação. O Mombaça e o Massai podem ser utilizados para fenação, no entanto necessita de revolvimento. O intervalo de revolvimento a cada duas horas proporcionou melhor taxa de desidratação em relação aos demais intervalos testados.
	
	\vspace{\onelineskip}
	
	\noindent
	\textbf{Palavras-chave}: Brachiaria brizantha. Cynodon dactylon. Fenação.
	
\end{document}
