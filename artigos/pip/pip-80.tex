\documentclass[article,12pt,onesidea,4paper,english,brazil]{abntex2}

\usepackage{lmodern, indentfirst, nomencl, color, graphicx, microtype, lipsum}			
\usepackage[T1]{fontenc}		
\usepackage[utf8]{inputenc}		

\setlrmarginsandblock{2cm}{2cm}{*}
\setulmarginsandblock{2cm}{2cm}{*}
\checkandfixthelayout

\setlength{\parindent}{1.3cm}
\setlength{\parskip}{0.2cm}

\SingleSpacing

\begin{document}
	
	\selectlanguage{brazil}
	
	\frenchspacing 
	
	\begin{center}
		\LARGE METODOLOGIAS DE AVALIAÇÃO DE POLÍTICAS PÚBLICAS: UMA REVISÃO
		DA LITERATURA\footnote{Trabalho realizado dentro da área de Conhecimento CNPq: 60200006 ADMINISTRAÇÃO.}
		
		\normalsize
		David Lucas da Silva Ferreira\footnote{Bolsista (modalidade de Ensino Superior), davidlucas1988@hotmail.com, Campus Porto Velho Zona
			Norte} 
		Thiago Pacife\footnote{Orientador (a), thiago.lima@ifro.edu.br, Universidade Federal de Rondônia} 
		Esiomar Andrade S. Filho\footnote{Co-orientador (a), esiomar.silva@ifro.edu.br, Campus Porto Velho Zona Norte} \\
		Rwrsilany Silva\footnote{Colaborador (a), rwrsilany.silva@ifro.edu.br, Campus Porto Velho Zona Norte}
		Gustavo Melazi Girardi\footnote{Colaborador (a), gustavo.girardi@ifro.edu.br, Campus Guajará Mirim} 
	\end{center}
	
	\noindent Entende-se a avaliação de políticas públicas como uma ferramenta de melhoramento
	da gestão pública visto que sua utilização provoca uma maior transparência na
	utilização dos recursos públicos, e, além disso, uma maior exigência por eficácia nas
	atividades executadas. As politicas publicas executadas pelo Estado devem sempre
	ter o intuito de sanar algum problema da sociedade, seja ele social ou econômico, e a
	avaliação vêm exatamente qualificar e quantificar os resultados alcançados pelas
	políticas e verificar se o problema que inicialmente a suscitou foi efetivamente sanado.
	As avaliações são instrumentos para fortalecimento da democracia moderna e um
	instrumento que procura atender as expectativas dos Stakeholders das políticas
	públicas por accountability. Este estudo de revisão da literatura tem por objetivo
	identificar quais são as principais metodologias para avaliação de políticas públicas
	utilizadas pela literatura especializada, assim como, a própria metodologia de
	pesquisa utilizada para realização desses estudos, no período de 2011 a 2016.
	Ademais, tem como intuito o fornecimento de um arcabouço teórico metodológico
	suficiente para a fundamentação de futuras avaliações e estudos acadêmicos sobre
	avaliação de políticas públicas. Neste contexto, a pesquisa do portfólio de trabalhos
	foi realizada com auxílio da ferramenta publish or perish e a amostra utilizada foi de
	noventa e oito trabalhos, dentre mil produções filtradas pelo programa, onde
	setecentos e trinta trabalhos foram eliminados na primeira fase de análise - análise do
	título - e outros cento e sessenta e oito trabalhos foram eliminados na segunda fase,
	por meio da análise de conteúdo. A técnica de revisão bibliográfica utilizada foi a
	revisão sistemática. Os resultados revelam que não há grandes divergências entre
	autores quanto aos modelos de avaliação de políticas públicas, tendo, sua maioria,
	escolhido entre duas vertentes: timing (antes e depois) e função (metas, processos e
	impactos), sendo o método mais utilizado pela literatura, a avaliação por Função. Por
	fim, a grande produção de estudos focados em avaliar políticas públicas refletem de
	forma ideal o momento político e a crescente politização que vem ocorrendo no país,
	onde se busca cada vez mais uma maior prestação de contas por parte do Estado e
	de suas ações. No que concerne à metodologia científica para realização dos estudos avaliativos, constatamos que a maior frequência foram estudos de caso positivistas
	com viés quantitativo.
	
	\vspace{\onelineskip}
	
	\noindent 
	\textbf{Palavras-chave}:Políticas Públicas. Avaliação. Avaliação de políticas públicas.

	
\end{document}
