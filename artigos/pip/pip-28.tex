\documentclass[article,12pt,onesidea,4paper,english,brazil]{abntex2}

\usepackage{lmodern, indentfirst, nomencl, color, graphicx, microtype, lipsum}			
\usepackage[T1]{fontenc}		
\usepackage[utf8]{inputenc}		

\setlrmarginsandblock{2cm}{2cm}{*}
\setulmarginsandblock{2cm}{2cm}{*}
\checkandfixthelayout

\setlength{\parindent}{1.3cm}
\setlength{\parskip}{0.2cm}

\SingleSpacing

\begin{document}
	
	\selectlanguage{brazil}
	
	\frenchspacing 
	
	\begin{center}
		\LARGE CARECAS: HISTÓRIA POR TRÁS DA HISTÓRIA\footnote{Trabalho realizado dentro da área de Ciências Humanas (CNPq), com financiamento do Instituto Federal de Educação, Ciência e Tecnologia de Rondônia, e do CNPq, com o edital nº 38/2016/PROPESP.}
		
		\normalsize
	Rayssa Rossatt de Souza Xavier\footnote{Bolsista (PIP-EM), rayssarossatt@gmail.com, Campus Cacoal.} 
	Elisa Camargo Gomes\footnote{Colaborador(a), elisacg21@gmail.com, Campus Cacoal.} 
		Shayenny Dias Felicio de Almeida\footnote{Colaborador(a), shayennydias@hotmail.com, Campus Cacoal.} 
		Sérgio Nunes de Jesus\footnote{Orientador(a), sergio.nunes@ifro.edu.br, Campus Cacoal.} 
	\end{center}
	
	\noindent O presente projeto de pesquisa tem como ancoragem a análise do rock tradicional, que passou por uma forte ruptura, originando o punk rock, conceituado como uma manifestação cultural e social, concatenado com o rock psicodélico, que emergiu no início da década de 1970 e trouxe à sociedade juvenil urbana elementos como simplicidade e agressividade, disseminadas por meio da música e atitudes desprezadas pela ideologia da época. Atitudes estas que, no Reino Unido, culminaram em uma revolução musical e cultural caótica sob a tensão da Guerra do Vietnã. Consequentemente, bandas oriundas de movimentos que ocorreram na época acima citada influenciaram diversas partes do mundo. Sabe-se que, devido a esse fator histórico, ao chegar ao Brasil, se destoaram e originaram três novas vertentes: os Carecas do ABC, Carecas do Subúrbio e Carecas do Brasil; inicialmente, representavam a cultura proletária e suburbana. Para tanto, o trabalho foi desenvolvido nos encontros dos grupos de pesquisa Língua(gem), cultura e sociedade: saberes e práticas discursivas na Amazônia, sob a orientação do professor Sérgio Nunes de Jesus, com as alunas do 3º ano, do curso Técnico em Agroecologia Integrado ao Ensino Médio, do Campus Cacoal, do Instituto Federal de Educação, Ciência e Tecnologia de Rondônia/IFRO, com o projeto PDA 2016, com base em pesquisas bibliográficas e de campo. Partindo da perspectiva cronológica que remete aos punks e a emergência dos Carecas do ABC, Carecas do Subúrbio e Carecas do Brasil, é válido ressaltar que, a imigração dos dialetos, costumes e ideologias neonazistas para o Brasil desencadearam alguns dos movimentos mais expressivos da história nacional, uma vez que incitaram a juventude trabalhadora a iniciarem uma sociedade fora dos padrões que buscava, antes de tudo, criticar o elemento social, a política e, culturalmente, o sistema brasileiro. Assim sendo, reconhece-se que há grande influência na liberdade de expressão na sociedade contemporânea.
	
	\vspace{\onelineskip}
	
	\noindent
	\textbf{Palavras-chave}: Movimentos. Música. Ideologia. \\
	\textbf{Fonte de financiamento}: IFRO e CNPq.
	
\end{document}
