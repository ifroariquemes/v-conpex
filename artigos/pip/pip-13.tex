\documentclass[article,12pt,onesidea,4paper,english,brazil]{abntex2}

\usepackage{lmodern, indentfirst, nomencl, color, graphicx, microtype, lipsum}			
\usepackage[T1]{fontenc}		
\usepackage[utf8]{inputenc}		

\setlrmarginsandblock{2cm}{2cm}{*}
\setulmarginsandblock{2cm}{2cm}{*}
\checkandfixthelayout

\setlength{\parindent}{1.3cm}
\setlength{\parskip}{0.2cm}

\SingleSpacing

\begin{document}
	
	\selectlanguage{brazil}
	
	\frenchspacing 
	
	\begin{center}
		\LARGE AVALIAÇÃO DA QUALIDADE DAS SEMENTES DE Enterolobium timbouva\footnote{Trabalho realizado dentro da área da ciência agrária com financiamento Instituto Federal de Ensino,
			Ciência e Tecnologia – IFRO}
		
		\normalsize
		Laryssa Loren Tiburço\footnote{Bolsista, laryssa.tiburco@gmail.com, Campus Ji-Paraná} 
		Maria Elessandra Araujo\footnote{Orientadora, elessadra.cg@gmail.com, Campus Ji-Paraná} 
	Andreza Pereira Mendonça\footnote{Co-orientadora, Mendonça.andreza@gmail.com, Campus Ji-Paraná} 
		 
	\end{center}
	
	\noindent A espécie Enterolobium timbouva é uma alternativa econômica em algumas regiões,
	pois sua madeira leve pode ser utilizada para fabricação de barcos e canoas, além
	da possibilidade de utilização na recuperação de áreas degradadas. No entanto a
	multiplicação dessa espécie é por meio de sementes, apesar disso, trabalhos
	experimentais enfocando as condições ideais de armazenamento de suas sementes,
	são quase inexistentes. A determinação da viabilidade de sementes de Enterolobium
	timbouva pelo teste de germinação é relativamente demorada, especialmente se for
	necessário superar a dormência, o que estenderia o período de avaliação para, no
	mínimo, 15 dias, diante disso o teste de tetrazólio tem se mostrado uma alternativa
	promissora pela qualidade e rapidez na determinação da viabilidade e do vigor das
	sementes. Neste contexto este projeto foi realizado com o objetivo do emprego do
	teste bioquímico de tetrazólio. O presente trabalho foi conduzido no laboratório de
	sementes e viveiros, e Laboratório de Biologia do Instituto Federal de Rondônia -
	Campus, Ji-Paraná. Foram utilizadas sementes de Enterolobium timbouva coletadas
	em áreas circunvizinhas ao município de Ji-Paraná. Foram avaliadas diferentes
	condições de preparo e coloração das sementes, analisando os aspectos dos
	tecidos, bem como a intensidade e uniformidade de coloração. A eficiência das
	várias condições empregadas para o teste de tetrazólio foi avaliada comparando os
	resultados destas com os do teste de germinação. O delineamento experimental
	utilizado nas diferentes etapas foi o inteiramente casualizado com quatro repetições.
	O software utilizado na análise foi o ASSISTAT, e as médias, após análise de
	variância, comparadas pelo teste de Tukey a 5\% de probabilidade. Comparando os
	diferentes tratamentos (concentração e períodos) com a germinação verificou-se
	diferença significativa, em os valores de viabilidades avaliados nos tratamentos
	(concentração e períodos) foram muito inferiores ao teste de germinação.
	Numericamente observa-se que o valor mais próximo da taxa de germinação (98\%),
	obteve-se quando as sementes foram imersas na solução de tetrazólio na
	concentração de 0,01\% pelo período de 8 horas (75\%), mostrando-se pouco
	eficiente para avaliar a viabilidade de sementes de Enterolobium timbouva.
	
	\vspace{\onelineskip}
	
	\noindent
	\textbf{Palavras-chave}: Enterolobium timbouva, Tetrazólio, Avaliação.
	
\end{document}
