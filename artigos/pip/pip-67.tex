\documentclass[article,12pt,onesidea,4paper,english,brazil]{abntex2}

\usepackage{lmodern, indentfirst, nomencl, color, graphicx, microtype, lipsum}			
\usepackage[T1]{fontenc}		
\usepackage[utf8]{inputenc}		

\setlrmarginsandblock{2cm}{2cm}{*}
\setulmarginsandblock{2cm}{2cm}{*}
\checkandfixthelayout

\setlength{\parindent}{1.3cm}
\setlength{\parskip}{0.2cm}

\SingleSpacing

\begin{document}
	
	\selectlanguage{brazil}
	
	\frenchspacing 
	
	\begin{center}
		\LARGE A MUSICOTERAPIA NO TRATAMENTO DE DOENÇAS MENTAIS: DA GÊNESE
		
	Geotecnologias no Ensino Médio: A inserção do software Google Earth no
	ensino de Geografia no Instituto Federal de Rondônia. Campus Cacoal\footnote{Trabalho realizado dentro da área de Conhecimento CNPq: 7.08.00.00-6. Geografia. Não houve
		financiamento.}
		
		\normalsize
	Bianca Rafaella Marques de Andrade\footnote{Aluna do Curso de Técnico em Agropecuária, no Campus Cacoal, do Instituto Federal de Educação,
		Ciência e Tecnologia de Rondônia – IFRO. E-mail: biancarafaella181@gmail.com.} 
	Geovana Oliveira da Silva3\footnote{
		Aluna do Curso de Técnico em Agropecuária, no Campus Cacoal, do Instituto Federal de Educação,
		Ciência e Tecnologia de Rondônia – IFRO. E-mail: gioverapb045@outlook.com}
	Jenifer Laurinda dos Anjos
	
	Oliveira\footnote{Aluna do Curso de Técnico em Agropecuária, no Campus Cacoal, do Instituto Federal de Educação,
		Ciência e Tecnologia de Rondônia – IFRO. E-mail: jenifer.laurinda@gmail.com} 
	Maria Clara Adorno Aram\footnote{Aluna do Curso de Técnico em Agropecuária, no Campus Cacoal, do Instituto Federal de Educação,
		Ciência e Tecnologia de Rondônia – IFRO. E-mail: clara\_aran18@icloud.com.} 
	Ayrton Schupp Pinheiro Oliveira\footnote{Professor/Orientador de Geografia no Campus Cacoal, do Instituto Federal de Educação, Ciência e
		Tecnologia de Rondônia – IFRO. E-mail: ayrtonspoliveira@gmail.com.} 
	 
	\end{center}
	
	\noindent Entender o espaço no campo da ciência geográfica é de suma importância para
	pensarmos as relações sociais sobre o principal objeto de estudo da Geografia. Com
	o advento das tecnologias sobre a sociedade moderna na década de 70 até o
	período contemporâneo, especialmente como comenta Milton Santos, no período
	técnico cientifico informacional, a humanidade vem experimentando um novo modo
	de vivência a partir dos avanços tecnológicos que são introduzidos constantemente
	sobre espaço, e sendo um destes que compõem os dispositivos tecnológicos, a
	localização e a possibilidade de ver o mundo a partir de um clique, outros países e
	localidades. Uma destas tecnologias se encontra no Software livre Google Earth. O
	objetivo da pesquisa visa aplicar de forma qualitativa e apresentar dados da
	aplicação do Google Earth sobre o ensino de Geografia no Instituto Federal de
	Rondônia/Campus Cacoal. O estudo contou com a participação das discentes do 2o
	ano do Curso Integrado em Agroecologia. Na metodologia foram aplicadas leituras
	bibliográficas sobre o tema das Geotecnologias, e aplicação do software aos alunos
	foi executado em um minicurso na Semana Agrotecnológica do ano corrente e
	também na disciplina de Geoprocessamento do 3o ano do Curso de Agroecologia,
	sendo feitos trabalhos envolvendo, localização, contexto histórico, topografia da
	região e delimitação de áreas. A aplicação das geotecnologias, qualitativamente
	apresentou resultados satisfatórios. Com a produção de mapas dos mais variados
	assuntos aplicados no campo da Geografia, demonstrando o grau de aceitação do
	Google Earth para o ensino de Geografia, em ambiente escolar, e como a sua
	aplicação pode ser progressivamente utilizado para determinar os conteúdos de
	ensino no campo geográfico com mais precisão e entendimento dos alunos.
	
	\vspace{\onelineskip}
	
	\noindent
	\textbf{Palavras-chave}:Educação. Geotecnologias. Google Earth.
	
\end{document}
