\documentclass[article,12pt,onesidea,4paper,english,brazil]{abntex2}

\usepackage{lmodern, indentfirst, nomencl, color, graphicx, microtype, lipsum}			
\usepackage[T1]{fontenc}		
\usepackage[utf8]{inputenc}		

\setlrmarginsandblock{2cm}{2cm}{*}
\setulmarginsandblock{2cm}{2cm}{*}
\checkandfixthelayout

\setlength{\parindent}{1.3cm}
\setlength{\parskip}{0.2cm}

\SingleSpacing

\begin{document}
	
	\selectlanguage{brazil}
	
	\frenchspacing 
	
	\begin{center}
		\LARGE PERFIL ANTROPOMÉTRICO E BIOQUÍMICO DOS ESTUDANTES DO IFRO
		CAMPUS PORTO VELHO ZONA NORTE\footnote{Trabalho realizado dentro da área de Conhecimento CNPq: Ciências da saúde com financiamento
			do Instituto Federal de Rondônia IFRO.}
		
		\normalsize
		Nathalia Thais Bukoski Silva\footnote{Bolsista PIBIC-EM, naty.thais65@gmail.com, Campus Porto Velho Calama} 
		Rodrigo Lopes da Silva\footnote{Colaborador, rodrigoslopes.sl@gmail.com, Campus Porto Velho Zona Norte} 
		Thiago Pacife de Lima\footnote{Orientador, thiago.lima@ifro.edu.br, Campus Porto Velho Calama} 
		
	\end{center}
	
	\noindent De acordo com o perfil dos alunos do Campus Porto Velho Zona Norte, dos 90
	alunos matriculados em 2014/2, foi possível verificar que 14\% possui até 21 anos,
	41\% estão entre 22 a 30 anos, 42\% entre 31 a 50, e 3\% acima de 50 anos,
	demonstrando que são pessoas com mais experiência de vida. Embora seja um
	processo natural, o envelhecimento submete o organismo a diversas alterações.
	Considerando que o IFRO busca contribuir para o pleno desenvolvimento do
	estudante, bem como sua permanência e êxito no processo educativo, promovendo
	o atendimento das necessidades do estudante não apenas no âmbito educacional,
	mas também contemplando aspectos fundamentais como a saúde e, ao considerar
	que o objetivo dos cursos técnicos e tecnológicos é capacitar cidadãos para o
	mercado de trabalho, desenvolver ações que promovam a saúde do estudante
	resultará não apenas em melhores resultados acadêmicos, mas também refletirá no
	desempenho laboral, assim, este trabalho buscou traçar o perfil antropométrico e
	bioquímico dos estudantes do IFRO Campus Zona Norte matriculados entre 2014 e
	2015. Para definição dos dados antropométricos/nutricionais foram realizadas
	aferição de peso, estatura e circunferência da cintura, e cálculo do índice de massa
	corpórea (IMC=kg/m2) com 155 estudantes. Os dados bioquímicos foram coletados
	através dos seguintes exames: Lipidograma, TGO/TGP, EAS, Ureia, entre outros.
	Entre os participantes 60\% eram mulheres e 40\% homens. A idade média foi de 29
	anos numa amplitude de 18 a 51. Os dados do IMC revelaram que os homens
	apresentam melhor índice de massa corpórea (25) em relação às mulheres (27). Em
	relação aos exames bioquímicos os valores encontrados na maioria dos casos
	estavam dentro do esperado, todavia houveram casos que foi necessária orientação
	pela enfermeira para consulta médica e tratamento dos quadros clínicos. No tocante
	ao rendimento acadêmico foi possível correlacionar que os alunos que dormem 8
	horas ou mais e possuem bons hábitos de saúde apresentaram melhor rendimento
	acadêmico em relação aos demais. Embora sejam dados preliminares e hajam
	outras variáveis que poderão ser consideradas, estudos dessa natureza contribuem
	para mostrar que bons hábitos de saúde podem contribuir inclusive para melhor
	rendimento acadêmico.
	
	\vspace{\onelineskip}
	
	\noindent
	\textbf{Palavras-chave}: Perfil. Saúde. Rendimento acadêmico.
	
	\noindent
	\textbf{Fonte de Financiamento}: Instituto Federal de Rondônia – IFRO, Edital n$^{\circ}$
	38/2016/PROPESP/IFRO.
	
\end{document}
