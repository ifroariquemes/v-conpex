\documentclass[article,12pt,onesidea,4paper,english,brazil]{abntex2}

\usepackage{lmodern, indentfirst, nomencl, color, graphicx, microtype, lipsum}			
\usepackage[T1]{fontenc}		
\usepackage[utf8]{inputenc}		

\setlrmarginsandblock{2cm}{2cm}{*}
\setulmarginsandblock{2cm}{2cm}{*}
\checkandfixthelayout

\setlength{\parindent}{1.3cm}
\setlength{\parskip}{0.2cm}

\SingleSpacing

\begin{document}
	
	\selectlanguage{brazil}
	
	\frenchspacing 
	
	\begin{center}
		\LARGE PROPRIEDADES FÍSICAS DO SOLO SOB DIFERENTES SISTEMAS DE MANEJOS EM AMBIENTE AMAZÔNICO\footnote{Trabalho realizado dentro da área de Ciências Agrárias com financiamento do CNPq/ IFRO.}
		
		\normalsize
		Matheus Lorrah Costa de Oliveira 
		Edson Souza Martins Júnior\footnote{Bolsistas (PIBIC EM), barttoxa@gmail.com, edson.souza.ifro@gmail.com, campus Ariquemes} 
		Thainá Alves Andrade\footnote{Colaboradora, thaina.taa@hotmail.com, campus Ariquemes} \\
		Lenita Aparecida Conus Venturoso\footnote{Orientadora, lenita.conus@ifro.edu.br, campus Ariquemes}
		Luciano dos Reis Venturoso\footnote{Co-orientador, luciano.venturoso@ifro.edu.br, campus Ariquemes} 
	\end{center}
	
	\noindent Os sistemas intensivos de exploração agropecuária e as condições edafoclimáticas
	são fatores que proporcionam a degradação do solo, sendo o aumento da densidade
	e da compactação uma das principais consequências do manejo inadequado. O
	grande desafio tem sido desenvolver sistemas capazes de recuperar áreas
	degradada, conciliando conservação ambiental com sustentabilidade econômica.
		Assim, o objetivo do estudo foi avaliar a densidade de um Latossolo Vermelho-
	Amarelo distrófico em diferentes sistemas de manejos do solo em Ariquemes, RO. A
	pesquisa foi realizada no Instituto Federal de Rondônia, campus Ariquemes, e
	conduzida em delineamento inteiramente casualizado, com três repetições. Os
	tratamentos foram constituídos pelos sistemas de manejo do solo: preparo
	convencional (milho), plantio direto (soja), fruticultura solteira (goiaba), fruticultura
	consorciada (cupuaçu e seringueira), floresta plantada (teca), pastagem (braquiária)
	e mata nativa. A densidade do solo foi determinada em cada área, em trincheiras
	representativas, com a coleta de amostras com estrutura preservada de solo em
	anéis volumétricos, nas profundidades de 0,00-0,05; 0,05-0,10; 0,10-0,15; 0,15-0,20;
	0,20-0,25; 0,25-0,30; 0,30-0,35 e 0,35-0,40 m. Observou-se que os sistemas de
	manejos alteraram a densidade do solo nas profundidades estudadas, exceto na
	camada superficial. A mata nativa apresentou as menores densidades nas
	profundidades analisadas. A densidade do solo foi maior em todos os manejos nas
	profundidades 0,05-0,10 m e 0,10-0,15 m evidenciando uma compactação mais
	acentuada nessa camada. Na fruticultura solteira os valores foram superiores na
	maioria das camadas, evidenciando o feito dos tratos culturais que são realizados
	manualmente nessa área. Nas condições do presente estudo, os sistemas de
	manejos avaliados influenciam a densidade do solo.
	
	\vspace{\onelineskip}
	
	\noindent
	\textbf{Palavras-chave}: Densidade do solo. Atributos físicos. Solos amazônicos.
	
\end{document}
