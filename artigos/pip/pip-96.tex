\documentclass[article,12pt,onesidea,4paper,english,brazil]{abntex2}

\usepackage{lmodern, indentfirst, nomencl, color, graphicx, microtype, lipsum}			
\usepackage[T1]{fontenc}		
\usepackage[utf8]{inputenc}		

\setlrmarginsandblock{2cm}{2cm}{*}
\setulmarginsandblock{2cm}{2cm}{*}
\checkandfixthelayout

\setlength{\parindent}{1.3cm}
\setlength{\parskip}{0.2cm}

\SingleSpacing

\begin{document}
	
	\selectlanguage{brazil}
	
	\frenchspacing 
	
	\begin{center}
		\LARGE PROJETO CARTOGRAFIA EDUCACIONAL:
		
		ANÁLISE DE INDICADORES DE CURSOS SUPERIORES DO IFRO\footnote{Trabalho realizado dentro da área Ciências Humanas/CNPq com financiamento do IFRO/CNPq
			(Edital no 35 PIP/IFRO/CNPq).}
		
		\normalsize
		Julian Alves de Queiroz\footnote{Bolsista (PIBIC), julianqueiroz@hotmail.com, Campus Porto Velho Calama.} 
		Rosa Martins Costa Pereira\footnote{Orientadora, rosa.martins@ifro.edu.br, Pró-Reitoria de Desenvolvimento Institucional/Reitoria.} 
		Gilberto Paulino da Silva\footnote{Coorientador, Gilberto.paulino@ifro.edu.br, Pró-Reitoria de Desenvolvimento Institucional/Reitoria.} 
		 
	\end{center}
	
	\noindent Por meio de um estudo exploratório-descritivo, realizou-se essa pesquisa cujo
	procedimento metodológico principal foi a extração de dados e construção de
	tabelas contendo os indicadores. Os relatórios de indicadores foram gerados de
	forma padronizada, pela extração centralizada no MEC e, posteriormente, validados
	com os dados do IFRO. A pesquisa teve como objetivos analisar indicadores de
	desempenho acadêmico do IFRO no período de 2009 a 2015, sistematizar
	indicadores de eficácia, eficiência, evasão e retenção dos cursos técnicos e de
	graduação (modalidade presencial) do IFRO, elaborar tabelas a partir de dados
	primários extraídos de sistemas, elaborar relatório por indicador, produzir relatório
	final com análise dos indicadores definidos para a pesquisa e calculados pelas
	instituições da Rede Federal de EPCT em cumprimento aos Acórdãos do Tribunal
	de Contas da União. As atividades desenvolvidas, neste Plano de Trabalho,
	consistiram na realização da extração de dados de todos os cursos superiores do
	IFRO com ciclos concluídos e ofertados no período estudado. Os dados analisados
	neste trabalho referem-se à extração realizada pelo Sistema Nacional de
	Informações da Educação Profissional e Tecnológica (SISTEC) no dia 26 de
	setembro de 2016 as 16h09min com uma amostra de 9 (nove) cursos de graduação.
	Em síntese, os resultados indicam que: 1. O IFRO disponibilizou 1960 (mil
	novecentos e sessenta) vagas para graduação, tendo matriculado no total 2381
	(dois mil trezentos e oitenta e um) alunos, distribuídos em 1080 (mil e oitenta) vagas
	para licenciatura, 400 (quatrocentos) vagas para bacharelado e 480 (quatrocentos e
	oitenta) vagas para tecnólogo; 2. Nos cursos de graduação, 59\% dos alunos
	matriculados inicialmente acabam desistindo, trancando ou transferindo-se para
	outra instituição; 3. Nos dados consolidados de todas as unidades constatou-se um
	alto índice de retenção do fluxo escolar (30\%) e 4. No IFRO, apenas 11\% dos alunos
	matriculados inicialmente, concluem a graduação.
	
	\vspace{\onelineskip}
	
	\noindent
	\textbf{Palavras-chave}: IFRO. Indicadores. Cursos Superiores.
	
\end{document}
