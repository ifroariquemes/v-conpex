\documentclass[article,12pt,onesidea,4paper,english,brazil]{abntex2}

\usepackage{lmodern, indentfirst, nomencl, color, graphicx, microtype, lipsum}			
\usepackage[T1]{fontenc}		
\usepackage[utf8]{inputenc}		

\setlrmarginsandblock{2cm}{2cm}{*}
\setulmarginsandblock{2cm}{2cm}{*}
\checkandfixthelayout

\setlength{\parindent}{1.3cm}
\setlength{\parskip}{0.2cm}

\SingleSpacing

\begin{document}
	
	\selectlanguage{brazil}
	
	\frenchspacing 
	
	\begin{center}
		\LARGE Comercialização de flores ornamentais no segmento varejista no
		município de Ji-Paraná, RO\footnote{Trabalho realizado dentro da (área de Conhecimento CNPq: Recursos Florestais.}
		
		\normalsize
		Gabriel Alves Abrão\footnote{gabriel.alves.3123@gmail.com Discente do Curso Técnico em Florestas – IFRO – Campus
			Ji-Paraná.} 
	Gustavo de Souza Costa\footnote{gustavodesouza.yt@gmail.com Discente do Curso Técnico em Florestas – IFRO – Campus
		Ji-Paraná.} 
	Janice Ferreira do Nascimento\footnote{Docente do Curso Técnico em Florestas – IFRO – Campus Ji-Paraná,
		janice.nascimento@ifro.edu.br.} 
	
	\end{center}
	
	\noindent Chama-se de plantas ornamentais, aquelas que são utilizadas para enfeitar determinados
	espaços e muitas das vezes, recorrendo à utilização dessas, para trazer o ‘natural’ à ambientes
	urbanos que perderam tal característica. O objetivo deste trabalho foi identificar as espécies de
	plantas ornamentais mais comercializadas no segmento varejista na cidade de Ji-Paraná, RO.
	A coleta de dados foi realizada por meio de entrevista, sendo estas, destinadas aos donos e/ou
	responsáveis dos viveiros e floriculturas, gerando dados coletados de 5 maiores pontos de
	comércio, sendo feita no ano de 2016. De acordo com as informações cedidas pelos
	entrevistados, as espécies de plantas mais procuradas pelos consumidores são: A Rosa do
	deserto (\textit{Adenium sp.}), nativa do continente Africano e de países Árabes que detém 45\% do
	total; seguida pela Rosa comum (\textit{Rosa sp.}), que é a flor de corte mais comercializadas no
	mundo e representa 23\% das compras nas floriculturas Ji-Paranaenses; as Orquídeas
	(\textit{Orchidaceae}) típicas de regiões tropicais, encontradas em todos os continentes, exceto na
	Antártica, apresenta grande variedade de espécies nativas do Brasil e abrange 17\% da procura
	pelos consumidores e por fim a Primavera (\textit{Boutextgainvillea sp.}) é uma planta arbustiva no qual
	sua principal característica de ornamentação não se encontram em suas flores, mas sim em
	um conjunto de folhas de cores vibrantes que se localizam-se em torno das flores, sendo essa
	solicitadas em 10\% dos casos de compra. Mediante os dados apresentados, percebe-se uma
	preferência na procura da rosa do deserto e esse fato é justificado por sua beleza cênica e por
	sua baixa carência de cuidados, já que a mesma é nativa de regiões que apresentam pouca
	disponibilidade de água, desta maneira, não necessita de regas diárias e nem de um substrato
	com alta carga nutricional apresentando também uma grande tolerância à luz solar. Já as rosas
	comuns, por serem tipicamente uma planta de corte e geralmente cultivadas para fins
	decorativos de curto período de duração, só apresentam um aumento na sua compra em datas
	comemorativas.
	
	\vspace{\onelineskip}
	
	\noindent
	\textbf{Palavras-chave}: Paisagismo, Viveiro, Floricultura.
	
\end{document}
