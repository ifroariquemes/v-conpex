\documentclass[article,12pt,onesidea,4paper,english,brazil]{abntex2}

\usepackage{lmodern, indentfirst, nomencl, color, graphicx, microtype, lipsum}			
\usepackage[T1]{fontenc}		
\usepackage[utf8]{inputenc}		

\setlrmarginsandblock{2cm}{2cm}{*}
\setulmarginsandblock{2cm}{2cm}{*}
\checkandfixthelayout

\setlength{\parindent}{1.3cm}
\setlength{\parskip}{0.2cm}

\SingleSpacing

\begin{document}
	
	\selectlanguage{brazil}
	
	\frenchspacing 
	
	\begin{center}
		\LARGE EFEITO DE DOSES DE NITROGÊNIO NA CULTURA DO MILHO\footnote{Trabalho realizado dentro da (área de Conhecimento CNPq: Ciências Agrárias) com financiamento do IFRO, Campus Ariquemes.}
		
		\normalsize
		Lucas Souza Markoviscz,\footnote{Bolsista (IC ET), hiriaqthiariane@gmail.com, Campus Ariquemes.} 
	Anderson Ferreira de Aquino,\footnote{Bolsista (IC ET), anderson.aquino.1999@gmail.com, Campus Ariquemes.} 
		Luciano dos Reis Venturoso,\footnote{Orientador, luciano.venturoso@ifro.edu.br, Campus Ariquemes.} 
		Lenita Aparecida Conus Venturoso\footnote{Co-orientadora, lenita.conus@ifro.edu.br, Campus Ariquemes.} 
	\end{center}
	
	\noindent O nitrogênio é o elemento mineral requerido em maior quantidade para a cultura do milho, e sem o mesmo a planta não consegue expressar todo seu potencial produtivo. No entanto, este nutriente tem custo elevado e alternativas como o uso de inoculantes contendo bactérias fixadoras de nitrogênio, como \textit{Azospirillum} poderiam resultar em uma economia importante para o agricultor. Diante do exposto, objetivou-se avaliar a utilização da bactéria diazotrófica \textit{Azospirillum brasilense} associada a doses crescentes de adubação nitrogenada no desenvolvimento e produtividade de milho na safrinha, assim como o potencial de economia do fertilizante nitrogenado combinado a inoculação. O experimento foi conduzido, em Latossolo Vermelho Amarelo Distrófico, na área experimental do Instituto Federal de Rondônia, campus Ariquemes, em cultivo de safrinha. Foi adotado o delineamento experimental de blocos casualizados, em arranjo fatorial 4 x 5, com quatro repetições. Foi utilizado quatro métodos de inoculação da bactéria A. brasilense: via sementes, foliar, no sulco de plantio e uma testemunha; e cinco doses de fertilizante nitrogenado, 0, 40, 80, 120 e 160 kg.ha$^{-1}$ de N, na forma de ureia. A cultura foi semeada em parcelas contendo quatro linhas de 5 m de comprimento, espaçadas 0,8 m entre si. Foram avaliados os caracteres vegetativos e reprodutivos da cultura. Não foi verificado efeito simples da adubação nitrogenada para os caracteres vegetativos da cultura. Com relação aos componentes do rendimento, houve aumento no comprimento das espigas com o uso da maior dose de nitrogênio. Para o número de grãos por fileira foi observado certa inconsistência dos dados, pois os maiores resultados foram obtidos nas doses de 0, 80 e 160 kg.ha$^{-1}$. Para o rendimento de grãos, destaca-se a maior dose de nitrogênio, na qual proporcionou 3.089,1 kg.ha$^{-1}$ de grãos, ainda que, sem diferir estatisticamente da dose de 120 kg.ha$^{-1}$ de nitrogênio (2.146,8 kg.ha$^{-1}$).
	
	\vspace{\onelineskip}
	
	\noindent
	\textbf{Palavras-chave}: \textit{Zea mays}. Doses de nitrogênio. Bactérias fixadoras de nitrogênio.\\
	\textbf{Fonte de Financiamento}: Instituto Federal de Rondônia, Campus Ariquemes
	
\end{document}
