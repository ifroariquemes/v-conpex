\documentclass[article,12pt,onesidea,4paper,english,brazil]{abntex2}

\usepackage{lmodern, indentfirst, nomencl, color, graphicx, microtype, lipsum}			
\usepackage[T1]{fontenc}		
\usepackage[utf8]{inputenc}		

\setlrmarginsandblock{2cm}{2cm}{*}
\setulmarginsandblock{2cm}{2cm}{*}
\checkandfixthelayout

\setlength{\parindent}{1.3cm}
\setlength{\parskip}{0.2cm}

\SingleSpacing

\begin{document}
	
	\selectlanguage{brazil}
	
	\frenchspacing 
	
	\begin{center}
		\LARGE DISCUSSÕES DE GÊNERO NO ESPAÇO ESCOLAR: UM LEVANTAMENTO DAS PRODUÇÕES CIENTÍFICAS NOS IFS DAS REGIÕES NORTE E NORDESTE\footnote{Trabalho realizado dentro da (área de Conhecimento CNPq: Ciências Humanas) com financiamento da DEPESP, Campus Vilhena.}
		
		\normalsize
		Pedro Ebert Santos\footnote{Colaborador, cephstedel@gmail.com, Campus Vilhena.} 
		Maria Consuêlo Moreira\footnote{Orientadora, maria.moreira@ifro.edu.br, Campus Vilhena.} 
		
	\end{center}
	
	\noindent O presente projeto trabalhou com as discussões referentes a temática de gênero através do levantamento de dados a respeito dos projetos e produções realizadas a cerca desse tema nos IFs das regiões Norte e Nordeste. A pesquisa é de natureza qualiquantitativa, dividida em quatro fases, sendo a primeira baseada em revisão teórica, com leitura e discussão de livros e artigos relacionados à temática de gênero. Na segunda fase, foi feito um levantamento da produção científica nos campi dos IFs do Norte e Nordeste, com pesquisa online e em anais de eventos. A terceira fase compreende a tabulação e interpretação dos dados coletados. Por fim, na quarta fase os dados coletados foram utilizados para a produção de um artigo científico encaminhado para publicação e exposição em eventos. Com a presente pesquisa, pretende-se criar meios para que sejam desenvolvidos ações na busca da igualdade e equidade nas relações de gênero dentro do IFRO, Campus Vilhena. Também acredita-se que a pesquisa servirá para o desenvolvimento do senso crítico dos alunos, através da desconstrução de equívocos e visões errôneas relacionados a questão de gênero. Foi analisado que os IFs da região nordeste tem produzido e publicado uma maior quantidade de trabalhos cuja temática seja relacionada a gênero. Vale salientar que levou-se em consideração em se ter uma maior quantidade de IFs no nordeste (186 campi) em relação a região norte (65 campi). Em sua maioria, os artigos encontrados durante a pesquisa foram de publicações nas edições do CONNEPI, outros trabalhos foram publicados em edições do PRÊMIO CONSTRUINDO A IGUALDADE DE GÊNERO.
	
	\vspace{\onelineskip}
	
	\noindent
	\textbf{Palavras-chave}: Gênero. Sexualidade. Escola.
	
\end{document}
