\documentclass[article,12pt,onesidea,4paper,english,brazil]{abntex2}

\usepackage{lmodern, indentfirst, nomencl, color, graphicx, microtype, lipsum}			
\usepackage[T1]{fontenc}		
\usepackage[utf8]{inputenc}		

\setlrmarginsandblock{2cm}{2cm}{*}
\setulmarginsandblock{2cm}{2cm}{*}
\checkandfixthelayout

\setlength{\parindent}{1.3cm}
\setlength{\parskip}{0.2cm}

\SingleSpacing

\begin{document}
	
	\selectlanguage{brazil}
	
	\frenchspacing 
	
	\begin{center}
		\LARGE ENTRE O REGIONAL E O GLOBAL: O DESAFIO DO CONTROLE DO ZIKA VÍRUS NA FRONTEIRA AMAZÔNICA\footnote{Trabalho realizado dentro da área Ciências Humanas, com financiamento do CNPq e PROPESP/IFRO.}
		
		\normalsize
	Brenda Gomes Romão,\footnote{Bolsista (PIBIC – EM), brendagms06@gmail.com Campus Porto Velho Calama.} 
		Xênia de Castro Barbosa\footnote{Orientadora, xenia.castro@ifro.edu.br Campus Porto Velho Calama.} 
	
	\end{center}
	
	\noindent Este resumo visa comunicar os resultados da pesquisa “Entre o Regional e o Global: um estudo dos desafios pertinentes à emergência do Zika Vírus na Amazônia brasileira”, desenvolvida no Núcleo de Estudos Históricos e Literários do IFRO (NEHLI-IFRO), no período de agosto de 2016 a julho de 2017. O estudo tratou dos desafios impostos pelo vírus da Zika (ZIKV) sobre a Amazônia Legal, com ênfase sobre a problemática da vigilância epidemiológica nas fronteiras. O objetivo geral foi analisar os desafios impostos pelo Zika vírus às políticas públicas de saúde na Região Amazônica. O estudo foi desenvolvido em abordagem qualitativa e configura-se como pesquisa bibliográfico-documental, inserindo-se no campo da História das Paisagens. Sua contribuição consiste em apresentar dados sistemáticos e atuais sobre a Zika, problema recente e ainda pouco estudado nas perspectivas históricas e geográficas. A principal base teórica referencia-se em Barcellos (2009), Haesbaert (1999), Becker (1985), Valle, Pimenta e Aguiar (2016). A Zika é favorecida pelo trânsito de pessoas e produtos em transportes intermodais constituindo-se um desafio adicional à gestão das fronteiras. O estudo alerta para a necessidade de aperfeiçoamento do sistema de vigilância epidemiológica na fronteira Amazônica e o desenvolvimento de políticas públicas integradas, que contemplem não apenas o combate ao vetor, mas a promoção integral da saúde. Aponta ainda para a Zika enquanto problema complexo, típico da modernidade (na qual as promessas de desenvolvimento e prosperidade se viram limitadas a determinadas regiões do planeta, não se efetivando de modo global).
	
	\vspace{\onelineskip}
	
	\noindent
	\textbf{Palavras-chave}: Zika. Amazônia Legal. Fronteira. \\
	\textbf{Fonte de financiamento}: CNPq. PROPESP/IFRO.
	
\end{document}
