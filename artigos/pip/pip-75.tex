\documentclass[article,12pt,onesidea,4paper,english,brazil]{abntex2}

\usepackage{lmodern, indentfirst, nomencl, color, graphicx, microtype, lipsum}			
\usepackage[T1]{fontenc}		
\usepackage[utf8]{inputenc}		

\setlrmarginsandblock{2cm}{2cm}{*}
\setulmarginsandblock{2cm}{2cm}{*}
\checkandfixthelayout

\setlength{\parindent}{1.3cm}
\setlength{\parskip}{0.2cm}

\SingleSpacing

\begin{document}
	
	\selectlanguage{brazil}
	
	\frenchspacing 
	
	\begin{center}
		\LARGE LEVANTAMENTO DE EPÍFITAS VASCULARES NA ZONA RURAL DE
		
		COLORADO DO OESTE, RONDÔNIA\footnote{Trabalho realizado dentro da área de Conhecimento CNPq: Ciências Biológicas}
		
		\normalsize
		Jackeline dos Santos Gonçalves\footnote{Acadêmica de Ciências Biológicas, jacksantosgoncalves@gmail.com, Campus Colorado do Oeste} 
		Dhara Thays Lopes\footnote{Acadêmica de Ciências Biológicas, dharathayslopea@gmail.com, Campus Colorado do Oeste} 
		Ranieli dos Anjos de Souza Muler\footnote{Orientador(a), ranieli.muler@ifro.edu.br, Campus Colorado do Oeste} 
		
	\end{center}
	
	\noindent As epífitas são plantas que para sobreviver necessitam se hospedar em outros
	vegetais. Por diversas razões são consideradas excelentes bioindicadores quanto ao
	grau de preservação do ambiente local, principalmente por possuírem a capacidade
	de reter partículas da atmosfera, especialmente as poluidoras. O hábito de vida
	desta categoria de plantas caracteriza-se pelo comensalismo entre vegetais, no qual
	uma espécie dependente (a epífita) se beneficia do substrato proporcionado pela
	outra (o forófito), retirando nutrientes diretamente da atmosfera, sem desenvolver
	estruturas parasitárias. Este trabalho teve por objetivo realizar um levantamento
	florístico de espécies epífitas em um fragmento florestal do município de Colorado
	do Oeste/Rondônia, em uma área de 2,84 hectares caracterizada pela formação
	vegetal do tipo Estacional Semidecidual em condições antropizadas. Para a
	obtenção dos dados para o levantamento florístico foi utilizado o método do
	caminhamento durante o mês de junho de 2017, em que as espécies registradas
	foram classificadas segundo as categorias de posicionamento no estrato e
	calculados o índice de diversidade de Shannon e de equabilidade de Pielou. Neste
	estudo, foram encontrados 10 indivíduos epífitos pertencentes às famílias
	Orchidaceae (4 ind.), Bromeliaceae (3) e Polypodiaceae (2), perfazendo uma
	densidade de 3,52 ind.ha-1
	
	. Os espécimes identificados no herbário do Instituto
	Federal de Rondônia campus Colorado do Oeste foram Aspasia variegata Lindl.,
	Oeceoclades maculata Lindley, 1821, Lockhartia imbricata (Lam.) Hoehne e
	Phlebodium araneosum (M.Martens \& Galeotti) Mickel \& Beitel, sendo que três
	indivíduos foram identificados apenas em nível de família (Bromeliaceae). A
	classificação em categorias ecológicas apontou a predominância de holoepífitas
	características, cujas espécies predominaram no fuste alto do forófito. O índice de
	diversidade Shannon estimado na amostragem foi de H’= 1,277 nats, o que indica
	baixa diversidade florística de epífitas na área de estudo quando comparado com
	outros trabalhos. A equabilidade de Pielou (J= 0,9212) demonstrou homogeneidade
	na abundância das espécies. Os resultados apontaram uma baixa riqueza e uma
	baixa distribuição vertical que pode estar associada a perturbações causadas pela
	ação antrópica, a alta incidência de luz solar sobre a área e à fase seca
	característica do clima local.
	
	\vspace{\onelineskip}
	
	\noindent
	\textbf{Palavras-chave}: Epífitas. Rondônia. Diversidade.
	
\end{document}
