\documentclass[article,12pt,onesidea,4paper,english,brazil]{abntex2}

\usepackage{lmodern, indentfirst, nomencl, color, graphicx, microtype, lipsum}			
\usepackage[T1]{fontenc}		
\usepackage[utf8]{inputenc}		

\setlrmarginsandblock{3cm}{3cm}{*}
\setulmarginsandblock{3cm}{3cm}{*}
\checkandfixthelayout

\setlength{\parindent}{1.3cm}
\setlength{\parskip}{0.2cm}

\SingleSpacing

\begin{document}
	
	\selectlanguage{brazil}
	
	\frenchspacing 
	
	\begin{center}
		\LARGE A IMPORTÂNCIA DA LEITURA E ESCRITA PARA A INTERPRETAÇÃO\footnote{Trabalho realizado na área de conhecimento: Linguística, letras e Artes (8.00.00.00-2 com
			financiamento de recursos do DEPESP/IFRO-Campus Vilhena.}
		
		\normalsize
		Layra Fabian Borba Rodrigues\footnote{Aluna bolsista do curso Técnico em Eletromecânica Integrado ao Ensino Médio –
			layrafab@gmail.com - IFRO- campus Vilhena.} 
		Adriani Pereira de Lima\footnote{Aluna bolsista de Licenciatura em Matemática – adrianilima@hotmail.com - IFRO – campus Vilhena.} 
		Rosiane Alves\footnote{Aluna colaboradora de Licenciatura em Matemática – rosianealves44@gmail.com - IFRO – campus
			Vilhena.} 
		Sandra Aparecida Fernandes Lopes Ferrari\footnote{Professora Orientadora do projeto de pesquisa – sandra@ifro.edu.br - IFRO- campus Vilhena.} 
	\end{center}
	
	\noindent Este trabalho destina-se a apresentar os resultados alcançados com a pesquisa
	intitulada Do Ensino Médio ao Ensino Superior: perspectivas sobre a importância de
	leitura e escrita para a interpretação matemática, realizada no IFRO, campus
	Vilhena, projeto aprovado pelo Edital 14/2016/DEPESP. O projeto fundamentou-se
	nos postulados dos Novos Estudos de Letramento, especificamente nas concepções
	de Mary Lea e Brian Street (2013) sobre ensino superior e letramentos acadêmicos,
	como também no pensamento de BRITO, (2003) para quem o letramento é visto
	como ensino da leitura, da redação, dos usos, do saber sobre a língua e dos valores
	que se constitui sobre ela. No IFRO, essa pesquisa e uma anterior a ela, feitas com
	alunos da licenciatura em matemática e dos cursos integrados ao ensino médio,
	respectivamente, mostraram que tais alunos não têm o hábito de ler ou escrever
	regularmente. Concomitante a isso, viu-se nos resultados de provas diagnósticas a
	eles aplicadas, que os mesmos possuem dificuldades de interpretação de problemas
	práticos. Entende-se, portanto, que a leitura e a escrita influenciam diretamente no
	desempenho acadêmico. A partir dos resultados alcançados com essas pesquisas,
	serão feitas propostas de intervenção que busquem suscitar nos alunos, tanto nos
	de ensino médio como nos de graduação, o gosto pela leitura, de modo que ler
	torne-se um hábito prazeroso, e não doloroso para eles. Com isso, será possível
	desenvolver a capacidade de interpretação e escrita, influenciando positivamente a
	aprendizagem e conhecimento. A pesquisa pode fazer com que as escolas e
	universidades tenham alunos muito mais prósperos.
	
	\vspace{\onelineskip}
	
	\noindent
	\textbf{Palavras-chave}: Letramento; Leitura; Escrita.
	
\end{document}
