\documentclass[article,12pt,onesidea,4paper,english,brazil]{abntex2}

\usepackage{lmodern, indentfirst, nomencl, color, graphicx, microtype, lipsum}			
\usepackage[T1]{fontenc}		
\usepackage[utf8]{inputenc}		

\setlrmarginsandblock{2cm}{2cm}{*}
\setulmarginsandblock{2cm}{2cm}{*}
\checkandfixthelayout

\setlength{\parindent}{1.3cm}
\setlength{\parskip}{0.2cm}

\SingleSpacing

\begin{document}
	
	\selectlanguage{brazil}
	
	\frenchspacing 
	
	\begin{center}
		\LARGE DA TEORIA À PRÁTICA ACADÊMICA: UM PROSPECTO METODOLÓGICO AO
		ENSINO – PESQUISA – EXTENSÃO NO CAMPUS CACOAL-IFRO\footnote{Trabalho realizado dentro da área de Conhecimento CNPq: Agronomia, com financiamento do
			Campus Cacoal-IFRO.}
		
		\normalsize
		Saimon Eler Bermond\footnote{Bolsista Ensino Superior, Discente do Curso Tecnologia em Agronegócio, email
			saimonbermond@hotmail.com, Campus Cacoal-IFRO.} 
		Darlene Magalhães Teixeira\footnote{Colaborador(a), Discente do Curso Tecnologia em Agronegócio, email
			darlene\_19872011@hotmail.com, Campus Cacoal-IFRO.} \\
		Sérgio Nunes de Jesus\footnote{Orientador(a), Departamento de Pesquisa/DEPESP, IFRO, email sergio.nunes@ifro.edu.br, Campus
			Cacoal-IFRO.} 
		ngrid Letícia
		Menezes Barbosa\footnote{Co-orientador(a), Professora e pesquisadora do IFRO, email ingrig.leticia@ifro.edu.br, Campus
			Cacoal-IFRO.} 
	\end{center}
	
	\noindent O presente projeto foi desenvolvido tendo como base a elaboração de um
	Minidicionário com terminologias técnicas e, simultaneamente, científicas do
	agronegócio, com traduções de termos mais empregados na área, da língua
	portuguesa para a língua inglesa (através de contextualização). A partir da
	realização do trabalho tornou-se possível a elaboração de um material didático
	(minidicionário) que auxilia não só estudantes do Agronegócio, mas também da
	Agroecologia, Agropecuária e Zootecnia, no Campus Cacoal-IFRO. Como todas e
	quaisquer áreas das ciências – o Agronegócio possui inúmeras terminologias que,
	geralmente são desconhecidos em áreas correlatas como a Agroecologia,
	Agropecuária, Zootecnia, Administração, Engenharia de Produção, Biologia entre
	outras e pelo consumidor leigo que constantemente se depara com essas
	terminologias em seu cotidiano. Nessa perspectiva, pensou-se num material didático
	que tornasse possível interações técnico-científicas que se aproximam do
	entendimento do acadêmico e do consumidor leigo, e que esses tivessem
	conhecimento sobre insumos comumente utilizados ao longo de sua vida. O projeto
	foi desenvolvido, pelos acadêmicos do quarto período de Agronegócio sob
	orientação do professor Sérgio Nunes de Jesus. Foram utilizados computadores e
	internet, como suporte adicional na pesquisa dos termos, uma vez sendo realizada a
	comparação com os dicionários da língua portuguesa e inglesa, facilitando, assim, a
	contextualização interpretativa do ‘corpus-técnico’ pesquisado. A produção desse
	material didático facilitou não só a compreensão dos seus interlocutores, mas
	também para aqueles que são a mola propulsora para a área agrícola-administrativa
	– a população consumidora dos produtos elencados no material produzido com esse
	trabalho. A elaboração desse minidicionário trouxe benefícios de fundamental
	importância, tanto para a formação de novos profissionais do setor, bem como, para
	aqueles que já atuam nesse mercado tão promissor para o país, o agronegócio.
	
	\vspace{\onelineskip}
	
	\noindent
	\textbf{Palavras-chave}: Minidicionário. Terminologias. Agronegócio. 
	
	\noindent
	\textbf{Fonte de financiamento}: IFRO Campus Cacoal. 
	
\end{document}
