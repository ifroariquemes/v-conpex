\documentclass[article,12pt,onesidea,4paper,english,brazil]{abntex2}

\usepackage{lmodern, indentfirst, nomencl, color, graphicx, microtype, lipsum,textcomp}			
\usepackage[T1]{fontenc}		
\usepackage[utf8]{inputenc}		

\setlrmarginsandblock{2cm}{2cm}{*}
\setulmarginsandblock{2cm}{2cm}{*}
\checkandfixthelayout

\setlength{\parindent}{1.3cm}
\setlength{\parskip}{0.2cm}

\SingleSpacing

\begin{document}
	
	\selectlanguage{brazil}
	
	\frenchspacing 
	
	\begin{center}
		\LARGE QUALIDADE FISIOLÓGICA DE SEMENTES DE Brachiaria brizantha cv. Marandú APÓS MISTURA COM FERTILIZANTE NPK\footnote{Área de avaliação: Agronomia.}
		
		\normalsize
	Diego Magnum Maifrede dos Santos\footnote{Discente no Curso Técnico em Agropecuária integrado ao Ensino Médio, diegomaifrede15@gmail.com, Campus Ariquemes} 
	Luciane da Cunha Codognoto\footnote{Engenheira agrônoma, luciane.codognoto@ifro.edu.br, Campus Ariquemes.} 
		Thassiane Telles Conde\footnote{Docente em Química, thassiane.conde@ifro.edu.br, Campus Ariquemes.} 
	Lucas Pedro Cipriani\footnote{Técnico em agropecuária, lucas.cipriani@ifro.edu.br, Campus Ariquemes.} 
	\end{center}
	
	\noindent Para facilitar a semeadura de espécies forrageiras é comum a mistura de fertilizantes com sementes, principalmente na implantação de sistemas consorciados. O tempo em que sementes de Brachiaria ficam misturadas com fertilizante pode diminuir o poder germinativo e o desenvolvimento da plântula. Assim, o presente trabalho teve como objetivo verificar o efeito de dez tempos de mistura (0, 3, 6, 12, 24, 36, 48, 72, 96 e 120 horas) do fertilizante granulado NPK 04-30-16 com sementes de B. brizantha cv. Marandú, na sua qualidade fisiológica (germinação e comprimento de plântula). O experimento foi conduzido em laboratório, no IFRO, Campus Ariquemes, em delineamento experimental inteiramente casualizado. A proporção da mistura utilizada foi de 28 g de fertilizante para 1 g de sementes viáveis. Para quantificar a germinação e o comprimento de plântula, três repetições de 50 sementes por tratamento, foram distribuídas entre folhas de papel germitest, umedecidos com água destilada na proporção de 2,5:1 e mantidas em câmara germinadora do tipo BOD, à temperatura de 25 ±1º C. Do sétimo ao 21º dia após a semeadura dos tratamentos, efetuou-se a contagem de germinação e medidas de comprimento de plântula normal. Foram realizadas análise de variância e as médias comparadas pelo teste de Scott-knott, a 5\% de probabilidade. Os tratamentos 0 e 120 horas de mistura, respectivamente, apresentaram germinações equivalentes a 71,33 e 58,00\%. Entretanto, os resultados não evidenciaram (p>0,05) efeito do fertilizante à germinação, nos tempos de contato avaliados. Para comprimento médio de plântula, o tratamento 0 hora de mistura destacou-se (138,66 mm), diferindo significativamente dos demais (p<0,05). Os tratamentos 36, 48, 72, 96 e 120 horas de mistura apresentaram os menores comprimentos de plântula, não diferindo significativamente entre si (p>0,05). O modelo de regressão polinomial de terceira ordem permitiu ajustes para a variável comprimento médio de plântula em função dos tempos de mistura. Assim, para que a perda na qualidade fisiológica não prejudique o comprimento de plântula em sementes de B. Brizantha misturadas ao fertilizante avaliado recomenda-se, sempre que possível, a semeadura imediata (0 hora de mistura).  
	
	\vspace{\onelineskip}
	
	\noindent
	\textbf{Palavras-chave}: Comprimento. Qualidade fisiológica. Semeadura.
	
\end{document}
