\documentclass[article,12pt,onesidea,4paper,english,brazil]{abntex2}

\usepackage{lmodern, indentfirst, nomencl, color, graphicx, microtype, lipsum}			
\usepackage[T1]{fontenc}		
\usepackage[utf8]{inputenc}		

\setlrmarginsandblock{2cm}{2cm}{*}
\setulmarginsandblock{2cm}{2cm}{*}
\checkandfixthelayout

\setlength{\parindent}{1.3cm}
\setlength{\parskip}{0.2cm}

\SingleSpacing

\begin{document}
	
	\selectlanguage{brazil}
	
	\frenchspacing 
	
	\begin{center}
		\LARGE ECOLOGIA E DIVERSIDADE DE PRIMATAS NA MATA DO TIRO DE GUERRA
		EM COLORADO DO OESTE, RONDÔNIA\footnote{Trabalho realizado dentro da área de Conhecimento CNPq: Ciências Biológicas.}
		
		\normalsize
		Shaiene de Medeiros Vieira\footnote{Acadêmica de Ciências Biológicas, shaienemedeiros4@gmail.com , Campus Colorado do Oeste.} 
		Ranieli dos Anjos de Souza Muler\footnote{Orientador(a), ranieli.muler@ifro.edu.br, Grupo de Pesquisas Espaciais (GREES), Campus Colorado
			do Oeste.} 
		
	\end{center}
	
	\noindent A diversidade de primatas está estimada em 399 espécies para o Brasil, com 92
	apresentando ocorrência no bioma Amazônico. Devido às projeções de aumento do
	desmatamento e consequentemente fragmentação de habitats, tanto a flora quanto
	a fauna encontram-se sob fortes pressões ambientais impulsionadas pelas ações
	antrópicas. Baseado nisto, este estudo objetivou avaliar a ecologia e a diversidade
	de primatas de um pequeno fragmento de mata no centro urbano de Colorado do
	Oeste, que encontra-se aos cuidados das forças do Exército. A metodologia utilizada
	foi caminhada diurna para visualização dos primatas existentes na área de estudo,
	durante o mês de junho de 2017. Foram identificadas três espécies de primatas, o
	Macaco Prego ( \textit{Cebus apella} (Linnaeus, 1758)), Bugio ( \textit{Alouatta puruensis}
	Lönnberg, 1941) e o Macaco-de-Cheiro ( \textit{Saimiri cf. ustus} I. Geoffroy, 1843)
	pertencentes às famílias Cebidae, Atelidae e Callitrichidae respectivamente. A
	espécie mais abundante foi \textit{C. apella} , cuja dieta generalista deste gênero pode ser
	um dos fatores que possibilitam a sobrevivência deste grupo em pequenas áreas. Já
	\textit{Saimiri cf. ustus} apresentou o registro de apenas um indivíduo fêmea adulta. 
	\textit{A.puruensis} possui uma dieta folívoro-frugívora, e é considerado como o gênero de
	maior distribuição geográfica de primatas neotropicais, conseguindo viver nos mais
	distintos ambientes, em pequenos fragmentos, inclusive os antropizados. As
	populações de \textit{C. apella} e \textit{A. puruensis} apresentaram indivíduos tanto adultos
	quanto jovens. Devido às espécies estarem em constante contato com os servidores
	locais, apresentam convívio dócil com os humanos e entre si, uma vez que, recebem
	alimentos suplementares três vezes por semana para subsidiar a sobrevivência
	destes grupos no mesmo habitat. Apesar de ser uma reserva urbana, encontra-se
	em ótimo estado de conservação, uma vez que a visitação é controlada e o
	ambiente totalmente cercado. O entendimento sobre a ecologia deste grupo deve
	ser ampliado para subsidiar medidas de conservação e preservação, por
	encontram-se ameaçados, principalmente, devido às forçantes antrópicas.
	
	\vspace{\onelineskip}
	
	\noindent
	\textbf{Palavras-chave}: Primatas. Ecologia. Rondônia.
	
\end{document}
