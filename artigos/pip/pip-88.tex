\documentclass[article,12pt,onesidea,4paper,english,brazil]{abntex2}

\usepackage{lmodern, indentfirst, nomencl, color, graphicx, microtype, lipsum}			
\usepackage[T1]{fontenc}		
\usepackage[utf8]{inputenc}		

\setlrmarginsandblock{2cm}{2cm}{*}
\setulmarginsandblock{2cm}{2cm}{*}
\checkandfixthelayout

\setlength{\parindent}{1.3cm}
\setlength{\parskip}{0.2cm}

\SingleSpacing

\begin{document}
	
	\selectlanguage{brazil}
	
	\frenchspacing 
	
	\begin{center}
		\LARGE PERCEPÇÃO DO ALUNO ACERCA DE SUA PERMANÊNCIA NO ENSINO
		SUPERIOR: UM ESTUDO DE CASO NO IFRO CAMPUS ZONA NORTE\footnote{Trabalho realizado dentro da área de Conhecimento CNPq: Ciências Sociais Aplicadas, financiado
			com recursos do Edital no 15/2016/IFRO/DEPESP/ZONA NORTE, de 31 de maio de 2016.}
		
		\normalsize
		Maria Beatriz Souza Pereira\footnote{Bolsista, Maria Beatriz Souza Pereira, mbe.pereira@gmail.com, Porto Velho Zona Norte} 
		Danielly Eponina Santos Gamenha\footnote{Bolsista, Danielly Eponina Santos Gamenha, daniellysantos70@gmail.com, Porto Velho Zona Norte} 
		Danielli Vacari de Brum\footnote{Coordenador(a), Danielli Vacari de Brum3, danielli.brum@ifro.edu.br, Porto Velho Zona Norte} 
		
	\end{center}
	
	\noindent O presente artigo baseia-se numa análise, a partir de métodos qualitativos,
	buscando-se as razões de permanência dos alunos do Curso Superior em Gestão
	Pública do Instituto Federal de Educação, Ciência e Tecnologia de Rondônia –
	Campus Porto Velho Zona Norte. A coleta de dados foi obtida através de
	questionário estruturado e a população investigada foi composta por 180
	acadêmicos matriculados e frequentes no ano de 2016 com amostra mínima (n) em
	função do erro (e) constituída por 80 alunos. O software utilizado para a apuração e
	análise estatística foi o Sphinx Léxica. Após análise dos resultados, observou-se que
	o ingresso no curso de Gestão Pública deu-se para obtenção do nível superior
	(37,5\%) e para formação na carreira pública (48,8\%). Considerando as visitas
	
	técnicas como fatores de motivação e interesse, 39\% dos respondentes consideram-
	nas suficientes. Infelizmente, 40\% dos alunos já pensaram em desistir do curso e
	
	32\% destes concordam que há falta de incentivos por parte da instituição na
	participação de projetos de pesquisa e extensão. Os principais fatores que levam os
	alunos a permanecerem são os laços com o IFRO (71,3\%), o relacionamento entre
	professores (90\%) e servidores (74\%) e a motivação familiar (52,5\%). Portanto, as
	relações humanas, o apoio obtido na família e os laços criados na instituição
	contribuem de maneira significativa para a permanência no curso. Com relação à
	avaliação dos programas de permanência ofertados pela instituição, a partir de
	editais de concorrência da CAED – Coordenação de Apoio ao Educando, onde se
	leva em consideração renda per capita, ingresso por ação afirmativa, membro
	familiar com deficiência e núcleo familiar com idosos e crianças, 47,6\% dos
	beneficiários os qualificam como bom/excelente e 55\% concordam que os auxílios
	ofertados contribuem para a sua permanência. Enfim, apesar da CAED em parceria
	com a Direção de Ensino influenciar na melhoria de vida profissional e pessoal dos
	alunos, podemos concluir que estes departamentos precisam reconsiderar e reforçar
	os programas já ofertados, oferecendo dinâmicas e acompanhamento
	individualizado e humanizado para que seus alunos permaneçam em suas
	graduações até a conclusão do curso independentemente das razões pessoais que
	em muitos casos os levam à evasão.
	
	\vspace{\onelineskip}
	
	\noindent
	\textbf{Palavras-chave}: Permanência Escolar. Cursos Superiores. IFRO.
\end{document}
