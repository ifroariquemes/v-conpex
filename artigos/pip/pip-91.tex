\documentclass[article,12pt,onesidea,4paper,english,brazil]{abntex2}

\usepackage{lmodern, indentfirst, nomencl, color, graphicx, microtype, lipsum}			
\usepackage[T1]{fontenc}		
\usepackage[utf8]{inputenc}		

\setlrmarginsandblock{2cm}{2cm}{*}
\setulmarginsandblock{2cm}{2cm}{*}
\checkandfixthelayout

\setlength{\parindent}{1.3cm}
\setlength{\parskip}{0.2cm}

\SingleSpacing

\begin{document}
	
	\selectlanguage{brazil}
	
	\frenchspacing 
	
	\begin{center}
		\LARGE PERFIL DOS ESTUDANTES DO IFRO CAMPUS PORTO VELHO ZONA NORTE\footnote{Trabalho realizado dentro da área de Conhecimento CNPq: Ciências Sociais Aplicadas com
			financiamento do Instituto Federal de Rondônia IFRO.}
		
		\normalsize
		Rodrigo Lopes da Silva\footnote{Bolsista PIBIC-EM, rodrigoslopes.sl@gmail.com, Campus Porto Velho Zona Norte} 
		Fernanda Ruschel Cremonese Colen\footnote{Co-Orientadora, fernanda.ruschel@ifro.edu.br, Campus Porto Velho Zona Norte} 
		Thiago Pacife de Lima\footnote{Orientador, thiago.lima@ifro.edu.br, Campus Porto Velho Calama} 
	
	\end{center}
	
	\noindent A Coordenação de Assistência ao Educando - CAED do Campus Porto Velho Zona
	Norte tem como objetivo prestar apoio aos estudantes visando contribuir com o
	acesso, a permanência e o êxito dos mesmos, na perspectiva de equidade,
	produção de conhecimento, melhoria do desempenho acadêmico e da qualidade de
	vida. Tendo em vista a liberação dos recursos de assistência estudantil e, frente à
	atual conjuntura da educação brasileira, na qual a aprendizagem apresenta índices
	insatisfatórios, elevados níveis de repetência e evasão, verifica-se a necessidade de
	analisar o perfil socioeconômico e cultural dos alunos na perspectiva de
	compreender sua trajetória social, econômica e estudantil, o que demonstrará, em
	parte, o perfil de nossos futuros profissionais. O objetivo do trabalho foi conhecer em
	detalhes as necessidades e carências dos alunos que ingressam na instituição
	visando pensar e planejar as políticas de assistência estudantil que serão
	implementadas para viabilizar a igualdade de oportunidades entre os estudantes e
	contribuir para a melhoria do desempenho acadêmico, combatendo de forma efetiva
	as situações de retenção e evasão. Trata-se de uma pesquisa de campo com
	abordagem quanti-qualitativa através da aplicação de questionário on-line para
	mapear o Perfil Socioeconômico, Cultural e dados de saúde dos estudantes
	matriculados entre 2014 e 2016 no Campus Porto Velho Zona Norte nas
	modalidades presencial e Ead. Ao todo foram convidados aproximadamente 4.500
	alunos sendo retirada uma amostra probabilística aleatória de 1332 estudantes. Os
	dados coletados foram analisados por meio de estatística descritiva com utilização
	dos Softwares Microsoft Excel e XLSTAT. Os resultados revelaram que 72\% dos
	estudantes são do sexo feminino e 71\% autodeclarados pardos ou pretos e 26\%
	brancos ou amarelos. Quanto à renda familiar, 74\% exercem atividade remunerada
	e 92\% declararam receber até 1,5 salário mínimo por mês. Quanto à escolaridade
	90\% são oriundos integralmente de escolas públicas. Os resultados reafirmam a
	importância do estudo do perfil dos estudantes uma vez que a maior parte destes se
	enquadram no perfil de possíveis beneficiários da Assistência Estudantil, podendo a
	instituição direcionar as ações buscando facilitar o acesso aos estudantes que mais
	necessitam de auxílio seja este financeiro ou não.
	
	\vspace{\onelineskip}
	
	\noindent
	\textbf{Palavras-chave}:Perfil socioeconômico. Equidade. Assistência estudantil.
	
	\noindent
	\textbf{Fonte de Financiamento}: Instituto Federal de Rondônia – IFRO, Edital n$^{\circ}$
	38/2016/PROPESP/IFRO.
	
\end{document}
