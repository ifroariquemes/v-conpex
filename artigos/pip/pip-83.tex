\documentclass[article,12pt,onesidea,4paper,english,brazil]{abntex2}

\usepackage{lmodern, indentfirst, nomencl, color, graphicx, microtype, lipsum}			
\usepackage[T1]{fontenc}		
\usepackage[utf8]{inputenc}		

\setlrmarginsandblock{2cm}{2cm}{*}
\setulmarginsandblock{2cm}{2cm}{*}
\checkandfixthelayout

\setlength{\parindent}{1.3cm}
\setlength{\parskip}{0.2cm}

\SingleSpacing

\begin{document}
	
	\selectlanguage{brazil}
	
	\frenchspacing 
	
	\begin{center}
		\LARGE MINIDICIONÁRIO PORTUGUÊS -- INGLÊS: TERMINOLOGIAS TÉCNICAS E\\ CIENTÍFICAS DO AGRONEGÓCIO\footnote{Trabalho realizado dentro da área de Conhecimento CNPq: Agronomia, com financiamento do
			Campus Cacoal-IFRO.}
		
		\normalsize
		Thays Klitzke dos Santos\footnote{Bolsista Ensino Superior, Discente do Curso Tecnologia em Agronegócio, email
			thays.klitzke.ifro@gmail.com, Campus Cacoal-IFRO.} 
		Darlene Magalhães Teixeira\footnote{Colaborador(a), Discente do Curso Tecnologia em Agronegócio, email
			darlene\_19872011@hotmail.com, Campus Cacoal-IFRO} \\
		Sérgio Nunes de Jesus\footnote{Orientador(a), Departamento de Pesquisa/DEPESP, IFRO, email sergio.nunes@ifro.edu.br, Campus
			Cacoal-IFRO.} 
		Davys Sleman de Negreiros\footnote{Co-orientador(a), Professor e pesquisador do IFRO, email davys.negreiros@ifro.edu.br, Campus
			Cacoal-IFRO.} 
	\end{center}
	
	\noindent O presente projeto foi desenvolvido com o intuito de elaborar um
	Minidicionário como suporte para o entendimento e interação dos interlocutores a
	partir da tradução dos termos do Agronegócio da língua portuguesa para a língua
	inglesa. Uma vez que só é possível vivenciar fatos e acontecimentos tecnológicos a
	partir das noções que se preestabelecem entre teoria e prática social previamente
	conhecidas em seu processo de interação técnico-teórico por um entendimento
	universalista na língua (gem), contudo, esse trabalho possibilitou a elaboração do
	Minidicionário como suporte para o entendimento e interação dos interlocutores a
	partir da tradução dos termos do Agronegócio e áreas afins. Observa-se que o
	Agronegócio possui inumeras terminologias, muitas delas desconhecidas, tanto pela
	comunidade acadêmica, bem como, pelo consumidor leigo que constantemente se
	depara com essas terminologias, nas compras de produtos em supermercados,
	açougues, feirões de agricultores, consultorias administrativas e outras. Nesse
	sentido, pensou-se produzir um material didático que beneficie não só alunos, mas
	também a comunidade consumidora em geral. O projeto foi desenvolvido, a partir da
	coleta de dados para a elaboração do Minidicionário. A busca por palavras do
	âmbito agrário teve como base a Biblioteca do Campus Cacoal-IFRO, onde foram
	realizadas as pesquisas em materiais didáticos como dicionários Inglês-Português, e
	dicionários técnicos que serviram de embasamento técnico na compreensão das
	terminologias específicas. O material foi disponibilizado, em meio eletrônico, para
	maior praticidade de acesso dos usuários como suporte didático no âmbito
	acadêmico para a compreensão linguística dos termos técnico-científicos do
	português e do inglês. O material didático produzido falicitou a compreensão dessas
	terminologias pelos seus interlocutores e também por aqueles que diariamente se
	deparam com os mesmos em seu respectivo meio, de trabalho ou mesmo de público
	consumidor em geral. Entende-se que, a elaboração desse minidicionário com
	terminologias técnicas e científicas do agronegócio traduzidas para a língua inglesa
	proporcionou benefícios importantes, tanto para a formação de novos profissionais
	do setor das ciências agrárias, bem como, para o público geral.
	
	\vspace{\onelineskip}
	
	\noindent
	\textbf{Palavras-chave}: Minidicionário. Terminologias. Agronegócio.
	  
	\noindent
	\textbf{Fonte de Financiamento}:IFRO Campus Cacoal.
	
\end{document}
