\documentclass[article,12pt,onesidea,4paper,english,brazil]{abntex2}

\usepackage{lmodern, indentfirst, nomencl, color, graphicx, microtype, lipsum}			
\usepackage[T1]{fontenc}		
\usepackage[utf8]{inputenc}		

\setlrmarginsandblock{2cm}{2cm}{*}
\setulmarginsandblock{2cm}{2cm}{*}
\checkandfixthelayout

\setlength{\parindent}{1.3cm}
\setlength{\parskip}{0.2cm}

\SingleSpacing

\begin{document}
	
	\selectlanguage{brazil}
	
	\frenchspacing 
	
	\begin{center}
		\LARGE EFEITOS DA IMPLANTAÇÃO DO PROJETO LUCAS DO RIO VERDE LEGAL NAS ÁREAS DE APP’S LOCAL\footnote{Trabalho realizado dentro da área de Conhecimento CNPq: Geofísica.}
		
		\normalsize
		MACHADO, A.,\footnote{Autor principal, alayzzamachado@gmail.com, Campus Colorado do Oeste.} 
	BORGES, B.R.S.,\footnote{Colaborador, brunoborges0615@gmail.com, Campus Colorado do Oeste.} 
		MOURA, V.\footnote{Orientador, valdir.moura@ifro.edu.br, Campus Colorado do Oeste.} 
		
	\end{center}
	
	\noindent Uma das áreas que tem tornado o foco de importantes discussões é a degradação ambiental, que possui entre várias causas, o avanço do agronegócio. O município de Lucas do Rio Verde desenvolveu uma iniciativa única no país: o projeto Lucas do Rio Verde Legal, cuja finalidade é a promoção da regularização socioambiental das propriedades rurais do município, compatibilizando o desenvolvimento agropecuário e a conservação ambiental da região. Assim, objetivou-se avaliar a eficácia da implantação do Projeto Lucas do Rio Verde Legal sob a área de vegetação do município de Lucas do Rio Verde/ MT. Utilizou-se uma série de imagens temporais dos anos de 1990, 1995, 2000, 2005, 2010 e 2016 disponíveis na plataforma do INPE e o software SPRING para o processamento das imagens. As imagens foram coletadas na órbita/ponto 227/69 com os sensores TM e OLI. Utilizou-se a classificação superviosinada, algoritmo Battacharya e as classes de uso do solo determinadas foram: floresta, agricultura, pastagem e água. Após a classificação, foi realizada a interpretação das imagens para discriminar os diferentes tipos de uso do solo. O resultado final do processo foi uma carta temática com as informações necessárias para o alcance do objetivo do estudo em questão. Os dados obtidos a partir do processamento de imagens referente ao uso do solo no município demonstraram desmatamento intenso até o ano de 2005 (nível crítico de 70\% de desmatamento). A partir de então, no ano de 2006 o projeto foi implantado. No ano de 2016, quando comparado ao ano de 2005, houve uma elevação de 10,5\% na área total de florestas do município. Mesmo com uma redução na área de agricultura, o potencial agrícola do município não foi reduzido, isso se deve basicamente a utilização de técnicas e tecnologias adequadas para a obtenção de bons índices produtivos aliados á conservação e restauração das florestas. O presente trabalho é pioneiro em se tratando de avaliar os efeitos da implantação do projeto Lucas do Rio Verde Legal, onde ressalta-se a eficiência do mesmo no sentido de amenizar/frear o processo de degradação ambiental, com potencial para se estender ao longo do território brasileiro.
	
	\vspace{\onelineskip}
	
	\noindent
	\textbf{Palavras-chave}: Degradação ambiental. Lucas do Rio Verde Legal. Agronegócio.
	
\end{document}
