\documentclass[article,12pt,onesidea,4paper,english,brazil]{abntex2}

\usepackage{lmodern, indentfirst, nomencl, color, graphicx, microtype, lipsum, textcomp}			
\usepackage[T1]{fontenc}		
\usepackage[utf8]{inputenc}		

\setlrmarginsandblock{2cm}{2cm}{*}
\setulmarginsandblock{2cm}{2cm}{*}
\checkandfixthelayout

\setlength{\parindent}{1.3cm}
\setlength{\parskip}{0.2cm}

\SingleSpacing

\begin{document}
	
	\selectlanguage{brazil}
	
	\frenchspacing 
	
	\begin{center}
		\LARGE ESTÁGIOS DE MUDANÇA DE COMPORTAMENTO PARA ATIVIDADE FÍSICA DOS MORADORES DA COMUNIDADE PURUZINHO\footnote{Trabalho realizado dentro da área de Educação Física com financiamento da Fundação Rondônia de Amparo ao Desenvolvimento das Ações Científicas e Tecnológicas e à Pesquisa do Estado de Rondônia (FAPERO).}
		
		\normalsize
	Amanda Carolina Candido Silva,\footnote{Voluntária, amandaccandidosilva@gmail.com, Campus Porto Velho Calama.} 
	Mel Naomí da Silva Borges,\footnote{Voluntária, melborges85@gmail.com, Campus Porto Velho Calama.} 
		Maria Enísia Soares de Souza,\footnote{Colaboradora, enisiasoares@gmail.com, Campus Porto Velho Calama.} 
	Matheus Magalhães Paulino Cruz,\footnote{Colaborador, matheus.cruz@ifro.edu.br, Campus Porto Velho Calama.} 
		Olakson Pinto Pedrosa,\footnote{Colaborador, olakson.pedrosa@ifro.edu.br, Campus Porto Velho Calama.}
		Tiago Lins de Lima,\footnote{Colaborador, tiago.lins@ifro.edu.br, Campus Porto Velho Calama.}
		Xênia de Castro Barbosa,\footnote{Colaboradora, xenia.castro@ifro.edu.br, Campus Porto Velho Calama.}
		Iranira Geminiano Melo.\footnote{Orientadora, iranira.melo@ifro.edu.br, Campus Porto Velho Calama.}
		
	\end{center}
	
	\noindent O questionário de Estágios de Mudança de Comportamento (EMC) sugere que indivíduos com comportamentos não saudáveis, tais como fumar ou ter estilo de vida sedentário, podem ser situados em uma série de estágios que representam seu nível de prontidão para a mudança de comportamento. A avaliação desses estágios é decisiva para a escolha de estratégias mais adequadas e eficazes para o aumento e a manutenção da motivação, favorecendo a mudança efetiva de comportamento, que levaria a melhores hábitos de estilo de vida, melhorando a qualidade de vida tanto individual, quanto coletiva. O estudo objetiva analisar os estágios de mudança de comportamento para atividade física em comunidade ribeirinha. O estudo foi realizado na comunidade ribeirinha do lago Puruzinho, localizado à margem esquerda do rio Madeira e a 20 km do município de Humaitá-Amazonas, participaram da pesquisa 54 moradores, com idade variando de 13 a 73 anos (33,2±14,40). A coleta de dados se deu a partir da utilização do formulário Estágios de Mudança de Comportamento (EMC) - Atividade Física. Após a coleta de dados estes foram tabulados no Software Microsoft Excel 2010 e Excel Xlstat 2014. Os resultados indicam que, considerando apenas os adultos, 55\% dos homens e 77\% das mulheres encontram-se em estágio de contemplação, 27\% dos homens e nenhuma mulher encontram-se em estágio de manutenção, 16\% dos homens e 22\% das mulheres encontram-se em estágio de pré-contemplação. Considerando os dados obtidos constatou-se que a maioria dos pesquisados encontram-se em estágio de contemplação, no total de 61\%. Neste estágio o indivíduo percebe o problema e tem sérias intenções de modificar seu comportamento (nos próximos seis meses, ao menos). Todavia, ainda não se compromissou com a tomada de decisão efetiva. O sujeito não se sente pronto para mudar e ainda não coloca em ambivalência as barreiras e malefícios de ser sedentário com os benefícios de ser ativo fisicamente. É necessário que os moradores da comunidade adotem melhores hábitos de estilo de vida, e sugere-se a criação de programas de intervenção para atividade física e disponibilização de ambientes voltados para a prática de atividade física, visto que estes são inexistentes no local.
	
	\vspace{\onelineskip}
	
	\noindent
	\textbf{Palavras-chave}: Estilo de Vida. Hábitos. Atividade Física.
	
\end{document}
