\documentclass[article,12pt,onesidea,4paper,english,brazil]{abntex2}

\usepackage{lmodern, indentfirst, nomencl, color, graphicx, microtype, lipsum,textcomp}			
\usepackage[T1]{fontenc}		
\usepackage[utf8]{inputenc}		

\setlrmarginsandblock{2cm}{2cm}{*}
\setulmarginsandblock{2cm}{2cm}{*}
\checkandfixthelayout

\setlength{\parindent}{1.3cm}
\setlength{\parskip}{0.2cm}

\SingleSpacing

\begin{document}
	
	\selectlanguage{brazil}
	
	\frenchspacing 
	
	\begin{center}
		\LARGE UTILIZAÇÃO DO TESTE DE DISCO DIFUSÃO NA DETERMINAÇÃO DO PERFIL DE SUSCEPTIBILIDADE DE S. AUREUS FRENTE A DIFERENTES EXTRATOS NATURAIS\footnote{Trabalho realizado dentro da Ciências Biológicas, com financiamento do IFRO.}
		
		\normalsize
	Ronaldo Julio Silva Rufino\footnote{Colaboradores, email ronaldojuliorufino@gmail.com, Campus Colorado do Oeste.} 
	Lizianne de Matos Emerick\footnote{Colaboradores, email lizianne.emerick@ifro.edu.br, Campus Colorado do Oeste.} 
	Natália Conceição\footnote{Orientadora, email natalia.conceicao@ifro.edu.br, Campus Colorado do Oeste.} 
	Camila Budim Lopes\footnote{Co-orientadora, email Camila.lopes@ifro.edu.br, Campus Colorado do Oeste.} 
	\end{center}
	
	\noindent A técnica de disco difusão é uma das principais técnicas utilizadas para avaliar qualitativamente o perfil de susceptibilidade de isolados clínicos frente a diversos antimicrobianos. No entanto, a utilização desta metodologia para avaliar extratos naturais ainda é escassa, devido à falta de padronização específica para esta finalidade, sendo necessários novos estudos acerca desta temática. Assim, o objetivo do presente trabalho foi avaliar o teste de disco difusão na determinação do perfil de susceptibilidade de Staphylococcus aureus frente a diferentes extratos naturais. Os extratos aquosos utilizados foram extraídos das seguintes plantas, Piper aduncun e Piper medium (jaborandi-falso), Pachyrrhizus tuberosus (feijão macuco), obtidos por decocção. O controle positivo para verificar a presença do halo de inibição de crescimento do microrganismo foi o antimicrobiano enrofloxacina (10\%) e, como controle negativo foi utilizado solução salina. As amostras de S. aureus foram obtidas de animais previamente diagnosticados com mastite bovina clínica. A identificação em espécie foi realizada por meio de testes bioquímicos convencionais. Os discos foram impregnados com os extratos e colocados sobre as placas de Petri, utilizando-se uma pinça metálica. As placas foram incubadas em estufa bacteriológica por 48h a 37ºC. De acordo com os resultados obtidos, houve a presença de halos de inibição de crescimento quando se utilizou extratos de P. aduncun e P. medium, mostrando que estas plantas apresentam uma atividade antimicrobiana em potencial. Portanto pode-se concluir que, embora ainda seja pouco utilizada nos testes com extratos naturais, o teste de disco de difusão mostrou-se eficiente para determinar o perfil de susceptibilidade das amostras de S. aureus frente aos diferentes extratos analisados. 
	\vspace{\onelineskip}
	
	\noindent
	\textbf{Palavras-chave}: Inibição Bacteriana. Piper. Mastite.
		\vspace{\onelineskip}
		
	\noindent
	\textbf{Fonte de financiamento}: IFRO.
	
\end{document}
