\documentclass[article,12pt,onesidea,4paper,english,brazil]{abntex2}

\usepackage{lmodern, indentfirst, nomencl, color, graphicx, microtype, lipsum}			
\usepackage[T1]{fontenc}		
\usepackage[utf8]{inputenc}		

\setlrmarginsandblock{2cm}{2cm}{*}
\setulmarginsandblock{2cm}{2cm}{*}
\checkandfixthelayout

\setlength{\parindent}{1.3cm}
\setlength{\parskip}{0.2cm}

\SingleSpacing

\begin{document}
	
	\selectlanguage{brazil}
	
	\frenchspacing 
	
	\begin{center}
		\LARGE AVALIAÇÃO DOS MÉTODOS DE QUEBRA DE DORMÊNCIA EM SEMENTES DE \MakeUppercase{Cassia fistula} L\footnote{Trabalho realizado dentro da área de Recursos Florestais e Engenharia Florestal, com financiamento do Instituto Federal de Rondônia, Campus Ji-Paraná por meio do edital 35 de 2016}
		
		\normalsize
	Marta Betânia Ferreira Carvalho\footnote{Bolsista Iniciação Cientifica EM, martabetania99@gmail.com, Campus Ji-Paraná} 
		Maria Elessandra Rodrigues Araujo\footnote{Orientador(a), maria.elessandra@ifro.edu.br, Campus Ji-Paraná} \\
		Andreza Pereira Mendonça\footnote{Co-orientador(a), andreza.mendoca@ifro.edu.br, Campus Ji-Paraná} 
	
	\end{center}
	
	\noindent 
	Cassia fistula popularmente conhecida como canafistula, chuva de ouro ou cassia-imperial, é uma árvore pertencente à família Leguminoseae, possui origem asiática, mas também pode ser encontrada em algumas regiões do Brasil, possuindo alta adaptabilidade, sendo muito importante na ornamentação e possuindo alto valor terapêutico. No entanto as sementes desta espécie apresentam dormência, dificultando a germinação da mesma que se torna lenta e desuniforme, pela ocorrência de mecanismos que impedem a troca gasosa o que dificulta o desenvolvimento do embrião, para obtenção do melhor desenvolvimento da semente é recomendado que se faça a superação da dormência. Objetivou-se avaliar diferentes métodos de quebra de dormência. O trabalho foi conduzido no laboratório de sementes do IFRO - Campus, Ji-Paraná. Utilizou-se os seguintes métodos para superar a dormência das sementes de Cassia fistula; $T_1$ – Testemunha; $T_2$ – escarificação com lixa n$\circ$ 80, $T^3$ - escarificação com lixa n$\circ$ 80, seguida de repouso em água por 24 horas; $T_4$ e $T_5$ - Ácido sulfúrico por 3 e 5 minutos; $T_6$ - Desponte da parte posterior ao eixo embrionário; $T_7$ - Desponte da parte posterior ao eixo embrionário, seguida de repouso em água por 24 horas; $T_8$ e $_9$ – Ácido clorídrico por 3 e 5 minutos. O delineamento experimental utilizado nas diferentes etapas foi o inteiramente casualizado com quatro repetições. O software utilizado na análise foi o ASSISTAT, e as médias, após análise de variância, comparadas pelo teste de Tukey a 5\% de probabilidade. Embora não havendo diferença significativa entre a escarificação física e química, as sementes que sofreu desponte ($T_6$) e as que foram escarificadas com lixa ($T_2$) ambas embebidas em água apresentaram maiores valores de germinação (95 e 98\%, respectivamente) não diferindo estatisticamente entre si. Obteve-se o maior comprimento de plântula quando se trabalhou com Ácido clorídrico por um período de 3 minutos (7,60 cm). Observou-se um maior acumulo de matéria seca para as plântulas que foram tratadas com lixa embebidas em água por 24 horas ($T_3$) e embebidas em ácido clorídrico por um período de 5 minutos ($T^9$) com resultados superiores aos demais métodos estudados 1,1 e 1,02 gramas respectivamente.
	
	\vspace{\onelineskip}
	
	\noindent
	\textbf{Palavras-chave}: Dormência. Germinação. Sementes. \\
	\textbf{Fonte de financiamento}: Instituto Federal de Rondônia, Campus Ji-Paraná por meio do edital 35 de 2016. 
	
\end{document}
