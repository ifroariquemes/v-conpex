\documentclass[article,12pt,onesidea,4paper,english,brazil]{abntex2}

\usepackage{lmodern, indentfirst, nomencl, color, graphicx, microtype, lipsum}			
\usepackage[T1]{fontenc}		
\usepackage[utf8]{inputenc}		

\setlrmarginsandblock{2cm}{2cm}{*}
\setulmarginsandblock{2cm}{2cm}{*}
\checkandfixthelayout

\setlength{\parindent}{1.3cm}
\setlength{\parskip}{0.2cm}

\SingleSpacing

\begin{document}
	
	\selectlanguage{brazil}
	
	\frenchspacing 
	
	\begin{center}
		\LARGE CARACTERIZAÇÃO DA PRODUÇÃO DE LEITE EM COLORADO DO OESTE – RO, \\SEGUNDO A IN 62/2011\footnote{Trabalho realizado dentro da Ciências agrarias com financiamento do IFRO.}
		
		\normalsize
	Lucas Eduardo Aguiar\footnote{Bolsista PIBIC EM, lucaseduardoaguiar@gmail.com, Campus Colorado do Oeste.} 
	Cristiane Reis Martins\footnote{Bolsista PIBIC Af, cristianereismartins@gmail.com, Campus Colorado do Oeste.} \\
		Sanlley Rafael Ferreira\footnote{Colaborador, sanlleyrafaelferreira,Campus, Colorado do Oeste.} 
		Nélio Ranieli Ferreira de Paula\footnote{Orientador, nelio.ferreira@ifro.edu.br, Campus Colorado do Oeste.} 
	\end{center}
	
	\noindent A produção de leite em Rondônia, vem aumentando ano após anos. O Município de Colorado do Oeste, que está localizado ao sul do estado de Rondônia, tem como base da sua economia, a bovinocultura leiteira. Porém existem poucos estudos quanto aos parâmetros da produção local. Atualmente, existem pequenos produtores que não investem no melhoramento do rebanho. Diante do exposto, buscou-se ampliar os estudos no município, para obter dados dos produtores estabelecendo parâmetros de realidade de produção por meio de elaboração de questionário socioeconômico avaliados em 3 diferentes níveis (pequena, média e grande) de propriedades do município. O questionário aplicado abrangeu questões sobre o rebanho leiteiro acerca da produção de leite, caracterizações gerais da propriedade, produtividade e manejo sanitário do rebanho, comércio e destinação do leite produzido. Observou-se que dentre os 3 produtores analisados que a produção diária do produtor “A” foi de 250 litros, se sobressaindo em comparação com os outros produtores “B” e “C” com 180 e 85 litros respectivamente. Isso se deve a adoção de novas tecnologias (pastagem irrigada), em sua propriedade. Para o preço pago pelo leite o Produtor “A” obteve um valor de 0,96 R\$ e os produtores “B” 0,85 R\$ e “C” 0,90 R\$, esta diferença se deve os diferentes locais de fornecimento da produção. Já para o número de vacas em lactação, observamos que o Produtor “B” possuía a maior quantia de vacas em lactação 36 vacas e os produtores “A” e “C” possuem 19 e 11 vacas respectivamente, porém o produtor “B” não necessariamente possui a maior produção diária de leite. Observou-se também, na caracterização da propriedade que todos possuíam assistência técnica seja por empresas particulares produtores “B” e “C” ou projeto balde cheio da EMBRAPA produtor “A”. E para a estrutura do curral, todos eram cobertos, com foço, ordenha feita de forma mecânica e presença de tanque resfriador na propriedade. Contudo, os objetivos da pesquisa foram alcançados de forma satisfatória, pois todos atenderam as especificações segundo a IN 62/2011.
	
	\vspace{\onelineskip}
	
	\noindent
	\textbf{Palavras-chave}: Produção. Leite. Produtividade. \\
	\textbf{Fonte de Financiamento}: IFRO
	
\end{document}
