\documentclass[article,12pt,onesidea,4paper,english,brazil]{abntex2}

\usepackage{lmodern, indentfirst, nomencl, color, graphicx, microtype, lipsum, textcomp}			
\usepackage[T1]{fontenc}		
\usepackage[utf8]{inputenc}		

\setlrmarginsandblock{2cm}{2cm}{*}
\setulmarginsandblock{2cm}{2cm}{*}
\checkandfixthelayout

\setlength{\parindent}{1.3cm}
\setlength{\parskip}{0.2cm}

\SingleSpacing

\begin{document}
	
	\selectlanguage{brazil}
	
	\frenchspacing 
	
	\begin{center}
		\LARGE ESTUDO FITOQUÍMICO DO EXTRATO DE SEMENTES DE Bowdichia
		virgilioides (SUCUPIRA-PRETA)\footnote{Trabalho realizado dentro da Química com financiamento do CNPq/IFRO.}
		
		\normalsize
		Quésia de Andrade Aguiar,\footnote{Bolsista(IC-EM), quesiadeandrade78@gmail.com, Campus JI-Paraná.} 
	Carla Karoline Bueno Souza,\footnote{Colaboradora, carla.karoline1204@gmail.com, Campus JI-Paraná.} 
			José Assis gomes de Brito,\footnote{Orientador, jose.assis@ifro.edu.br, professor do IFRO Campus JI-Paraná.} 
		Fernando Cotinguiba da Silva\footnote{Co-orientador(a), fernando@correio.nppn.ufrj.br, professor do IPPN/UFRJ.} 
	\end{center}
	
	\noindent A sucupira preta se destaca por seu uso popular no tratamento de diversas doenças,
	entre elas, a diabetes, bronquite, diarreia e em inflamações uterinas. Este projeto
	realizou um o estudo fitoquímico da espécie Bowdichia virgilioides, (sucupira-preta),
	com o objetivo de identificar metabólitos secundários no extrato de sua semente. As
	sementes foram coletadas no município de Guiratinga, estado do Mato Groso, e
	envidadas para análise nos laboratórios do Instituto Federal de Rondônia, Campus
	Ji-Paraná. Após secas, as sementes foram trituradas e submetidas a extração com
	etanol para obtenção do extrato bruto. Este foi então submetido ao processo de
	partição líquido-liquido que resultou na obtenção de três partições, a hexanica, a
	diclorometanica e acética. Foi selecionada uma das partições, a hexanica, e esta foi
	então submetida à técnicas cromatográficas usuais, como a cromatografia em
	coluna aberta, usando sílica como fase estacionária, e a cromatografia em camada
	delgada, para fins de isolamento dos compostos. Posteriormente, os compostos
	obtidos por meio do fracionamento na cromatografia em coluna foram submetidos a
	identificação estrutural por meio da técnica de cromatografia gasosa acoplada a
	espectrômetro de massas - CGMS. Os resultados das análises foram comparados
	com o bando de dados da biblioteca NIST do aparelho, o qual identificou diversos
	compostos, dentre os quais destacamos a Norethindrone, Cholestan-3-ol,2-
	methylene, Spirost-8-en-11-one, 3-hydroxy-, (3$\beta$,5$\alpha$,14$\beta$,20$\beta$,22$\beta$,25R), Stigmasterol,
	Sitosterol, Retinol, Glycine, N-[(3$\alpha$,5$\beta$)-24-oxo-3-[(trimethylsilyl)oxy]cholan-24-yl]-,
	methyl ester, Ethyl iso-allocholate e o Trans-Geranilgeraniol. De acordo com a
	literatura consultada, tais compostos apresentam atividades diversas como
	antioxidantes, anti-inflamatória, anti-osteoartrítica, estrogênica, antioxidante, antiinfecciosa,
	como também alguns mostram-se atuantes em importantes processos
	biológicos, como na biossíntese de outros compostos. Desta forma, os resultados
	das análises demonstraram que a fração hexanica, do extrato de sementes da
	sucupira preta, apresenta vários compostos que possuem propriedades
	farmacológicas.
	
	\vspace{\onelineskip}
	
	\noindent
	\textbf{Palavras-chave}: Fitoquímica, Bowdichia virgilioides, metabólitos secundários
	\textbf{Fonte de financiamento}: CNPq/IFRO
	
\end{document}
