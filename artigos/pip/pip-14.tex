\documentclass[article,12pt,onesidea,4paper,english,brazil]{abntex2}

\usepackage{lmodern, indentfirst, nomencl, color, graphicx, microtype, lipsum}			
\usepackage[T1]{fontenc}		
\usepackage[utf8]{inputenc}		

\setlrmarginsandblock{2cm}{2cm}{*}
\setulmarginsandblock{2cm}{2cm}{*}
\checkandfixthelayout

\setlength{\parindent}{1.3cm}
\setlength{\parskip}{0.2cm}

\SingleSpacing

\begin{document}
	
	\selectlanguage{brazil}
	
	\frenchspacing 
	
	\begin{center}
		\LARGE AVALIAÇÃO DA QUALIDADE DE MUDAS DE CANA FÍSTULA FORMADAS A PARTIR DE COMPOSTOS ORGÂNICOS.\footnote{Engenharia Florestal com financiamento do edital 35 de 2016}
		
		\normalsize
		Milene Queiroz Brunaldi Lima\footnote{Bolsista (acadêmico), milabrunaldi@gmail.com, Campus Ji-Paraná} 
		Polyana Barros Nascimento Carvalho\footnote{Bolsista (acadêmico), polyanabarrosnc@gmail.com, Campus Ji-Paraná} 
		Andreza Mendonça\footnote{Orientador(a), andreza.mendonca@ifro.edu.br, Campus Ji-Paraná} 
	Maria Elessndra Rodrigues Araujo\footnote{Co-orientador(a), maria.elessandra@ifro.edu.br, Campus Ji-Paraná} 
	\end{center}
	
	\noindent A espécie Cassia fistula L., é uma espécie florestal exótica comumente conhecida
	por chuva-de-ouro ou cana imperial, é uma leguminosa Caesalpinoideae do gênero
	Cassia. A espécie canafístula tem sido indicada para recuperação de áreas
	degradadas. Contudo pouco se sabe do desenvolvimento das mudas no viveiro.
	Portanto, o objetivo do trabalho foi avaliar o desenvolvimento das mudas de
	canafístulas sob diferentes misturas de substratos a partir do uso de resíduos
	orgânicos. As sementes foram coletadas em áreas circunvizinhas a Ji-Paraná. As
	sementes foram beneficiadas e sua dormência quebrada pelo método de desponte.
	Após a dormência das sementes serem quebradas elas foram semeadas em areia
	lavada. As mudas foram submetidas a diferentes misturas compostos orgânicos e
	avaliadas após 120 dias no viveiro a 50\% de sombreamento. Para caracterizar o
	substrato que resultou na produção de mudas mais vigorosas, os substratos foram
	avaliados considerando o desenvolvimento diferencial das plantas, com base na
	resposta nos parâmetros morfológicos (comprimento da parte aérea, diâmetro do
	colo e a relação entre diâmetro e altura). Em todos os parâmetros avaliados o
	substrato formado a partir da compostagem casca de mandioca + leucena teve os
	melhores resultados em relação aos demais tratamentos testados. Houve diferenças
	significativas nas médias da altura das mudas entre os substratos testados, sendo
	que valores mais altos foram constatados quando se utilizou o substrato proveniente
	da compostagem de casca de mandioca +leucena, provavelmente este substrato
	permitiu um maior acumulo de reservas.
	
	\vspace{\onelineskip}
	
	\noindent
	\textbf{Palavras-chave}: Mudas, Cana fistula, Avaliação biométrica.
	
\end{document}
