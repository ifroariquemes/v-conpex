\documentclass[article,12pt,onesidea,4paper,english,brazil]{abntex2}

\usepackage{lmodern, indentfirst, nomencl, color, graphicx, microtype, lipsum}			
\usepackage[T1]{fontenc}		
\usepackage[utf8]{inputenc}		

\setlrmarginsandblock{2cm}{2cm}{*}
\setulmarginsandblock{2cm}{2cm}{*}
\checkandfixthelayout

\setlength{\parindent}{1.3cm}
\setlength{\parskip}{0.2cm}

\SingleSpacing

\begin{document}
	
	\selectlanguage{brazil}
	
	\frenchspacing 
	
	\begin{center}
		\LARGE AVALIAÇÃO DE CARACTERÍSTICAS VEGETATIVAS DE CULTIVARES DE SOJA COM TRANSGENIA\\RR E INTACTA\footnote{Trabalho realizado dentro da (área de Conhecimento CNPq: Ciências Agrárias) com financiamento do CNPq / IFRO}
		
		\normalsize
	Gabriela Alves Corrêa\footnote{Colaboradora, bicacorrd@gmail.com, Campus Ariquemes} 
		Lucas Belarmino dos Santos Almeida\footnote{Bolsista (PIBIC EM), lucas261198.lb@gmail.com, Campus Ariquemes} 
		Gabriel da Silva Stoinski\footnote{Bolsista (PIBIC EM), gabrielstoinsk@gmail.com, Campus Ariquemes} \\
		Lenita Aparecida Conus Venturoso\footnote{Orientadora, lenita.conus@ifro.edu.br, Campus Ariquemes} 
		Luciano dos Reis Venturoso\footnote{Co-orientador, luciano.venturoso@ifro.edu.br, Campus Ariquemes}
		
	\end{center}
	
	\noindent O aumento no cultivo de soja no país, ligado a importância dessa commoditie no agronegócio nacional, impulsionou o desenvolvimento de materiais geneticamente modificados. Diante deste cenário, tornou cada vez mais necessários estudos que possam determinar cultivares com melhor adaptação a cada região edafoclimática. Objetivou-se com a pesquisa, avaliar o desempenho de dezesseis cultivares de soja transgênica, no município de Ariquemes. O ensaio experimental foi conduzido na área experimental do Instituto Federal de Rondônia, Campus Ariquemes. Adotou-se o delineamento experimental de blocos casualizados, com quatro repetições. O preparo do solo foi executado de modo convencional. Foram cultivados os materiais, BRBMG 12-008, BRB 11-15660, BRB 11-16058, BRBMG 12-0019, BRi 12-20669, BRi 12-20675, BRB 12-20587, BRASRR 12-13375, BRGO 11-3814-3, BRS 8890RR, BRRY 45-16349, P98Y11, M 8210 IPRO, P98Y30, TMG 132 RR e BRSMG 850GRR. As cultivares foram semeadas em parcelas contendo quatro linhas de 5 m de comprimento, espaçadas entre si por 0,45 m e densidade seguindo as recomendações de cada material. Foram avaliados a população de plantas, ciclo, percentual de acamamento, altura de plantas no florescimento, altura na colheita e inserção da primeira vagem. Foi verificado efeito significativo para as variáveis população de plantas, ciclo, altura de plantas no florescimento e na colheita e inserção da primeira vagem. Não houve materiais que se destacaram para todas as variáveis vegetativas. A população variou de 63.750 a 161.250 plantas.ha-1, ficando abaixo do indicado para alguns materiais. O ciclo fenológico das cultivares apresentou variação quando comparado aos resultados de outras regiões, indicando a particularidade com as condições edafoclimáticas do estado. Os materiais P98Y11, BRB 11-16058, BRS 8890RR, P98Y30 e M 8210 IPRO apresentaram altura de plantas abaixo da indicada para colheita mecanizada.
	
	\vspace{\onelineskip}
	
	\noindent
	\textbf{Palavras-chave}: Glycine max, Cultivares transgênicas, Soja em Rondônia. \\
	\textbf{Fonte de Financiamento} CNPq e Instituto Federal de Rondônia.
	
\end{document}
