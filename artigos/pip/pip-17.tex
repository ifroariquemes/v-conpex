\documentclass[article,12pt,onesidea,4paper,english,brazil]{abntex2}

\usepackage{lmodern, indentfirst, nomencl, color, graphicx, microtype, lipsum}			
\usepackage[T1]{fontenc}		
\usepackage[utf8]{inputenc}		

\setlrmarginsandblock{2cm}{2cm}{*}
\setulmarginsandblock{2cm}{2cm}{*}
\checkandfixthelayout

\setlength{\parindent}{1.3cm}
\setlength{\parskip}{0.2cm}

\SingleSpacing

\begin{document}
	
	\selectlanguage{brazil}
	
	\frenchspacing 
	
	\begin{center}
		\LARGE AVALIAÇÃO DE DIFERENTES COMPOSTOS ORGÂNICOS NA PRODUÇÃO
		DE BETERRABA (BETA VULGARIS L.) EM AMBIENTE PROTEGIDO\footnote{Trabalho realizado dentro das Ciências Agrárias com financiamento do DEPESP}
		
		\normalsize
		Dayane Barbosa Pereira\footnote{Bolsista, (Ensino superior), dayane\_barbosa13@hotmail.com, Campus Colorado do Oeste} 
		João Vinicius Caetano Plentz\footnote{Bolsista, (Ensino Médio), viniciuscaetanoplentz@gmail.com, Campus Colorado do Oeste} 
	Darllan Junior Luiz Santos Ferreira de Oliveira\footnote{Colaborador, darllan\_junior58@hotmail.com, Campus Colorado do Oeste} 
		Luiz Cobiniano de Melo Filho\footnote{Orientador, luiz.cobiniano@ifro.edu.br, Campus Colorado do Oeste}
		Marcos Aurélio Anequine Macedo5\footnote{Co-orientador, marcos.anequine@ifro.edu.br, Campus Colorado do Oeste} 
	\end{center}
	
	\noindent A beterraba (Beta vulgaris L.) é pertencente à família quenopodiácea na qual sua
	parte comestível é a raiz tuberosa representada pelo formato globular, seu sistema
	radicular é caracterizado como pivotante, alcançando a raiz principal até 60 cm de
	profundidade. O objetivo foi de avaliar diferentes compostos orgânicos em
	ambiente protegido para a produção de beterraba no Cone Sul do estado de
	Rondônia. O solo ao qual foi estudado é classificado como Gleissolo. O
	delineamento experimental foi em blocos casualizados com quatro repetições e
	nove tratamentos.Testemunha, adubação, fertipeixe, biofertilizante,compostagem,
	fertipeixe+compostagem, biofertilizante+compostagem, fertipeixe+biofertilizante,
	biofertilizante+fertipeixe+compostagem, com um total de 144 vasos, constituídos
	assim por uma planta. As mudas de beterraba foram produzidas em bandejas de
	isopor de 200 cédulas e irrigadas diariamente por microaspersão. Após 30 dias da
	semeadura, as mesmas foram transplantadas para os vasos, onde permaneceram
	até o ponto de colheita em casa de vegetação. A adubação de base foi realizada
	de acordo com a análise de solo, exceto na testemunha que não houve nenhum
	tipo de adubação. As aplicações dos produtos foram realizadas diretamente no
	solo, semanalmente, nas dosagens de 6 ml de biofertilizante e 2ml de fertipeixe,
	em um volume de 250 ml de água. As beterrabas foram colhidas, levadas para
	laboratório e feita as avaliações de produtividade e diâmetro, utilizando balança
	digital e o paquímetro. Concluiu-se que os tratamentos bio+ferti+comp e
	compostagem demonstraram melhores resultados.
	
	\vspace{\onelineskip}
	
	\noindent
	\textbf{Palavras-chave}: Biofertilizante. Fertipeixe. Compostagem. \\
	\textbf{Fonte de financiamento}: DEPESP
	
\end{document}
