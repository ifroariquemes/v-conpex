\documentclass[article,12pt,onesidea,4paper,english,brazil]{abntex2}

\usepackage{lmodern, indentfirst, nomencl, color, graphicx, microtype, lipsum}			
\usepackage[T1]{fontenc}		
\usepackage[utf8]{inputenc}		

\setlrmarginsandblock{2cm}{2cm}{*}
\setulmarginsandblock{2cm}{2cm}{*}
\checkandfixthelayout

\setlength{\parindent}{1.3cm}
\setlength{\parskip}{0.2cm}

\SingleSpacing

\begin{document}
	
	\selectlanguage{brazil}
	
	\frenchspacing 
	
	\begin{center}
		\LARGE MULTIPLICAR OU DIVIDIR: \\CONTRIBUIÇÕES À PRÁTICA PEDAGÓGICA PARA A CONSTRUÇÃO DO CONHECIMENTO MATEMÁTICO NOS ANOS INICIAIS DA EDUCAÇÃO BÁSICA.\footnote{Trabalho realizado dentro da área de Conhecimento CNPq: Matemática.}
		
		\normalsize
		Patrícia Feitosa Basso Miranda\footnote{Mestre pelo Programa de Pós Graduação Mestrado Profissional e Rede Nacional – PROFMAT no Polo da Universidade Federal de Rondônia. patricia.basso@ifro.edu.br, Campus Porto Velho - Zona Norte.} 
		Prof. Dr. Marinaldo Felipe da Silva\footnote{Orientador, dr.marinaldo@hotmail.com, UNIR.} 
	\end{center}
	
	\noindent No ensino da Matemática o modo como o conhecimento é conduzido aos alunos é
	uma das questões mais relevantes para o processo de ensino-aprendizagem,
	podendo influenciar diretamente na qualidade deste. A aprendizagem na Matemática
	tem apresentado baixos índices de proficiências em avaliações nacionais,dados do
	Sistema de Avaliação da Educação Básica (SAEB) evidenciam que os alunos dos
	anos iniciais do Ensino Fundamental têm mostrado dificuldades na aprendizagem
	dos conceitos matemáticos, deste modo este trabalho teve como objetivo destacar
	os principais conteúdos, competências e habilidades não assimiladas pelos alunos
	concluintes do 5o ano na disciplina de Matemática, segundo a percepção dos
	professores que lecionam no 6o ano, e socializar os resultados obtidos juntamente
	aos professores das séries iniciais, tendo-se utilizado a pesquisa-ação como
	metodologia. A análise foi realizada junto às escolas públicas da cidade de Porto
	Velho – RO, e para tal, foi dividida em três etapas. A primeira etapa se deu com a
	confecção e aplicação de um questionário aos professores do 6o ano. Tal
	questionário objetivou averiguar as principais dificuldades dos alunos em relação às
	competências e habilidades na disciplina de Matemática que não foram assimiladas
	no Ensino Fundamental I. Os dados coletados foram tratados estatisticamente com
	suporte do software EPI INFO versão 3.5.2. A segunda etapa se deu em razão da
	aplicação in loco de uma oficina com as professoras na Escola Estadual de Ensino
	Fundamental Maria Carmosina, onde ocorreu a socialização dos dados e aplicação
	de atividades voltadas para a melhoria do ensino. Na última etapa foi aplicado outro
	questionário que visava avaliar dentre outros, o nível de satisfação a respeito das
	temáticas desenvolvidas na oficina, procurando incentivá-las a utilizar de tais
	práticas em sala de aula. Diante dos dados analisados, foi possível constatar que há
	dificuldade de aprendizagem em todas as competências/habilidades pesquisadas na
	visão dos professores que responderam aos questionários, os alunos não obtiveram
	e/ou não assimilaram o conteúdo adequadamente até o término do 5o ano,deixando
	nítida a precária situação do ensino da Matemática na rede pública do município de
	Porto Velho.
	
	\vspace{\onelineskip}
	
	\noindent
	\textbf{Palavras-chave}: Matemática. Competências. Habilidades.
	
\end{document}