\documentclass[article,12pt,onesidea,4paper,english,brazil]{abntex2}

\usepackage{lmodern, indentfirst, nomencl, color, graphicx, microtype, lipsum}			
\usepackage[T1]{fontenc}		
\usepackage[utf8]{inputenc}		

\setlrmarginsandblock{2cm}{2cm}{*}
\setulmarginsandblock{2cm}{2cm}{*}
\checkandfixthelayout

\setlength{\parindent}{1.3cm}
\setlength{\parskip}{0.2cm}

\SingleSpacing

\begin{document}
	
	\selectlanguage{brazil}
	
	\frenchspacing 
	
	\begin{center}
		\LARGE CONTRIBUIÇÕES DO INSTITUTO FEDERAL DE RONDÔNIA EM QUALIFICAÇÃO E FORMAÇÃO:\\
		UM ESTUDO DOS EGRESSOS DO CURSO TÉCNICO EM QUÍMICA INTEGRADO AO ENSINO MÉDIO DO \textit{CAMPUS} CALAMA.\footnote{Trabalho realizado dentro da área de Conhecimento CNPq: 7.08.00.00-6.}
		
		\normalsize
		Ênio Gomes da Silva\footnote{Doutorando: Ênio Gomes da Silva: Professor do Instituto Federal de Educação Ciência e Tecnologia de Rondônia - \textit{Campus} Ariquemes, e-mail: enio.gomes@ifro.edu.br} 
		Ivanise Maria Rizzatti\footnote{Orientadora: Ivanise Maria Rizzatti: Professora da Universadade Estadual de Roraima – UERR, e-mail: niserizzatti@gmail.com} 
	\end{center}
	
	\noindent É evidente a existência de duas visões acerca do curso técnico em química integrado ao
	ensino médio: uma fragmentada e outra integrada. Essas dimensões, quando pensadas em
	uma formação técnica, surgem importantes reflexões, por exemplo: os conhecimentos
	relevantes para uma adequada formação na perspectiva integrada ao curso de técnico em
	química de nível médio, ressaltando que essa perspectiva é baseada no Projeto Pedagógico
	do Curso (PPC). Assim consideramos oportuno levantar a seguinte questão: A proposta
	curricular do ensino médio integrado ao curso técnico em química ofertada pelo IFRO se
	efetiva na formação de um técnico em química capaz de atuar no mercado de trabalho
	existente na região? Nossa justificativa para este estudo surgiu quando, licenciado em
	química e professor do IFRO, percebi a dificuldade que os alunos do curso técnico em
	química têm em relacionar os assuntos de química com a sua formação profissional, pois
	para o aprendizado da química é preciso experimentar os conceitos científicos, teóricos e
	suas ferramentas em atividades práticas contextualizadas que envolvem processos
	similares, àqueles presentes na pesquisa laboratorial onde a ciência real acontece de fato.
	Diante desta observação nosso problema de pesquisa está baseado na proposta curricular
	do ensino médio integrado ao técnico em química ofertada pelo IFRO Campus Calama se
	efetiva na formação de um técnico em química capaz de atuar no mercado de trabalho
	existente na região? Para tanto, tomaremos como objetivo geral será verificar se a
	qualificação proporcionada pelo IFRO Campus Calama aos egressos do Curso Técnico em
	Química Integrado ao Ensino Médio contribui para sua inserção na vida profissional. O
	caminho metodológico desta pesquisa foi fundamentado na abordagem qualitativa por se
	tratar de uma pesquisa de cunho social, por meio da qual investigaremos os processos junto
	a pessoas, lugares, ideias que resultaram principalmente em dados descritivos (BOGDAN E
	BIKLEN, 2013). Finalizando, as considerações finais e/ou recomendações para melhoria do
	referido curso em estudo.
	
	\vspace{\onelineskip}
	
	\noindent
	\textbf{Palavras-chave}: Qualificação. Técnico em Química. Ensino de Química.
	
\end{document}