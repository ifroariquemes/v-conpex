\documentclass[article,12pt,onesidea,4paper,english,brazil]{abntex2}

\usepackage{lmodern, indentfirst, nomencl, color, graphicx, microtype, lipsum}			
\usepackage[T1]{fontenc}		
\usepackage[utf8]{inputenc}		

\setlrmarginsandblock{2cm}{2cm}{*}
\setulmarginsandblock{2cm}{2cm}{*}
\checkandfixthelayout

\setlength{\parindent}{1.3cm}
\setlength{\parskip}{0.2cm}

\SingleSpacing

\begin{document}
	
	\selectlanguage{brazil}
	
	\frenchspacing 
	
	\begin{center}
		\LARGE DESENVOLVIMENTO DE UMA PLATAFORMA EDUCACIONAL DE APOIO AO ENSINO E APRENDIZAGEM DE ROBÓTICA À LUZ DA PEDAGOGIA DE PROJETOS\footnote{Trabalho realizado dentro da Educação.}
		
		\normalsize
		Ricardo Bussons da Silva\footnote{Trabalho realizado dentro da Educação. 2 Mestrando, ricardo.bussons@ifro.edu.br, IFRO - Campus Porto Velho - Calama} 
		Marinaldo Felipe da Silva\footnote{Orientador, felipe@unir.br, UNIR – Campus BR}  
	\end{center}
	
	\noindent Esta pesquisa teve como objetivo principal o desenvolvimento de um módulo
	didático para auxiliar no processo de ensino e aprendizagem de Robótica aliada a
	Pedagogia de Projetos. Para isso, iniciou-se um levantamento bibliográfico acerca
	do tema ensino de programação de hardware e módulos didáticos comercias para o
	estudo de microeletrônica, robótica e áreas afins. Em sequência, realizou-se uma
	entrevista semiestruturada com professores atuantes nos cursos técnicos integrados
	em Eletrotécnica e Informática do Instituto Federal de Educação, Ciência e
	Tecnologia de Rondônia – IFRO no Campus Porto Velho – Calama, com o intuito de
	se obter informações referentes às suas experiências profissionais como docentes e
	principais dificuldades apresentadas pelos alunos nas disciplinas que envolvam
	programação e/ou montagem de hardware, para a partir de então, elaborar um novo
	recurso didático, que reúna as principais características técnicas e, sobretudo,
	pedagógicas, julgadas pelos professores, mais importantes, e que possibilitem a
	redução das limitações proporcionadas pelos módulos didáticos atuais e auxiliem
	docentes em suas aulas práticas, de forma a melhorar a contextualização do
	conteúdo, reduzir o nível de abstração requerido do aluno nas experimentações,
	aproximar estas práticas de situações reais do cotidiano de um profissional técnico
	das áreas relacionadas a Eletricidade, e consequentemente proporcionar ao aluno
	um conhecimento mais significativo. Neste sentido, o equipamento aqui proposto
	baseia-se em sistemas embarcados à microcontroladores PIC, Arduino e Raspberry,
	podendo ainda, ser controlado por sistemas não embarcados, garantindo sua
	utilização em disciplinas de Eletrônica e Eletricidade Básica, bem como, a disciplina
	de Física, da Base Nacional Comum Curricular. Trata-se de um sistema que simula
	processos industriais automatizados utilizando esteiras transportadoras, sensores de
	posicionamento e identificação de objetos, além de interfaces homem-máquina
	audiovisuais (displays LCD, LED’s, matriz de LED’s, buzzers, entre outros). Com
	isso, acredita-se que este projeto, contribua com o desenvolvimento da criatividade,
	senso crítico, autonomia do aluno frente a resolução de problemas, e ainda auxilie o
	professor quanto a motivação do aluno, visto o caráter lúdico do equipamento.
	
	\vspace{\onelineskip}
	
	\noindent
	\textbf{Palavras-chave}: Material Didático. Robótica Educacional. Pedagogia de Projetos..
	
\end{document}