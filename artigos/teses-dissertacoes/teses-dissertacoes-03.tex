\documentclass[article,12pt,onesidea,4paper,english,brazil]{abntex2}

\usepackage{lmodern, indentfirst, nomencl, color, graphicx, microtype, lipsum}			
\usepackage[T1]{fontenc}		
\usepackage[utf8]{inputenc}		

\setlrmarginsandblock{2cm}{2cm}{*}
\setulmarginsandblock{2cm}{2cm}{*}
\checkandfixthelayout

\setlength{\parindent}{1.3cm}
\setlength{\parskip}{0.2cm}

\SingleSpacing

\begin{document}
	
	\selectlanguage{brazil}
	
	\frenchspacing 
	
	\begin{center}
		\LARGE FUNDAMENTAÇÃO METODOLÓGICA NA PRÁTICA DA CONTEXTUALIZAÇÃO
		DO ENSINO DE MATEMÁTICA NAS SÉRIES INICIAIS DE UMA ALDEIA PAITER
		SURUÍ
		
		\normalsize
		Antonio Ferreira Neto\footnote{Bolsista: Antonio Ferreira Neto. Bolsa Pesquisador, antonio.f.neto@ifro.edu.br , \textit{Campus} Cacoal.} 
		José Roberto Linhares de Mattos\footnote{Orientador, jrlinhares@gmail.com , \textit{Universidade Federal Fluminense}.} 
	\end{center}
	
	\noindent Neste resumo abordamos resultados parciais de uma pesquisa de tese ainda em
	andamento. Tomando a etnomatemática, em uma abordagem histórico-cultural,
	ajudando a entender melhor a capacidade de um povo criar e coletivizar
	representações da realidade. Neste trabalho o objetivo principal é investigar, através
	de uma perspectiva etnomatemática, a cultura matemática da etnia Paiter Suruí no
	cotidiano das aldeias, observando como o professor indígena Paiter Suruí
	desenvolve suas atividades. A pesquisa tem caráter qualitativo, utilizando como
	instrumentos observação participante e entrevistas a moradores, alunos e
	professores das aldeias. No decorrer da pesquisa notamos diferentes maneiras que
	podem ser usadas para ensinar conceitos da Matemática tomando como viés a
	cultura de um povo, em um ambiente formal e informal, minimizando barreiras na
	aprendizagem e tornando-a mais significativa. Destacamos a importância do
	conhecimento matemático para os indígenas e a utilização desses saberes na
	manutenção das aldeias (venda de produtos, construção, plantios, entre outros).
	Verificamos quais os conteúdos de matemática são mais utilizados e que os auxiliam
	no seu trabalho e no desenvolvimento das aldeias. Durante a pesquisa
	evidenciamos o processo interdisciplinar como ferramenta indispensável na
	superação das dificuldades de ensinar nas aldeias (falta de material adequado, falta
	de alguns recursos didáticos etc.). A contextualização da matemática, auxilia o
	educando e o educador a entender a realidade holística do saber fazer, passar a
	entender fatos que observados de outro ângulo tornam-se menos complicados,
	instiga à pesquisa e, por conseguinte, a descoberta, numa volta a sua origem.
	Ensinar matemática para um povo da floresta, usando como plataforma sua cultura
	com seus significados e formas diversificadas, é tirá-lo da aprendizagem entre
	quatro paredes e expô-lo a um cenário inovador em que a teoria se confunde com a
	realidade, imergindo o aluno em formas e modelos compreensíveis.
	
	\vspace{\onelineskip}
	
	\noindent
	\textbf{Palavras-chave}: Contextualização. Educação escolar indígena. Etnomatemática.
	
\end{document}