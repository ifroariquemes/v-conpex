\documentclass[article,12pt,onesidea,4paper,english,brazil]{abntex2}

\usepackage{lmodern, indentfirst, nomencl, color, graphicx, microtype, lipsum}			
\usepackage[T1]{fontenc}		
\usepackage[utf8]{inputenc}		

\setlrmarginsandblock{2cm}{2cm}{*}
\setulmarginsandblock{2cm}{2cm}{*}
\checkandfixthelayout

\setlength{\parindent}{1.3cm}
\setlength{\parskip}{0.2cm}

\SingleSpacing

\begin{document}
	
	\selectlanguage{brazil}
	
	\frenchspacing 
	
	\begin{center}
		\LARGE ESTATIVA COM SISTEMA\\AUTOMATIZADO DE DISPARO\footnote{Trabalho realizado dentro da área de Engenharias, com financiamento do Instituto Federal de Educação, Ciência e Tecnologia – IFRO – \textit{Campus} Porto Velho Calama (2016).}
		
		\normalsize
		Jean Peixoto Campos\footnote{Orientador, jean.campos@ifro.edu.br, \textit{Campus} Porto Velho Calama.} 
		Carlos Augusto Bauer Aquino\footnote{Co-orientador, carlos.augusto@ifro.edu.br, \textit{Campus} Porto Velho Calama.} 
		Vlademir Fernandes de Oliveira Júnior\footnote{Co-orientador, vlademir.fernandes@ifro.edu.br, \textit{Campus} Porto Velho Calama.} 
		Hudson Alberto Vieira Oliveira\footnote{Bolsista (PIBIC), hudsonalbrt@gmail.com, \textit{Campus} Porto Velho Calama.}
		Vinícius Rian Rodrigues da Silva\footnote{Bolsista (PIBIC), vinicius.rianpvh@gmail.com, \textit{Campus} Porto Velho Calama.}\\
		Giovani Patrick Bevilacqua\footnote{Colaborador, giovani.fisica@gmail.com, \textit{Campus} Porto Velho Calama.}
		José Loureiro Curvelo Filho\footnote{Colaborador, curvelopvh@hotmail.com, \textit{Campus} Porto Velho Calama.} 
	\end{center}
	
	\noindent Em vista da necessidade identificada junto ao Instituto de Criminalística (IC) da
	Polícia Técnica Científica do Estado de Rondônia (POLITEC), a presente pesquisa
	visou subsidiar a construção de um sistema de disparo automatizado para armas de
	fogo. Assim, foi desenvolvido um equipamento que atendesse aos requisitos
	técnicos e agilizassem os procedimentos periciais, com vistas a promover maior
	segurança nos testes de balística realizados. Desta forma, o IFRO – Campus Porto
	Velho Calama, se engajou no estudo do problema e na criação do equipamento que
	suprisse o especificado. O método tomou por base conceitos da pesquisa
	observacional e experimental e suas devidas técnicas associadas. Foram criados
	diversos protótipos com diferentes materiais (isopor, madeira, metal) para uma
	visualização dos elementos fixos e móveis, bem como do sistema de operação do
	equipamento denominado estativa. Após testes e observações dos peritos do IC,
	foram realizados os ajustes sugeridos que levaram à versão atual, em que se
	implementou o sistema automatizado de disparo planejado. A partir da necessidade
	identificada, iniciou-se uma pesquisa por equipamentos existentes com finalidade
	similar em que não retornou nenhum resultado, tanto no INPI como em diversos
	bancos de patentes. Assim, passou-se ao processo de criação da estativa. Seguindo
	os preceitos dos métodos da pesquisa, implementou-se as etapas: (i) criação de um
	modelo para identificar os elementos fixos e móveis, bem como suas disposições no
	conjunto; (ii) criação de um modelo para aperfeiçoamento do sistema de fixação e
	ajuste de altura tanto para armas curtas como longas; (iii) construção do primeiro
	protótipo para testes do sistema de fixação e de disparo e (iv) versão atual com
	adequações nos sistemas de fixação, altura e disparo automatizado. Com o intuito
	de aperfeiçoar a segurança aos peritos, a construção da estativa se balizou pelo
	método observacional e experimental. Desenvolvendo-se por uma série de etapas,
	algumas com interação dos peritos, em busca de um produto que incorpora
	materiais resistentes, sistema elétrico e design específico com o objetivo de prover
	segurança aos testes de balística. Ou seja, a estativa com sistema automatizado de
	disparo é um equipamento para testes em balística que permite a fixação de armas
	de fogo de diversos modelos e acionamento automático de disparo.
	
	\vspace{\onelineskip}
	
	\noindent
	\textbf{Palavras-chave}: Estativa. Balística. Automação.
	
\end{document}