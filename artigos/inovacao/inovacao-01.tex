\documentclass[article,12pt,onesidea,4paper,english,brazil]{abntex2}

\usepackage{lmodern, indentfirst, nomencl, color, graphicx, microtype, lipsum}			
\usepackage[T1]{fontenc}		
\usepackage[utf8]{inputenc}		

\setlrmarginsandblock{2cm}{2cm}{*}
\setulmarginsandblock{2cm}{2cm}{*}
\checkandfixthelayout

\setlength{\parindent}{1.3cm}
\setlength{\parskip}{0.2cm}

\SingleSpacing

\begin{document}
	
	\selectlanguage{brazil}
	
	\frenchspacing 
	
	\begin{center}
		\LARGE APLICATIVO PARA SÉRIES INFINITAS\\COM RAZÕES MÚLTIPLAS\footnote{Trabalho realizado dentro da Ciências Exatas e da Terra com financiamento da PROPESP – IFRO.}
		
		\normalsize
		Gleison Guardia\footnote{Autor, gleison.guardia@ifro.edu.br, \textit{Campus} Ji-Paraná.} 
		Michel da Silva\footnote{Co-Autor, michel.silva@ifro.edu.br, \textit{Campus} Ji-Paraná.} 
		Milena Rufini de Andrade\footnote{Bolsista, drak.milena@gmail.com, \textit{Campus} Ji-Paraná.} 
	\end{center}
	
	\noindent O estudo das séries matemáticas proporcionam um avanço nos conceitos de análise
	e probabilidade em demandas nos mercados futuros. As séries, desde a sua
	descoberta, com avanços feitos por Gauss e Newton, limitam-se a estudar os efeitos
	de uma única razão influenciando as projeções dos dados; visto que, uma série real
	poderia apresentar outros fatores determinantes para a simulação, que durante sua
	construção são imprescindíveis na geração dos dados, mais eficiente e sensata
	seria a construção histórica que absorvesse essas particularidades. A proposta do
	trabalho foi o desenvolvimento de um aplicativo para dispositivos móveis, que usa
	como princípio não uma série aritmética ou geométrica simples, mas aquela que
	permite a influência de \textit{n} razões na sua projeção. Baseando-se em conceitos
	defendidos por Bisognin (2014), Luckesi (2012), Muniz (2008) e Polya (2013), a
	equipe composta inicialmente por um professor de Matemática, um de Informática e
	dois alunos, sendo um do curso Técnico em Informática integrado ao Ensino Médio e
	outro do Tecnólogo em Análise e Desenvolvimento de Sistemas, buscaram meios
	matemáticos para modificar as tradicionais fórmulas da Progressão Aritmética como
	também a da Progressão Geométrica, permitindo a inserção de várias razões; após
	algum tempo de ataque as referidas estruturas, a equipe final, obteve êxito em
	desenvolver conceitos matemáticos inéditos, que permitem o uso de razões
	múltiplas influenciando uma série geral, servindo de base para a construção dos
	algorítimos de processamento do aplicativo, dando assim a confiabilidade ao
	funcionamento para qualquer variável, embasada no poder demostrativo da
	matemática. O aplicativo produzido, permite ao usuário simular qualquer ambiente
	onde necessite verificar projeções futuras de mercado ou sistema lógico, que precise
	de um ou mais fatores influentes, resultando num valor preciso temporal ou a soma
	de dados para a data. Este conceito permite previsões mais rápidas e precisas para
	situações diversas, onde em um único processamento, absorve vários senários de
	simulações em separado, economizando tempo, hora máquina e usos excessivos de
	correlação e análises estatísticas.
	
	\vspace{\onelineskip}
	
	\noindent
	\textbf{Palavras-chave}: Séries. Razões Múltiplas. Aplicativo.
	
\end{document}