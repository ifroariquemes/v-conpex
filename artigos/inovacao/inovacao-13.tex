\documentclass[article,12pt,onesidea,4paper,english,brazil]{abntex2}

\usepackage{lmodern, indentfirst, nomencl, color, graphicx, microtype, lipsum, textcomp}			
\usepackage[T1]{fontenc}		
\usepackage[utf8]{inputenc}		

\setlrmarginsandblock{2cm}{2cm}{*}
\setulmarginsandblock{2cm}{2cm}{*}
\checkandfixthelayout

\setlength{\parindent}{1.3cm}
\setlength{\parskip}{0.2cm}

\SingleSpacing

\begin{document}
	
	\selectlanguage{brazil}
	
	\frenchspacing 
	
	\begin{center}
		\LARGE PRODUÇÃO DE FERMENTADOS SECOS E SUAVES “VINHO” DE CAMU-CAMU
		(\MakeUppercase{\textit{Myrciaria dubia}}), ARAÇÁ-BOI (\MakeUppercase{\textit{Eugenia stipitata}}) E\\CAJÁ-MANGA (\MakeUppercase{\textit{Spondias dulcis}}),\\FRUTOS TIPICOS DA REGIÃO NORTE
		
		\normalsize
		Daniella da Silva Sousa\footnote{Bolsista, Licenciatura em Química, daniellaesousa@gmail.com, \textit{Campus} Jí-Paraná.} 
		Renato André Zan\footnote{Coordenador, Professor EBTT de Química email: renato.zan@ifro.edu.br , \textit{Campus} Ji-Paraná.} 
	\end{center}
	
	\noindent Uma das atividades mais antigas que esteve presente na vida do
	homem desde as primeiras civilizações é a produção e o consumo de bebidas
	alcoólicas. O vinho se define por uma bebida alcoólica resultante da fermentação do
	mosto (sumo do fruto antes de terminada a fermentação) simples da uva – \textit{Vitis
	vinifera} - sã, fresca e madura. Alguns autores afirmam que este nome “vinho” seja
	apenas usado para bebida proveniente da uva, pois seu nome científico traduzido do
	latim significa videira. De acordo com a legislação, Decreto-Lei N. 2.499/38 Cap. VI,
	os produtos obtidos pela fermentação alcoólica de frutas frescas, maduras, é
	permitido o nome de vinho, seguido da declaração expressa de sua natureza, no
	rótulo, em caracteres nítidos e de igual tamanho. Ex.: vinho de laranja. Atualmente
	existem vários tipos de vinhos de frutas, e, para este projeto foram escolhidos alguns
	frutos da região norte: Camu-Camu (\textit{Myrciaria dubia}), Araçá-boi (\textit{Eugenia stipitata}) e Cajá-Manga (\textit{Spondias dulcis}), com os objetivos de valorizar os frutos desta região, propor uma nova possibilidade de renda a sitiantes da região, que tem contato com esses frutos, divulgar os frutos da região através da produção de uma bebida
	apreciada no mundo todo. Avaliar algumas características
	físico-químicas da bebida produzida em relação ao teor alcoólico, oBrix, densidade,
	pH, para verificação dos seus produtos finais. Os métodos de analise utilizados
	estavam de acordo com as técnicas do Instituto Adolfo Lutz, a produção e as
	análises dos vinhos foram realizadas no laboratório do IFRO, Campus Ji-Paraná. As
	etapas utilizadas para a produção do vinho: 1. DESPOLPA: Etapa em que a polpa
	do fruto é extraída e triturada. 2. FERVURA: É responsável por separar o liquido
	para a fervura. 3. ADIÇÃO DE SACAROSE: Momento em que é adicionado a
	sacarose até obter o oBrix desejado. 4. FERMENTAÇÃO: Responsável pela variação
	alcoólica, onde um conjunto de reações enzimaticamente são controladas. A glicose
	é uma das substâncias mais empregadas pelos microrganismos como ponto de
	partida na fermentação. 5. FILTRAÇÃO: Separação de impurezas, retirada de
	partículas indesejadas da bebida. 6. MATURAÇÃO: Processo de amadurecimento
	da bebida. Durante o processo de fermentação foram
	realizadas as análises experimentais para controle e acompanhamento de
	densidade, pH, acidez total, °Brix, teor alcoólico, e ao final obteve-se resultados
	satisfatórios, de acordo com o esperado (padrão estipulado pela legislação).
	 Após todo o desenvolvimento do projeto, pode-se ter certeza do grande
	potencial desses frutos na produção de vinhos sendo que no final desta pesquisa
	todos os vinhos foram enviados para registro de patente, para assegurar os direitos
	da pesquisa.
	
	\vspace{\onelineskip}
	
	\noindent
	\textbf{Palavras-chave}: Vinho. Fermentação. Frutos.
	
\end{document}