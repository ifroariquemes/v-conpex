\documentclass[article,12pt,onesidea,4paper,english,brazil]{abntex2}

\usepackage{lmodern, indentfirst, nomencl, color, graphicx, microtype, lipsum}			
\usepackage[T1]{fontenc}		
\usepackage[utf8]{inputenc}		

\setlrmarginsandblock{2cm}{2cm}{*}
\setulmarginsandblock{2cm}{2cm}{*}
\checkandfixthelayout

\setlength{\parindent}{1.3cm}
\setlength{\parskip}{0.2cm}

\SingleSpacing

\begin{document}
	
	\selectlanguage{brazil}
	
	\frenchspacing 
	
	\begin{center}
		\LARGE DOMINÓ DE CARTAS DE ISOMERIA\footnote{Trabalho realizado dentro da área de Ciências Exatas e da Terra.}
		
		\normalsize
		Patrick Cruz Moreira da Silva\footnote{Aluno do Curso Técnico em Química – IFRO - \textit{Campus} Ji-Paraná.} 
		Larissa Ferreira Soares\footnote{Aluna do Curso Técnico em Química – IFRO - \textit{Campus} Ji-Paraná.} 
		Gabriella Saldanha Simplicio\footnote{Aluna do Curso Técnico em Química – IFRO - \textit{Campus} Ji-Paraná.} 
		Larissa Dos Santos Silva\footnote{Aluna do Curso Técnico em Química – IFRO - \textit{Campus} Ji-Paraná.}
		Gabrielly Correia de Jesus\footnote{Aluna do Curso Técnico em Química – IFRO - \textit{Campus} Ji-Paraná}
		Camila Ellen Ferrerira Oliveira\footnote{Aluna da Licenciatura em Química – IFRO - \textit{Campus} Ji-Paraná}
		Alecsandra de Oliveira Souza\footnote{Orientador (a), alecsandra.souza@ifro.edu.br,– IFRO - \textit{Campus} Ji-Paraná.} 
	\end{center}
	
	\noindent Um fenômeno bem conhecido na área da química é a isomeria, que consiste em
	dois ou mais compostos orgânicos diferentes, mas que apresentam mesmas
	fórmulas moleculares e diferentes fórmulas estruturais. Os conceitos de isomeria se
	aplicam em varias áreas de conhecimento do cotidiano. As adversidades de seu
	ensino e aplicações começam no ensino médio, onde os alunos aprendem a teoria e
	de uma forma bem estática. Em algumas escolas os alunos nem chegam a estudar
	o conteúdo. Aos alunos que chegam a estudar o conteúdo, suas principais dúvidas
	são em como diferenciar as classificações isoméricas, visto que este é um conteúdo
	vasto, e que apresenta detalhes importantes para diferenciação de suas
	classificações. Assim o jogo didático proposto, dominó de cartas de isomeria, auxilia
	na aprendizagem de isomeria, sendo uma metodologia alternativa no processo de
	ensino e aprendizagem do conteúdo. O material propostos busca a interação da
	teoria com a aplicação prática dos conceitos da isomeria acerca de classificações,
	assim, estimula um maior aprendizado, fixação e interação social. Possuindo 28
	cartas, sendo 7 peças-chave e outras 21 peças com duas estruturas químicas
	isômeras aleatórias em cada peça, um total de 42 estruturas químicas, forma-se o
	dominó de isômeros em cartas de baralho. As peças chaves condizem às
	classificações isoméricas, sendo: Funcional, Esqueletal, Posicional, Tautomérica,
	Espacial Cis-Trans, Espacial E-Z e Óptico. Cada classe dessa conta com seis
	isômeros, que ao longo de cada partida forma caminhos como no jogo de dominó
	convencional. Com uma aplicabilidade muito dinâmica, esse jogo pode ser utilizado
	em sala de aula, como metodologia alternativa de aplicação do conteúdo, podendo
	ter valor de atividades de fixação de conteúdo a verificação de aprendizagem dos
	mesmos, visto que proporciona a revisão de todos os tipos de isomeria desta forma
	além de sair das mesmas didáticas convencionais do ensino os alunos adquirem a
	possibilidades de possuir versões do jogo em suas casas, possibilitando praticar
	com os amigos, estudando para provas e vestibulares, tudo isso de uma maneira
	diferente e inovadora.
	
	\vspace{\onelineskip}
	
	\noindent
	\textbf{Palavras-chave}: Isomeria. Jogo. Didáticas alternativas.
	
\end{document}