\documentclass[article,12pt,onesidea,4paper,english,brazil]{abntex2}

\usepackage{lmodern, indentfirst, nomencl, color, graphicx, microtype, lipsum}			
\usepackage[T1]{fontenc}		
\usepackage[utf8]{inputenc}		

\setlrmarginsandblock{2cm}{2cm}{*}
\setulmarginsandblock{2cm}{2cm}{*}
\checkandfixthelayout

\setlength{\parindent}{1.3cm}
\setlength{\parskip}{0.2cm}

\SingleSpacing

\begin{document}
	
	\selectlanguage{brazil}
	
	\frenchspacing 
	
	\begin{center}
		\LARGE DESENVOLVIMENTO DE APLICATIVO DE VIAGENS NA REGIÃO AMAZÔNICA
		
		\normalsize
		Danielle Menezes Marrieli\footnote{Danielle Menezes Marrieli, dani.teceletro2016@gmail.com, Porto Velho-Calama} 
		Jonas Wellington Menezes de Andrade\footnote{Jonas Wellington Menezes de Andrade, jonaswellington1@outlook.com, Porto Velho-Calama} 
		Jorge Vandeman de Jesus Torres\footnote{Jorge Vandeman de Jesus Torres, jorgevandeman109@gmail.com, UNIR} 
		Matheus Alves da Silva\footnote{Matheus Alves da Silva, matheusalvesmg9@gmail.com, Porto Velho-Calama}
		Samuel Clementino de Oliveira Rocha\footnote{Samuel Clementino de Oliveira Rocha, samucorocha1@gmail.com}
		Thiago Andrey Azevedo Reis\footnote{Thiago Andrey Azevedo Reis, thiagoandrey11@gmail.com, UNIR} 
	\end{center}
	
	\noindent TRAVEL'S foi uma ideia criada durante o evento StartUp Weekend em Porto Velho em 2017. A TRAVEL'S é um aplicativo que contribuirá, a partir do uso da tecnologia
	digital no desenvolvimento do turismo na região norte do país. Nossa missão é
	assegurar que viajantes possam conhecer, e visitar diversos pontos turísticos na
	região através de informações confiáveis. Da mesma forma, contribuímos para o
	mercado do turismo e com a competitividade do turismo nacional. Logo, a TRAVEL'S
	vem para transformar e ajudar a reconstruir o turismo em nossa região. A TRAVEL'S
	espera se torna uma grande empresa no setor de turismo. Inicialmente os esforços
	da organização serão focados a coletas de dados e organização de informações
	referentes ao turismo nos estados. Após apuração dos dados, partimos para a o
	desenvolvimento do aplicativo. O objetivo principal é fomentar o turismo na região e
	possibilitar que o turista conheça, viva e compartilhe a beleza de vários lugares,
	através de uma rede que conecta viajantes, reunindo os principais pontos turísticos
	espalhados pela região. O aplicativo também vem preparado para auxiliar o usuário
	no planejamento da viagem, disponibilizando os recursos de compra de passagens,
	reservas de hotéis, reservas de restaurantes, atrações locais e localização de
	serviços essenciais (Hospital, Central de Polícia, Bancos, etc.). Com o aumento de
	usuários mobile as buscas e reservas feitas on-line vem crescendo cada vez mais, o
	uso de aplicativos como agências de viagens tem sido essencial pelos turistas, para
	a organização e definição de seus roteiros de turismo e utilização de serviços de
	hospedagem.
	
	\vspace{\onelineskip}
	
	\noindent
	\textbf{Palavras-chave}: StartUp. Turismo. Amazônia. Rede Social.
	
\end{document}