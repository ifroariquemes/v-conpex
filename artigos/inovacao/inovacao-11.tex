\documentclass[article,12pt,onesidea,4paper,english,brazil]{abntex2}

\usepackage{lmodern, indentfirst, nomencl, color, graphicx, microtype, lipsum}			
\usepackage[T1]{fontenc}		
\usepackage[utf8]{inputenc}		

\setlrmarginsandblock{2cm}{2cm}{*}
\setulmarginsandblock{2cm}{2cm}{*}
\checkandfixthelayout

\setlength{\parindent}{1.3cm}
\setlength{\parskip}{0.2cm}

\SingleSpacing

\begin{document}
	
	\selectlanguage{brazil}
	
	\frenchspacing 
	
	\begin{center}
		\LARGE FABRICAÇÃO DE CERVEJAS DO TIPO APA (\MakeUppercase{\textit{American Pale Ale}}) PRODUZIDAS
		A PARTIR DE CAMU-CAMU (\MakeUppercase{\textit{Myrciaria dubia}}),\\ BACUPARI ANÃO (\MakeUppercase{\textit{Garcinia
		brasiliensis}}) FRUTAS TIPICAS DA REGIÃO NORTE
		
		\normalsize
		Débora Taisa Keller da Silva\footnote{Bolsista, Licenciatura em Química, deborakeller01@gmail.com, \textit{Campus} Jí-Paraná.} 
		Renato André Zan\footnote{Coordenador, Professor EBTT de Química, email:renato.zan@ifro.edu.br, \textit{Campus} Jí-Paraná.}
	\end{center}
	
	\noindent Desde a antiguidade as bebidas alcoólicas sempre estiveram presente na vida do homem. A Região Norte possui uma grande variedade de frutos a serem
	explorados, onde a produção da cerveja artesanal com o fruto camu-camu (\textit{Myrciaria
	dubia}) e bacupari anão (\textit{Garcinia brasiliensis}) têm uma grande finalidade na área
	tecnológica de inovação, onde temos como objetivos a valorização dos frutos da
	região e fonte de renda para agricultores, que tem contato com esses frutos, divulgar
	os frutos regionais através da produção de uma bebida apreciada no mundo todo.
	Os ingredientes utilizados para a fabricação foram; malte agrária, malte
	Monique tipo 2, cereal Barley, lúpulo e fermento e a polpa do fruto de camu-camu
	(\textit{Myrciaria dubia}) e bacupari anão (\textit{Garcinia brasiliensis}). Iniciou-se moendo os maltes e cereais, em seguida foi adicionada água destilada em uma panela, quando
	a temperatura da água chegou a 70$^\circ$C, os maltes e cereais foram adicionados, a
	temperatura foi mantida constante em 66$^\circ$C durante uma hora, após esse
	procedimento realizou-se o teste do iodo, logo obteve uma recirculação de liquido,
	em seguida foi filtrado. O líquido foi transferido para um balde. Após esse processo
	foi transferido novamente para a panela, e adicionado o lúpulo e fervido por 50
	minutos com a tampa da panela aberta. Em seguida foi submetido ao banho de gelo
	e deixado esfriar até chegar em 30$^\circ$C, em seguida adicionado o fermento.
	Transferiu-se o liquido para um balde de fermentação. O balde com o mosto
	permaneceu fechado durante 15 dias. Após esse período o balde foi aberto, o liquido
	filtrado, e foi transferido para uma panela, e fervido com açúcar. Em seguida
	envasado e maturado. Durante todo o processo, ao comparar os
	resultados obtidos, tanto na fabricação quanto nas analises físico-químicas da
	produção da cerveja teve resultados satisfatório, confirmando que os resultados são
	bem próximos do padrão estipulado. Pode-se confirmar que o fruto tem propriedades
	químicas essenciais para uma boa produção de bebidas alcoólicas. Sendo assim,
	pode ser usada comercialmente na fabricação de cerveja. Após todo o
	desenvolvimento do projeto, pode-se ter certeza do grande potencial desses frutos
	na produção de cervejas sendo que no final desta pesquisa todos os processos de
	produção foram enviados para registro de patente, para assegurar os direitos da
	pesquisa.
	
	\vspace{\onelineskip}
	
	\noindent
	\textbf{Palavras-chave}: Cerveja. Frutas. Fabricação.
	
\end{document}