\documentclass[article,12pt,onesidea,4paper,english,brazil]{abntex2}

\usepackage{lmodern, indentfirst, nomencl, color, graphicx, microtype, lipsum}			
\usepackage[T1]{fontenc}		
\usepackage[utf8]{inputenc}		

\setlrmarginsandblock{2cm}{2cm}{*}
\setulmarginsandblock{2cm}{2cm}{*}
\checkandfixthelayout

\setlength{\parindent}{1.3cm}
\setlength{\parskip}{0.2cm}

\SingleSpacing

\begin{document}
	
	\selectlanguage{brazil}
	
	\frenchspacing 
	
	\begin{center}
		\LARGE CAIXA ELETRÔNICA PARA ENSINO DE LÓGICA E MATEMÁTICA\footnote{Trabalho realizado dentro da (área de Conhecimento CNPq: Ciência da Computação (10300007))
		com financiamento do Instituto Federal de educação, Ciência e Tecnologia de Rondônia - IFRO.}
		
		\normalsize
		Roberto Simplício\footnote{Orientador, roberto.simplício@ifro.edu.br, \textit{Campus} Vilhena.} 
		Clayton Ferraz Andrade\footnote{Colaborador, clayton.andrade@ifro.edu.br, \textit{Campus} Vilhena.} 
		Rodrigo Stiz\footnote{Colaborador, rodrigo.stiz@ifro.edu.br, \textit{Campus} Vilhena.} 
		Claudia Aparecida Prates\footnote{Colaboradora, claudia.prates@ifro.edu.br, \textit{Campus} Vilhena.} 
	\end{center}
	
	\noindent Tem-se observado que muitos dos alunos, embora tenham conhecimentos
	matemáticos oriundos do ensino básico ou médio, para resolver problemas
	matemáticos, ainda encontram dificuldades na passagem do raciocínio matemático
	para o correspondente computacional, dificuldades na passagem do raciocínio
	intuitivo, ainda que matemático, para uma linguagem de programação dificultando
	dessa forma, o aprendizado de lógica de programação. Observa-se que uma grande
	parcela dos alunos encontra. Com o objetivo de contribuir na minimização dessa
	problemática, foi construída uma caixa eletrônica para o ensino de lógica e
	matemática. Trata-se de um dispositivo didático de ensino de lógica de programação
	e princípios matemáticos, para uso em instituições públicas e privadas, do ensino
	básico, médio, médio técnico ou graduação. O referido dispositivo didático tem como
	finalidade o ensino de lógica de programação, eletroeletrônica embarcada e
	princípios matemáticos, tais como: geometria, plano cartesiano, sistema de
	coordenadas, álgebra, soma, produto, subtração, divisão, álgebra booleana. É um
	instrumento eletrônico construído baseado na plataforma Arduino acondicionado em
	uma caixa de material resistente, medindo 10 cm x 15 cm, onde estão dispostos
	ledes de diversas cores, botões de acionamento e um pequeno alto-falante. Sua
	aplicação no recente minicurso de Arduino, no EIIFRO, no mês de agosto de 2017,
	realizado no \textit{Campus} Vilhena mostrou ser uma excelente ferramenta de apoio que
	amplia as possibilidades e diversidades de ensino, promove a criatividade dos
	alunos e estimula a construção do conhecimento em novas áreas, além de tornar o
	fazer pedagógico mais dinâmico e diversificado. Também contribui para o
	desenvolvimento do raciocínio lógico-matemático e ao desenvolvimento de
	estruturas cognitivas fundamentais para o aprendizado da lógica de programação.
	
	\vspace{\onelineskip}
	
	\noindent
	\textbf{Palavras-chave}: Arduino. Lógica. Matemática.
	
\end{document}