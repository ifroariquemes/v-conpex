\documentclass[article,12pt,onesidea,4paper,english,brazil]{abntex2}

\usepackage{lmodern, indentfirst, nomencl, color, graphicx, microtype, lipsum}			
\usepackage[T1]{fontenc}		
\usepackage[utf8]{inputenc}		

\setlrmarginsandblock{2cm}{2cm}{*}
\setulmarginsandblock{2cm}{2cm}{*}
\checkandfixthelayout

\setlength{\parindent}{1.3cm}
\setlength{\parskip}{0.2cm}

\SingleSpacing

\begin{document}
	
	\selectlanguage{brazil}
	
	\frenchspacing 
	
	\begin{center}
		\LARGE EVASÃO ESCOLAR: UM ESTÁGIO NUM CENTRO DE REFERÊNCIA EM PESQUISAS EDUCACIONAIS\footnote{1Trabalho realizado dentro da área de Educação, com financiamento do IFRO/ NII/ARINT.}
		
		\normalsize
	Edivan Moura de Deus\footnote{Bolsista PIPEEX, edivanmdd@gmail.com, IFRO-Campus Ariquemes.} 
	Clotilde Tânia Rodrigues Luz\footnote{Orientadora no IFRO, tania.luz@ifro.edu.br, IFRO-Campus Ariquemes.} 
	Anabela mesquita Teixeira Sarmento\footnote{Orientadora no IPP, sarmento@iscap.ipp.pt, IPP-Instituto Politécnico do Porto-ISCAP-Instituto Superior de Contabilidade e Administração do Porto.} 
	\end{center}
	
	\noindent Entende-se por evasão escolar uma situação em que o aluno abandona a escola por alguma razão, e a matrícula não é refeita no próprio ano letivo ou no seguinte, ou seja, não há prosseguimento dos estudos até a sua conclusão. O principal objetivo da pesquisa realizada no âmbito da mobilidade internacional foi a obtenção de novos conhecimentos acerca da evasão escolar, como também de novas metodologias a serem aplicadas no ensino, principalmente como medidas paliativas no enfrentamento do problema. No Instituto Politécnico do Porto - IPP, a pesquisa ocorreu de forma didática, com o apoio da orientadora Anabela Mesquita Teixeira Sarmento, recomendando que escolas de educação básica do Porto fossem visitadas para conhecer a realidade dos estudantes locais. Durante a pesquisa, houve uma revisão de literatura em que se verificou que a evasão escolar tende a ser considerada como uma mazela educacional a ser remediada e para isso necessita ser destrinchada e observada em cada ângulo possível. Muitas vezes, a “cura” pode ser o mais simples dos remédios. Medidas simples podem ser aplicadas, desde oficinas de esportes, culinária, filmes, entre outras, contanto que os alunos sejam os principais protagonistas. Outra medida é a diversificação de metodologias criando espaços onde o aluno possa interagir e demonstrar seu posicionamento. Durante o intercâmbio, houve a oportunidade de conhecer aspectos relevantes da cultura portuguesa, como arquitetura, arte, museus e pessoas. Percebe-se que há problemas que afetam as diversas culturas do globo e a evasão escolar é uma destas, muitas vezes causada por problemas como vulnerabilidade socioeconômica, violência, necessidade de trabalhar para contribuir na renda familiar etc. Estes são fatores que estão presentes na região norte e que podem ser apontados como causas da evasão escolar. Atualmente ainda é necessária a pesquisa e a aplicação de políticas públicas para que os índices de evasão escolar continuem a baixar. Após o retorno ao Brasil, junto com a orientadora no IFRO, Clotilde Tânia Rodrigues Luz, debateu-se a possibilidade de se trabalhar com a investigação documental sobre evasão escolar em Ariquemes/Rondônia, tendo assim um fragmento dos dados de evasão escolar do município.
	
	\vspace{\onelineskip}
	
	\noindent
	\textbf{Palavras-chave}: Evasão escolar. Ensino. Pesquisa.
	
\end{document}
