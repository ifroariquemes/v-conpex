\documentclass[article,12pt,onesidea,4paper,english,brazil]{abntex2}

\usepackage{lmodern, indentfirst, nomencl, color, graphicx, microtype, lipsum}			
\usepackage[T1]{fontenc}		
\usepackage[utf8]{inputenc}		

\setlrmarginsandblock{2cm}{2cm}{*}
\setulmarginsandblock{2cm}{2cm}{*}
\checkandfixthelayout

\setlength{\parindent}{1.3cm}
\setlength{\parskip}{0.2cm}

\SingleSpacing

\begin{document}
	
	\selectlanguage{brazil}
	
	\frenchspacing 
	
	\begin{center}
		\LARGE AGRICULTURA FITOSSANITÁRIA DE PEQUENOS PRODUTORES\footnote{Trabalho realizado dentro da área de Finanças, com financiamento da ARINT/IFRO.}
		
		\normalsize
		BISPO, E.Q\footnote{Bolsista PIPEEX, eduardoqueoma@gmail.com, Campus Vilhena.} 
		ARENHARDT, V.\footnote{Orientadora IFRO, valeria.arenhardt@ifro.edu.br, Campus Vilhena.} 
		CONDE, C.I.C.\footnote{Orientador UNAL, cicardozoc@unal.edu.co, Sede Palmira.} 
	\end{center}
	
	\noindent A agricultura de pequenos produtores em Palmira, na Colômbia, é um share muito representativo para a região do Valle del Cauca, pois a região é rica no cultivo de hortifrúti, e gera emprego e renda para essa região, abastece o comércio local, feiras móveis e o mercado regional. O foco é nas hortaliças, parte principal dessa apresentação. Ao analisar o processo do controle de suas receitas e despesas e trabalhar com questionários abertos e levantamento de dados em órgãos públicos da região, constatou-se que esses pequenos produtores não tinham controle financeiro algum de seu trabalho. Após esse minucioso trabalho de levantamento de dados, sugerimos aos pequenos produtores de hortaliças a tentativa de controlar suas entradas e saídas, para sabermos a lucratividade e/ou lucro no exercício de quatro meses. Foram distribuídas aos produtores planilhas com toda a programação financeira diária, toda detalhada, para analisar a situação financeira dos produtores. Conforme o desenvolvimento do trabalho, solicitou-se à Secretaria da Fazenda e ao departamento de finanças na prefeitura de Palmira a informações sobre a existência de leis que fomentem incentivos fiscais, isenções, prioridade em participações licitatórias, como é feito aqui no Brasil. Os órgãos não nos forneceram as informações solicitadas, pois não há nenhuma lei que ajude o pequeno produtor rural com incentivos fiscais. Ao longo do trabalho de estágio, os produtores começaram a se organizar em suas finanças, adequando a sua realidade, aprendendo a executar um fluxo de caixa, sabendo o custo inicial e final de um pé de alface, por exemplo. Antes do estágio, não se tinha qualquer tipo de controle financeiro, sendo de grande êxito para uma projeção financeira, pois após esse trabalho de estágio, alguns pequenos produtores se organizaram e conseguiram aumentar a sua produção, investir em imobilizado e em mão-de-obra especializada, gerando emprego e renda para toda a região.
	
	\vspace{\onelineskip}
	
	\noindent
	\textbf{Palavras-chave}: Finanças. Pequenos Produtores. Hortaliças.
	
\end{document}
