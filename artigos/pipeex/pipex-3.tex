\documentclass[article,12pt,onesidea,4paper,english,brazil]{abntex2}

\usepackage{lmodern, indentfirst, nomencl, color, graphicx, microtype, lipsum}			
\usepackage[T1]{fontenc}		
\usepackage[utf8]{inputenc}		

\setlrmarginsandblock{2cm}{2cm}{*}
\setulmarginsandblock{2cm}{2cm}{*}
\checkandfixthelayout

\setlength{\parindent}{1.3cm}
\setlength{\parskip}{0.2cm}

\SingleSpacing

\begin{document}
	
	\selectlanguage{brazil}
	
	\frenchspacing 
	
	\begin{center}
		\LARGE CARACTERIZAÇÃO QUÍMICA DE AMINOÁCIDOS PRESENTES EM ALGAS
		MARINHAS – EXPERIÊNCIAS DE ESTÁGIO EM PESQUISA\footnote{Trabalho realizado dentro da área das Ciências Exatas e da Terra.}
		
		\normalsize
	Cristina Maria Fernandes Delerue Alvim de Matos\footnote{Orientadora, cmm@isep.ipp.pt, Instituto Politécnico do Porto.} 
	Cristina Maria Dias Soares\footnote{Co-orientadora,CMDSS@isep.ipp.pt, Instituto Politécnico do Porto.} 
	Gabriel Henrique Abrantes Holanda\footnote{Bolsista (estágio em pesquisa – mobilidade estudantil), gabrielhenrique2802@gmail.com, Campus
		Porto Velho Calama.} 
	Márcia Bay\footnote{Orientadora, márcia.bay@ifro.edu.br, Campus Porto Velho Calama.} 
	\end{center}
	
	\noindent Gabriel Henrique Abrantes Holanda, estudante do curso Técnico em Química, do
	Instituto Federal de Rondônia – IFRO, Campus Porto Velho Calama, foi selecionado
	através do edital número nº 63/NII/ARINT/2016 para realização, durante o segundo
	semestre letivo de 2016, de atividade de estágio em pesquisa no GRAQ (Grupo de
	Reação e Análises Químicas), centro de investigação do ISEP, Instituição Superior
	de Engenharia que compõe o quadro de instituições do Instituto Politécnico do Porto
	(IPP), na Cidade do Porto, Portugal. Dentro do GRAQ, o estudante Gabriel Holanda
	foi inserido, na condição de estagiário e aprendiz, no projeto IcanSea, onde
	trabalhou de forma mais específica, como auxiliar, no estudo dos aminoácidos
	presentes em algas marinhas da costa portuguesa, sob co-orientação da Dra.
	Cristina Soares, orientação da Dra. Cristina Matos e da professora Márcia Bay, do
	Instituto Federal de Rondônia. A importância do estudo realizado se deve ao fato de
	que, uma vez sendo os blocos construtores das proteínas, os aminoácidos, em
	especial aqueles não sintetizados pelo organismo humano, são de suma importância
	e por isso devem ser ingeridos através da alimentação. Dentro desse contexto,
	algas marinhas podem ser fontes alternativas em potencial de aminoácidos para a
	supressão de necessidade proteica. Para a realização do estudo, o método
	escolhido consistiu no uso da técnica de Cromatografia Líquida de Alta Eficiência em
	fase reversa (HPLC) com detecção por fluorescência e absorção na região do
	UV/Visível. A partir dos primeiros cromatogramas, resultados interessantes foram
	obtidos, tais como a detecção do aminoácido triptofano, pois mesmo se tratando de
	uma substancia instável em meio ácido, o seu pico foi identificado no cromatograma,
	o que normalmente representa um desafio na área de ciência dos alimentos. Vale
	ressaltar que, além da atividade principal em laboratório, outras atividades
	complementares fora realizadas, tais como a participação como ouvinte em aulas,
	em eventos científicos e culturais. Por fim, vale ressaltar que a experiência de
	intercâmbio como um todo enriquece a visão daquele que o vivencia o que propicia
	o compartilhamento de experiências e conhecimentos, objetivo principal do PIPEEX,
	e que o estudante em questão vem realizando desde o seu retorno ao Brasil.
	
	\vspace{\onelineskip}
	
	\noindent
	\textbf{Palavras-chave}: Algas marinhas. Aminoácidos. Intercâmbio acadêmico.
	 
	 \vspace{\onelineskip}
	 
	 \noindent
	 \textbf{Fonte de Financiamento}: IFRO/ARINT.
\end{document}
