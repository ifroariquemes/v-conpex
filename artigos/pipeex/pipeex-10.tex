\documentclass[article,12pt,onesidea,4paper,english,brazil]{abntex2}

\usepackage{lmodern, indentfirst, nomencl, color, graphicx, microtype, lipsum}			
\usepackage[T1]{fontenc}		
\usepackage[utf8]{inputenc}		

\setlrmarginsandblock{2cm}{2cm}{*}
\setulmarginsandblock{2cm}{2cm}{*}
\checkandfixthelayout

\setlength{\parindent}{1.3cm}
\setlength{\parskip}{0.2cm}

\SingleSpacing

\begin{document}
	
	\selectlanguage{brazil}
	
	\frenchspacing 
	
	\begin{center}
		\LARGE OBTENÇÃO DE UM CULTIVAR MELHORADO DE CILANTRO (\textit{CORIANDRUM SATIVUM}) NA REGIÃO DO VALLE DEL CAUCA – PALMIRA COLÔMBIA\footnote{Trabalho realizado através do Programa de Internacionalização de Pesquisa, Ensino e Extensão – PIPEEX, 2016, \textit{Universidad Nacional de Colombia} – UNAL, em Palmira, Colômbia.}
		
		\normalsize
		Paulo Ricardo Pastore Silva\footnote{Bolsista PIPEEX, prpastore97@gmail.com, \textit{Campus} Colorado do Oeste.} 
		Marcos Aurelio Anequine de Maceno\footnote{Professor Orientador, marcos.anequine@ifro.edu.br, \textit{Campus} Colorado do Oeste.} 
		Mario Augusto Garcia Davila\footnote{Professor Co-orientador, magarciada@unal.edu.co, UNAL – Sede Palmira.}
	\end{center}
	
	\noindent O Melhoramento genético é o processo de seleção intencional de um material
	genético de qualquer ser vivo, para obtenção de indivíduos com características que
	se tem interesse. Com o intuito de uma melhoria na produtividade, um alto
	desempenho contra doenças e um rápido período para colheita de Cilantro
	(\textit{Coriandrum sativum}) na Colômbia, foi realizado um estudo sobre o melhoramento
	genético de tal cultura. Com o intuito de melhorar a produtividade desta planta, um
	experimento foi realizado para mudar geneticamente as variedades resistentes e
	pouco produtivas. Este projeto foi desenvolvido pelo Professor Dr. Armando Zapata,
	responsável pelo Centro de Pesquisa da UNAL – CEUNP. As linhagens estudadas
	foram obtidas da polinização da variedade Precoso pela variedade Bogotano,
	gerando assim 10 “variedades” das plantas polinizadas, chamadas de PRECOSO
	S2, PRECOSO 25 S2, PRECOSO 31 S2, PRECOSO 29 S2, PRECOSO 30 S2,	PRECOSO 19 S2, PRECOSO 23 S2, PRECOSO 34 S2, PRECOSO 15 S2, PRECOSO 4 S2 e PRECOSO 28 S2. O trabalho tinha como objetivo avaliar as	variedades melhoradas e foi selecionado para o Programa de Internacionalização da Pesquisa, Ensino e Extensão do IFRO – PIPEEX através do edital no 63/2016. 
	As variedades foram plantadas em sementeiras e, logo após seu crescimento, foram
	transplantadas para pequenos vasos de plástico. Posteriormente, foi feito o plantio
	em canteiros com comprimento de 5,5m, cada variedade separada em seu próprio
	canteiro; o espaçamento utilizado entre plantas foi de 10cm. Com o objetivo de
	analisar o crescimento vegetativo, que inclui a altura das plantas (medidas que vão
	do nível do solo até a última inflorescência), a quantidade de hastes por planta, e o	índice de susceptibilidade a doenças de solo, o trabalho foi finalizado e resultou em que as melhores variedades de cilantro melhorado entre as variedades BOGOTANO e PRECOSO foram as PRECOSO 29 S2, PRECOSO 25 S2 e PRECOSO 19 S2.
	Como o clima de Palmira é semelhante ao da região Norte brasileira, as variedades
	são ótimas para produtores que desejam uma maior produtividade e um alto
	desempenho contra fungos de solo.
	
	\vspace{\onelineskip}
	
	\noindent
	\textbf{Palavras-chave}: Produtividade. Variedades. Crescimento Vegetativo. Doenças de Solo.
	
\end{document}