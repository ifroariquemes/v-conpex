\documentclass[article,12pt,onesidea,4paper,english,brazil]{abntex2}

\usepackage{lmodern, indentfirst, nomencl, color, graphicx, microtype, lipsum}			
\usepackage[T1]{fontenc}		
\usepackage[utf8]{inputenc}		

\setlrmarginsandblock{2cm}{2cm}{*}
\setulmarginsandblock{2cm}{2cm}{*}
\checkandfixthelayout

\setlength{\parindent}{1.3cm}
\setlength{\parskip}{0.2cm}

\SingleSpacing

\begin{document}
	
	\selectlanguage{brazil}
	
	\frenchspacing 
	
	\begin{center}
		\LARGE IDENTIFICAÇÃO E SELEÇÃO DE \\BACTÉRIAS ENDÓFITAS ANTAGONISTAS DE \textit{PSEUDOMONAS SAVASTANOI PV. SAVASTANOI}\footnote{Trabalho realizado dentro da área de Conhecimento CNPq: Agrárias – Agrobiotecnologia com financiamento IFRO/ARINT.}
		
		\normalsize
		Jessica Pagung\footnote{Bolsista PIPEEX, jessicapagung18@gmail.com, \textit{Campus} Colorado do Oeste.} 
		Larah Drielly Santos Herrera\footnote{Bolsista PIPEEX, herrera.larah@gmail.com, \textit{Campus} Colorado do Oeste.} \\
		Rafael Henrique Pereira dos Reis\footnote{Orientador, rafael.reis@ifro.edu.br, \textit{Campus} Colorado do Oeste.} 
		Paula Cristina Santos Baptista\footnote{Co-orientadora, pbaptista@ipb.pt, Instituto Politécnico de Bragança (IPB).} 
	\end{center}
	
	\noindent A tuberculose, causada pela bactéria \textit{Pseudomonas savastanoi pv. Savastanoi}	constitui uma das principais doenças da oliveira. Esta bactéria produz tumores,	sobretudo nos ramos e troncos da oliveira, causando uma redução do seu vigor e	consequentemente da produtividade. 
	Visando explorar as potencialidades de bactérias endofíticas na luta biológica contra esta doença, o Instituto Politécnico de
	Bragança, junto à equipe de pesquisa do laboratório de agrobiotecnologia, realizam
	diversos trabalhos acerca desta problemática. Assim, este trabalho é oriundo do
	período de mobilidade internacional, acompanhando uma pesquisa de doutorado, na
	qual foram isoladas bactérias de ramos e folhas de duas cultivares com diferentes
	susceptibilidades à tuberculose. Estas foram submetidas à identificação molecular
	pela sequenciação da sub unidade 16S do rRNA e posterior seleção das bactérias
	antagonistas a \textit{savastonoi pv.} através do desenvolvimento de alguns protocolos para	testes de inibição. A ação antagonista dos isolados obtidos foi avaliada em placas de
	petri pelo método da cultura em meio Luria- Bertani (LB), em Ágar Mueller-Hinton
	(MHA) (pH 7,3), em meio com extrato de folhas, caules e nódulos, também em meio
	Kings Medium B (KB) (pH 7; NaOH), que permite a visualização da florescência da
	Pseudomonas \textit{savastanoi pv. savastanoi (Pss)}, assim como em meio líquido de LB
	em micro tubos e falcons, realizando diferentes testes de concentrações bacterianas
	com variados testes de aplicação, na tentativa de identificação do melhor método de
	visualização de inibição do crescimento da \textit{Pseudomona}, uma vez que não há
	publicações descrevendo a melhor forma para realização deste. Assim, os estudos
	realizados com as bactérias endofíticas e epifíticas são de suma importância para
	validar ou não o uso destes organismos isolados no combate a esse patógeno. E
	ainda que no Brasil o cultivo de Oliveiras (\textit{Olea europaea}) seja escasso, as técnicas
	obtidas proporcionaram o desenvolvimento de pesquisas similares no \textit{Campus}
	Colorado do Oeste, demonstrando a relevância do programa para o IFRO e para a
	vida acadêmica dos discentes contemplados.
	
	\vspace{\onelineskip}
	
	\noindent
	\textbf{Palavras-chave}: Mobilidade Internacional. Isolamento Bacteriano. Teste de Inibição.
	
\end{document}