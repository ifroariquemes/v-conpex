\documentclass[article,12pt,onesidea,4paper,english,brazil]{abntex2}

\usepackage{lmodern, indentfirst, nomencl, color, graphicx, microtype, lipsum}			
\usepackage[T1]{fontenc}		
\usepackage[utf8]{inputenc}		

\setlrmarginsandblock{2cm}{2cm}{*}
\setulmarginsandblock{2cm}{2cm}{*}
\checkandfixthelayout

\setlength{\parindent}{1.3cm}
\setlength{\parskip}{0.2cm}

\SingleSpacing

\begin{document}
	
	\selectlanguage{brazil}
	
	\frenchspacing 
	
	\begin{center}
		\LARGE O PAPEL DAS MÍDIAS SOCIAIS NA\\FORMAÇÃO PROFISSIONAL\footnote{Trabalho realizado dentro da área de Ciências Humanas com financiamento do PIPEEX.}
		
		\normalsize
		Laura Rafaela da Silva Viana\footnote{Bolsista do PIPEEX, laura.rafaela3@gmail.com, \textit{Campus} Porto Velho Calama.} 
		Antônio dos Santos Júnior\footnote{Orientador, antonio.junior@ifro.edu.br, \textit{Campus} Porto Velho Calama.} 
		Sheylla Chediak\footnote{Co-orientadora, sheylla.chediak@ifro.edu.br, \textit{Campus} Porto Velho Calama.} 
		Anabela Mesquita\footnote{Co-orientadora, sarmento@iscap.ipp.pt, Instituto Superior de Contabilidade e Administração do Porto - ISCAP.} 
	\end{center}
	
	\noindent O início do século XXI foi marcado pela inovação das tecnologias de segunda
	geração da \textit{internet} e o consequente surgimento de novos modelos de comunicação
	eletrônica, mais interativos e instantâneos. A tecnologia pode influenciar a maneira
	como as pessoas convivem, aprendem e se comportam. É neste contexto que se
	inserem as mídias sociais, um dos principais meios de comunicação atual, cujo uso
	gera novas formas de relações pessoais e profissionais. Quanto a este último, nos
	preocupou investigar a maneira como o comportamento \textit{online} dos estudantes em
	formação pode interferir na sua vida profissional no que tange a inserção no
	mercado de trabalho e de maneira mais geral ao compartilhamento de
	conhecimentos acadêmicos. A pesquisa foi realizada no Instituto Superior de
	Contabilidade e Administração do Porto – ISCAP e caracterizou-se por levantamento
	de referencial bibliográfico, coleta e análise de dados. Dez estudantes do instituto
	foram entrevistados e os resultados mostraram que apenas 50\% deles utilizam as
	redes sociais virtuais para promoção no mercado de trabalho. Neste sentido,
	discutem-se pontos como a necessidade de orientação escolar e se sugere uma
	importante ligação entre a visibilidade profissional, aprendizagem compartilhada e
	engajamento social. Durante o período de Mobilidade Internacional realizada no
	Programa PIPEEX, os resultados foram apresentados no Congresso Internacional
	de Secretariado e Assessoria (CISA 2016). Concomitante à realização da pesquisa,
	também houve a participação em aulas de mestrado, conferências, palestras e em
	grupo artístico. Quanto ao plano de regresso do intercâmbio, as atividades já
	realizadas se caracterizam principalmente por apresentações aos estudantes que
	são realizadas de forma mais intensa em um projeto de extensão e publicação de
	um artigo científico. A experiência proporcionada pelo PIPEEX contribuiu
	significativamente para o crescimento acadêmico da estudante e para a criação de
	um olhar mais diferente e amplo sobre as pessoas e o mundo.
	
	\vspace{\onelineskip}
	
	\noindent
	\textbf{Palavras-chave}: Mídias Sociais. Comportamento \textit{Online}. Vida profissional. PIPEEX.
	
\end{document}