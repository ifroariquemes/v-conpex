\documentclass[article,12pt,onesidea,4paper,english,brazil]{abntex2}

\usepackage{lmodern, indentfirst, nomencl, color, graphicx, microtype, lipsum}			
\usepackage[T1]{fontenc}		
\usepackage[utf8]{inputenc}		

\setlrmarginsandblock{2cm}{2cm}{*}
\setulmarginsandblock{2cm}{2cm}{*}
\checkandfixthelayout

\setlength{\parindent}{1.3cm}
\setlength{\parskip}{0.2cm}

\SingleSpacing

\begin{document}
	
	\selectlanguage{brazil}
	
	\frenchspacing 
	
	\begin{center}
		\LARGE GENÉTICA E BIOTECNOLOGIA COM AMOSTRAS DE \textit{\MakeUppercase{Capsicum chinense}}\footnote{Trabalho realizado na área de Genética, com financiamento do Pipeex / IFRO.}
		
		\normalsize
		Gean Carlos de Souza Albuquerque\footnote{Gean Carlos de Souza Albuquerque, geancarlosalbuquerque@gmail.com, \textit{Campus} Colorado do	Oeste.} 
		Ruben Dario Rojas Pantoja\footnote{Ruben Dario Rojas Pantoja, rdrojas@unal.edu.co, \textit{Universidad Nacional de Colombia}.} \\
		José de Anchieta Almeida da Silva\footnote{José de Anchieta Almeida da Silva, jose.silva@ifro.edu.br, \textit{Campus} Colorado do Oeste.} 
		Creuci Maria Caetano\footnote{Creuci Maria Caetano, cmcaetano@unal.edu.co, \textit{Universidad Nacional de Colombia}.} 
	\end{center}
	
	\noindent As plantas com gênero \textit{Capsicum} são representadas por 6 espécies como sendo
	cultivares com alto valor econômico. Entre eles está \textit{Capsicum chinense}, que foi
	utilizado na pesquisa. As plantas dessa espécie são de grande valor econômico no
	Brasil, havendo assim a necessidade de estudos para promover seu melhoramento
	genético. A utilização de marcadores moleculares permite acelerar a exploração da
	diversidade e selecionar genótipos de \textit{Capsicum} que sejam mais importantes
	economicamente. O estágio teve como objetivo principal conhecer algumas técnicas
	moleculares básicas na área de melhoramento genético. O estágio ocorreu no
	laboratório de biotecnologia vegetal da \textit{Universidad Nacional de Colombia}, sede
	Palmira (UNAL), com a colaboração do Ms. Ruben Dario Rojas Pantoja, através da
	utilização das técnicas em extração de DNA em nitrogênio líquido e outras técnicas
	de análise, processadas pelo protocolo modificado estabelecido por Doyle \& Doyle,
	1987. Com o material coletado, fez-se a medição da concentração de DNA diluído e
	levado para as demais análises com marcadores SSR. Para a utilização do DNA
	diluído, foram escolhidos marcadores pré-estabelecidos pela literatura utilizada,
	sendo os marcadores diluídos antes de serem mesclados ao DNA e inseridos pelo
	coquetel estabelecido também pela literatura utilizada. A técnica de extração foi feita
	a partir de um buffer indicado na metodologia Doyle \& Doyle. Em sequência, a
	pesquisa levou ao aprendizado de novas técnicas para preparar as amostras de
	\textit{Capsicum} e para submeter os primers utilizados sendo levados ao termociclador. Ao
	saírem do termociclador, as amostras foram levadas para câmara de eletroforese em
	gel de agarose para verificar a qualidade das amostras retiradas e saber se foram
	amplificadas. Por fim, foram enviadas para a universidade de Cornell, nos Estados
	Unidos, para análise e posterior avaliação dos resultados. As amostras
	apresentaram boa qualidade e foram enviadas para análise no exterior. O período
	de entrega dos resultados excederia ao da mobilidade. Portanto, a leitura dos
	resultados infelizmente se deu apenas pelo Colaborador que estava presente na
	UNAL.
	
	\vspace{\onelineskip}
	
	\noindent
	\textbf{Palavras-chave}: \textit{Capsicum}. Marcadores. Biotecnologia.
	
\end{document}