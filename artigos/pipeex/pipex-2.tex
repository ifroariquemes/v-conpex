\documentclass[article,12pt,onesidea,4paper,english,brazil]{abntex2}

\usepackage{lmodern, indentfirst, nomencl, color, graphicx, microtype, lipsum}			
\usepackage[T1]{fontenc}		
\usepackage[utf8]{inputenc}		

\setlrmarginsandblock{2cm}{2cm}{*}
\setulmarginsandblock{2cm}{2cm}{*}
\checkandfixthelayout

\setlength{\parindent}{1.3cm}
\setlength{\parskip}{0.2cm}

\SingleSpacing

\begin{document}
	
	\selectlanguage{brazil}
	
	\frenchspacing 
	
	\begin{center}
		\LARGE AVALIAÇÃO DE LINHAS AVANÇADAS DE TOMATE (SOLANUM
		LYCOPERSICUM) CHONTO E CHERRY\footnote{Trabalho realizado dentro da Agronomia com financiamento do ARINT/IFRO.}
		
		\normalsize
		G. T. K. Viotto\footnote{Bolsista PIPEEX, thiemyviotto@gmail.com, Campus Colorado do Oeste.} 
		M. A. A. Macêdo\footnote{Orientador IFRO, marcos.anequine@ifro.edu.br, Campus Colorado do Oeste.} 
		M. A. Garcia\footnote{Orientador UNAL, magarciada@unal.edu.co, Sede Palmira.} 
	\end{center}
	
	\noindent O tomate Solanum lycopersicum é produzido e consumido no mundo todo, ganhando destaque por
	ser a segunda hortaliça mais produzida no mundo. Sua importância é de cunho econômico e social
	pelo volume de produção e os inúmeros empregos gerados no decorrer da cadeia produtiva na
	Colômbia. Este alto volume de produção tem uma problemática desencadeada pela busca de
	cultivares que se adequam ao mercado consumidor colombiano. O mercado consumidor de tomate
	estabelece frutos de alta qualidade, com tamanho e diâmetros proporcionais, além de coloração
	avermelhada. Este presente estudo trouxe como objetivo encontrar variedades de tomates tipo
	chonto e cherry que se adequariam ao mercado consumidor. Os testes foram realizados no Centro
	Experimental da Universidad Nacional de Colombia - UNAL– Sede Palmira (CEUNP). Foram testadas
	22 variedades, sendo 13 do tipo chonto e 9 do tipo cherry. As variedades de tomate tipo chonto
	utilizadas foram: TIA19M, MA11, MAT27T1, T2/MAIS, CHONTO TMNF12, CHONTO TMLF12, CHONTO
	TILC, MA9TI, MA9, MA7TI, RACIMO, MA23, MA27. E do tipo cherry foram: CH5, CH21AC426, CH7,
	HORTIFRESCO, CH4/AC, CH2, TRAVIESA, CH6/A44688, CH11A341. Os parâmetros de qualidade
	avaliados foram a altura de floração das plantas durante o estágio vegetativo, e, no pós-colheita, a
	contagem de frutos por planta, a pesagem aferindo o diâmetro e o comprimento dos frutos, além da
	contagem dos lóculos e a observação da coloração. Algumas variedades apresentaram bons
	resultados em números de frutos por planta, mas não apresentaram resultados satisfatórios em
	relação a peso e diâmetro dos frutos, já outras apresentaram todos os parâmetros desejados. Ao
	final, executou-se uma comparação com os frutos que já estão no mercado comercial e os
	produzidos na CEUNP, para poder verificar se as variedades testadas se encaixam ao mercado
	consumidor. As variedades do tipo chonto que se adequam ao mercado consumidor são: T2/MAIS e
	CHONTO TILC. Os materiais cherry que podem ser inseridos no mercado consumidor são: TRAVIESA,
	CH11A341, CH2, CH4/AC, HORTIFRESCO, CH21AC426, CH6/A44688, CH5. Esta pesquisa foi realizada
	por meio de mobilidade estudantil pelo Programa PIPEEX, do Instituto Federal de Educação, Ciência e
	Tecnologia de Rondônia.
	
	\vspace{\onelineskip}
	
	\noindent
	\textbf{Palavras-chave}: Tomate Solanum lycopersicum. Comercialização. Qualidade.
	
\end{document}
