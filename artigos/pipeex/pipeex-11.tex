\documentclass[article,12pt,onesidea,4paper,english,brazil]{abntex2}

\usepackage{lmodern, indentfirst, nomencl, color, graphicx, microtype, lipsum}			
\usepackage[T1]{fontenc}		
\usepackage[utf8]{inputenc}		

\setlrmarginsandblock{2cm}{2cm}{*}
\setulmarginsandblock{2cm}{2cm}{*}
\checkandfixthelayout

\setlength{\parindent}{1.3cm}
\setlength{\parskip}{0.2cm}

\SingleSpacing

\begin{document}
	
	\selectlanguage{brazil}
	
	\frenchspacing 
	
	\begin{center}
		\LARGE SELEÇÃO \textit{IN VITRO} E \textit{IN VIVO} DE FUNGOS ENDOFÍTICOS ANTAGONISTAS DE \textit{FUSICLADIUM OLEAGINEUM}\footnote{Trabalho realizado dentro da área de Conhecimento CNPq: Agrárias - Agrobiotecnologia com financiamento ARINT/IFRO.}
		
		\normalsize
		Larah Drielly Santos Herrera\footnote{Bolsista Pipeex – Programa de Internacionalização da Pesquisa, Ensino e Extensão, herrera.larah@gmail.com , \textit{Campus} Colorado do Oeste.} 
		Jessica Pagung\footnote{Colaboradora, jessicapagung18@gmail.com, \textit{Campus} Colorado do Oeste.} 
		Rafael Henrique Pereira dos Reis\footnote{Orientador, rafael.reis@ifro.edu.br, \textit{Campus} Colorado do Oeste.} 
		Paula Cristina Baptista\footnote{Co-orientadora, pbaptista@ipb.edu.br, Instituto Politécnico de Bragança – IPB.} 
	\end{center}
	
	\noindent A doença olho-de-pavão, causada pelo fungo \textit{Fusicladium oleagineum}, constitui uma das doenças mais comuns da oliveira. Levando em consideração os danos econômicos causados pelo ataque desta doença, faz-se necessário uma constante busca por meios de controle para a mesma. O presente trabalho é fruto do	acompanhamento de uma pesquisa de doutorado, realizada no laboratório de agrobiotecnologia do Instituto Politécnico de Bragança – PT, durante o período de mobilidade, de setembro a dezembro de 2016. O trabalho objetivou realizar a
	seleção de fungos endofíticos antagonistas do agente causal da doença olho-de-
	pavão, utilizando bioensaios e ensaios em estufa, além de compreender e praticar
	as técnicas utilizadas nos bioensaios \textit{in vitro} e possibilitar a aquisição e o
	compartilhamento de conhecimento nas áreas de estudo/pesquisa. As atividades
	realizadas durante o período de mobilidade foram baseadas na seleção de fungos
	endofíticos, aplicando-se duas metodologias distintas: bioensaios (\textit{in vitro}) e ensaios	na estufa (\textit{in vivo}). Nos bioensaios, folhas da cv. Madural foram submetidas a 3 tratamentos: 1) inoculadas com o endofítico; 2) inoculadas com F. \textit{oleagineum}; 3)
	inoculadas com endofítico + \textit{Fusicladium oleagineum}. Após inoculação do presente
	fungo investigado, foram avaliadas ao longo do tempo, a incidência e a severidade
	do mesmo, de forma a determinar a capacidade do endófito de reduzir a progressão
	da doença. Nos ensaios de estufa, plântulas de oliveira da cv. Madural foram
	submetidas aos mesmos tratamentos anteriormente referidos. Ainda que no Brasil o
	cultivo de Oliveiras (\textit{Olea europaea}) seja escasso, as técnicas em estudos para
	promover o controle da presente doença, através de fungos endofíticos podem ser
	aplicadas no controle de doenças que apresentam a capacidade de se desenvolver
	em culturas de importância econômica para o país. Neste aspecto, levando em
	consideração a importância deste estudo, as atividades desenvolvidas durante o
	período de mobilidade possibilitaram o desenvolvimento de pesquisas similares, no
	\textit{Campus} Colorado do Oeste.
	
	\vspace{\onelineskip}
	
	\noindent
	\textbf{Palavras-chave}: Controle. Doença Olho-de-Pavão. Olival. Genética.
	
\end{document}