\documentclass[article,12pt,onesidea,4paper,english,brazil]{abntex2}

\usepackage{lmodern, indentfirst, nomencl, color, graphicx, microtype, lipsum}			
\usepackage[T1]{fontenc}		
\usepackage[utf8]{inputenc}		

\setlrmarginsandblock{2cm}{2cm}{*}
\setulmarginsandblock{2cm}{2cm}{*}
\checkandfixthelayout

\setlength{\parindent}{1.3cm}
\setlength{\parskip}{0.2cm}

\SingleSpacing

\begin{document}
	
	\selectlanguage{brazil}
	
	\frenchspacing 
	
	\begin{center}
		\LARGE CLIMA ORGANIZACIONAL:\\RELATO DE EXPERIÊNCIA DE PESQUISA NA CIDADE DO PORTO, EM PORTUGAL\footnote{Trabalho realizado dentro da área de Conhecimento CNPq: Administração Pública e de Empresas, Ciências Contábeis e Turismo.}
		
		\normalsize
	Kelly Cristiane Catafesta\footnote{Bolsista PIPEEX; Aluna do curso de graduação de Tecnologia em Gestão Pública, kelly.catafesta@gmail.com, Campus Porto Velho Zona Norte.} 
	Higor Cordeiro de Souza\footnote{Professor (a) orientador (a), higor.souza@ifro.edu.br Campus Porto Velho Zona Norte.} 
	Viviana Meirinhos\footnote{Professor (a) orientador (a), vivianameirinhos@iscap.ipp.pt Instituto Superior de Administração e Contabilidade do Porto.} 
	\end{center}
	
	\noindent O clima organizacional é um fenômeno perceptual duradouro, construído com base na experiência, multidimensional e compartilhado pelos membros de uma unidade de organização, cuja função principal é orientar e regular os comportamentos individuais de acordo com os padrões determinados por ela. O objetivo proposto para a pesquisa inicialmente foi identificar a influência da cultura e clima organizacional na motivação para o trabalho. No decorrer da pesquisa teórica, observou-se que para aplicar uma pesquisa de clima organizacional seria necessário desenvolver uma ferramenta que atendesse a contento a ideia de multidimensionalidade do construto, visto que a maioria das pesquisas abordam dimensões como, por exemplo, satisfação e bem-estar do trabalhador. Na construção do questionário, cada dimensão foi conceituada e seus objetivos definidos. Realizamos pré-testes com o questionário e foram verificados pontos de difícil compreensão nas questões, como dupla interpretação, repetição de propostas, dentre outros. Também observamos que o tempo necessário para leitura e resposta variou entre 6 a 10 minutos. Com algumas correções e os ajustes feitos, passamos para a fase de testes com os funcionários, na qual participaram cinco pessoas de diferentes setores da organização. Para aplicação definitiva foi escolhida a abordagem pessoal na entrega dos questionários, visando oportunizar uma explanação sobre a pesquisa e seus objetivos, bem como estimular a participação. Os questionários foram entregues em envelopes fechados, posto que ele representa avaliação do ambiente de trabalho, da instituição e dos funcionários. Dos 60 funcionários da instituição, 34 efetuaram respostas ao questionário e os resultados demandaram uma análise embasada com maior robustez na literatura. Do período da pesquisa cumprido na cidade do Porto, em Portugal, aprofundamos o estudo do referencial teórico, elaboramos e aplicamos o questionário. No retorno ao Brasil, a continuidade dos estudos ocorreu voltada à interpretação dos dados e sua caracterização junto à literatura. A pesquisa realizada destaca-se pelo questionário produzido abrangendo distintas dimensões que caracterizam o construto e a ferramenta pode ser utilizada em qualquer instituição que queira avaliar clima organizacional. No IFRO, o projeto de pesquisa de clima organizacional em todas suas unidades foi apresentado à administração e aguarda autorização para início da aplicação.
	
	\vspace{\onelineskip}
	
	\noindent
	\textbf{Palavras-chave}: Clima Organizacional. Organização. Trabalho.
	
\end{document}
