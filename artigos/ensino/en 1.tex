\documentclass[article,12pt,onesidea,4paper,english,brazil]{abntex2}

\usepackage{lmodern, indentfirst, nomencl, color, graphicx, microtype, lipsum}			
\usepackage[T1]{fontenc}		
\usepackage[utf8]{inputenc}		

\setlrmarginsandblock{2cm}{2cm}{*}
\setulmarginsandblock{2cm}{2cm}{*}
\checkandfixthelayout

\setlength{\parindent}{1.3cm}
\setlength{\parskip}{0.2cm}

\SingleSpacing

\begin{document}
	
	\selectlanguage{brazil}
	
	\frenchspacing 
	
	\begin{center}
		\LARGE A ESCRITA LITERÁRIA COM ALUNOS DO \\CURSO TÉCNICO INTEGRADO:\\
		DESAFIOS E POSSIBILIDADES\footnote{Trabalho realizado dentro da área de Conhecimento do CNPq: Letras.}
		
		\normalsize
		Denise Mota\footnote{Orientanda: email: denisemota72@gmail.com. Câmpus Cacoal.} 
	Shelly Braum\footnote{Colaboradora: email: shelly.braum@ifro.edu.br, Câmpus Cacoal.} 
	Andreia dos Santos Oliveira\footnote{Orientadora: email: andreia.oliveira@ifro.edu.br, Câmpus Cacoal.} 
		\end{center}
	
	\noindent Propor situações que contribuam com o desenvolvimento de habilidades de escrita é
	um desafio no primeiro ano do ensino médio, pois recebemos muitos alunos que
	apresentam problemas em utilizar estratégias de escrita básica, tais como a
	constituição de parágrafos, elementos de coesão e coerência textual. O desafio tornase
	ainda maior, quando se pretende trabalhar com a escrita de textos literários, haja
	vista que exige maiores habilidades dos alunos por se tratar da composição de um
	gênero especial: o literário, com todas as suas especificidades.Com o intuito de
	contribuir com a superação dessas complexidades e a construção da competência
	de escrita, elaborou-se o projeto de ensino em questão que objetiva sanar as
	dificuldades relacionadas à escrita por meio da produção de livros literários. O
	trabalho está sendo desenvolvido no ano de 2017 e consta das seguintes etapas:
	leitura de obras literárias e identificação das especificidades do texto literário e
	processo de composição do livro, planejamento da narrativa a ser produzida,
	levando em consideração elementos, tais como: público alvo, tema, foco narrativo,
	personagens e enredo, produção dos gêneros biografias referentes aos autores e
	resumos. No momento em que este trabalho foi proposto muitos alunos ficaram
	entusiasmados, porém assustados com a ideia de escrever um livro, entretanto, à
	medida que as orientações eram feitas, eles superaram as angústias e começaram a
	perceber o prazer de compor seus primeiros livros literários. O projeto ainda está em
	fase de execução, porém já temos os primeiros resultados, tendo em vista que os
	textos já foram escritos, revisados e reescritos, as biografias e resumos produzidos.
	Ao final destas etapas, encantamo-nos com a qualidade das produções. Livros com
	histórias surpreendentes, instigantes e convidativas foram produzidos e muitos
	alunos sentiram-se tão encantados com essa atividade que pretendem dedicar-se
	ainda mais à atividade de escrita literária. No momento, os alunos estão na última
	fase da proposta que se trata da ilustração das obras e organização dos textos para
	a finalização. O projeto está sendo desenvolvido com 80 alunos matriculados nos
	primeiros anos dos cursos técnicos integrados de Agroecologia e Agropecuária.
	
	\vspace{\onelineskip}
	
	\noindent
	\textbf{Palavras-chave}: Escrita literária. Ensino técnico integrado. Desafios e possibilidades.
	
\end{document}
