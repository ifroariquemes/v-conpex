\documentclass[article,12pt,onesidea,4paper,english,brazil]{abntex2}

\usepackage{lmodern, indentfirst, nomencl, color, graphicx, microtype, lipsum}			
\usepackage[T1]{fontenc}		
\usepackage[utf8]{inputenc}		

\setlrmarginsandblock{2cm}{2cm}{*}
\setulmarginsandblock{2cm}{2cm}{*}
\checkandfixthelayout

\setlength{\parindent}{1.3cm}
\setlength{\parskip}{0.2cm}

\SingleSpacing

\begin{document}
	
	\selectlanguage{brazil}
	
	\frenchspacing 
	
	\begin{center}
		\LARGE MATEMÁTICA PARA ELETROTÉCNICOS
		
		\normalsize
		Suelene da Silva Batista\footnote{Suelene da Silva Batista, suelene.batista@ifro.edu.br, Campus Porto Velho Calama.} 
	Ricardo Bussons da Silva\footnote{Ricardo Bussons da Silva, ricardo.bussons@ifro.edu.br, Campus Porto Velho Calama.} 
	Dr. Marinaldo Felipe da Silva\footnote{Ricardo Bussons da Silva, ricardo.bussons@ifro.edu.br, Campus Porto Velho Calama.} 
	\end{center}
	
	\noindent O Instituto Federal de Educação, Ciência e Tecnologia de Rondônia (IFRO) -
	Campus Porto Velho Calama, oferta o Curso Técnico em Eletrotécnica Integrado ao
	Ensino Médio, modalidade que tem por objetivo promover a habilitação profissional
	técnica de nível médio, e a conclusão da última etapa da educação básica por meio
	da integração curricular das disciplinas comum do Ensino Médio as disciplinas
	específicas do curso. O Curso Técnico em Eletrotécnica é vinculado ao Eixo
	Industrial do Catálogo Nacional dos Cursos Técnicos, esta formação exige a
	construção de conhecimentos sólidos da matemática, construídas nas etapas do
	ensino fundamental. Entretanto, com base na avaliação de desempenho acadêmico
	dos alunos do curso, constatamos o baixo rendimento das turmas nas disciplinas
	que envolviam conhecimentos prévios da matemática, ocasionados pela falta de prérequisitos
	referentes à construção dos conceitos básicos da matemática, em
	consequência disto, o curso apresentava alto índice de retenção nas disciplinas, o
	que comprometia significativamente o êxito dos estudantes em seu processo de
	escolarização. Desta forma, como estratégia para melhorar a aprendizagem dos
	alunos, desenvolvemos no ano de 2016, o projeto de ensino “Matemática para
	Eletrotécnico” que visava trabalhar conteúdos básicos da disciplina, direcionados ao
	curso. O projeto foi desenvolvido com a participação dos professores da disciplina
	de matemática e professores das áreas técnicas, verificamos que estas ações
	contribuíram para o resgate dos conhecimentos que não foi aprendido, e sua relação
	com a área de formação na obtenção de conhecimentos os futuros profissionais.
	
	\vspace{\onelineskip}
	
	\noindent
	\textbf{Palavras-chave}: Matemática. Eletrotécnica. Aprendizagem.
	
\end{document}
