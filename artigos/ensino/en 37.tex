\documentclass[article,12pt,onesidea,4paper,english,brazil]{abntex2}

\usepackage{lmodern, indentfirst, nomencl, color, graphicx, microtype, lipsum}			
\usepackage[T1]{fontenc}		
\usepackage[utf8]{inputenc}		

\setlrmarginsandblock{2cm}{2cm}{*}
\setulmarginsandblock{2cm}{2cm}{*}
\checkandfixthelayout

\setlength{\parindent}{1.3cm}
\setlength{\parskip}{0.2cm}

\SingleSpacing

\begin{document}
	
	\selectlanguage{brazil}
	
	\frenchspacing 
	
	\begin{center}
		\LARGE REDES DE JOGOS\footnote{Trabalho realizado dentro da área de Educação.}
		
		\normalsize
	Juliane Martinez Galiano\footnote{Coordenadora, juliane.martinez@ifro.edu.br, Campus Ariquemes.} 
	Fernando Bispo Cavalcante\footnote{Colaborador, fbispocavalcante@gmail.com, Campus Ariquemes.} 
	Juber Miranda Cunha\footnote{Colaborador, miranda.juber@gmail.com, Campus Ariquemes.} 
	\end{center}
	
	\noindent Os jogos estão presentes em nossas vidas, seja para competição, diversão, treino
	ou aprendizado. O jogo quando aplicado a educação é um instrumento motivador do
	processo de ensino/aprendizagem, pois permite desenvolver uma aula lúdica,
	diferente, dinâmica e atrativa para o aluno. O jogo propicia um desenvolvimento
	global nas áreas cognitiva, afetiva, social, moral e motora, e contribui para a
	estruturação da autonomia, criatividade, responsabilidade e cooperação. Esse
	projeto de ensino visa desafiar o aluno a desenvolver um jogo físico sobre a matéria
	“Equipamentos de Redes” da disciplina de Redes de Computadores I. A finalidade
	do projeto é estimular a aprendizagem dos alunos, fazendo com que eles
	desenvolvam seu lado criativo e cooperativo ao elaborar um jogo com regras claras
	e objetivas, ao mesmo tempo em que constrói o seu conhecimento sobre a matéria.
	Esse projeto conta com dois alunos colaboradores que irão desenvolver seu jogo,
	avaliar os demais e produzir documentos com os resultados alcançados. Para isso,
	a primeira etapa do projeto será a explicação do conteúdo pela docente referente a
	“Equipamentos de Redes de Computadores” e os alunos farão um seminário sobre o
	tema. Em seguida, montar as equipes que irão produzir os jogos. As equipes
	deverão ter no máximo 5 alunos e deverão pesquisar sobre a matéria designada,
	planejar as regras e produzir/confeccionar o jogo. Após a confecção, em uma data
	específica, as duas turmas se reunirão e disponibilizarão o jogo para que todos os
	colegas possam utilizá-lo e avaliá-lo. Poderão participar alunos do 2º ano A e B do
	curso de Manutenção e Suporte em Informática. O resultado esperado é que a partir
	dessa atividade o discente consiga determinar regras ao elaborar qualquer projeto e,
	também, possa desenvolver a sua criatividade em projetos que vão além do
	computador. Para isso, o produto final será um jogo físico e uma avaliação dos
	colaboradores (através de relatórios, artigos, banners, etc). O projeto encontra-se
	em andamento, na fase de confecção e já com data prevista para apresentação dos
	jogos. Esse projeto foi aplicado, também, no ano anterior e resultou apenas no
	desenvolvimento do jogo, sem avaliação de colaboradores.
	
	\vspace{\onelineskip}
	
	\noindent
	\textbf{Palavras-chave}: Jogos. Redes de Computadores. Educação.
	
\end{document}
