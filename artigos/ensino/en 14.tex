\documentclass[article,12pt,onesidea,4paper,english,brazil]{abntex2}

\usepackage{lmodern, indentfirst, nomencl, color, graphicx, microtype, lipsum}			
\usepackage[T1]{fontenc}		
\usepackage[utf8]{inputenc}		

\setlrmarginsandblock{2cm}{2cm}{*}
\setulmarginsandblock{2cm}{2cm}{*}
\checkandfixthelayout

\setlength{\parindent}{1.3cm}
\setlength{\parskip}{0.2cm}

\SingleSpacing

\begin{document}
	
	\selectlanguage{brazil}
	
	\frenchspacing 
	
	\begin{center}
		\LARGE IMPACTOS AMBIENTAIS PROVOCADOS PELO DESMATAMENTO: PERCEPÇÃO DOS ESTUDANTES DO CURSO TÉCNICO DE FLORESTAS\footnote{Informações sobre o resumo.}
		
		\normalsize
	Ester Niza de Oliveira Peres\footnote{Bolsista e-mail: esternizaoliveira@gmail.com.Discente do curso técnico de Florestas-
		INSTITUTO FEDERAL RONDÔNIA- Câmpus Ji-Paraná.} 
		Janice Ferreira do Nascimento\footnote{Colaborador(a), janice.nascimento@ifro.edu.br.Câmpus Ji-Paraná.} 
	\end{center}
	
	\noindent Todos os impactos ambientais provocados pelo desmatamento são consequência de
	ações incorretas dos seres humanos. São muitos os que desmatam para aumentar
	suas áreas de criação e plantio. Além disso, o processo de desenvolvimento com a
	construção de rodovias e estradas, se torna um agravante na perda de habitat
	natural para os animais. É necessário que a população conheça e entenda o
	impacto que o desmatamento traz para o meio ambiente de uma forma geral, o que
	pode ser realizado através de projetos de educação ambientais, palestras e
	envolvimento da população nas questões ambientais. Esta pesquisa foi realizada
	através de revisão bibliográfica e aplicação de questionário com 40 (quarenta)
	alunos do 1º ano do curso de técnico em floresta e 20 (vinte) alunos do 4º ano do
	mesmo curso, todos os alunos estudam no Instituto Federal de Rondônia. Com a
	aplicação do questionário e com os dados obtidos, pode- se concluir que a
	percepção dos estudantes sobre o tema “desmatamento” é um pouco subjetiva, pois
	houve algumas diferenças de opiniões, não se tem uma consciência firmada sobre o
	assunto. Para a maioria dos estudantes o desmatamento é somente a retirada de
	árvores, matas e florestas, que assim dizendo seria a destruição da natureza,
	portanto com esses dados tende-se, a saber, que os estudantes de fato não têm
	uma opinião formada sobre o que é o desmatamento, há uma porcentagem mínima
	da falta de conhecimento sobre o tema abordado.
	
	\vspace{\onelineskip}
	
	\noindent
	\textbf{Palavras-chave}: Impacto Ambiental. Desmatamento.Meio Ambiente.
	
\end{document}
