\documentclass[article,12pt,onesidea,4paper,english,brazil]{abntex2}

\usepackage{lmodern, indentfirst, nomencl, color, graphicx, microtype, lipsum}			
\usepackage[T1]{fontenc}		
\usepackage[utf8]{inputenc}		

\setlrmarginsandblock{2cm}{2cm}{*}
\setulmarginsandblock{2cm}{2cm}{*}
\checkandfixthelayout

\setlength{\parindent}{1.3cm}
\setlength{\parskip}{0.2cm}

\SingleSpacing

\begin{document}
	
	\selectlanguage{brazil}
	
	\frenchspacing 
	
	\begin{center}
		\LARGE O USO DA TECNOLOGIA GOOGLE FERRAMENTAS ALIADO AOS MÉTODOS DE ENSINO EM SALA DE AULA AMPLIA A QUALIDADE DO ACOMPANHAMENTO DE TRABALHOS\footnote{Trabalho realizado dentro da Área de Ciências Exatas.}
		
		\normalsize
	Ilma Rodrigues de Souza Fausto\footnote{Coordenadora do Projeto. Professora Orientadora. Docente/Analista de Sistemas - Campus Ji-Paraná – IFRO. (69)99209-1078 Currículo Lattes disponível em:http://lattes.cnpq.br/3193486844184524.} 
	\end{center}
	
	\noindent Sabe-se que o Google Drive é uma ferramenta de gestão, podem representar um ótimo papel na
	educação. Basta que elas sejam utilizadas da maneira correta. Se o professor tem interesse em aliar
	tecnologia e aprendizado para atrair o interesse dos estudantes. A aprendizagem por projetos integra o
	compartilhamento de informações, o processo facilita as experiências de aprendizado dirigidas pelos
	próprios alunos, o que aumenta sua autonomia e segurança dentro da sala de aula. Então se percebe
	que o uso de ferramentas para apoiar o professor em suas aulas é de suma importância para o
	acompanhamento real dos trabalhos desenvolvidos em sala de aula inclusive TCC, aplicando a
	aprendizagem orientada. A ferramenta Google Drive permite que o grupo de alunos compartilhe com
	o professor, desde o início do trabalho, desta forma o professor consegue ver de forma assíncrona o
	que cada integrante do grupo desenvolveu, permitindo que esse acompanhamento ocorra também para
	orientações sobre o desenvolvimento do conteúdo. A ferramenta possui: Documentos Google;
	Apresentações Google; Formulários Google, Planilhas Google e Formulários Google, com essas
	possibilidades os alunos e o professor trabalham de uma maneira mais justa, pois aquelas informações
	de que um trabalhou mais que o outro fica evidenciado no resumo de atividades da ferramenta,
	inclusive quanto tempo o aluno ficou trabalhando no documento. Sabe-se que há uma economia
	relevante na impressão de papéis com a adoção dessa ferramenta, uma proposta importante para as
	novas diretrizes do campus. Em relação ao Google Agenda como ferramenta de organização para os
	alunos e professores, em agendamentos de tarefas, seminários e provas e lembre-os com avisos
	enviados pelo próprio sistema. Percebo nas orientações de trabalhos que estas ferramentas incentivam
	os estudantes a realizarem brainstorms colaborativos além de permitir a visualização de ideias e o
	planejamento de projetos. Uso a ferramenta há dez anos e recomendo, todos os trabalhos desse tempo
	estão lotados na ferramenta, sem utilizar o espaço da máquina do professor.
	
	\vspace{\onelineskip}
	
	\noindent
	\textbf{Palavras-chave}: Google Drive. Ferramentas. Professor. Aluno.
	
	\noindent
	\textbf{Fonte de financiamento}: IFRO.
\end{document}
