\documentclass[article,12pt,onesidea,4paper,english,brazil]{abntex2}

\usepackage{lmodern, indentfirst, nomencl, color, graphicx, microtype, lipsum,textcomp}			
\usepackage[T1]{fontenc}		
\usepackage[utf8]{inputenc}		

\setlrmarginsandblock{2cm}{2cm}{*}
\setulmarginsandblock{2cm}{2cm}{*}
\checkandfixthelayout

\setlength{\parindent}{1.3cm}
\setlength{\parskip}{0.2cm}

\SingleSpacing

\begin{document}
	
	\selectlanguage{brazil}
	
	\frenchspacing 
	
	\begin{center}
		\LARGE PRÁTICAS INTERDISCIPLINARES COMO METODOLOGIA DE SUPERAR A
		FRAGMENTAÇÃO DE CONTEÚDO\footnote{Trabalho realizado dentro da (área de Conhecimento CNPq: 7.08.04.02-8 Métodos e Técnicas de
			Ensino).}
		
		\normalsize
	Danieli de Sá Neiva Cardoso\footnote{Técnica Administrativa do IFRO Campus Colorado do Oeste. Co-Orientadora de projeto de pesquisa IFRO Campus Colorado do Oeste. e-mail: danieli.cardoso@ifro.edu.br.} 
	Licimara da Silva Nicola\footnote{Técnica Administrativa do IFRO Campus Colorado do Oeste. Co-Orientadora de projeto de pesquisa IFRO Campus Colorado do Oeste. e-mail: licimara.nicola@ifro.edu.br.} 
	Márcia Jovani de Oliveira Nunes\footnote{Supervisora Pedagógica do IFRO Campus Colorado do Oeste. Orientadora de projeto de pesquisa IFRO Campus Colorado do Oeste. e-mail: marcia.nunes@ifro.edu.br.} 
	Maria Aparecida Costa Oliveira\footnote{Supervisora Pedagógica do IFRO Campus Colorado do Oeste. Orientadora de projeto de pesquisa
		IFRO Campus Colorado do Oeste. e-mail: maria.oliveira@ifro.edu.br.} 
	\end{center}
	
	\noindent A interdisciplinaridade exige compreensão, ação, planejamentos e atividades de
	forma coletiva. FAZENDA (2002) descreve que o educador precisa estar em
	constante formação, a fim de produzir novidades que tornem as aulas e o processo
	de ensino aprendizagem mais instigante e significativo aos alunos visando obter
	melhores resultados. Assim, este trabalho é resultado de um projeto de pesquisa
	sobre as atividades interdisciplinares desenvolvidas no Campus Colorado, que
	objetivou investigar se ocorrem melhorias na qualidade do processo ensino
	aprendizagem a partir da interdisciplinaridade. Após análise das ementas e Projeto
	Pedagógico do curso, alguns professores aderiram à proposta e desenvolveram
	atividades articuladas, nas disciplinas de: Arte, Biologia, Educação Física, Filosofia,
	Física, Geografia, História, Espanhol, Inglês, Língua Portuguesa, Matemática,
	Mecanização Agrícola, Produção Animal II, Produção Vegetal II, Química, Sociologia
	e Topografia. Para a investigação foi utilizado análise em bancos de dados,
	conselho de classe, reuniões do NAPNE e pesquisa com os alunos dos segundos
	anos do Curso Técnico em Agropecuária Integrado ao Ensino Médio através de
	questionário, 138 alunos participaram, dentre esses 93,5\% declaram que
	observaram os docentes de sua turma, trabalhando disciplinas em conjunto, 2,9\%
	relataram que não observaram essa metodologia e 3,6\% consideraram que talvez
	elas aconteceram. Quanto à melhor compreensão do conteúdo devido às atividades
	interdisciplinares, 59,4\% dos alunos afirmaram que sim, facilita a aprendizagem,
	23,2\% declararam que não observaram alterações quanto ao aprendizado e 17,4\%
	ficaram em dúvida sobre a questão. Sobre à contribuição desta metodologia para
	melhorar a aprendizagem, 92,8 \% declararam que tiveram mais facilidades em
	desenvolver as atividades propostas e apenas 7,2\% sentiram dificuldades em
	acompanhar o planejamento e aplicação da metodologia. Desta forma observamos
	que a interdisciplinaridade vem tentando florescer timidamente em nosso Campus,
	rompendo com o tradicionalismo, quebrando barreiras e preconceitos, enfrentando
	muitas dificuldades, mas também criando oportunidades de trabalhos. Sendo assim
	acompanhar e assessorar o planejamento, desenvolvimento e avaliação das
	atividades interdisciplinares foi o maior objetivo deste projeto de pesquisa, a fim de
	promover práticas de ensino exitosas.
	
	\vspace{\onelineskip}
	
	\noindent
	\textbf{Palavras-chave}: Aprendizagem. Interdisciplinar. Ensino-aprendizagem.
	
\end{document}
