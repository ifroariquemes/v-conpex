\documentclass[article,12pt,onesidea,4paper,english,brazil]{abntex2}

\usepackage{lmodern, indentfirst, nomencl, color, graphicx, microtype, lipsum}			
\usepackage[T1]{fontenc}		
\usepackage[utf8]{inputenc}		

\setlrmarginsandblock{2cm}{2cm}{*}
\setulmarginsandblock{2cm}{2cm}{*}
\checkandfixthelayout

\setlength{\parindent}{1.3cm}
\setlength{\parskip}{0.2cm}

\SingleSpacing

\begin{document}
	
	\selectlanguage{brazil}
	
	\frenchspacing 
	
	\begin{center}
		\LARGE O TEATRO COMO FORMA DE EXPRESSÃO, COMUNICAÇÃO, REFLEXÃO E
		CATALIZADOR DO ENSINO
		
		\normalsize
		Andressa Evellyn de Freitas\footnote{Bolsista, andressaevellynnn@gmail.com, Campus Ariquemes.} 
		Dayana Gonçalves da Costa\footnote{Bolsista, dayanagoncalvesdacosta@gmail.com, Campus Ariquemes.} 
		Manoel Sampaio Schiavi\footnote{Colaborador, Manoel.schiavi@ifro.edu.br, Campus Ariquemes.} 
	Renivaldo Oliveira Fortes\footnote{Colaborador, renivaldo.fortes@ifro.edu.br, Campus Ariquemes} 
   Oscar Costa Borche
	\end{center}
	
	\noindent A proposta do projeto consiste em proporcionar uma vivência em Arte, mais
	especificamente na área ligada ao Teatro. Relacionar informações, técnicas e
	conceitos, valorizar as diversidades étnicas, sociais, culturais e políticas de
	Rondônia. Melhorar a autoestima valorizando-se enquanto sujeito, despertar o
	interesse e respeito pela sua própria produção, dos colegas e de outros artistas,
	criar um campo propício para o ensino, análise e reflexão. Constituir e estruturar um
	grupo de estudos teatrais com a participação da comunidade interna do campus.
	A arte tem sido proposta como instrumento fundamental de educadores, ocupando
	historicamente papéis diversos, desde Platão, que a considerava como base de toda
	a educação natural. (PCN Arte, pag. 83)
	O teatro no processo de formação do ser, cumpre não só função integradora, mas
	dá oportunidade para que o indivíduo se aproprie crítica e construtivamente dos
	conteúdos sociais e culturais de sua comunidade mediante trocas com os seus
	grupos. No dinamismo da experimentação, da fluência criativa propiciada pela
	liberdade e segurança, o indivíduo pode transitar livremente por todas as
	emergências internas integrando imaginação, percepção, emoção, intuição, memória
	e raciocínio. A arte geralmente é entendida como a atividade humana ligada a
	manifestações de ordem estética, educacional e comunicativa, realizada a partir da
	percepção, das emoções e das ideias, com o objetivo de estimular essas instâncias
	da consciência, dando um significado único e diferente para cada ser humano.
	O Projeto de ensino denominado “O Teatro Como Forma de Expressão,
	Comunicação, Reflexão e Catalizador do Ensino”, propõe uma estética teatral
	fundamentada no ator e na plateia, e principalmente nessa relação com a plateia,
	assim como no Laboratório de Grotowski, permite anular a necessidade de
	iluminação e cenário. A forma de pensar desse artista contribui para a articulação
	didático-pedagógica e dos conteúdos a partir da seleção de textos para uma
	montagem cênica, produção de esquetes, exercícios de improvisação e expressão
	corporal. O projeto apossa-se do termo laboratório cunhado pelo pesquisador para
	abrir um campo de aprendizagem aprofundada da linguagem teatral.
	
	\vspace{\onelineskip}
	
	\noindent
	\textbf{Palavras-chave}: Educação Musical. Ensino Coletivo. Ensino integrado.
	
		\vspace{\onelineskip}
	
	\noindent
	\textbf{Fonte de financiamento}: IFRO - PROEN.
\end{document}
