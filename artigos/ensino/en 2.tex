\documentclass[article,12pt,onesidea,4paper,english,brazil]{abntex2}

\usepackage{lmodern, indentfirst, nomencl, color, graphicx, microtype, lipsum,textcomp}			
\usepackage[T1]{fontenc}		
\usepackage[utf8]{inputenc}		

\setlrmarginsandblock{2cm}{2cm}{*}
\setulmarginsandblock{2cm}{2cm}{*}
\checkandfixthelayout

\setlength{\parindent}{1.3cm}
\setlength{\parskip}{0.2cm}

\SingleSpacing

\begin{document}
	
	\selectlanguage{brazil}
	
	\frenchspacing 
	
	\begin{center}
		\LARGE A IMPORTÂNCIA DA HIGIENIZAÇÃO DAS MÃOS INTEGRADA À ATIVIDADE
		PEDAGÓGICA LABORATORIAL E AGROINDUSTRIAL\footnote{Trabalho realizado dentro de práticas exitosas no ensino.}
		
		\normalsize
	Josiane Capellaro Varela\footnote{Discente do curso Técnico em Alimentos, josy\_capellaro@hotmail.com, Campus Colorado do Oeste.} 
		Donizete Alves de Lima Júnior\footnote{Discente do curso Técnico em Alimentos donny\_alves@hotmail.com, Campus Colorado do Oeste.} \\
	Jessica Capellaro Varela\footnote{Discente do curso Técnico em Alimentos, jessicacvarela@hotmail.com, Campus Colorado do Oeste} 
	Nélio Ranieli Ferreira de Paula\footnote{Orientador e professor, nelio.ferreira@ifro.edu.br, Campus Colorado do Oeste.} 
	\end{center}
	
	\noindent Por meio do ensino prático de Higienização das Mãos, realizado na disciplina de
	Higiene e Controle de Qualidade na Indústria de Alimentos, foi possível a percepção
	da importância desta atividade para a indústria alimentícia, uma vez que esta técnica
	atende os preceitos básicos de saúde e segurança encontrados nas legislações
	vigentes. Tendo como objetivo geral, a promoção e discussão de conhecimento
	técnico e pedagógico a respeito da importância de uma correta higienização das
	mãos e corporal, sendo uma prática fundamental antes de qualquer procedimento,
	independentemente de sua área de atuação. Além disso, ocorreu a socialização de
	conhecimento dos fatores intrínsecos e extrínsecos que são interligados com a
	higienização e controlam o desenvolvimento microbiano, influenciando assim, na
	qualidade e segurança alimentar. Dentro do ensino prático, foram realizados
	diversos processos metodológicos, como: Pré-lavagem das mãos (apenas com
	água), Lavagem com detergente, Realização dos movimentos das palmas das
	mãos, entrelace dos dedos, esfregue das unhas, articulações, punhos (15 a 20
	segundos) enxague, secagem com papel toalha, e por último, anti-sepsia com álcool
	70\%. No desenvolvimento desta metodologia, foi aplicado o princípio das boas
	práticas de manuseio e produção de alimentos, importantíssimo na prevenção de
	toxinfecções, garantindo alimentos seguros e inócuos ao mercado consumidor.
	Também foram utilizados outros materiais específicos, como: álcool 70\%,
	Detergente, Água, Papel toalha, Placas de Petri, Estufa, Autoclave, Contador de
	Colônias, Marcadores entre outros. Houve 100\% de aprendizagem, pois ficou
	evidente que a higienização das mãos é uma medida individual simples, fácil, menos
	dispendiosa e rápida para a sua execução. Sendo essencial para a qualidade e
	saúde dos produtos oferecidos ao consumidor.
	
	\vspace{\onelineskip}
	
	\noindent
	\textbf{Palavras-chave}: Higienização das mãos. Alimentos. Aprendizagem.
	
\end{document}
