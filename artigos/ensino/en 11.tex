\documentclass[article,12pt,onesidea,4paper,english,brazil]{abntex2}

\usepackage{lmodern, indentfirst, nomencl, color, graphicx, microtype, lipsum,textcomp}			
\usepackage[T1]{fontenc}		
\usepackage[utf8]{inputenc}		

\setlrmarginsandblock{2cm}{2cm}{*}
\setulmarginsandblock{2cm}{2cm}{*}
\checkandfixthelayout

\setlength{\parindent}{1.3cm}
\setlength{\parskip}{0.2cm}

\SingleSpacing

\begin{document}
	
	\selectlanguage{brazil}
	
	\frenchspacing 
	
	\begin{center}
		\LARGE GERADOR K.M.M.L 2.0\footnote{Trabalho realizado dentro das Ciências Exatas e da Terras.}
		
		\normalsize
		Maria Laura Felix\footnote{Colaboradora, email: marialaurafelix212@gmail.com, Campus do IFRO de Ji-Paraná.} 
		Kleyton Morais\footnote{Colaborador, email: kjmorais@outlook.com, Campus do IFRO de Ji-Paraná.} 
	Marco Aurélio de Jesus\footnote{Orientador, email: marco.aurelio@ifro.edu.br, Campus do IFRO de Ji-Paraná.} 
	\end{center}
	
	\noindent O Reator de Plasma criado em 2015 pelo Dr. Keshe, um físico nuclear especialista
	em energia livre, energia do plasma e anti-gravidade, é um sistema auto nutritivo
	constituído por um campo magnético rotativo plasmático dentro de bobinas de fio de
	cobre que produz grafeno e por consequência gera energia. Têm-se como objetivo
	explicar como ocorre a geração de energia, mostrar que futuramente essa tecnologia
	pode ser uma alternativa para o aproveitamento de energia de um ambiente e
	promover a interdisciplinaridade entre Química e Física. O experimento foi realizado
	no Laboratório de Física, do campus IFRO de Ji-Paraná, pelos colaboradores,
	ambos estão cursando técnico em Química de nível médio e supervisionado pelo
	orientador. Para produzir o Reator foi necessário um recipiente de plástico de 500ml
	com tampa, quatro bobinas de cobre, 250ml de um refrigerante e 21,8g de NaOH.
	Fixou-se as bobinas nos pequenos furos na tampa do recipiente, em seguida foi feita
	a reação do refrigerante com o NaOH separadamente em um recipiente de vidro.
	Depois de feita a reação foi transferida para o recipiente com as bobinas. A solução
	ficou no recipiente fechado durante 1hora e depois foi retirada e com a ajuda de um
	multímetro realizou-se a primeira medição. Depois da analise do experimento
	percebeu-se que a cada 6 horas o multímetro mudava de positivo para negativo,
	concluiu-se então que quando ele estava na fase negativa ele estava
	“recarregando”, ele não liberava qualquer energia, pois a estava acumulando a partir
	do ambiente, já na fase positiva ocorria um “descarregamento” passando a liberar a
	eletricidade captada do ambiente. O reator conseguiu gerar 60 miliVolts. O
	dispositivo foi apresentado para uma turma do 3° ano do curso técnico em Química,
	que observaram a demonstração e as explicações. Os alunos interagiram fazendo
	perguntas e demonstrando interesse pelo tema, o que leva a concluir que a prática
	surtiu o efeito didático esperado. Considerando ainda que princípios físicos foram
	estudados por meio de reações químicas, conclui-se que o dispositivo promoveu a
	interdisciplinaridade entre as duas ciências.
	
	\vspace{\onelineskip}
	
	\noindent
	\textbf{Palavras-chave}: Física. Gerador. Plasma.
	
\end{document}
