\documentclass[article,12pt,onesidea,4paper,english,brazil]{abntex2}

\usepackage{lmodern, indentfirst, nomencl, color, graphicx, microtype, lipsum}			
\usepackage[T1]{fontenc}		
\usepackage[utf8]{inputenc}		

\setlrmarginsandblock{2cm}{2cm}{*}
\setulmarginsandblock{2cm}{2cm}{*}
\checkandfixthelayout

\setlength{\parindent}{1.3cm}
\setlength{\parskip}{0.2cm}

\SingleSpacing

\begin{document}
	
	\selectlanguage{brazil}
	
	\frenchspacing 
	
	\begin{center}
		\LARGE PROJETO “VIVÊNCIAS INTERDISCIPLINARES”
		INTEMPERISMO, PH SOLO (POTENCIAL HIDROGÊNIO IÔNICO) E EROSÃO\footnote{Trabalho realizado dentro da área de Conhecimento CNPq: Ciência Humana.}
		
		\normalsize
	Claudia C. Coimbra\footnote{Professora PEBTT, claudia.coimbra@ifro.edu.br, Campus Ariquemes.} 
	Cinthia C.M. Silva\footnote{Professora substituta, cinthia.matias@ifro.edu.br, Campus Ariquemes.} 
	Soraia S. Martins\footnote{Professora substituta, Soraia.martins@ifro.edu.br, Campus Ariquemes.} 
	Quezia S. Rosa\footnote{Professora PEBTT, quezia.rosa@ifro.edu.br, Campus Ariquemes.} 
	\end{center}
	
	\noindent Os conteúdos que tem o solo como elemento central acontecem em
	diversas disciplinas e o Instituto Federal de Educação, Ciência e Tecnologia de
	Rondônia (IFRO) mostra-se como ambiente propício para
	propagação de conhecimentos acerca da importância ecológica, social e econômica
	da conservação dos solos, tornando-se interessante tratar do assunto de uma forma
	integrada, permitindo ao aluno a percepção da complementaridade dos diversos
	aspectos em relação ao solo. O objetivo desse projeto foi apresentar uma proposta
	didática para o Ensino Médio integrado, a partir de uma abordagem multi, inter e
	intradisciplinar, correlacionando o estudo do intemperismo, a erosão e a análise do
	pH do solo. Estes conteúdos foram apresentados na forma de aulas práticas,
	devidamente ilustrados, sempre com o intuito de explicitar a relação das diferentes
	variáveis abordadas nas referidas disciplinas, correlacionando-as com a preservação
	e a sustentabilidade do meio ambiente. O projeto foi realizado na área do Instituto
	Federal de Educação, Ciência e Tecnologia de Rondônia - Campus Ariquemes e
	proximidades em dois dias (17/08/2017 - quinta-feira e 18/08/2017 – sexta-feira) em
	todos os horários do turno matutino com somatória de 8 aulas trabalhadas pelas
	docentes Cláudia Coimbra (Geografia), Cínthia Silva (Solos) e Soraia Martins
	(Química) nos primeiros anos do curso técnico de agropecuária (1º A (26 alunos), 1º
	B (28 alunos) e º1ºC (26 alunos) totalizando 80 alunos). Para a execução do projeto
	foi realizado um remanejamento dos horários de aulas dos docentes. As aulas foram
	realizadas da seguinte forma: Aula prática para observação do intemperismo
	físico/químico e tipos de erosões com o registro fotográfico/ responsável professora
	Cláudia Coimbra - disciplina de Geografia. Recolhimento de amostras de solos em
	áreas distintas para processamento em laboratório de biologia/ responsável
	professora Cínthia Silva - disciplina de solos. Aula experimental no laboratório de
	química para realização das análises de PH do solo/ responsável professora Soraia
	Martins da disciplina de química. E a disciplina de Orientação para a Prática
	Profissional e Pesquisa/ responsável professora Quezia Rosa, como catalisador,
	para que os alunos possam perceber com maior clareza orientação para a
	elaboração de um relatório de desempenho prático, onde os alunos deverão
	descrever como se deu a execução das aulas teóricas e práticas. Espera-se com o
	projeto fornecer informações básicas sobre os conceitos e processos de origem e
	formação dos solos, bem como sua importância para os ecossistemas naturais e
	agrícolas. Deste modo, acredita-se que se estará priorizando o que assegura os
	PCNs, pois Interdisciplinaridade significa interdependência, interação e comunicação
	entre campos do saber, ou disciplinas, o que possibilita a integração do
	conhecimento em áreas significativas.
	
	\vspace{\onelineskip}
	
	\noindent
	\textbf{Palavras-chave}: Interdisciplinaridade. Intemperismo. Solo. Erosão.
	
\end{document}
