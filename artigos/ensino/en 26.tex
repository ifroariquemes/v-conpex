\documentclass[article,12pt,onesidea,4paper,english,brazil]{abntex2}

\usepackage{lmodern, indentfirst, nomencl, color, graphicx, microtype, lipsum,textcomp}			
\usepackage[T1]{fontenc}		
\usepackage[utf8]{inputenc}		

\setlrmarginsandblock{2cm}{2cm}{*}
\setulmarginsandblock{2cm}{2cm}{*}
\checkandfixthelayout

\setlength{\parindent}{1.3cm}
\setlength{\parskip}{0.2cm}

\SingleSpacing

\begin{document}
	
	\selectlanguage{brazil}
	
	\frenchspacing 
	
	\begin{center}
		\LARGE OLIMPÍADAS DO CONCRETO
		
		\normalsize
	Rayssa Forte Lopes\footnote{Bolsista, rayssa-forte@outlook.com, Campus Vilhena.} 
	José Vitor dos Santos Silva\footnote{Bolsista, vitor99094900@outlook.com, Campus Vilhena.} 
	Junior Batista Duarte\footnote{Orientador(a), junior.duarte@ifro.edu.br, Campus Vilhena.} 
	\end{center}
	
	\noindent O estudo do concreto é extremamente vasto e muitas vezes uma exposição teórica
	do tema não é suficiente para abordar diversos aspectos. Portanto, o objetivo desse
	projeto foi sanar essa deficiência, de tal forma que desse aos alunos um pouco da
	vivência de laboratório. Esse projeto teve como público-alvo os alunos do curso
	técnico em edificações. Inicialmente os alunos deveriam definir um traço conforme
	as equações da dosagem passadas em sala, depois deveriam pesar os
	componentes, só então eles deveriam lançar os componentes na betoneira. Os
	ensaios realizados foram: amostragem do concreto, consistência da colher,
	abatimento de tronco de cone, massa específica, moldagem de corpos de prova e
	ensaio de resistência a compressão, esse último só foi feito com a turma do 1° B
	edificações, como o concreto demora 28 dias para ficar pronto os outros grupos
	apenas assistiram esse primeiro ensaio. As categorias que os alunos concorreram
	foram: maior abatimento e maior resistência. Como material foi usado cimento, areia,
	brita e água e como equipamentos foi usado uma betoneira, carrinho de mão, colher
	de pedreiro, fôrmas para corpo de prova cilíndrico (10x20), prensa hidráulica entre
	outros. No abatimento o primeiro colocado foi um grupo do terceiro ano e no critério
	resistência foi um grupo do primeiro B. Houve uma participação massiva por parte
	dos alunos, houve muito empenho e todos os objetivos do projeto foram alcançados.
	O interessante foi a reação na sala de aula antes e depois dos ensaios, perguntas
	simples que os alunos tinham dificuldade de responder antes dos ensaios, como
	“Descreva o processo do sluptest”, eram respondidas com muita facilidade após o
	ensaio. Com isso é possível concluir que trazer a prática pra sala de aula tem
	excelentes resultados.
	
	\vspace{\onelineskip}
	
	\noindent
	\textbf{Palavras-chave}: Olimpíadas. Concreto. Dosagem.
	
\end{document}
