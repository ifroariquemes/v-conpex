\documentclass[article,12pt,onesidea,4paper,english,brazil]{abntex2}

\usepackage{lmodern, indentfirst, nomencl, color, graphicx, microtype, lipsum}			
\usepackage[T1]{fontenc}		
\usepackage[utf8]{inputenc}		

\setlrmarginsandblock{2cm}{2cm}{*}
\setulmarginsandblock{2cm}{2cm}{*}
\checkandfixthelayout

\setlength{\parindent}{1.3cm}
\setlength{\parskip}{0.2cm}

\SingleSpacing

\begin{document}
	
	\selectlanguage{brazil}
	
	\frenchspacing 
	
	\begin{center}
		\LARGE PARA ALÉM DA PRÁTICA: ESTABELECENDO RELAÇÕES COM A
		COMUNIDADE E EMPREGADORES\footnote{Trabalho realizado dentro da área de Conhecimento CNPq/CAPES: Métodos e Técnicas de Ensino.}
		
		\normalsize
	Quézia da Silva Rosa\footnote{Coordenadora, quezia.rosa@ifro.edu.br, Campus Ariquemes.} 
	Mirian de Oliveira Bertotti \,\,
	Rodrigo Henrique Santoro\footnote{Colaborador, rodrigohsantoro@gmail.com, Campus Ariquemes.} 
	\end{center}
	
	\noindent Buscando vincular conhecimentos teóricos do contexto escolar com a prática no
	campo, a visita técnica é usada como recurso metodológico para dinamizar as aulas,
	ampliar os conhecimentos desenvolvidos em sala e estreitar a relação entre o
	estudante de Aquicultura e a futura área de atuação. Quando é possível, além de
	observar, praticar, abre-se a oportunidade de estabelecer laços com a comunidade e
	estreitar vínculos com empresas e/ou produtores que poderão levar ao estágio e
	quiçá, um emprego. O educando acompanhado de seu professor, poderá
	demonstrar que o conhecimento científico é aliado do conhecimento prático e que
	deve servir para melhorar os diversos aspectos relacionados ao cotidiano. Essa
	metodologia, por muitas vezes, pode ser o caminho para que os estudantes tenham
	os primeiros contatos com o universo profissional podendo abrir as portas do
	emprego quando for possível à empresa e/ou produtor reconhecer o potencial dos
	educandos. No Curso Técnico em Aquicultura do Campus Ariquemes, essa
	metodologia tem sido amplamente adotada com a visão de que é possível usar a
	prática para dar visibilidade à instituição e à capacidade dos alunos. Como o curso é
	ofertado no período noturno as visitas técnicas são realizadas nos sábados letivos e,
	em algumas situações, mais de um professor acompanha a turma, pois a atividade
	pode tornar-se uma prática interdisciplinar, permitindo assim, a contextualização e
	(re)significação dos conteúdos. Pós-visita opta-se por elaborar ou um relatório
	técnico, ou um debate sobre todas as atividades trabalhadas e conhecimentos
	obtidos. Em uma região que é a maior produtora de pescados do maior estado
	produtor de peixes de cativeiro do País, existe ainda muito a ser explorado, assim o
	próximo passo é estabelecer parcerias com associações, cooperativas e pequenos
	proprietários que são ainda mais carentes de assistência técnica e pode ser o novo
	filão onde os profissionais poderão atuar. A troca de experiências entre os alunos,
	professores e profissionais da área é uma boa opção de aprendizado, agrega
	valores incalculáveis tanto para a trajetória pessoal, quando para a trajetória
	profissional, além de motivar os educandos cada vez mais para prosseguirem no
	curso e atuarem na área.
	
	\vspace{\onelineskip}
	
	\noindent
	\textbf{Palavras-chave}: Aquicultura. Piscicultura. Visitas Técnicas.
	
\end{document}
