\documentclass[article,12pt,onesidea,4paper,english,brazil]{abntex2}

\usepackage{lmodern, indentfirst, nomencl, color, graphicx, microtype, lipsum}			
\usepackage[T1]{fontenc}		
\usepackage[utf8]{inputenc}		

\setlrmarginsandblock{2cm}{2cm}{*}
\setulmarginsandblock{2cm}{2cm}{*}
\checkandfixthelayout

\setlength{\parindent}{1.3cm}
\setlength{\parskip}{0.2cm}

\SingleSpacing

\begin{document}
	
	\selectlanguage{brazil}
	
	\frenchspacing 
	
	\begin{center}
		\LARGE MERGULHANDO NAS PALAVRAS: A LEITURA COMO FONTE DE PRAZER E
		PENSAMENTO AUTÔNOMO\footnote{Trabalho realizado dentro da área de Conhecimento CNPq: Linguística, Letras e Arte.}
		
		\normalsize
	Mirian de O. Bertotti\footnote{Coordenadora, mirian.bertotti@ifro.edu.br, Campus Ariquemes.} 
	Emanuele Esteffani Cavalcante\footnote{Colaboradora; emanuele.esteffani@gmail.com, Campus Ariquemes.} 
	Danielly Akemi de Souza Higuti\footnote{Colaboradora, daniellyakeme@gmail.com, Campus Ariquemes.} 
	Jelen Orana Marques Prado de Almeida\footnote{Colaboradora, jelenprado@gmail.com, Campus Ariquemes.} 
     Júlia Moraes Bratek\footnote{Colaboradora, julia.bratek@hotmail.com, Campus Ariquemes.} 
	\end{center}
	
	\noindent A leitura é uma parte importante de uma aprendizagem construída através de
	relações. Não nascemos lendo, adquirimos essa habilidade ao longo do tempo de
	diversas formas. Desta maneira, a escola desponta como umas das instituições
	fundamentais para que esse processo ocorra. Precisamos aprender a ler para
	aprender a indagar, a desenvolver opinião própria, aprender a ouvir, entender, falar
	e compreender, estes são atos que tornam perceptível a importância da leitura na
	vida do indivíduo, pois o diálogo começa a partir do momento em que entramos em
	contato consigo mesmo. “Mergulhando nas Palavras" é um projeto que consiste em
	fomentar a leitura de textos literários. Tem como objetivo ampliar o repertório
	linguístico e literário, propiciando espaço e momentos prazerosos para discente e
	docente dentro do campus Ariquemes. O projeto surgiu através da observação dos
	próprios discentes de que em horários vagos, muitos liam pelos corredores sem
	conforto algum, já que a infraestrutura do instituto não permitia obter a prática
	prazerosa da leitura. Deste modo, observando o espaço disponível na biblioteca do
	campus Ariquemes, voluntários organizaram um local aconchegante para a leitura. A
	inauguração aconteceu no dia 06/04/2017, o “Mergulhando” possui um acervo com
	235 exemplares literários e 265 exemplares de HQs, ambos recebidos através de
	campanhas de doação com a comunidade escolar. Desde a inauguração, o local
	recebe cerca de 300 visitantes mensais e 30 diários, o ambiente está disponível de
	segunda a sexta-feira, nos horários de almoço e no período noturno para que os
	alunos residentes possam usufruir do espaço. No início das atividades contava com
	04 discentes e um grupo de docentes, nos dias atuais são 25 colaboradores sendo
	05 docentes e 20 discentes, estes atuam sem obtenção de nota e/ou bolsas de
	incentivo, são responsáveis por atividades como catalogação, divulgação,
	organização, manutenção e desenvolvimento de atividades motivadoras à leitura.
	Ações como o “Mergulhando nas Palavras” realizadas dentro de uma instituição com
	vistas à formação técnica mostram-nos que é possível pensarmos em uma
	educação omnilateral, na qual a leitura tem papel essencial para o desenvolvimento
	do pensar autônomo dos educandos.
	
	\vspace{\onelineskip}
	
	\noindent
	\textbf{Palavras-chave}: Leitura.Aprendizagem.Educação.
	
\end{document}
