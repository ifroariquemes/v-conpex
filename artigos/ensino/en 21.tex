\documentclass[article,12pt,onesidea,4paper,english,brazil]{abntex2}

\usepackage{lmodern, indentfirst, nomencl, color, graphicx, microtype, lipsum}			
\usepackage[T1]{fontenc}		
\usepackage[utf8]{inputenc}		

\setlrmarginsandblock{2cm}{2cm}{*}
\setulmarginsandblock{2cm}{2cm}{*}
\checkandfixthelayout

\setlength{\parindent}{1.3cm}
\setlength{\parskip}{0.2cm}

\SingleSpacing

\begin{document}
	
	\selectlanguage{brazil}
	
	\frenchspacing 
	
	\begin{center}
		\LARGE MOMENTOS EDUCATIVOS QUE CONTRIBUEM NO PROCESSO DE ENSINO APRENDIZAGEM\footnote{Trabalho realizado dentro da (área de Conhecimento CNPq: 70800006 Educação).}
		
		\normalsize
	Elisângela de Carvalho Franco\footnote{Orientadora/Coordenadora, elisangela.franco@ifro.edu.br, Campus Ariquemes.} 
	Maria Ângela J. Maschio\footnote{Colaboradora, maria.angela@ifro.edu.br, Campus Ariquemes.} 
	Marli Pereira\footnote{Colaboradora, marli.pereira@ifro.edu.br, Campus Ariquemes.} 
	\end{center}
	
	\noindent Sabe-se que um dos principais elementos que impulsionam a aprendizagem é a
	motivação. A partir de situações sejam elas lúdicas ou reflexivas é possível
	desenvolver momentos prazerosos que tornam o ambiente de ensino um lugar
	propício ao desejo de aprender. Desta forma, há uma variedade de métodos que
	auxiliam os estudantes a compreender os procedimentos de ensino que são
	transmitidos no contexto escolar de forma (intra) interdisciplinar. Entre esses
	métodos há o diálogo e a reflexão. No processo de aprendizagem é fundamental
	proporcionar momentos ao diálogo sobre as prováveis causas que estejam
	influenciando no ensino-aprendizagem. Essas intervenções auxiliam na reflexão dos
	fatores e nas medidas que possam ser tomadas na resolução dos problemas. Assim,
	o presente trabalho visa apresentar ações que foram desenvolvidas pela equipe
	multidisciplinar da Coordenação de Assistência ao Educando (CAED), Campus
	Ariquemes, por meio do Serviço de Orientação Educacional (SOE), na forma de
	momentos educativos com os estudantes dos cursos técnicos integrados. Para isso,
	a metodologia foi qualitativa de natureza expositiva e dialógica, tendo como
	procedimentos o desenvolvimento de palestras com exposição de vídeos
	motivacionais. O público-alvo foram os estudantes dos cursos técnicos de
	Agropecuária, Alimentos e Informática contando, também, com o apoio e a
	participação de profissionais das áreas de Psicologia, Pedagogia, Nutrição,
	Enfermagem, Serviço Social, Assistência de Alunos e de apoio administrativo. Para
	o desenvolvimento da atividade foram utilizados os resultados da Ata do Conselho
	de Classe do primeiro semestre para a coleta de diagnóstico das causas que mais
	influenciaram no desempenho escolar. Posteriormente, tabulou-se um plano de
	trabalho com a escala de programação para quatro dias, com turmas de 55
	estudantes, intercalados dos três cursos técnicos. Ao término participaram dos
	momentos educativos mais de 180 estudantes. Enfim, a ação conseguiu atingir os
	estudantes à reflexão de mudanças significativas quanto às situações que
	influenciaram no primeiro semestre letivo no desempenho educacional. Portanto,
	espera-se que para o decorrer do segundo semestre aja à melhoria dos
	procedimentos atitudinais e, principalmente, na aprendizagem dos estudantes para o
	bem-estar do processo de ensino-aprendizagem.
	
	\vspace{\onelineskip}
	
	\noindent
	\textbf{Palavras-chave}: Motivação. Aprendizagem. Diálogo. Reflexão.
	
\end{document}
