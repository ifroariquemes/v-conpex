\documentclass[article,12pt,onesidea,4paper,english,brazil]{abntex2}

\usepackage{lmodern, indentfirst, nomencl, color, graphicx, microtype, lipsum}			
\usepackage[T1]{fontenc}		
\usepackage[utf8]{inputenc}		

\setlrmarginsandblock{2cm}{2cm}{*}
\setulmarginsandblock{2cm}{2cm}{*}
\checkandfixthelayout

\setlength{\parindent}{1.3cm}
\setlength{\parskip}{0.2cm}

\SingleSpacing

\begin{document}
	
	\selectlanguage{brazil}
	
	\frenchspacing 
	
	\begin{center}
		\LARGE PROMOVENDO A QUALIDADE DE VIDA NO AMBIENTE
		ESCOLAR\footnote{Trabalho realizado dentro da área de Conhecimento CNPq 7.08.00.00-6.}
		
		\normalsize
		Maria Ângela Justino Maschio\footnote{Coordenadora, maria.angela@ifro.edu.br, Campus Ariquemes.} 
		Débora de Mattos Branth\footnote{Colaboradora, debora.branth@ifro.edu.br, Campus Ariquemes.} 
		Osmar Martins\footnote{Colaborador, osmar.martins@ifro.edu.br, Campus Ariquemes.} 
	Jósé Fábio Xavier\footnote{Colaborador, fábio.xavier@ifro.edu.br, Campus Ariquemes.} 
	\end{center}
	
	\noindent Higiene é o conjunto de conhecimentos e técnicas que visam promover a saúde e
	evitar as doenças, por isso para ter boa saúde é necessário ter bons hábitos de
	higiene corporal, mental, ambiental e alimentar. Uma vez que a Residência
	Estudantil é uma extensão da casa dos estudantes, uma das principais
	preocupações da Instituição é em relação à integralidade da saúde e qualidade de
	vida dos mesmos. Sendo assim, esse projeto teve o objetivo de sensibilizar os
	estudantes residentes sobre a necessidade de cuidarem bem de si próprios e do
	meio em que vivem para terem uma qualidade de vida saudável fora de seus lares.
	Para tal, foi utilizada como metodologia a realização de: 1) rodas de conversas, com
	a participação da coordenação da CAED, assistente social e nutricionista; 2) visitas
	in loco à Residência Estudantil, semanalmente, pela assistente social; 3) palestra
	sobre higiene e saúde, ministrada pela assistente social; 4) palestra sobre doenças
	respiratórias, ministrada pelo enfermeiro. Todas as atividades foram realizadas
	dentro no campus, no período noturno, e alcançou a participação de 81 estudantes
	residentes. Após a realização do projeto verificou-se uma mudança positiva de
	comportamento em relação aos hábitos diários de higiene, conforme observadas
	durante as visitas in loco. Também foi identificado que os estudantes passaram a ter
	a rotina de manter o ambiente asseado, o quarto organizado, banheiros
	higienizados, mesas do refeitório limpas, bem como o compromisso em realizar seus
	deveres dentro da residência com assiduidade. Sendo assim, ao educar para a
	saúde e para a higiene, de forma contextualizada e sistemática, a escola contribui de
	forma decisiva na formação de cidadãos capazes de atuar em favor da melhoria dos
	níveis de saúde pessoal e da coletividade.
	
	\vspace{\onelineskip}
	
	\noindent
	\textbf{Palavras-chave}: Educação. Saúde. Qualidade de vida.
	
\end{document}
