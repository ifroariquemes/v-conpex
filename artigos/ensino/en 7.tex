\documentclass[article,12pt,onesidea,4paper,english,brazil]{abntex2}

\usepackage{lmodern, indentfirst, nomencl, color, graphicx, microtype, lipsum}			
\usepackage[T1]{fontenc}		
\usepackage[utf8]{inputenc}		

\setlrmarginsandblock{2cm}{2cm}{*}
\setulmarginsandblock{2cm}{2cm}{*}
\checkandfixthelayout

\setlength{\parindent}{1.3cm}
\setlength{\parskip}{0.2cm}

\SingleSpacing

\begin{document}
	
	\selectlanguage{brazil}
	
	\frenchspacing 
	
	\begin{center}
		\LARGE ELABORAÇÃO DE UM MAPA DE RISCOS NO SETOR ADMINISTRATIVO,
		MARCENARIA, CANTINA E ENFERMARIA DO IFRO CAMPUS JI-PARANÁ\footnote{Trabalho realizado dentro das Ciências Agrárias.}
		
		\normalsize
		Daniel Ferro Nobre de Lima\footnote{Discente do Curso Técnico em Florestas Integrado ao Ensino Médio, daniel\_ferro2011@hotmail.com, IFRO – Campus Ji-Paraná.} 
		Raimundo Gomes da Silva Junior\footnote{Docente do Curso Técnico em Florestas Integrado ao Ensino Médio, raimundo.junior@ifro.edu.br, IFRO - Campus Ji-Paraná.} 
	Lorena de Souza Tavares\footnote{Docente do Curso Técnico em Florestas Integrado ao Ensino Médio, lorena.tavares@ifro.edu.br,
		IFRO - Campus Ji-Paraná.} 
	\end{center}
	
	\noindent O mapa de risco é uma representação gráfica dos riscos presentes no
	local de trabalho capazes de oferecer danos à saúde e à vida do trabalhador. O
	mesmo é elaborado para evitar a ocorrência de acidentes e garantir a segurança de
	toda a equipe durante a realização do trabalho diário dentro de uma empresa. Assim
	como qualquer prevenção, um mapa de risco tem como objetivo final reduzir o
	número de acidentes de trabalho e danos à saúde do trabalhador. A presente
	pesquisa teve como objetivo geral analisar os riscos existentes no setor
	administrativo, enfermaria, cantina e marcenaria do IFRO Campus Ji-Paraná para a
	elaboração de um mapa de risco no mesmo. Para tanto, utilizou-se como
	metodologia a revisão bibliográfica e a aplicação de um questionário auto avaliativo
	para os trabalhadores da descrita área. Através da revisão e do questionário, foram
	encontrados fatores como: Queixas recorrentes entre os trabalhadores, satisfação
	com o ambiente de trabalho, doenças ocasionadas pelo local, aparelhos incômodos,
	sugestões, entre outros. Em todas as salas do setor administrativo, cantina,
	enfermaria e marcenaria houveram altos índices de riscos de acidente, reclamações
	e sugestões, entre outros. O campus revela inúmeros dados referente à ocorrência
	de riscos, divididos em grupos denominados: Riscos de acidentes, riscos biológicos,
	químicos, físicos e ergonômicos, sendo possível, através da análise desses dados,
	compreender os fatores que possam ter influenciado, bem como os relacionados ao
	estilo de vida dos trabalhadores. Com base nos dados obtidos pode-se observar a
	necessidade da elaboração de um mapa de risco para compreensão, aproximação e
	intervenção de problemáticas decorridas de tais riscos e que possam estar
	acontecendo dentro da empresa. Haja vista que, com a identificação dos riscos,
	integração e conscientização dos funcionários, o mapa de riscos se torne uma
	importante ferramenta de benefícios para a equipe e para a imagem da instituição.
	
	\vspace{\onelineskip}
	
	\noindent
	\textbf{Palavras-chave}: Mapa de risco, Acidentes de trabalho, Empresa.
	
\end{document}
