\documentclass[article,12pt,onesidea,4paper,english,brazil]{abntex2}

\usepackage{lmodern, indentfirst, nomencl, color, graphicx, microtype, lipsum}			
\usepackage[T1]{fontenc}		
\usepackage[utf8]{inputenc}		

\setlrmarginsandblock{2cm}{2cm}{*}
\setulmarginsandblock{2cm}{2cm}{*}
\checkandfixthelayout

\setlength{\parindent}{1.3cm}
\setlength{\parskip}{0.2cm}

\SingleSpacing

\begin{document}
	
	\selectlanguage{brazil}
	
	\frenchspacing 
	
	\begin{center}
		\LARGE A MÚSICA COMO INSTRUMENTO DE CONVERGÊNCIA DE SABERES\footnote{Projeto de ensino aprovado no edital 159/2017/IFRO com financiamento da Pró-Reitoria de Ensino.}
		
		\normalsize
	Guilherme Rafael Crisostomo Castelo\footnote{Bolsista, guilhermerafaelcrisostomocaste@gmail.com, Campus Ariquemes.} 
		Amanda Pereira Milan\footnote{Bolsista, amandapmilan@hotmail.com, Campus Ariquemes..} 
	Oscar Costa Borche\footnote{Colaborador, oscar.borche@ifro.edu.br, Campus Ariquemes.} \\
	Letícia Araújo Brandão\footnote{Colaborador, leticia.brandao@ifro.edu.br, Campus Ariquemes.} 
 Alessandro Eleutério de Oliveira\footnote{Colaborador, alessandro.oliveira@ifro.edu.br, Campus Ariquemes.} 
Manoel Sampaio Schiavi\footnote{Orientador, manoel.schiavi@ifro.edu.br, Campus Ariquemes.} 
	\end{center}
	
	\noindent O projeto de ensino “A música como instrumento de convergência de saberes”
	consiste na formação de grupos musicais escolares que trabalhem a prática vocal e
	instrumental através do ensino coletivo, explorando o repertório da música popular
	brasileira em consonância com os conteúdos escolares de disciplinas do ensino
	médio. O projeto visa incentivar e encorajar o estudo da música através do repertório
	musical brasileiro engajando os participantes na elaboração e execução de um
	concerto didático com temáticas referentes a formação, identidades e características
	culturais brasileiras. O repertório musical será selecionado tanto por parte da equipe
	do projeto quanto por parte dos participantes promovendo um espaço de encontro
	entre estudantes e servidores no qual se estude a música em seu aspecto amplo de
	apreciação, composição e performance. Neste espaço de socialização e troca de
	experiências dentre os participantes pretende-se fomentar a produção de ações
	culturais que além de promover espaços para expressão e criação artística,
	busquem refletir sobre a identidade cultural brasileira. A metodologia do projeto
	trabalhará de forma integrada as áreas de apreciação e pesquisa de repertório, junto
	com atividades de composição e performance, utilizando o modelo C(L)A(S)P de
	Keith Swanwick que propõe o ensino integrado de música em suas diversas áreas.
	No dado momento, o projeto encontra-se em fase de execução com a organização
	dos ensaios e formação dos grupos instrumentais e vocais; treinamento dos
	bolsistas; definição da temática e escolha do repertório; elaboração do roteiro para o
	concerto didático com a interação das disciplinas de história, sociologia e artes. A
	temática escolhido foi o processo de ocupação do estado de Rondônia em suas
	diferentes fases. Durante as reuniões para definição da temática, os participantes
	apresentaram interesse em desenvolver um fazer musical contextualizado, crítico e
	participativo, inclusive com propostas de relatos orais de histórias de familiares que
	participaram no processo de formação do estado. Sendo assim, o projeto contribui
	para um maior envolvimento da comunidade escolar através do ensino da música
	levando em consideração os aspectos de composição, apreciação e performance,
	possibilitando formas de atuação de educação musical dentro do ensino técnico
	integrado.
	
	\vspace{\onelineskip}
	
	\noindent
	\textbf{Palavras-chave}: Educação Musical. Ensino Coletivo. Ensino integrado.
	
	\vspace{\onelineskip}
	
	\noindent
	\textbf{Fonte de financiamento}: IFRO - PROEN.
	
\end{document}
