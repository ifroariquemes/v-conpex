\documentclass[article,12pt,onesidea,4paper,english,brazil]{abntex2}

\usepackage{lmodern, indentfirst, nomencl, color, graphicx, microtype, lipsum}			
\usepackage[T1]{fontenc}		
\usepackage[utf8]{inputenc}		

\setlrmarginsandblock{2cm}{2cm}{*}
\setulmarginsandblock{2cm}{2cm}{*}
\checkandfixthelayout

\setlength{\parindent}{1.3cm}
\setlength{\parskip}{0.2cm}

\SingleSpacing

\begin{document}
	
	\selectlanguage{brazil}
	
	\frenchspacing 
	
	\begin{center}
		\LARGE PRÁTICA PROFISSIONAL E INTEGRAÇÃO CURRICULAR: UMA REALIDADE
		POSSÍVEL\footnote{Trabalho realizado dentro da área de Conhecimento CNPq/CAPES: Métodos e Técnicas de Ensino.}
		
		\normalsize
	Francyelle Ruana Faria da Silva\footnote{Coordenadora, francyelle.faria@ifro.edu.br. Campus Ariquemes.} 
	Quézia da Silva Rosa\footnote{Colaboradora, quezia.rosa@ifro.edu.br, Campus Ariquemes.} \\
	Mirian de Oliveira Bertotti\footnote{Colaboradora, mirian.bertotti@ifro.edu.br, Campus Ariquemes.} 
	Vagson Ferreira Cação\footnote{Colaborador, vagson.caçao@ifro.edu.br, Campus Ariquemes.} 
	\end{center}
	
	\noindent O processo de ensino aprendizagem requer metodologias que permitam a
	superação da reprodução para produção do conhecimento, se torna relevante então,
	a integração curricular. No ensino técnico, essa necessidade é mais latente, pois o
	educando precisa compreender de que modo as disciplinas se relacionam e
	culminando no conhecimento necessário para o mundo do trabalho. Com o ensino
	técnico integrado ao ensino médio, uma lacuna se abriu entre as disciplinas do
	núcleo comum e do núcleo técnico, pois é comum a visão de que um forma o
	cidadão e o outro forma o trabalhador. Por acreditar que isso é uma falácia, este
	projeto objetivou integrar as disciplinas de Língua Portuguesa, Matemática,
	Produção Animal e Orientação para Prática Profissional e Pesquisa a fim de
	demonstrar a aplicabilidade de um conteúdo para além dos limites das disciplinas e
	demonstrar que é possível formar o cidadão e o trabalhador utilizando para essa
	integração a execução de um projeto de criação de minhocas e produção de húmus.
	O projeto foi realizado com as três turmas de primeiro ano do Curso Técnico em
	Agropecuária e foi constituído por quatro etapas: 1) aula interdisciplinar com todos
	os professores apresentando o projeto e como cada disciplina contribuiria para a
	construção do conhecimento; 2) Instruções às turmas sobre a execução do
	minhocário e a produção de húmus; 3) Apresentação do Relatório de Desempenho
	Prático a ser utilizado para apresentar os resultados; e 4) Correção do relatório e
	apresentação oral dos resultados obtidos. A participação de cada disciplina ficou
	assim estabelecida: Português, correção dos aspectos gramaticais do relatório;
	Matemática, cálculo da quantidade de minhocas em razão da área utilizada;
	Produção Animal, orientação sobre a criação de minhocas e produção de húmus; e
	OPPP elaboração do roteiro e das normas metodológicas a serem adotados no
	relatório. O resultado foi a apropriação dos conteúdos, gerando conhecimento que
	proporcionou a produção húmus a partir da criação das minhocas permitindo,
	inclusive, a comercialização do produto. Além disso, foram capazes de elaborarem
	um relatório técnico-científico atendendo os critérios estabelecidos. Deste modo
	percebeu-se que a integração curricular é possível e benéfica à docentes e
	discentes.
	
	\vspace{\onelineskip}
	
	\noindent
	\textbf{Palavras-chave}: Interdisciplinaridade. Ensino Técnico. Minhocário.
	
\end{document}
