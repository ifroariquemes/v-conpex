\documentclass[article,12pt,onesidea,4paper,english,brazil]{abntex2}

\usepackage{lmodern, indentfirst, nomencl, color, graphicx, microtype, lipsum}			
\usepackage[T1]{fontenc}		
\usepackage[utf8]{inputenc}		

\setlrmarginsandblock{2cm}{2cm}{*}
\setulmarginsandblock{2cm}{2cm}{*}
\checkandfixthelayout

\setlength{\parindent}{1.3cm}
\setlength{\parskip}{0.2cm}

\SingleSpacing

\begin{document}
	
	\selectlanguage{brazil}
	
	\frenchspacing 
	
	\begin{center}
		\LARGE MASTER IFRO:COMIDA E CULTURA HISPÂNICAS NO CAMPUS CACOAL\footnote{Trabalho realizado dentro da área da Educação, com financiamento do IFRO.}
		
		\normalsize
		Amanda da Silva Prado\footnote{Bolsista Amanda da Silva Prado, amandaprado0320@gmail.com, Campus Cacoal.} 
	Regis Marlon Santos da Silva\footnote{Bolsista Regis Marlon Santos da Silva, marlonsantos69816@gmail.com, Campus Cacoal.} 
	Iramaia Grespan Ferreira\footnote{Colaboradora Iramaia Grespan Ferreira, iramaia.grespan@ifro.edu.br, Campus Cacoal.} 
	Andreia dos Santos Oliveira\footnote{Colaboradora Andreia dos Santos Oliveira, andreia.oliveira@ifro.edu.br, Campus Cacoal.} 
	Shelly Braum\footnote{Orientadora, Shelly Braum, shelly.braum@ifro.edu.br, Campus Cacoal .} 
	\end{center}
	
	\noindent O presente trabalho visa promover, através do aspecto cultural comida, conhecimento acerca das
	culturas hispânicas e seus hábitos alimentares. Isso se dá uma vez que o espanhol é
	predominantemente nosso idioma vizinho e o nosso desconhecimento quanto aos seus costumes é
	notório. Esta lacuna busca ser sanada neste projeto à medida que os discentes dos segundos anos
	dos três cursos de Ensino Médio Técnico Integrado do campus Cacoal (Agroecologia, Agropecuária e
	Informática) promovem uma manhã de degustação de comidas e bebidas típicas dos vinte e um
	países que tem o espanhol como idioma oficial, produzem vídeos para evidenciar o uso da linguagem
	oral e elaboram textos e apresentações sobre as características de cada país e seus costumes.
	Também deverão decorar de acordo com os costumes do país representado, englobando mais um
	aspecto cultural. Logo, mostra-se necessário a abordagem do projeto, como mecanismo para
	envolver a comunidade com a língua de forma dinâmica. O Multiculturalismo pauta este projeto, já
	que visa a integração e a coexistência de diversas culturas de forma respeitosa, mas ao mesmo
	tempo crítica (SILVA, 1999). Quanto às concepções de língua estrangeira, linguagem e seus usos,
	baseamo-nos na Abordagem Comunicativa, uma vez que seu foco é no sentido, no significado e na
	interação proposital entre os indivíduos na língua estrangeira. Como alguns passos do projeto já
	foram dados, os estudantes já puderam conhecer um pouco mais dos vinte e um países, já
	escolheram as receitas que serão produzidas e estão em fase de preparação da decoração. Essas
	etapas proporcionam não só aos acadêmicos como também aos servidores do instituto, maior
	interação com a língua e a cultura de países hispano falantes. Portanto, o mesmo se torna
	imprescindível para o bom desenvolvimento acadêmico e para capacitação de profissionais para o
	mercado de trabalho, sendo também instrumento chave de interdisciplinaridade.
	
	\vspace{\onelineskip}
	
	\noindent
	\textbf{Palavras-chave}: Modelo. Resumo. Conpex.
	
		\vspace{\onelineskip}
	
	\noindent
	\textbf{Fonte de financiamento}: IFRO.
	
\end{document}
