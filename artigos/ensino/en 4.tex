\documentclass[article,12pt,onesidea,4paper,english,brazil]{abntex2}

\usepackage{lmodern, indentfirst, nomencl, color, graphicx, microtype, lipsum}			
\usepackage[T1]{fontenc}		
\usepackage[utf8]{inputenc}		

\setlrmarginsandblock{2cm}{2cm}{*}
\setulmarginsandblock{2cm}{2cm}{*}
\checkandfixthelayout

\setlength{\parindent}{1.3cm}
\setlength{\parskip}{0.2cm}

\SingleSpacing

\begin{document}
	
	\selectlanguage{brazil}
	
	\frenchspacing 
	
	\begin{center}
		\LARGE ATIVIDADES INTERDISCIPLINARES NO CURSO TÉCNICO EM AGROPECUÁRIA
		INTEGRADO AO ENSINO MÉDIO\footnote{1Trabalho realizado dentro da (área de Conhecimento CNPq: 7.08.04.02-8 Métodos e Técnicas de
			Ensino).}
		
		\normalsize
	Rafaela Freitas da Silva\footnote{2Discente do Curso Técnico em Agropecuária. Bolsista CNPq, Modalidade PIBIC – IFRO Campus
		Colorado do Oeste. e-mail: rafaelafreitasdasilva15@gmail.com.} 
		João Paulo da Silva Cavasani\footnote{3Discente do Curso Técnico em Agropecuária. Bolsista CNPq, Modalidade projeto de pesquisa - IFRO Campus Colorado do Oeste. e-mail: joaopauloifro@gmail.com.} 
	Marcia Jovani de Oliveira Nunes\footnote{4Supervisora Pedagógica do IFRO Campus Colorado do Oeste. Orientadora de projeto de pesquisa IFRO Campus Colorado do Oeste. e-mail: marcia.nunes@ifro.edu.br.} 
	Maria Aparecida Costa Oliveira\footnote{5Supervisora Pedagógica do IFRO Campus Colorado do Oeste. Orientadora de projeto de pesquisa
		IFRO Campus Colorado do Oeste. e-mail: maria.oliveira@ifro.edu.br.} 
	\end{center}
	
	\noindent Quando o educador se dispõe trabalhar com interdisciplinaridade busca uma nova
	postura frente ao processo de ensino e aprendizagem, uma mudança de atitude que
	visa a um conhecimento global e amplo. Desta forma as atividades interdisciplinares
	promovem possibilidades para formação de um sujeito integral, pois incentiva
	docente e alunos a uma postura diferenciada e ao aperfeiçoamento em outras áreas
	de conhecimento. Este trabalho resulta do projeto de pesquisa intitulado “Ações
	Integradas, e suas contribuições para melhoria na qualidade do processo de ensino
	e aprendizagem no Campus Colorado do Oeste”. A pesquisa voltada às atividades
	interdisciplinares objetiva, de acordo com FAZENDA (2002) olhar com indagação os
	docentes em suas atividades de planejamento do ensino, para colaborar através de
	orientações e sugestões com as práticas que favoreçam o êxito da aprendizagem.
	Apresentamos o resultado das ações interdisciplinares desenvolvidas em Língua
	Portuguesa, História e Arte, com os alunos dos 2º anos do curso Técnico em
	Agropecuária. Para análise, utilizamos a observação do planejamento das aulas, das
	atividades em sala e das apresentações de explanação oral, identificamos
	características interdisciplinares nas produções dos alunos como: imagens, folders,
	vídeo, teatro e debate. Finalizamos com a aplicação de questionário às turmas
	envolvidas. Em Língua Portuguesa, foram realizadas leituras dos clássicos da
	literatura brasileira. Arte participou das produções, apresentações, confecção de
	materiais e teatro. A História localizou o período literário das obras em linha
	cronológica. As apresentações artísticas ocorreram no auditório, com exposição de
	paródias musicais. A investigação destas atividades demonstrou que professores e
	alunos conseguiram visualizar que a articulação do saber entre as disciplinas
	favorece tanto o processo de ensino quanto a compreensão dos conteúdos devido a
	contextualização que permite a vivência global de uma realidade a partir das
	experiências cotidianas do aluno, do professor e da sociedade.
	
	\vspace{\onelineskip}
	
	\noindent
	\textbf{Palavras-chave}: Práticas Pedagógicas. Projeto. Interdisciplinaridade.
	
\end{document}
