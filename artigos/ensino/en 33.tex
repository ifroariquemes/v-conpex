\documentclass[article,12pt,onesidea,4paper,english,brazil]{abntex2}

\usepackage{lmodern, indentfirst, nomencl, color, graphicx, microtype, lipsum}			
\usepackage[T1]{fontenc}		
\usepackage[utf8]{inputenc}		

\setlrmarginsandblock{2cm}{2cm}{*}
\setulmarginsandblock{2cm}{2cm}{*}
\checkandfixthelayout

\setlength{\parindent}{1.3cm}
\setlength{\parskip}{0.2cm}

\SingleSpacing

\begin{document}
	
	\selectlanguage{brazil}
	
	\frenchspacing 
	
	\begin{center}
		\LARGE PROJETO DE ENSINO - JORNAL NA ESCOLA (IF-IN:FORMA)\footnote{Projeto realizado dentro da (área de linguagens).}
		
		\normalsize
		Andressa Castro Priori de Souza\footnote{Coordenador (a), andressa.priori@ifro.edu.br, Campus Ariquemes.} 
	Andressa Evellyn de Freitas\footnote{Colaborador(a), andressaevellynnn@gmail.com, Campus Ariquemes.}\\ 
	Gabriele Almeida Lopes\footnote{Colaborador(a), gabialmeidalopes@outlook.com, Campus Ariquemes.} 
	Calebe de Paula Costa\footnote{Colaborador(a), caleps12@hotmail.com, Campus Ariquemes.} 
	\end{center}
	
	\noindent O conhecimento e entendimento dos gêneros textuais e tipologias é de suma
	importância para a formação do aluno/cidadão crítico e preparado para saber fazer o
	uso de sua língua materna nas mais variadas esferas e situações, uma vez que esse
	conhecimento lhe permitirá compreender que são os grupos sociais que criam esses
	gêneros e os utilizam com a finalidade de persuadir, informar, entreter, agir sobre o
	interlocutor, entre outros. Entretanto, os nossos alunos chegam com sérias
	deficiências no quesito leitura, interpretação e, consequentemente, produção textual.
	O projeto em questão surgiu da necessidade de contribuir com o desenvolvimento
	de habilidades de leitura e escrita dos alunos dos cursos técnicos em informática,
	alimentos e agropecuária do campus Ariquemes, que possuem dificuldades com o
	uso da língua materna. O objetivo principal desta ação é trabalhar em sala, durante
	as aulas de língua portuguesa, a leitura e a produção textual dos mais diversos
	gêneros. Sempre, ao final de cada bimestre, o professor de língua portuguesa de
	cada turma, aponta a melhor produção textual que será publicada no jornal da
	escola. Além dos textos selecionados, o jornal também tem a função de divulgar
	diversos assuntos de interesse da comunidade escolar, tais como: eventos
	socioculturais, homenagens a alunos que vem se destacando nos simulados,
	vestibulares e ENEM, dicas de estudos, produções poéticas, informes sobre
	eventos, entre outros. Os jornais tem periodicidade mensal, sendo feita uma tiragem
	de cinquenta exemplares todo final de mês. O projeto ainda está em andamento,
	mas já se pode notar resultados positivos dessa prática.
	
	\vspace{\onelineskip}
	
	\noindent
	\textbf{Palavras-chave}: Projeto de Ensino. Leitura. Escrita.
	
\end{document}
