\documentclass[article,12pt,onesidea,4paper,english,brazil]{abntex2}

\usepackage{lmodern, indentfirst, nomencl, color, graphicx, microtype, lipsum}			
\usepackage[T1]{fontenc}		
\usepackage[utf8]{inputenc}		

\setlrmarginsandblock{2cm}{2cm}{*}
\setulmarginsandblock{2cm}{2cm}{*}
\checkandfixthelayout

\setlength{\parindent}{1.3cm}
\setlength{\parskip}{0.2cm}

\SingleSpacing

\begin{document}
	
	\selectlanguage{brazil}
	
	\frenchspacing 
	
	\begin{center}
		\LARGE LEITURA E ESCRITA:\\UM DESAFIO A SER EXPLORADO\footnote{Trabalho realizado dentro da área Linguística, Letras e Artes.}
		
		\normalsize
	Luana dos Anjos\footnote{Aluna do primeiro ano do ensino médio integrado ao técnico em agropecuária,luana.seringueiras2017@gmail.com, Campus Colorado do Oeste.} 
	Maria Apª. Costa Oliveira\footnote{Colaborador(a), maria.oliveira@ifro.com.br, Campus Colorado do Oeste.} 
	Jesus Rodrigues da Penha\footnote{Orientador, prof-jesus@hotmail.com, Campus Colorado do Oeste.} 
	Mauro Sérgio Demício\footnote{Co-orientador, mauro.demicio@ifro.com.br, Campus Colorado do Oeste.} 
	\end{center}
	
	\noindent A literatura nas escolas deve despertar o gosto pela leitura e escrita, proporcionando
	fruição, alegria e encanto. Além disso, ela pode desenvolver a imaginação, os
	sentimentos, a emoção, a expressão e o movimento através de uma aprendizagem
	prazerosa (SAWULSKI, 2002). Nesse aspecto, o projeto visa buscar resultados
	significantes na superação das dificuldades encontradas na aprendizagem,
	estimulando o prazer pelo ato de ler e escrever. Considera ainda a
	interdisciplinaridade e a atuação de todos os alunos da turma nesse processo ao
	trabalhar com gênero literário que possibilita ao aluno a aquisição de competências
	leitoras, desenvolvendo análise literária, leitura e produção textual. O
	desenvolvimento do trabalho se deu por meio da apresentação do projeto aos
	professores, para articulação de ideias e ações, seguido de atividades aos alunos do
	1º ano do ensino médio integrado ao curso técnico em agropecuária do Instituto
	Federal de Rondônia, Campus Colorado do Oeste, resultando nos textos que cada
	discente apresentou no primeiro semestre. De cada leitura realizada, o aluno
	produziu uma análise literária e a digitou em sua pasta individual. Na última semana
	de cada bimestre os alunos apresentaram oralmente o livro que leram, relatando as
	características literárias deste. Ao final da execução do projeto, o professor
	orientador escolherá as produções do aluno que melhor demonstrou desempenho
	nas análises e, numa segunda etapa, enviará os textos deste aluno para a
	confecção do “Livro Escritor 2017”, que será a premiação a ser oferecida pela
	instituição, juntamente com certificado de participação no projeto. A participação dos
	alunos está se dando de forma voluntária, tendo, durante o primeiro semestre de
	2017, contado com a adesão de 105 alunos, o que demonstra boa aceitação. O
	projeto tem contribuído na formação do perfil de um leitor mais ativo e crítico, fator
	importante nesta etapa da escolarização básica, possibilitando habilidades de leitura
	e escrita com ênfase na reflexão crítica, sem desprezar o prazer possibilitado pela
	leitura.
	
	\vspace{\onelineskip}
	
	\noindent
	\textbf{Palavras-chave}: Leitura. Escrita. Análise.
	
\end{document}
