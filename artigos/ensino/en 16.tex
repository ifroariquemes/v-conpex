\documentclass[article,12pt,onesidea,4paper,english,brazil]{abntex2}

\usepackage{lmodern, indentfirst, nomencl, color, graphicx, microtype, lipsum}			
\usepackage[T1]{fontenc}		
\usepackage[utf8]{inputenc}		

\setlrmarginsandblock{2cm}{2cm}{*}
\setulmarginsandblock{2cm}{2cm}{*}
\checkandfixthelayout

\setlength{\parindent}{1.3cm}
\setlength{\parskip}{0.2cm}

\SingleSpacing

\begin{document}
	
	\selectlanguage{brazil}
	
	\frenchspacing 
	
	\begin{center}
		\LARGE KANBAN:FERRAMENTA DE ORGANIZAÇÃO ESTUDANTIL\footnote{Trabalho realizado dentro da (área de Conhecimento CNPq: Ciência Comunicação) com financiamento do (IFRO – Campus Ji-Paraná).}
		
		\normalsize
	Vicente Leonardo Cordeiro Kohler\footnote{Cursando o 4º ano do curso técnico em informática – Campus Ji-Paraná Instituto Federal de
		Educação Ciência e Tecnologia de Rondônia – IFRO, vicleonardock@gmail.com, Ji-Paraná.} 
	Reinaldo Lima Pereira\footnote{Professor - Campus Ji-Paraná Instituto Federal de Educação Ciência e Tecnologia de Rondônia –
		IFRO, reinaldo.pereira@ifro.edu.br, Ji-Paraná.} 
	\end{center}
	
	\noindent O acumulo de atividades escolares está cada vez mais presente no dia a dia,
	atualmente é algo presente na vida de todo estudante. Faz-se necessário, pois, a
	organização estudantil, que é de grande importância, visto que sua não utilização
	resulta em falta de tempo para conclusão atividades e até mesmo na perda de
	compromissos, o que pode levar ao fracasso escolar, seja no ensino médio,
	graduação, mestrado ou outras atividades que estejam relacionados a estudos.
	Nesse viés, pode-se se utilizar os princípios do quadro Kanban para desenvolver
	uma possível forma de solver o problema. Visto isso, a pesquisa tem como objetivo
	geral desenvolver um software que utilize os princípios do método Kanban,
	intentando prover uma solução para a organização de atividades estudantis por meio
	da utilização da tecnologia. Para o desenvolvimento do software na realização da
	pesquisa científica foi utilizado como metodologia o desenvolvimento gradual do
	sistema, desenvolvendo uma funcionalidade a cada vez que outra era finalizada; e
	sempre desenvolvendo uma que dependesse de outra funcionalidade ainda não
	concluída. A linguagem utilizada para o desenvolvimento foi o Java e a ferramenta
	utilizada foi o Eclipse Mars. O software desenvolvido atendeu aos objetivos iniciais,
	possibilitando organizar atividades estudantis, facilitando sua visualização e sua
	modificação, tal como edição ou conclusão de atividade. As funcionalidades gerais
	implementadas foram de visualização de atividades por tabela e calendário,
	gerenciamento de atividades, que abrange cadastro, edição e conclusão,
	gerenciamento de professores e matérias, possibilitando cadastro e edição ou
	exclusão, gerenciamento de atividades adicionadas à pendentes, organização de
	atividades concluídas. Por meio do software foi possível organizar atividades do dia
	a dia escolar, ocasionando a facilitação da vida estudantil.
	
	\vspace{\onelineskip}
	
	\noindent
	\textbf{Palavras-chave}: Kaban. Organização. Software.
	
\end{document}
