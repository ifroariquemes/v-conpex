\documentclass[article,12pt,onesidea,4paper,english,brazil]{abntex2}

\usepackage{lmodern, indentfirst, nomencl, color, graphicx, microtype, lipsum}			
\usepackage[T1]{fontenc}		
\usepackage[utf8]{inputenc}		

\setlrmarginsandblock{2cm}{2cm}{*}
\setulmarginsandblock{2cm}{2cm}{*}
\checkandfixthelayout

\setlength{\parindent}{1.3cm}
\setlength{\parskip}{0.2cm}

\SingleSpacing

\begin{document}
	
	\selectlanguage{brazil}
	
	\frenchspacing 
	
	\begin{center}
		\LARGE EMPREENDENDEDORISMO PARA O EMPREGADO OU PARA O EMPREGADOR?\footnote{Trabalho realizado dentro da área de Conhecimento CNPq/CAPES: Métodos e Técnicas de Ensino.}
		
		\normalsize
		Quezia da Silva Rosa\footnote{Coordenadora, quezia.rosa@ifro.edu.br. Campus Ariquemes.} 
	Mirian de Oliveira Bertotti\footnote{Colaboradora, mirian.bertotti@ifro.edu.br, Campus Ariquemes.} 
		Bruno Andrade Felipe Silva\footnote{Colaborador, bruninho22@gmail.com, Campus Ariquemes.} 
	\end{center}
	
	\noindent Ser empreendedor é ter habilidade de identificar oportunidades e saber aproveitá-las
	criando negócios capazes de gerar empregos e renda. Quando trabalhado de
	acordo com a ótica do empregado, denomina-se empreendedorismo interno, já na
	ótica do empregador, é empreendedorismo individual. Ao ser abordado no ensino
	técnico, surge uma questão: formar para ser empregado ou formar para empregar?
	Ao mesmo tempo em que se busca capacitar para que o aluno saia apto para se
	empregar no eixo de formação, a disciplina busca capacitá-lo para se tornar um
	empreendedor e posteriormente se tornar um empregador. A dificuldade de trabalhar
	a disciplina no ensino técnico é aliar a formação que ele deve ter como empregado à
	capacidade de aproveitar as oportunidades e tornar-se um empreendedor. No curso
	técnico em Aquicultura essa problemática foi trabalhada, aliando o ensino teórico à
	prática. A metodologia foi adotada no 1º bimestre do 2º período do curso utilizando
	um total de 20 aulas e foi dividida em 3 etapas. 1) Exposição do conteúdo: 8 aulas
	sobre o tema empreendedorismo, incluindo plano de negócio; 2) Elaboração do
	Plano de Negócio: 6 aulas com acompanhamento e orientação do professor às
	equipes; e 3) Apresentação: 6 aulas de exposição das ideias numa simulação de
	captação de recurso. No total, cinco negócios foram pensados e apresentados,
	todos relacionados à área de atuação do curso, versando sobre o ramo alimentício,
	produção e serviços. Foi possível através da disciplina, despertar no aluno, a
	percepção de que existem alternativas que vão além da empregabilidade. Todo o
	conhecimento adquirido no curso pode ser utilizado para que os alunos passem de
	possíveis empregados à possíveis empregadores.
	
	\vspace{\onelineskip}
	
	\noindent
	\textbf{Palavras-chave}: Ensino. Plano de Negócios. Técnico em Aquicultura.
	
\end{document}
