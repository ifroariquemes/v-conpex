\documentclass[article,12pt,onesidea,4paper,english,brazil]{abntex2}

\usepackage{lmodern, indentfirst, nomencl, color, graphicx, microtype, lipsum,textcomp}			
\usepackage[T1]{fontenc}		
\usepackage[utf8]{inputenc}		

\setlrmarginsandblock{2cm}{2cm}{*}
\setulmarginsandblock{2cm}{2cm}{*}
\checkandfixthelayout

\setlength{\parindent}{1.3cm}
\setlength{\parskip}{0.2cm}

\SingleSpacing

\begin{document}
	
	\selectlanguage{brazil}
	
	\frenchspacing 
	
	\begin{center}
		\LARGE PROJETO INTEGRADOR: CIÊNCIAS HUMANAS E SUAS TECNOLOGIAS\footnote{Trabalho realizado dentro da área de conhecimento de Ciências Humanas.}
		
		\normalsize
	Adriano Lopes Saraiva\footnote{Professor EBTT, adirano.saraiva@ifro.edu.br, IFRO Campus Porto Velho Calama.} 
	Flávio Leite Costa\footnote{Professor EBTT, flavio.leite@ifro.edu.br, IFRO Campus Porto Velho Calama.} 
	Gedeli Ferrazzo\footnote{Professora EBTT, gedeli.ferrazzo@ifro.edu.br, IFRO Campus Porto Velho Calama.} 
	Raimundo José dos Santos Filho\footnote{Professor EBTT, raimundo.santos@ifro.edu.br, IFRO Campus Porto Velho Calama.} 
	\end{center}
	
	\noindent O projeto integrador Ciências Humanas e suas Tecnologias, desenvolvido no
	primeiro semestre de 2017, nas turmas do 1° ano do Curso Técnico em Química
	integrado ao ensino Médio, teve como propósito a integração curricular das
	disciplinas de Filosofia, Sociologia, História e Geografia. Para tanto, procedeu-se
	com o planejamento docente com vistas a definir o componente curricular comum às
	disciplinas do eixo, bem como definir a metodologia e a avaliação conjunta das
	atividades. De acordo com o planejamento, as atividades foram desenvolvidas em
	duas etapas, a primeira referente ao processo de sistematização e estruturação de
	um roteiro e a segunda etapa, a criação de um documentário ou apresentação na
	“Semana de Educação para a Vida”, tendo como objetivo relacionar as mudanças
	ocorridas ao longo do tempo no espaço e na cultura humana, pela ação do homem.
	Durante o desenvolvimento do projeto foram necessárias constantes explanações
	quanto ao caráter integralizador do trabalho, haja vista a dificuldade dos discentes
	em conceber os conteúdos de forma integrada, uma vez que grande parte da
	trajetória escolar desses alunos se estabeleceu pela compreensão dos conteúdos de
	forma fragmentada e estanque. Contudo, no final foi possível garantir êxito na
	proposta de integração. Destaca-se que projeto integrador ciências humanas e suas
	tecnologias, possibilitou aos discentes maior compreensão na relação entre as
	distintas disciplinas e suas manifestações no cotidiano, uma vez que o objeto de
	trabalho não é visto apenas de forma unilateral, mas de forma conjunta, na
	totalidade do fenômeno ou do fato sociocultural. Para os docentes proponentes do
	projeto, foi possível constar que o projeto integrador se coloca como uma estratégia
	para cumprimento das ANP’s e que efetivamente cumpre com o objetivo de
	processo de ensino-aprendizagem.
	
	\vspace{\onelineskip}
	
	\noindent
	\textbf{Palavras-chave}: Projeto Integrador. Ciências Humanas e suas Tecnologias. Curso Técnico em Química integrado ao ensino Médio.
	
\end{document}
