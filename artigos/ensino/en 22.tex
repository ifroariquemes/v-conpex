\documentclass[article,12pt,onesidea,4paper,english,brazil]{abntex2}

\usepackage{lmodern, indentfirst, nomencl, color, graphicx, microtype, lipsum}			
\usepackage[T1]{fontenc}		
\usepackage[utf8]{inputenc}		

\setlrmarginsandblock{2cm}{2cm}{*}
\setulmarginsandblock{2cm}{2cm}{*}
\checkandfixthelayout

\setlength{\parindent}{1.3cm}
\setlength{\parskip}{0.2cm}

\SingleSpacing

\begin{document}
	
	\selectlanguage{brazil}
	
	\frenchspacing 
	
	\begin{center}
		\LARGE MURAL ONLINE\footnote{ Trabalho realizado dentro da área de ciências exatas e da terra: ciência da computação).}
		
		\normalsize
	Géder Gabriel Louback Cunha\footnote{Cursando o 4º ano do curso técnico em informática - Campus Ji-Paraná Instituto Federal de
		Educação Ciência e Tecnologia de Rondônia – IFRO, gedergabriel@gmail.com, Ji-paraná.} 
	Ilma Rodrigues de Souza Fausto\footnote{Professora Orientadora, Docente analista de sistemas - Campus Ji-Paraná Instituto Federal
		de Educação Ciência e Tecnologia de Rondônia – IFRO, Ilma.rodrigues@ifro.edu.br, Ji-paraná.} 
	\end{center}
	
	\noindent O grande uso das tecnologias móbiles, fez com que surgisse uma alta procura por
	aplicações com soluções para o dia a dia. Desse modo, é fato que celulares e
	smartphones estão sempre presentes nas vidas dos alunos. Nesse contexto, o objetivo
	deste trabalho é tratar sobre recursos e as contribuições do aplicativo “Mural Online”
	para disponibilizar o acesso rápido e eficiente as informações do mural do IFRO.
	Inicialmente foi feito uma pesquisa sobre aplicativos que permitisse a criação de jogos
	ou aplicativos mobiles, após um processo de escolha por facilidade de manuseio da
	ferramenta de criação, foi escolhido o site Fábrica de Aplicativos. Pesquisou-se sobre a
	ferramenta e como utilizá-la através de vídeos e sites e a própria ferramenta
	disponibiliza vídeos para se aprende utilizá-la, além de tutoriais interativos sendo de
	fácil aprendizagem não demoro aprende utilizá-la assim adiantando boa parte do
	processo para a criação do app. Foi desenvolvido primeiramente a parte gráfica do
	aplicativo, usando cores e logo referente ao IFRO, pois é um aplicativo destinado ao
	âmbito escolar específico do IFRO - Campus Ji-Paraná. Logo em seguida foi inserido
	informações, tais como eventos e editais, para demonstrar como seria realmente o
	aplicativo. Foi testado e apresentou o funcionamento esperado. Para finalização do
	aplicativo, foi necessário o pagamento da conta no site, pois é uma ferramenta paga. A
	pós o pagamento foi possível baixar o aplicativo. Por meio da tecnologia da informação
	foi possível facilitar o acesso as informações divulgadas pelo IFRO - Campus Ji-paraná.
	Entre outras divulgações de eventos, foi facilitado a divulgação do III Seminário de
	Ensino, Pesquisa e Extensão - 2017 ( SEPEX ), III Feira de Empreendedorismo, III
	Feira de Estágio e Negócios - 2017, III IFRO Profissões - 2017 e II Semana da Química
	-2017. O APP mostra que a divulgação de informação através de papéis tornou-se
	obsoleta no mundo digital grandes que é atualmente. Com isso, o resultado do projeto
	mostra que é possível tornar as divulgações escolares mais rápidas e dinâmicas.
	
	\vspace{\onelineskip}
	
	\noindent
	\textbf{Palavras-chave}: Mural. Online.
	
		\noindent
	\textbf{Fonte de financiamento}: IFRO Campus Ji-Paraná.
	
\end{document}
