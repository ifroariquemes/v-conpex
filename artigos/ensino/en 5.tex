\documentclass[article,12pt,onesidea,4paper,english,brazil]{abntex2}

\usepackage{lmodern, indentfirst, nomencl, color, graphicx, microtype, lipsum}			
\usepackage[T1]{fontenc}		
\usepackage[utf8]{inputenc}		

\setlrmarginsandblock{2cm}{2cm}{*}
\setulmarginsandblock{2cm}{2cm}{*}
\checkandfixthelayout

\setlength{\parindent}{1.3cm}
\setlength{\parskip}{0.2cm}

\SingleSpacing

\begin{document}
	
	\selectlanguage{brazil}
	
	\frenchspacing 
	
	\begin{center}
		\LARGE CONHECENDO O MEU LUGAR E AS MINHAS\\RAÍZES EDUCACIONAIS\footnote{Trabalho realizado dentro da área de conhecimento de linguística, letras e artes sem financiamento.}
		
		\normalsize
		Ingrid Dantas da Silva\footnote{Discente, ingriddantaspvh@hotmail.com, Campus Porto Velho Zona Norte.} 
		Emi Silva de Oliveira\footnote{Orientadora, emi.oliveira@ifro.edu.br, Campus Porto Velho Zona Norte.} 
	Ana Cláudia Dias Ribeiro\footnote{Co-orientadora, ana.ribeiro@ifro.edu.br, Campus Porto Velho Zona Norte.} 
	\end{center}
	
	\noindent O Curso Técnico em Cooperativismo Concomitante ao Ensino Médio do IFRO Campus Zona Norte,
	uma parceria entre o Instituto Federal de Rondônia e o Governo do Estado de Rondônia iniciou-se no
	dia 29/07/2016, e se apresenta como possibilidade formativa evidente e necessária no Estado de
	Rondônia por questões regionais, locais, socioeconômicas e ambientalmente sustentáveis. Conhecer
	a realidade cotidiana educacional é primordial, sendo que um dos maiores desafios dos cursos a
	distância é identificar o cerne de suas raízes educacionais. A produção textual é uma atividade
	essencial para o desenvolvimento intelectual de qualquer cidadão, especialmente se este se encontra
	numa fase escolar em que está prestes a especializar-se profissionalmente. A Antropologia cultural
	procura entender como as sociedades, dos primórdios até hoje e em diferentes regiões do mundo,
	produzem, reproduzem e materializam o saber, isto é, como as diferentes sociedades formam e
	transmitem o seu conhecimento acumulado ao longo dos tempos. Dito de outra forma, como elas
	formam e transmitem a sua memória social (Connerton 1989; Olick 2011). O domínio da expressão
	escrita requer muito esforço e dedicação não só do discente, que precisa realizar a tarefa de escrita e
	reescrita insistentemente, como também da escola, que deve buscar maneiras diversas e
	interessantes que motivem seus alunos a realizá-las. Primeiramente, foi apresentado o gênero
	memórias, visando a motivação, a reflexão e apresentação da importância das raízes culturais de um
	povo e sobre a manutenção da memória. Posteriormente, foi apresentado o projeto aos professores
	presenciais e alunos, explicitando o objetivo de caracterizar os ambientes da comunidade escolar
	como foi, como está sendo e como esperam que seja a educação no futuro. Além de fomentar uma
	reflexão sobre a importância das raízes educacionais, no sentido da afirmação de sua identidade, o
	projeto motivou trinta e cinco escolas dos Municípios de Ouro Preto do Oeste, Guajará-Mirim, Porto
	Velho, Rolim de Moura, Pimenta Bueno, Ji-Paraná e Cerejeiras, a produzir textos por meio da
	investigação dos acontecimentos educacionais de mais de quinhentos alunos com base no contexto
	sociocultural em que estão inseridas, a fim de, por meio de textos e desenhos, fazer o registro de sua
	identidade.
	
	\vspace{\onelineskip}
	
	\noindent
	\textbf{Palavras-chave}: Memórias. Meu lugar. Raízes educacionais.
	
\end{document}
