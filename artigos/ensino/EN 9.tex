\documentclass[article,12pt,onesidea,4paper,english,brazil]{abntex2}

\usepackage{lmodern, indentfirst, nomencl, color, graphicx, microtype, lipsum}			
\usepackage[T1]{fontenc}		
\usepackage[utf8]{inputenc}		

\setlrmarginsandblock{2cm}{2cm}{*}
\setulmarginsandblock{2cm}{2cm}{*}
\checkandfixthelayout

\setlength{\parindent}{1.3cm}
\setlength{\parskip}{0.2cm}

\SingleSpacing

\begin{document}
	
	\selectlanguage{brazil}
	
	\frenchspacing 
	
	\begin{center}
		\LARGE ENSINO APRENDIZAGEM ATRAVÉS DE EXPERIMENTOS CIENTÍFICOS
		CIRCUITOS ELÉTRICOS NA RODA DE LED\footnote{Trabalho realizado dentro da área de Conhecimento (CNPq) –Ensino/ Física sem financiamento.}
		
		\normalsize
		Larissa Kelly Oliveira Cuellar\footnote{Colaboradora, larissaoliveirac29@gmail.com, Campus Guajará-Mirim.} 
		Ana Vitória dos Santos Félix\footnote{Colaboradora, anavfelixs@gmail.com, Campus Guajará-Mirim.} 
	Gabriel Silva Marques\footnote{Colaborador, gabrielsilvamarques631@gmail.com, Campus Guajará-Mirim.} 
		Elcivan dos Santos Silva\footnote{Orientador(a), elcivan.silva@ifro.edu.br, Campus Guajará-Mirim.} 
	\end{center}
	
	\noindent A eletricidade está presente na vida cotidiana do ser humano desde a Grécia antiga,
	quando observou os primeiros fenômenos eletrostáticos. É impossível imaginar
	como seria nossa vida sem a eletricidade. A proposta apresentada nesse trabalho
	surgiu como consequência do projeto integrador “informática para todos” que foi
	desenvolvido no IFRO/Guajará-Mirim em 2016. O objetivo principal deste trabalho foi
	mostrar uma maneira intuitiva e prática de repassar para a comunidade escolar do
	IFRO/Guajará-Mirim, um jeito simples de entender como funciona um circuito
	elétrico. O Experimento foi realizado para facilitar a compreensão do conteúdo da
	disciplina de Eletroeletrônica Básica, do curso Técnico em Manutenção e Suporte
	em Informática Concomitante ao Ensino Médio Vespertino do IFRO/Guajará-Mirim.
	O Experimento foi desenvolvido em algumas etapas. Primeiro fez-se a pesquisa
	bibliográfica, em seguida confeccionou-se o aparato experimental circuito elétrico
	Roda de LED, no qual consistia de uma Roda feita de EVA com papel alumínio, com
	lâmpadas de LEDs postas em sua volta e um suporte feito de canos de PVC. Por
	último, apresentou-se o trabalho na forma de um seminário na sala de aula. A partir
	da observação dos aspectos resultantes do experimento, pôde-se analisar e
	identificar a grande utilidade do tema abordado na vida cotidiana dos alunos e das
	pessoas, assim como sua relevância e importância no que tange o auxílio e o
	desenvolvimento de novas ferramentas metodológicas, como elemento facilitador
	para o ensino-aprendizagem de assuntos, que até então, são considerados de difícil
	entendimento. Este experimento é trivial, efetivo e de baixo custo, mesmo já
	existente e registrado, foi de grande valia para que a turma compreendesse de uma
	forma prática e interativa o conteúdo repassado, além de compor um grande volume
	de informações que estão conectadas diretamente ao cotidiano dos indivíduos,
	tornando-o assim, bastante significativo para o ensino-aprendizagem.
	
	\vspace{\onelineskip}
	
	\noindent
	\textbf{Palavras-chave}: Eletroeletrônica. Circuitos Elétricos. LEDs.
	
\end{document}
