\documentclass[article,12pt,onesidea,4paper,english,brazil]{abntex2}

\usepackage{lmodern, indentfirst, nomencl, color, graphicx, microtype, lipsum,textcomp}			
\usepackage[T1]{fontenc}		
\usepackage[utf8]{inputenc}		

\setlrmarginsandblock{2cm}{2cm}{*}
\setulmarginsandblock{2cm}{2cm}{*}
\checkandfixthelayout

\setlength{\parindent}{1.3cm}
\setlength{\parskip}{0.2cm}

\SingleSpacing

\begin{document}
	
	\selectlanguage{brazil}
	
	\frenchspacing 
	
	\begin{center}
		\LARGE ESTUDO DA CAPILARIDADE EM \\ARGAMASSAS ADITIVADAS\footnote{Trabalho realizado dentro da área 30101018 MATERIAIS E COMPONENTES DE CONSTRUÇÃO
			com financiamento do IFRO (EDITAL Nº 76/2017/PVCAL - CGAB/IFRO, DE 03 DE MAIO DE 2017).}
		
		\normalsize
	Leandro de Souza Carvalho\footnote{Bolsista: aluno do curso técnico em edificações, leandroejm@gmail.com, Campus Porto Velho Calama.} 
	Emílio Gabriel Freire dos Santos\footnote{Colaborador: aluno do curso técnico em edificações, emiliogabriel564@gmail.com, Campus Porto Velho Calama.} \\
	Rafael Alves de Oliveira\footnote{Colaborador: aluno do curso técnico em edificações, rafa22ro@gmail.com, Campus Porto Velho Calama.} 
	Valéria Costa de Oliveira\footnote{Docente Orientador, valeria.oliveira@ifro.edu.br, Campus Porto Velho Calama.} 
	\end{center}
	
	\noindent A argamassa é uma mistura de aglomerantes, agregado miúdo e água, podendo ter
	ainda aditivos químicos e minerais. Os aditivos plastificantes reduzem a quantidade
	de água necessária para produção das argamassas e ajudam no tamponamento dos
	poros. O objetivo desta prática foi o estudo da influência dos aditivos plastificantes
	na redução da permeabilidade das argamassas de revestimentos, utilizando
	materiais de Porto Velho - RO, além do ensino sobre os materiais de construção
	envolvidos e tecnologias construtivas de revestimentos para os alunos do curso
	técnico em edificações. A metodologia adotada foi um programa experimental
	desenvolvido na disciplina de materiais de construção construído com aulas práticas
	no laboratório de Resistência dos materiais do Campus Porto Velho Calama.
	Utilizou-se para as dosagens das argamassas quatro aditivos plastificantes e um
	aditivo impermeabilizante, cimento CP IV e areia lavada de rio fina. As argamassas
	foram produzidas de forma atender um espalhamento inicial de 260±5 mm, medidos
	no equipamento denominado mesa de consistência (Flow Table), assim se
	determinou a demanda de água. Foram produzidas argamassas com três diferentes
	traços e teores de cimento (rico, intermediário e pobre). O teor dos aditivos foram
	determinados pelo percentual definido pelo fabricante dos produtos com base na
	massa de cimento. Aos 28 dias realizou-se os ensaios de absorção por capilaridade
	e coeficiente de capilaridade. Os ensaios de absorção por capilaridade e coeficiente
	de capilaridade foram realizados conforme as prescrições da Norma NBR
	15259:2005 (Argamassa para assentamento e revestimento de paredes e tetos -
	Determinação de absorção de água por capilaridade e do coeficiente de
	capilaridade). Os resultados mostraram que as argamassas produzidas somente
	com cimento e areia obtiveram valores de capilaridade variando de 9,0 g/dm².min1/2
	a 29,33 g/dm².min1/2. O aditivo impermeabilizante representou valores variando de
	5,20 g/dm².min1/2 a 18,20 g/dm².min1/2. Já para as argamassas aditivadas com os
	plastificantes resultaram em 2,40 g/dm².min1/2 a 12,50 g/dm².min1/2. Os traços ricos
	mostraram a influência do teor de cimento para redução da absorção capilar quando
	comparado aos demais traços e os benefícios dos aditivos plastificantes para
	redução da capilaridade.
	
	\vspace{\onelineskip}
	
	\noindent
	\textbf{Palavras-chave}: Capilaridade. Argamassa. Plastificantes.
	
\end{document}
