\documentclass[article,12pt,onesidea,4paper,english,brazil]{abntex2}

\usepackage{lmodern, indentfirst, nomencl, color, graphicx, microtype, lipsum,textcomp}			
\usepackage[T1]{fontenc}		
\usepackage[utf8]{inputenc}		

\setlrmarginsandblock{2cm}{2cm}{*}
\setulmarginsandblock{2cm}{2cm}{*}
\checkandfixthelayout

\setlength{\parindent}{1.3cm}
\setlength{\parskip}{0.2cm}

\SingleSpacing

\begin{document}
	
	\selectlanguage{brazil}
	
	\frenchspacing 
	
	\begin{center}
		\LARGE PRODUÇÃO DE MASSA VERDE E \\COMPRIMENTO DE PANÍCULA DE SORGO
		UTILIZANDO FERTILIZANTE DE VISCERAS DE PESCADO DA AMAZÔNIA
		
		\normalsize
	Fernando Silva Cardoso\footnote{Acadêmico de Agronomia IFRO, fernandoagroifro@gmail.com, Colorado do Oeste.} 
	André Luiz da Silva Baia\footnote{Acadêmico de Agronomia IFRO, andre.baia@outlook.com, Colorado do Oeste.} \\
	Juliana Pereira Mendes\footnote{Acadêmica de Agronomia IFRO, juliana.mendes@gmail.com, Colorado do Oeste.} 
	Rafael Henrique Pereira Reis\footnote{Professor D.Sc do IFRO-Campus Colorado do Oeste, rafael.reis@ifro.edu.br, Colorado do Oeste.} 
	\end{center}
	
	\noindent O sorgo apresenta alto potencial produtivo, tolerância a estresse hídrico, essas
	vantagens permitem seu cultivo em períodos de baixa pluviosidade, trazendo ao
	produtor uma alternativa para produção (grãos e silagem), ou simplesmente para
	cobertura de solo em sistema de plantio direto. No estado de Rondônia a atividade
	de piscicultura é forte, e no processamento do peixe os resíduos não possuem uma
	destinação adequada, o Ferti-peixe® veio como alternativa para minimizar os
	impactos negativos decorrentes desse processamento, apresentando baixo custo, e
	propriedades químicas e biológicas interessantes do ponto de vista agronômico. O
	objetivo deste trabalho foi avaliar o desempenho da cultura utilizando esse
	fertilizante alternativo, seguindo como parâmetro a recomendação do fabricante, o
	estudo ocorreu entre os meses de Agosto e Outubro de 2015 em uma área
	destinada a experimentos do IFRO-Campus de Colorado do Oeste. O arranjo
	experimental foi em blocos casualizados em esquema fatorial, com 4 repetições
	sendo que o fator 1 foi a dose de fertilizante (50\%;100\%; e 200\%), e o fator 2 o tipo
	de aplicação (no plantio e foliar), a aplicação foliar foi feita 30 dias após o plantio. A
	condução do experimento foi feita pela turma do quarto período de agronomia do
	campus sob orientação do professor Rafael H.P. Reis, e os dados obtidos foram
	submetidos a Análise de variância, utlizando o software ASSISTAT versão 7.7 pt
	(2017). As avaliações foram feitas 60 dias após o plantio, realizando a pesagem de
	10 plantas da área útil de cada parcela, e a medição do tamanho da panícula para
	posteriomente avaliar a produtividade de grãos, esses valores foram extrapolados
	para 1 (um) hectare. Através da análise de Variância observou-se a interação entre
	os fatores F1xF2 para o tamanho da panícula, enquanto que para o a produção de
	massa verde não houve interação entre os tratamentos. A aplicação quando feita em
	no plantio apresentou valores melhores, isso pode ser explicado pela ação
	microbiológica do produto no solo, fazendo com que haja uma melhor atuação de
	microorganismos benéficos a cultura. Entre blocos houve diferença significativa a
	nível de 1\% de probabilidade de erro para os dois tratamentos.
	
	\vspace{\onelineskip}
	
	\noindent
	\textbf{Palavras-chave}: Adubação alternativa. Ferti-peixe. Produção.
	
\end{document}
