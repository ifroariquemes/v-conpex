\documentclass[article,12pt,onesidea,4paper,english,brazil]{abntex2}

\usepackage{lmodern, indentfirst, nomencl, color, graphicx, microtype, lipsum}			
\usepackage[T1]{fontenc}		
\usepackage[utf8]{inputenc}		

\setlrmarginsandblock{2cm}{2cm}{*}
\setulmarginsandblock{2cm}{2cm}{*}
\checkandfixthelayout

\setlength{\parindent}{1.3cm}
\setlength{\parskip}{0.2cm}

\SingleSpacing

\begin{document}
	
	\selectlanguage{brazil}
	
	\frenchspacing 
	
	\begin{center}
		\LARGE CONSTRUÇÃO DO CONHECIMENTO EM FÍSICA ATRAVÉS DA
		EXPERIMENTAÇÃO\footnote{Trabalho realizado dentro da área de Conhecimento CNPq: Ciências Exatas e da Terra com
			financiamento do IFRO.}
		
		\normalsize
		Alinne Marques Torres Moreira\footnote{Bolsista (Discente do Curso Técnico em Agropecuária), email alinnemarquestorres16@gmail.com,
			Campus Colorado do Oeste.} 
	Diego Beltrame Orlandin\footnote{Bolsista (Discente do Curso Técnico em Agropecuária), email diegobeltrameorlandin@gmail.com,
		Campus Colorado do Oeste.} \\
		Wanderson Junior Lima do Nascimento\footnote{Bolsista (Discente do Curso Técnico em Agropecuária), email wandersonj2018@gmail.com, Campus Colorado do Oeste.} 
	Marcio Adolfo de Almeida\footnote{Orientador (Docente), email marcio.almeida@ifro.edu.br, Campus Colorado do Oeste.} 
	\end{center}
	
	\noindent O estudo da Física durante anos vem sendo encarado como algo meramente
	teórico, a forma como professores transmitem os conteúdos descrevendo apenas
	fórmulas matemáticas, demonstra não ter nada a ver com a realidade do aluno,
	consequência direta é o mau desempenho em sala de aula, quanto à avaliação da
	aprendizagem. O projeto buscou uma nova metodologia com o objetivo de aproximar
	teoria e prática, proporcionando aos alunos expor seus conhecimentos científicos
	por meio de experimentos relacionando-os aos conceitos de Física, oferecendo ao
	aluno uma ligação entre a disciplina ensinada na escola e a Física observada nos
	fenômenos naturais. O projeto foi desenvolvido com alunos dos 2º Anos do Curso
	Técnico em Agropecuária Integrado ao Ensino Médio do IFRO, Campus Colorado do
	Oeste - Rondônia durante os meses de julho e agosto de 2017. Inicialmente houve a
	escolha dos conteúdos envolvendo Termometria, Dilatação Térmica, Calorimetria,
	Mudanças de estado, Estudo dos Gases e Termodinâmica, divididos em equipes
	com pelo menos quatro integrantes por equipe, por turma, num total de quarenta
	experimentos. Logo depois de realizada a escolha dos conteúdos, os alunos
	prepararam a apresentação de um roteiro didático experimental com alguns
	elementos textuais para serem preenchidos por cada equipe contendo capa, folha
	de rosto, objetivos específicos, introdução, apresentação, materiais utilizados,
	montagem, resultados, conclusão e referenciais. Cada equipe confeccionou pelo
	menos um experimento pertinente ao tema abordado. Depois de tudo pronto as
	equipes apresentaram o roteiro acompanhado do experimento nas suas respectivas
	turmas nas aulas de Física, com apresentações de dois grupos por semana,
	totalizando oito grupos por sala. O encerramento do projeto ocorreu com a
	exposição dos experimentos no Rol do Centro de Convenções do referido Campus,
	envolvendo discentes e servidores. Diante do contexto abordado se percebe a
	capacidade de observar, interagir e entender os fenômenos naturais. Assim, o
	professor, ao mesmo tempo em que ministrar os conteúdos específicos de Física,
	possa desenvolver aulas experimentais valorizando a física prática, se posicionando
	como agente de ligação entre o ensino e a aprendizagem.
	
	\vspace{\onelineskip}
	
	\noindent
	\textbf{Palavras-chave}: Ensino.Experimentação.Física.
	
	\vspace{\onelineskip}
	
	\noindent
	\textbf{Fonte de financiamento}: IFRO - Campus Colorado do Oeste.
	
\end{document}
