\documentclass[article,12pt,onesidea,4paper,english,brazil]{abntex2}

\usepackage{lmodern, indentfirst, nomencl, color, graphicx, microtype, lipsum}			
\usepackage[T1]{fontenc}		
\usepackage[utf8]{inputenc}		

\setlrmarginsandblock{2cm}{2cm}{*}
\setulmarginsandblock{2cm}{2cm}{*}
\checkandfixthelayout

\setlength{\parindent}{1.3cm}
\setlength{\parskip}{0.2cm}

\SingleSpacing

\begin{document}
	
	\selectlanguage{brazil}
	
	\frenchspacing 
	
	\begin{center}
		\LARGE GRUPO VOCAL IFRO GM (MÚSICA/ EDUCAÇÃO MUSICAL)\footnote{Trabalho realizado na área de Artes/Música/Educação Musical sem financiamento.}
		
		\normalsize
	Marcelo Caires Luz\footnote{Professor EBTT 40 horas/DE, área Música, Campus Guajará Mirim, RO. e-mail:marcelo.luz@ifro.edu.br.}  
	\end{center}
	
	\noindent Hoje em dia, no geral, as pessoas desejam conhecer a linguagem musical de forma
	prática e acessível, visto que no passado, as aulas de música eram densas, cheias
	de códigos próprios, símbolos complicados, e pedagogicamente
	descontextualizadas da realidade, o que causava desmotivação no interessado logo
	no início da aprendizagem. Atualmente, educadores musicais realizam este trabalho
	fundamentado nos pedagogos contemporâneos e seus métodos ativos e
	interacionistas, tais como: Willems, Dalcroze, Kodaly, Orff, Suzuki, e outros, além
	dos brasileiros, Sá Pereira, Liddy Chiaffarelli, Villa-Lobos etc.; utilizam-se ainda os
	que propõe uma pedagogia aberta do ouvir e do improvisar, como Schafer e
	Koellreutter. Com isso, a experiência com música torna-se interessante, prazerosa,
	ritualística, sensorial, motora, emocional, ativando os sistemas neuro-auditivo, neurovisual
	e neuro-linguístico, além de promover uma atividade cognitiva intensa,
	colaborando na melhora do raciocínio lógico matemático e na percepção e
	aprendizagem linguística dos discentes. Por outro lado, a música na escola deve
	promover atividades práticas, vivenciais, com escuta participativa, fragmentos da
	dança, canto, com expressão/percussão corporal, o que desenvolve processos
	criativos e experimentais, além da percepção histórica e estética da Arte. O objetivo
	principal desta ação é estabelecer processo de sensibilização, promover a
	socialização e proporcionar uma prática musical dirigida utilizando o canto, a
	expressão cênica e a percussão corporal, além do uso de instrumentos musicais
	quando essenciais à montagem das cenas criadas. Quanto a metodologia utilizada,
	realizam-se dois encontros prático-expositivos de 90 minutos por semana, com
	turmas formadas nos contraturnos dos cursos regulares do campus. O trabalho vem
	sendo realizado a partir de abril em Guajará Mirim, e logo já se conseguiu colher
	resultados musicais estéticos, heterogêneos e plurais. Os alunos se apresentaram
	na certificação realizada do curso de Técnico de Manutenção e Suporte em
	Informática – turma do presencial e do EaD. Além disso, na Semana da Pátria,
	realizou-se um Concerto Cívico em que pôde-se executar os hinos pátrios e canções
	militares relevantes, acompanhados pela banda do 6º Batalhão de Infantaria de
	Selva. Nesta mostra, conseguimos reunir dentro da Catedral da cidade, 30 soldados
	que também foram preparados pelo mesmo processo, além dos 45 alunos já
	participantes do projeto no campus.
	
	\vspace{\onelineskip}
	
	\noindent
	\textbf{Palavras-chave}: Grupo Vocal. Música. Educação Musical.
	
\end{document}
