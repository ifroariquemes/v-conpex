\documentclass[article,12pt,onesidea,4paper,english,brazil]{abntex2}

\usepackage{lmodern, indentfirst, nomencl, color, graphicx, microtype, lipsum,textcomp}			
\usepackage[T1]{fontenc}		
\usepackage[utf8]{inputenc}		

\setlrmarginsandblock{2cm}{2cm}{*}
\setulmarginsandblock{2cm}{2cm}{*}
\checkandfixthelayout

\setlength{\parindent}{1.3cm}
\setlength{\parskip}{0.2cm}

\SingleSpacing

\begin{document}
	
	\selectlanguage{brazil}
	
	\frenchspacing 
	
	\begin{center}
		\LARGE PROJETO DE SOFTWARE INSTITUCIONAL NA DISCIPLINA DE
		DESENVOLVIMENTO DE SISTEMAS\footnote{Trabalho realizado dentro da área de Conhecimento CNPq: Engenharia de Software.}
		
		\normalsize
	Natanael Augusto Viana Simões\footnote{Orientador(a), natanael.simoes@ifro.edu.br, Campus Ariquemes.} 
	Alessandra Rodrigues Moreira\footnote{Colaboradora, alessandramoreirarodrigues@gmail.com, Campus Ariquemes.} 
	Ana Paula Andrade Guerreiro\footnote{Colaboradora, anapaulaguerreiroa@hotmail.com, Campus Ariquemes.} \\
	Jeferson Loose Benevitz\footnote{Colaborador, jeferson-loose@hotmail.com, Campus Ariquemes.} 
    Michael Jackson Pereira dos Santos\footnote{Colaborador, michael2fera@gmail.com, Campus Ariquemes.} 
	Rafael Sinigaglia Paiva\footnote{Colaborador, mocolirio@gmail.com, Campus Ariquemes.}
	\end{center}
	
	\noindent O ensino do Desenvolvimento de Sistemas tem sido um grande desafio em cursos
	relacionados a computação em todo o mundo, devido as atividades desenvolvidas
	nesta disciplina demandar a junção do conhecimento de diversas áreas da
	informática, tais como banco de dados, programação, engenharia de software e
	outras. Esta disciplina está sendo ofertada para a turma do 4° ano de Técnico em
	Informática, onde as ações exitosas foram executadas. A fim de permitir uma
	experiência prática que pudesse apoiar o processo de ensino, elegemos um
	problema do Campus Ariquemes. Passamos a trabalhar na solução computacional
	que irá resolver um problema real. Foi escolhido o Departamento de Extensão
	(DEPEX) como stakeholder tendo como cenário o acompanhamento de egressos.
	Não há um controle efetivo nem comunicação satisfatória com aqueles que
	concluíram seus cursos na Instituição devido a dificuldade em manter os dados
	atualizados. Nosso objetivo era projetar e desenvolver um programa de computador
	que permitisse não apenas a fácil atualização da vida egressa dos alunos, mas
	também que empresas parceiras pudessem buscar a qualquer tempo egressos com
	um perfil específico para oferecer trabalho. Durante o primeiro e segundo bimestres
	foram apresentados aos alunos os conceitos das práticas envolvidas no
	desenvolvimento de sistemas. Neste período quatro sistemas fictícios foram
	trabalhados onde estes elaboraram, para cada solução, as partes de um Documento
	de Especificação de Requisitos de Software (DERS). No terceiro bimestre foram
	levados até o DEPEX e CRA para realizarem o levantamento de requisitos e
	posteriormente documentaram os requisitos funcionais e não-funcionais, Diagrama
	de Casos de Uso e sua descrição formal. Para o quarto bimestre, os alunos
	codificarão o software de acordo com as especificações do documento elaborado
	abordando assuntos como Engenharia de Usabilidade, Versionamento,
	Metodologias de Desenvolvimento, Design Patterns e outros assuntos em conjunto
	com a prática.
	
	\vspace{\onelineskip}
	
	\noindent
	\textbf{Palavras-chave}:Software institucional. Projeto de Software. Desenvolvimento de Software.
	
\end{document}
