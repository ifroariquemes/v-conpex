\documentclass[article,12pt,onesidea,4paper,english,brazil]{abntex2}

\usepackage{lmodern, indentfirst, nomencl, color, graphicx, microtype, lipsum}			
\usepackage[T1]{fontenc}		
\usepackage[utf8]{inputenc}		

\setlrmarginsandblock{2cm}{2cm}{*}
\setulmarginsandblock{2cm}{2cm}{*}
\checkandfixthelayout

\setlength{\parindent}{1.3cm}
\setlength{\parskip}{0.2cm}

\SingleSpacing

\begin{document}
	
	\selectlanguage{brazil}
	
	\frenchspacing 
	
	\begin{center}
		\LARGE OBMEP: EM BUSCA DO OURO – 1ª FASE\footnote{Trabalho realizado dentro da 70804001–Educação/Ensino-Aprendizagem.}
		
		\normalsize
	Eder Regiolli Dias\footnote{Coordenador, eder.dias@ifro.edu.br, Campus Cacoal.} 
	Claudemir Miranda Barboza\footnote{Colaborador, claudemir.barbosa@ifro.edu.br, Campus Cacoal.} 
	Maily Marques Pereira\footnote{Colaboradora, maily.pereira@ifro.edu.br, Campus Cacoal.} 
	Jorge da Silva Werneck\footnote{Colaborador, jorge.werneck@ifro.edu.br, Campus Cacoal.} 
	Josirene Zalenski de Siqueira Carvalho\footnote{Colaboradora, josirene.carvalho@ifro.edu.br, Campus Cacoal.} 
	Samanta Margarida Milani\footnote{Colaboradora, samanta.margarida@ifro.edu.br, Campus Cacoal.} 
	Tiago Eutíquio Lemes Santana\footnote{Colaborador, tiagoeutiquio@gmail.com, Campus Cacoal.} 
	\end{center}
	
	\noindent A Olimpíada Brasileira de Matemática das Escolas Públicas (OBMEP) se destaca na
	mídia norteando várias políticas educacionais. Estudantes que tem bom
	aproveitamento nesta competição podem participar de programas como o de
	Iniciação Científica Junior (PIC), que é realizado por meio de uma rede nacional de
	professores em polos espalhados pelo país, um dos objetivos é despertar nos
	estudantes o gosto pela matemática e pela ciência em geral motivando-os na
	escolha profissional pelas carreiras científicas e tecnológicas. No IFRO-Campus
	Cacoal, desde sua criação em 2009, ainda não teve estudantes premiados com
	medalhas, somente com menção honrosa totalizando 12 certificados, com isso, o
	projeto buscou, além de reforçar o ensino da matemática através da resolução de
	problemas, também mudar o contexto atual do quadro de premiação nessa
	olimpíada, inserindo o Campus no cenário nacional, fazendo com que os estudantes
	tragam as primeiras medalhas, e com isso, possivelmente a premiação de
	professores. Ao executar o projeto verificamos um aumento da quantidade de
	estudantes que buscam aprimorar o conhecimento em matemática comparado ao
	ano anterior. Dentre todos os estudantes de ensino médio do campus,
	aproximadamente 20\% deles receberam certificado de participação no projeto
	conforme frequência nas aulas. Percebemos também que com o incentivo de ser
	premiado na 1ª fase com medalhas personalizadas muitos estudantes se dedicaram
	mais ao estudo. Utilizamos a metodologia de resolução de problemas, apresentando
	problemas de provas da OBMEP de anos anteriores, alguns retiramos dos Bancos
	de Questões, a fim de aumentar o rendimento na prova e a quantidade de
	estudantes premiados. Além das aulas presenciais, disponibilizamos semanalmente
	simulados on-line através do Ambiente Virtual de Aprendizagem (AVA). Criamos um
	mural do projeto, para divulgação de informações e fizemos uma contagem
	regressiva atualizando dia-a-dia até a data da prova, sendo o resultado divulgado
	uma semana após a realização da 1ª fase. O projeto alcançou um resultado
	satisfatório, pôde-se perceber uma melhora significativa dos estudantes
	participantes ao término da 1ª etapa do projeto, pois os mesmos, tiveram melhora
	nas notas bimestrais devido as aulas oferecidas com a metodologia usada. Nesse
	ano, 31 estudantes se classificaram para a 2ª fase da OBMEP.
	
	\vspace{\onelineskip}
	
	\noindent
	\textbf{Palavras-chave}: OBMEP. AVA. Olimpíada. CONPEX.
	
\end{document}
