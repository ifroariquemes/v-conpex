\documentclass[article,12pt,onesidea,4paper,english,brazil]{abntex2}

\usepackage{lmodern, indentfirst, nomencl, color, graphicx, microtype, lipsum}			
\usepackage[T1]{fontenc}		
\usepackage[utf8]{inputenc}		

\setlrmarginsandblock{2cm}{2cm}{*}
\setulmarginsandblock{2cm}{2cm}{*}
\checkandfixthelayout

\setlength{\parindent}{1.3cm}
\setlength{\parskip}{0.2cm}

\SingleSpacing

\begin{document}
	
	\selectlanguage{brazil}
	
	\frenchspacing 
	
	\begin{center}
		\LARGE PLANTAS MEDICINAIS UTILIZADAS NA\\ZONA RURAL DE COLORADO DO
		OESTE,\\NO ESTADO DE RONDÔNIA
		
		\normalsize
	Rafael José Silva de Freitas\footnote{(Acadêmico/IFRO)-rafaelsilvafreitas449@gmail.com.Graduando do Curso de Licenciatura em Ciências biológicas do IFRO, Campus Colorado do Oeste.} 
	Ricardo Mendes Ferreira\footnote{(Acadêmico/IFRO)- ricardomendes695@gmail.com.Graduando do Curso de Licenciatura em Ciências biológicas do IFRO, Campus Colorado do Oeste.} 
	Ranieli Dos Anjos de Souza\footnote{(Professora/IFRO)- ranieli.muler@ifro.edu.br.Formada em Biologia e Mestra em Geoprocessamento. Docente do curso de Licenciatura em Ciências Biológicas do IFRO, campus Colorado do Oeste.} 
	\end{center}
	
	\noindent Algumas pessoas fazem usos de plantas para fins medicinais,
	terapêuticos para tratamento de dores, alívio, cura e prevenções de algumas
	doenças, sendo uma técnica utilizada há muito tempo pela a humanidade. Apesar de
	não haver estudo mais aprofundado sobre sua eficiência terapêutica, a população
	utiliza seus conhecimentos tradicionais que são passados de geração em geração.
	O estudo tem como objetivo analisar a importância do conhecimento popular,
	tradicional e observar as limitações expostas às populações que vivem em zonas
	rurais.As coletas de dados aconteceram na zona rural de Colorado do
	Oeste no Estado de Rondônia, por meio de questionários previamente elaborados.
	Foram escolhidas aleatoriamente 25 unidades domiciliares próximos à BR 435 km
	11, onde os entrevistados receberam informações individualmente, em uma
	linguagem acessível e clara. Após a entrevista realizou-se a tabulação dos dados,
	com o auxilio de uma tabela.Através dos resultados obtidos pela
	pesquisa que apresentava sete espécies de plantas medicinais: alecrim, boldo,
	camomila, capim-limão, erva-doce, hortelã, macela, foi possível observar que o
	boldo é a planta medicinal mais utilizada e a segunda mais usada foi o hortelã. De
	acordo com os entrevistados, a forma optada de obter as plantas é aquela oriunda
	de seu próprio cultivo, pois ressaltaram a importância de cultivá-las em ambientes
	limpos e sem a utilização de agrotóxicos.
	
	
	\vspace{\onelineskip}
	
	\noindent
	\textbf{Palavras-chave}: Plantas medicinais. Prevenções de doenças. Eficiência terapêutica.
	
\end{document}
