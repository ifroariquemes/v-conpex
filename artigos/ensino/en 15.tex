\documentclass[article,12pt,onesidea,4paper,english,brazil]{abntex2}

\usepackage{lmodern, indentfirst, nomencl, color, graphicx, microtype, lipsum}			
\usepackage[T1]{fontenc}		
\usepackage[utf8]{inputenc}		

\setlrmarginsandblock{2cm}{2cm}{*}
\setulmarginsandblock{2cm}{2cm}{*}
\checkandfixthelayout

\setlength{\parindent}{1.3cm}
\setlength{\parskip}{0.2cm}

\SingleSpacing

\begin{document}
	
	\selectlanguage{brazil}
	
	\frenchspacing 
	
	\begin{center}
		\LARGE INTERNET DAS COISAS NO CONTROLE DA AUTOMAÇÃO RESIDENCIAL NO MUNICÍPIO DE JI-PARANÁ–RO\footnote{Trabalho realizado dentro do V Congresso de Pesquisa, Ensino e Extensão (Conpex).}
		
		\normalsize
	Gislaine Rodrigues Ribeiro\footnote{Acadêmica do 4º período do Curso Superior em Análise e Desenvolvimento de Sistemas -
		Campus Ji-Paraná. Instituto Federal de Educação, Ciência e Tecnologia de Rondônia - IFRO.
		gislainegemea1@gmail.com (69)99219-6078 Currículo Lattes disponível em:
		http://lattes.cnpq.br/9694167205065695.} 
	Lucas Pereira Tavares\footnote{Acadêmico do 4º período do Curso Superior em Análise e Desenvolvimento de Sistemas -
		Campus Ji-Paraná. Instituto Federal de Educação, Ciência e Tecnologia de Rondônia - IFRO.
		lucaspereiratavares@gmail.com (69)99339-1113 Currículo Lattes disponível em:
		http://lattes.cnpq.br/687864613604714.} 
	Ilma Rodrigues de Souza Fausto\footnote{Professora Orientadora. Docente/Analista de Sistemas - Campus Ji-Paraná
		Instituto Federal de Educação, Ciência e Tecnologia de Rondônia - IFRO
		(69)99209-1078 Currículo Lattes disponível em: http://lattes.cnpq.br/3193486844184524.} 
	\end{center}
	
	\noindent Com a evolução tecnológica, os processos cada vez mais deixam de ser manuais e
	convergem para a automatização nos mais diversos segmentos. Assim surge a
	Internet das Coisas - IoT, cuja finalidade é conectar objetos usados cotidianamente à
	rede mundial de computadores. Tendo em vista que IoT apresenta a integração de
	objetos, possibilita-se diversas formas de aplicação, como na automação residencial.
	O presente trabalho tem por objetivo identificar as empresas que atuam na
	automação residencial em Ji-Paraná - RO, no controle da iluminação e climatização
	por Internet das Coisas, bem como apontar as vantagens e desvantagens em ter
	uma “casa inteligente”. Tendo em vista que o conceito de IoT é recente, inicialmente
	realizou-se uma pesquisa exploratória acerca do assunto, após isso, efetuou-se uma
	pesquisa bibliográfica utilizando livros e a internet, além da pesquisa descritiva,
	através da realização de entrevista para identificar as empresas que fornecem o
	serviço de automação residencial para controle da iluminação e climatização. Como
	resultado, verificou-se que há apenas uma empresa na cidade que realiza a
	automação nas residências. Atualmente existem 7 casas com controle de
	climatização e iluminação automatizados, além de 3 em execução. No entanto, a
	expectativa é que este número aumente e mais pessoas interessem em investir
	nesse quesito tecnológico. Aqueles que aderiram à esta tecnologia relatam
	satisfação, salientando que muito além do conforto, está a comodidade
	proporcionada, melhorando a iluminação do ambiente, que fica também idealmente
	climatizada. O custo pode ainda ser uma limitação para disseminar o uso, embora o
	valor é semelhante ao restante do país e tem-se reduzido, viabilizando a
	implantação. Deste modo, constata-se que o setor é bastante inovador e promissor,
	tornando-se aos poucos parte da realidade da população, embora poucas empresas
	fornecem os serviços em Ji-Paraná.
	\vspace{\onelineskip}
	
	\noindent
	\textbf{Palavras-chave}: Automação. Residências. IoT.
	
\end{document}
