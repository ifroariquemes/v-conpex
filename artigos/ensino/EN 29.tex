\documentclass[article,12pt,onesidea,4paper,english,brazil]{abntex2}

\usepackage{lmodern, indentfirst, nomencl, color, graphicx, microtype, lipsum,textcomp}			
\usepackage[T1]{fontenc}		
\usepackage[utf8]{inputenc}		

\setlrmarginsandblock{2cm}{2cm}{*}
\setulmarginsandblock{2cm}{2cm}{*}
\checkandfixthelayout

\setlength{\parindent}{1.3cm}
\setlength{\parskip}{0.2cm}

\SingleSpacing

\begin{document}
	
	\selectlanguage{brazil}
	
	\frenchspacing 
	
	\begin{center}
		\LARGE PRÁTICA DE MICROBIOLOGIA INTEGRADA À ATIVIDADE PEDAGÓGICA NA
		AGROINDÚSTRIA DO IFRO CAMPUS COLORADO DO OESTE - RO\footnote{Trabalho realizado dentro de práticas exitosas no ensino.}
		
		\normalsize
	Donizete Alves de Lima Júnior\footnote{Discente do curso Técnico em Alimentos donny\_alves@hotmail.com, Campus Colorado do Oeste.} 
	Josiane Capellaro Varela\footnote{Discente do curso Técnico em Alimentos, josy\_capellaro@hotmail.com, Campus Colorado do Oeste.} 
	Nélio Ranieli Ferreira de Paula\footnote{Orientador e professor, nelio.ferreira@ifro.edu.br, Campus Colorado do Oeste.} 
	\end{center}
	
	\noindent Durante o ensino prático de Microbiologia Alimentar, realizado na disciplina de
	Microbiologia de Alimentos, foi possível a percepção e importância desta atividade
	para a indústria alimentícia, uma vez que esta técnica vem melhorar e atender os
	requisitos contemporâneos de saúde e segurança encontradas dentro dos
	parâmetros dos 5 S. Assim, o objetivo geral desta prática foi promover a discussão e
	fixação do conhecimento técnico-científico e pedagógico no laboratório agroindustrial
	no IFRO Campus Colorado do Oeste – RO sobre os fatores intrínsecos e
	extrínsecos que controlam o desenvolvimento microbiano das bactérias que possam
	estar presentes dentro de uma agroindústria. Dentro do ensino prático, foram
	realizados diversos processos metodológicos, como: Escolha do Local Avaliado,
	Metodologia e Materiais Utilizados, Preparo do Material, Esterilização, Meio de
	Cultura, Identificação, Coleta, Análises, Avaliação e discussão de resultados. Para o
	desenvolvimento desta metodologia, foi aplicada as boas práticas de análises
	microbiológicas para a indústria alimentar agregando saúde e segurança ao
	mercado consumidor. Também foram utilizados outros materiais específicos, como:
	álcool 70\%, Placas de Petri, Estufa, Autoclave, Contador de Colônias, Marcadores
	entre outros. Houve 100\% de aprendizagem, pois ficou evidente que o controle de
	atividade microbiológica dentro de uma agroindústria alimentar é essencial para a
	qualidade do produto final.
	
	\vspace{\onelineskip}
	
	\noindent
	\textbf{Palavras-chave}: Microbiologia. Alimentos. Aprendizagem.
	
\end{document}
